\setcounter{chapter}{4}
\chapter{Applications to Classical Mechanics}
\pagebreak[4]
\section{p153 - Exercise}
\begin{tcolorbox}
If $\mu^{\alpha}$ are the contravariant components of a unit vector in a surface $S$, show that $\mu^{\alpha}f_{\alpha}$ is the physical component of acceleration in the direction tangent to $S$ defined by $\mu^{\alpha}$.
\end{tcolorbox}
As we are in an Euclidean space we can interpret $a_{mn}\mu^{\alpha}f^{\alpha}$ as $\left|\mu\right|\left|f\right|\cos\theta $ with $\theta$ the angle between the two vectors. As $\left|\mu\right|=1$ we have
\begin{align}
a_{mn}\mu^{\alpha}f^{\alpha}&= \mu^{\alpha}f_{\alpha}\\
&= \left|f\right|\cos\theta 
\end{align}
which is the projection of the vector $f$ on the unit vector $\mu$.
$$\blacklozenge$$
\newpage

\section{p154 - Clarification to 5.226.}
\begin{tcolorbox}
$$\mathbf{\text{5.226.}\spatie v\dv{v}{s}=0,\quad \overline{\kappa}v^2=0}$$
Assuming that the particle is not at rest $v\ne 0$, and therefore $\overline{\kappa}=0$. \textit{\textbf{Since this implies that the curve is a geodesic}...}
\end{tcolorbox}
The assertion in bold is a direct consequence $$\mathbf{\text{2.513.}}\spatie \frac{\delta \dv{x^r}{s}}{\delta s}=0$$ 
As in $ \mathbf{5.233}$ we have $\frac{\delta \lambda^{\alpha}}{\delta s}=\frac{\delta \dv{x^{\alpha}}{s}}{\delta s}=0$, the considered curve follows the geodesic curve.
$$\blacklozenge$$
\newpage

\section{p155 - Exercise}
\begin{tcolorbox}
Show that in relativity the force $4$-vector $X^r$ lies along the first normal of the trajectory in space-time. Express the first curvature in terms of the proper mass $m$ of the particle and the magnitude $X$ of $ X^r$.
\end{tcolorbox}
Let us recall the first Frenet formula $\mathbf{2.705}$ without forgetting that the metric form is not positive-definite, $$\frac{\delta \lambda^r}{\delta s}=\kappa\nu^r,\quad \epsilon_{(1)}\nu_n\nu^n=1$$ As $\mathbf{5.299}$ $$m\frac{\delta \lambda^r}{\delta s}=X^r$$ it is clear that $X^r = m\kappa\nu^r$ and is collinear with the first normal.
\begin{align}
X^r &= m\kappa\nu^r\\
\times \quad a_{mr}X^m\quad\Rightarrow\spatie \underbrace{a_{mr}X^mX^r}_{=\left(X^1\right)^2+\left(X^2\right)^2+\left(X^3\right)^2-\left(X^4\right)^2} &= m\kappa \underbrace{a_{mr}\nu^m\nu^r}_{= \epsilon_{(1)}}
\end{align}
\textbf{$$\Rightarrow\spatie \kappa = \epsilon_{(1)}\frac{\left(X^1\right)^2+\left(X^2\right)^2+\left(X^3\right)^2-\left(X^4\right)^2}{m}
$$}
$$\blacklozenge$$
\newpage


\section{p156 - Clarification}
\begin{tcolorbox}
Interpretation of 
$$\mathbf{5.231.}\spatie M_{rs}=\epsilon_{rsn}M_n=z_rF_s-z_sF_r$$
\end{tcolorbox}
What do the $M_{rs}$ represent?
\begin{figure}[H]

\begin{tikzpicture}[scale=0.75]
\tikzstyle{left-hand-mirror} = [
    draw,
    postaction=decorate, 
    decoration={
        markings,
        mark=between positions 0.015 and 0.98 step 0.1072 with {\draw (0,0)--(60:3pt);}
    }
]  
\coordinate (O) at (0,0);
\coordinate (X) at (-5,-5);
\coordinate (Y) at (10,0);
\coordinate (Z) at (0,10);
\draw [-{Latex[length=3mm]}] (O) -- (X);
\draw [-{Latex[length=3mm]}] (O) -- (Y);
\draw [-{Latex[length=3mm]}] (O) -- (Z);
\node[label=north west:$z_1$] at (X) {};
\node[label=north east:$z_2$] at (Y) {};
\node[label=north west:$z_3$] at (Z) {};
\coordinate (P) at (5,3);
\draw [-{Latex[length=3mm]},ultra thick] (O) -- (P);
\node[label=north west:$P$] at (P) {};
\coordinate (F) at (8,7) {};
\draw [-{Latex[length=3mm]}, ultra thick] (P) -- (F);
\node[above right] at (F) {$\overrightarrow{F}$};
\coordinate (Fp) at (8-5,7-3) {};
\draw [-{Latex[length=3mm]},dashdotted,ultra thick] (O) -- (Fp);
\node[above right,] at (Fp) {$\overrightarrow{F^{'}}$};
\coordinate (Px) at (-2,-2) {};
\coordinate (Py) at (6.8,0) {};
\node[label=north west:$P_1$] at (Px) {};
\node[label=north west:$P_2$] at (Py) {};
\coordinate (Pp) at (5,-2) {};
\draw [dashed] (Pp) -- (P);
\draw [dashed] (Pp) -- (Px);
\draw [dashed] (Pp) -- (Py);
\coordinate (Fpp) at (3,-1) {};
\coordinate (Fx) at (-1,-1) {} {};
\coordinate (Fy) at (4,0) {} {};
\node[label=north west:$F_1$] at (Fx) {};
\node[label=north west:$F_2$] at (Fy) {};
\draw[fill = black]  (Fx) circle (0.1);
\draw[fill = black]  (Fy) circle (0.1);
\draw [dashed] (Fp) -- (Fpp);
\draw [dashed] (Fpp) -- (Fx);
\draw [dashed] (Fpp) -- (Fy);

\coordinate (Fppp) at (6,-1) {} {};
\coordinate (Pppp) at (2,-2) {} {} {};
\node[{anchor=north west }] at (Fppp) {$\overrightarrow{F_1}$};
\node[{anchor=south east }] at (Pppp) {$\overrightarrow{F_2}$};
\draw [-{Latex[length=3mm]}, ultra thick] (O) -- (Px);
\draw [-{Latex[length=3mm]},ultra thick] (O) -- (Py);
\draw [-{Latex[length=3mm]}, ultra thick] (Px) -- (Pppp);
\draw [-{Latex[length=3mm]},ultra thick] (Py) -- (Fppp);
%\node[label=north west:$K$] at (Fppp) {};
%\node[label=north west:$S$] at (Pppp) {};
\draw [dashed] (Fp) -- (Fpp);
\draw [dashed] (Fppp) -- (Fx);
\draw [dashed] (Pppp) -- (Fy);
%\filldraw[ultra thick, gray!10] (Px) -- (Pppp) -- (Fy) -- (O) -- (Px) -- cycle;
%\filldraw[ultra thick,gray!20] (Fx) -- (Fppp) -- (Py) -- (O) -- (Px) -- cycle;
%\draw[ultra thick, gray!80] (Px) -- (Pppp) -- (Fy) -- (O) -- (Px) -- cycle;
%\draw[ultra thick,gray!80] (Fx) -- (Fppp) -- (Py) -- (O) -- (Px) -- cycle;
\coordinate (Vp1) at (0,6) {} {};
\coordinate (Vp2) at (0,9) {} {} {};
\draw [-{Latex[length=3mm]}, ultra thick] (O) -- (Vp1);
\draw [-{Latex[length=3mm]}, ultra thick] (O) -- (Vp2);
\node[{anchor=north west }] at (Vp1) {$\overrightarrow{P_1}\times\overrightarrow{F_2}$};
\node[{anchor=north west }] at (Vp2) {-$\overrightarrow{P_2}\times\overrightarrow{F_1}$};
\draw[fill=white]  (Py) circle (0.1);
\draw[fill=white]  (Px) circle (0.1);
\draw[fill=white]  (P) circle (0.1);
\draw[decoration={markings, mark=at position 0.1 with {\arrow[scale = 1.5]{latex[]}}},
    postaction={decorate}](0,4.3) ellipse (1 and 0.2);
    \draw[decoration={markings, mark=at position 0.1 with {\arrow[scale = 1.5]{latex[reversed]}}},
    postaction={decorate}](0,7.3) ellipse (1 and 0.2);
 \coordinate (Qx) at (3.9,1.7) {} {};
\coordinate (Qy) at (8,3) {} {} {};
\draw [-{Latex[length=3mm]},dotted, ultra thick] (P) -- (Qx);
\draw [-{Latex[length=3mm]},dotted, ultra thick] (P) -- (Qy);
\node[{anchor=north west }] at (Qx) {$\overrightarrow{F_1}$};
\node[{anchor=north west }] at (Qy) {$\overrightarrow{F_2}$};
\end{tikzpicture}
\caption{Interpretation of the tensor moment $M_{12}$}
\label{fig:fig_p156_5320}
\end{figure}
Let's consider a mass point $P$ on which a force $\overrightarrow{F}$ is acting. The force has components $\left(F_x,F_y,F_z\right)$ in the  space $V^{'}_3$ (which is by the way not the space $V_3$ of the considered mass point).\\
Let's investigate the element $M_{12}$ of the \textit{tensor moment}.\\
$P_1F_2\overrightarrow{e_3}$ is the vector product $\overrightarrow{P_1}\times\overrightarrow{F_2}$ and is as such the torque of the component $F_2$ of $\overrightarrow{F}$ acting on the mass point situated at $P_1$. The origin being fixed, $\overrightarrow{F_2}$ tries to move $P_1$, clockwise along the $z_3$ axis. The same is true for the component $\overrightarrow{F_1}$ acting on the mass point situated at $P_2$, and is represented here by the vector $- \overrightarrow{P_2}\times\overrightarrow{F_1}$ ($\overrightarrow{F_1}$ tries to move  $P_2$, counter clockwise along the $z_3$ axis). \\
Hence, $P_1F_2-P_2F_1$ is the net force trying to move the point $P$ along the $z_3$ axis (i.e. in the plane $\parallel$ with the $z_3=0$ plane).
$$\blacklozenge$$
\newpage


\section{p156 - Clarification}
\begin{tcolorbox}
$$\mathbf{5.234.}\spatie \dv{h_r}{t}= M_r$$
\end{tcolorbox}
\begin{align}
h_r &= m\epsilon_{rmn}z_mv_n\\
\Rightarrow \spatie \dv{h_r}{t} &= m\epsilon_{rmn}\dv{z_m}{t} v_n+m\epsilon_{rmn}z_m\dv{v_n}{t}\\
&= m\underbrace{\epsilon_{rmn}v_m v_n}_{=0}+\underbrace{\epsilon_{rmn}z_mF_n}_{=M_r}\\
&=M_r
\end{align}
$$\blacklozenge$$
\newpage



\section{p158-159 - Clarification}
\begin{tcolorbox}
$$\mathbf{5.313.}\spatie \omega_{rs}= -\omega_{sr}$$ From 5.310 and the vector character of $v_r$ and $z_r$ (for transformations which do not change the origin), \textbf{it follows that $\omega_{rs} $ is a Cartesian tensor of second order}.
\end{tcolorbox}
Be 
\begin{align}
v^{}_r = -\omega^{}_{rn}z^{}_n
\end{align}
Considering orthogonal transformation in a flat space $z^{'}_m = A_{mr}z^{}_r+B_m$ with  $B_m=0$ as we consider only transformations which do not change the origin. Differentiation with the parameter $t$ gives 
\begin{align}
v^{'}_m &= A_{mr}v^{}_r\\
&= -\omega^{}_{rn}A_{mr}z^{}_n\\
\end{align}
But $z^{'}_q = A_{qr}z^{}_r\quad\Rightarrow \quad A_{qn}z^{'}_q = A_{qn}A_{qr}z^{}_r\quad\Rightarrow \quad A_{qn}z^{'}_q = z^{}_n$ 
Hence
\begin{align}
v^{'}_m &= -\omega^{}_{rn}A_{mr}z^{}_n\\
&= -\underbrace{\omega^{}_{rn}A_{mr}A_{qn}}_{\overset{\underset{\mathrm{def}}{}}{=}\omega_{mq}^{'}}z^{'}_q\\
v^{'}_m &= -\omega_{mq}^{'}z^{'}_q
\end{align}
$$\blacklozenge$$
\newpage


\section{p159 - Exercise}
\begin{tcolorbox}
Show that if a rigid body rotates about the point $z_r=b_r$ as fixed point, the velociy of a general point of the body is given by $$v_r=-\omega_{rm}\left(z_m-b_m\right)$$
\end{tcolorbox}
By $\mathbf{5.302. }$:
\begin{align}
\left(z^{(1)}_m-z^{(2)}_m\right)\left(dz^{(1)}_m-dz^{(2)}_m\right)=0
\end{align}
At the fixed point we have $z^{(2)}_m=b_m$ and $dz^{(2)}_m=0$, hence
\begin{align}
\left(z^{(1)}_m-b_m\right)\left(dz^{(1)}_m\right)=0\\
\Rightarrow\spatie z^{(1)}_mdz^{(1)}_m =b_mdz^{(1)}_m
\end{align}
As this is true for any point of the rigid mass, expanding (1) and using (3) we get when dividing by $dt$
\begin{align}
\left(z^{(2)}_m-b_m\right)v^{(1)}_m+\left(z^{(1)}_m-b_m\right)v^{(2)}_m=0
\end{align}
Taking twice the partial derivative $\frac{\partial^2}{\partial z^{(1)}_p\partial z^{(1)}_q}$ we get
\begin{align}
\left(z^{(2)}_m-b_m\right)\frac{\partial^2 v_m}{\partial z^{(1)}_p\partial z^{(1)}_q}=0
\end{align}
As this is true for any arbitrary point in the rigid body we get
\begin{align}
\frac{\partial^2 v_m}{\partial z^{(1)}_p\partial z^{(1)}_q}=0\\
\Rightarrow\spatie v_m= K_{mr}z_r+B_m
\end{align}
At the fixed point we have 
\begin{align}
 K_{mr}b_r+B_m =0
\end{align}
Plugging this in (7)
\begin{align}
 v_m= K_{mr}\left(z_r-b_m\right)
\end{align}
Putting $K_{mr}=-\omega_{mr}$ gives us indeed the asked expression.
$$\blacklozenge$$
\newpage


\section{p161 - Clarification}
\begin{tcolorbox}
$$\mathbf{5.325.}\spatie \Omega_{np}\sum \left(mf_nz_p\right)= \Omega_{np}\sum F_nz_p $$
and hence, since $\Omega_{np}$ is arbitrary,
$$\mathbf{5.326.}\spatie \sum m\left(f_nz_p-f_pz_n\right)= \sum \left(F_nz_p-F_pz_n \right)$$
\end{tcolorbox}
To be complete the following step should be inserted

\begin{align}
\Omega_{np}\sum \left(mf_nz_p\right)&= \Omega_{np}\sum F_nz_p \\
\text{As }\Omega_{np}\text{ is skew-symmetric:}\spatie -\Omega_{np}\sum \left(mf_pz_n\right)&= -\Omega_{np}\sum F_pz_n \\
\text{(1)+(2) }\spatie \Omega_{np}\sum m\left(f_nz_p-f_pz_n\right)&= \Omega_{np}\sum\left( F_nz_p -F_pz_n\right)
\end{align}
and hence, since $\Omega_{np}$ is arbitrary,
$$\mathbf{5.326.}\spatie \sum m\left(f_nz_p-f_pz_n\right)= \sum \left(F_nz_p-F_pz_n \right)$$
$$\blacklozenge$$
\newpage

\section{p161 - Clarification}
\begin{tcolorbox}
$$\mathbf{5.329.}\spatie h_{np}=\sum m\left(\omega_{nq} z_q z_p -\omega_{pq} z_qz_n\right)$$
$$= J_{npqr}\omega_{rq}$$
where
$$\mathbf{5.330.}\spatie J_{npqr}= \sum m\left(\delta_{nr}z_qz_p-\delta_{pr}z_nz_q \right)$$
\end{tcolorbox}
\begin{align}
h_{np}&=\sum m\left(\omega_{nq} z_q z_p -\omega_{pq} z_qz_n\right)\\
&=\sum m\left(\omega_{rq}\delta_{rn} z_q z_p -\omega_{rq} \delta_{rp}z_qz_n\right)\\
&=\omega_{rq}\sum m\left(\delta_{rn} z_q z_p -\delta_{rp}z_qz_n\right)\\
&=J_{npqr}\omega_{rq}
\end{align}
$$\blacklozenge$$
$$J^{-1}={\begin{pmatrix}{\dfrac {x}{r}}&{\dfrac {y}{r}}&{\dfrac {z}{r}}\\\\{\dfrac {xz}{r^{2}{\sqrt {x^{2}+y^{2}}}}}&{\dfrac {yz}{r^{2}{\sqrt {x^{2}+y^{2}}}}}&{\dfrac {-(x^{2}+y^{2})}{r^{2}{\sqrt {x^{2}+y^{2}}}}}\\\\{\dfrac {-y}{x^{2}+y^{2}}}&{\dfrac {x}{x^{2}+y^{2}}}&0\end{pmatrix}}$$
\newpage


\section{p186 - Exercise 1}
\begin{tcolorbox}
If a vector at the point with coordinates $\left(1,1,1\right)$ in Euclidean $3$-space has components $\left(3,-1,2\right)$, find the contravariant, covariant and physical components in spherical polar coordinates.
\end{tcolorbox}
The tensor $T_n$ to consider is $\left(3,-1,2\right) - \left(1,1,1\right)= \left(2,-2,1\right)$.\\
The Jacobian matrix for the transformation $z^n \rightarrow x^k$, evaluated at the point $\left(1,1,1\right)$ is 
\begin{align}
J_{\left(1,1,1\right)}&={\begin{pmatrix}{\dfrac {x}{r}}&{\dfrac {y}{r}}&{\dfrac {z}{r}}\\\\{\dfrac {xz}{r^{2}{\sqrt {x^{2}+y^{2}}}}}&{\dfrac {yz}{r^{2}{\sqrt {x^{2}+y^{2}}}}}&{\dfrac {-(x^{2}+y^{2})}{r^{2}{\sqrt {x^{2}+y^{2}}}}}\\\\{\dfrac {-y}{x^{2}+y^{2}}}&{\dfrac {x}{x^{2}+y^{2}}}&0\end{pmatrix}}\\
&=\begin{pmatrix}\dfrac {1}{\sqrt{3}}&\dfrac {1}{\sqrt{3}}&\dfrac {1}{\sqrt{3}}\\\\ \dfrac {1}{3\sqrt{2}}& \dfrac {1}{3\sqrt{2}}&-\dfrac {\sqrt{2}}{3}\\\\ -\dfrac {1}{2}&\dfrac {1}{2}&0\end{pmatrix}\\
\Rightarrow \spatie 
\begin{pmatrix}
r\\
\theta\\
\phi
\end{pmatrix}_{T^{'n}}&=\begin{pmatrix}\dfrac {1}{\sqrt{3}}&\dfrac {1}{\sqrt{3}}&\dfrac {1}{\sqrt{3}}\\\\ \dfrac {1}{3\sqrt{2}}& \dfrac {1}{3\sqrt{2}}&-\dfrac {\sqrt{2}}{3}\\\\ -\dfrac {1}{2}&\dfrac {1}{2}&0\end{pmatrix}\begin{pmatrix}
2\\
-2\\
1
\end{pmatrix}\\
&=\begin{pmatrix}
\dfrac {1}{\sqrt{3}}\\
-\dfrac {\sqrt{2}}{3}\\
-2
\end{pmatrix}
\end{align}
We have the metric tensor evaluated at $\left(1,1,1\right)$
\begin{align}
a_{mn} &= \begin{pmatrix}
1&0&0\\\\
0&r^2&0\\\\
0&0&r^2\sin^2\theta\\\\
\end{pmatrix}=\begin{pmatrix}
1&0&0\\\\
0&3&0\\\\
0&0&2\\\\
\end{pmatrix}\\
\Rightarrow \spatie 
\begin{pmatrix}
r\\
\theta\\
\phi
\end{pmatrix}_{T^{'}_n}&=\begin{pmatrix}
1&0&0\\\\
0&3&0\\\\
0&0&2\\\\
\end{pmatrix}\begin{pmatrix}
\dfrac {1}{\sqrt{3}}\\
-\dfrac {\sqrt{2}}{3}\\
-2
\end{pmatrix}\\
&=\begin{pmatrix}
\dfrac {1}{\sqrt{3}}\\
-\sqrt{2}\\
-4
\end{pmatrix}
\end{align}
And the physical components 
\begin{align}
\begin{pmatrix}
r\\
\theta\\
\phi
\end{pmatrix}_{T^{'}_{ph.}}&=\begin{pmatrix}
1&0&0\\\\
0&\frac{1}{\sqrt{3}}&0\\\\
0&0&\frac{1}{\sqrt{2}}\\\\
\end{pmatrix}\begin{pmatrix}
\dfrac {1}{\sqrt{3}}\\
-\sqrt{2}\\
-4
\end{pmatrix}\\
&=\begin{pmatrix}
\dfrac {1}{\sqrt{3}}\\
-\sqrt{\dfrac {{2}}{{3}}}\\
-2\sqrt{2}
\end{pmatrix}
\end{align}
Another way to find the physical components is to project orthogonally the tensor on the unit vectors of a local Cartesian coordinate system, oriented along the unit vectors $\overline{e}_r,\overline{e}_{\theta},\overline{e}_{\phi}$ corresponding to the vector $P \left(1,1,1\right)$ with modulus $\left|P \right|=\sqrt{3}$. 
We have for the tensor $T_n (2,-2,1)$ with modulus $\left|T_n \right|=3$ as component along $\overline{e}_r$:
\begin{align}
\left|T_n \right|\cos \alpha &= \left|T_n \right|\frac{\left<T_n,P  \right>}{\left|T_n \right|\left|P \right|}\\
&= \left|T_n \right|\frac{2-2+1}{\left|T_n \right|\left|P \right|}\\
&= \frac{1}{\sqrt{3}}
\end{align}
For the component along $\overline{e}_{\theta}$ we first have to determine the vector $\overline{e}_{\theta}$. As first equation we have the orthogonality condition with $\overline{e}_r$ and putting $\overline{e}_{\theta} = (a,b,c)$, get $\left<\overline{e}_r,\overline{e}_{\theta}  \right>=  a+b+c=0$. As $\overline{e}_{\theta}$ lies in the plane $(1,1,0)-(0,0,0)-(0,0,1)$ we can put $a=b$ and get $\overline{e}_{\theta} =  \frac{1}{\sqrt{6}}\left(1,1,-2\right)$ and get for the tensor $T_n (2,-2,1)$  as component along $\overline{e}_{\theta}$:
\begin{align}
\left|T_n \right|\cos \beta &= \left|T_n \right|\frac{\left<T_n,\overline{e}_{\theta}  \right>}{\left|T_n \right|}\\
&= \left|T_n \right|\frac{2-2-2}{\left|T_n \right|\sqrt{6}}\\
&= -\frac{\sqrt{2}}{\sqrt{3}}
\end{align}
For the component along $\overline{e}_{\phi}$ we first have to determine the vector $\overline{e}_{\phi}$. As first equation we have the orthogonality condition with the pair $\overline{e}_r,\overline{e}_{\theta}$  and  get $\overline{e}_{\phi} = \overline{e}_r \times \overline{e}_{\theta}  =  \frac{1}{\sqrt{3}\sqrt{6}}\left( -3,3,0\right)= \left( -\frac{1}{\sqrt{2}},\frac{1}{\sqrt{2}},0\right)$.\\
For the tensor $T_n (2,-2,1)$  as component along $\overline{e}_{\phi}$:
\begin{align}
\left|T_n \right|\cos \gamma &= \left|T_n \right|\frac{\left<T_n,\overline{e}_{\phi}  \right>}{\left|T_n \right|}\\
&= \left|T_n \right|\frac{-2-2}{\left|T_n \right|\sqrt{2}}\\
&= -\frac{4}{\sqrt{2}}\\
&= -2\sqrt{2}
\end{align}
giving
\begin{align}
\begin{pmatrix}
r\\
\theta\\
\phi
\end{pmatrix}_{T^{'}_{ph.}}
&=\begin{pmatrix}
\dfrac {1}{\sqrt{3}}\\
-\sqrt{\dfrac {{2}}{{3}}}\\
-2\sqrt{2}
\end{pmatrix}
\end{align}
as in (9).
$$\blacklozenge$$
\newpage

\section{p181 and p182 - Clarification Figures 13., 14. and 15.}
\begin{tcolorbox}
There are several ways to show the homeomorphism of the configuration space of a rigid body with fixed point.
\end{tcolorbox}
\begin{figure}[H]
    \centering
    \subfloat[]{\begin{tikzpicture}[scale=0.25]
\coordinate (O) at (0,0);
\node[label=south :$O$] at (O) {};
\coordinate (X) at (-9.5,-8) {} {};
\coordinate (Y) at (17,0) {} {};
\coordinate (Z) at (0,16.5) {} {};

\coordinate (X0) at (-5.35,-4.5) {} {} ;
\coordinate (Y0) at (9,0) {} ;
\coordinate (Z0) at (0,9) {} {} {} ;
\coordinate (XY) at (5,-4.5) {} {} {} ;
\coordinate (YZ) at (9,9) {} {} ;
\coordinate (XZ) at (-5.35,4.5) {} {};
\coordinate (XYZ) at (5,4.5) {} {} {};


\coordinate (HXY1) at (-0.64,-4.5) {} {} {};
\coordinate (HXY2) at (-0.64,4.5) {} {} {};
\coordinate (HZY1) at (4.5,9) {} {} {};
\coordinate (HZY2) at (4.5,0) {} {} {};


\draw [-{Latex[length=2mm]}] (O) -- (X);
\draw [-{Latex[length=2mm]}] (O) -- (Y);
\draw [-{Latex[length=2mm]}] (O) -- (Z);
\node[label=south east:$\theta$] at (X) {};
\node[label=south west:$\phi$] at (Y) {};
\node[label=south east:$\psi$] at (Z) {};
\node[label=north west:$\pi$] at (X0) {};
\node[label=south :$2\pi$] at (Y0) {};
\node[label=north east:$2\pi$] at (Z0) {};
%\node (Sb) [rectangle, minimum width=3cm, minimum height=1cm,draw=black, pattern color=black, pattern = north east lines]{} ;

\draw [] (X0) -- (XY);
\draw [] (Y0) -- (YZ);
\draw [] (Z0) -- (XZ);
\draw [] (X0) -- (XZ);
\draw [] (Y0) -- (XY);
\draw [] (XY) -- (XYZ);
\draw [] (XZ) -- (XYZ);
\draw [] (YZ) -- (XYZ);
\draw [] (YZ) -- (Z0);
\draw [] (O) -- (Z0);
\draw [] (O) -- (X0);
\draw [] (O) -- (Y0);

\draw[fill=white]  (X0) circle (0.1);
\draw[fill=white]  (Y0) circle (0.1);
\draw[fill=white]  (Z0) circle (0.1);

\draw[fill=white]  (XZ) circle (0.1);
\draw[fill=white]  (YZ) circle (0.1);
\draw[fill=white]  (XZ) circle (0.1);
\draw[fill=white]  (O) circle (0.1);
\draw[fill=white]  (XYZ) circle (0.1);

%\draw[pattern color=black, pattern = dots]  (O) rectangle (YZ) node (v0) {};
%\draw[pattern color=black, pattern = dots]  (X0) rectangle (XYZ) node (v1) {};
\coordinate (plane0) at (7.1,2.2) {};
\coordinate (plane1) at (-2.7,2.2) {};
\draw[fill=white]  (plane1) circle (0.1);
\draw[fill=white]  (plane0) circle (0.1);
%\draw[ decoration={markings, mark=at position 0.75 with {\arrow[scale = 1.]{Latex[length=3mm,reversed]}}},    postaction={decorate}](plane0) .. controls (29.5,-2) and (-28,-4) .. (plane1);
\draw[pattern color=black, pattern = dots]  (HXY1) node (v2) {} -- (HXY2) -- (HZY1) -- (HZY2) -- (HZY2) -- cycle;
\draw  (YZ) edge (XY);
\draw  (XYZ) edge (Y0);
\draw  (XZ) edge (O);
\draw  (Z0) edge (X0);
\node[label=north:$\mathbf{\Phi_{2\pi}}$] at (plane0) {};
\node[label=south:$\mathbf{\Phi_{0}}$] at (plane1) {};
\draw [ dashed,decoration={markings, mark=at position 0.55 with {\arrow[scale = 1.]{Latex[length=2mm]}}},    postaction={decorate}](plane1) arc(-70:261:-20cm and -5cm);
\end{tikzpicture}}
	\
    \subfloat[]{
\begin{tikzpicture}[scale=0.5]
\coordinate (O1) at (0,0);
\draw [thick] (O1) ellipse (6 and 0.6);
\coordinate (O2) at (0,-3);
\draw[dashed]  (O2) ellipse (6 and 0.6);
\coordinate (Om) at (0,-1.5);
\coordinate (O1s) at (0,0);
\draw[dashed]  (O1s) ellipse (3 and 0.3);
\coordinate (O2s) at (0,-3);
\draw[dashed]  (O2s) ellipse (3 and 0.3);
\coordinate (Oms) at (0,-1.5);
\draw[pattern color=black, pattern = dots]  (O1) ellipse (3 and 0.3);

%\draw  (O1) ellipse (1 and 0.1);
%\draw[dashed]  (O2) ellipse (1 and 0.1);
\coordinate (Plu) at (-6,0);
\coordinate (Pru) at (6,0);
\coordinate (Pld) at (-6,-3);
\coordinate (Prd) at (6,-3);
\coordinate (Plm) at (-6,-1.5);
\coordinate (Prm) at (6,-1.5);
\coordinate (Plds) at (-3,-3);
\coordinate (Prds) at (3,-3);
\coordinate (Plus) at (-3,0);
\coordinate (Prus) at (3,-0);

\coordinate (Qlu) at (-1,0);
\coordinate (Qru) at (1,0);
\coordinate (Qld) at (-1,-3);
\coordinate (Qrd) at (1,-3);
\draw [-{Latex[length=3mm]}] (Pld) -- (Plu);
\draw [] (Pru) -- (Prd);
\draw [dashed] (Plds) -- (Plus);
\draw [dashed] (Prds) -- (Prus);
%\draw [dashed] (Qlu) -- (Qld);
%\draw [dashed] (Qru) -- (Qrd);
\coordinate (Pu0) at (-1.5,-0.25) {};
\coordinate (Pd0) at (-1.5,-3.25) {};
\coordinate (Pu) at (-3.5,-0.5) {};
\coordinate (Pdl) at (-3.5,-3.5) {} {};
\coordinate (Pdr) at (3.5,-3.5) {} {};
\coordinate (Pmu) at (-3.5,-2) {} {};
\coordinate (Pml) at (-1.5,-1.5) {} {};
\draw[thick]  plot[ smooth,tension=.9] coordinates {(Pld) (Pdl) (Pdr) (Prd)};
\draw [dashed] (Plus) -- (Plu);
\draw[pattern color=gray!60, pattern = dots]  (Pu0) node (v2) {} -- (Pu) -- (Pdl) -- (Pd0) -- (Pu0) -- cycle;
\node[label=west:$\mathbf{\psi}$] at (Plm) {};
\node[label=north east :$\theta$] at (Pmu) {};
\draw  plot[ smooth,tension=.7] coordinates {(-3.5,0) (-3.3,-0.23) (-2.5,-0.35)};
\node[label=west :$\mathbf{\phi}$] at (-3.5,0) {};
\draw [-{Latex[length=1mm]}] (Pml) -- (Pmu);
\draw[fill=white]  (O1) circle (0.1);
\draw[fill=white]  (O2) circle (0.1);
%\draw[ decoration={markings, mark=at position 0.7 with {\arrow[scale = 1.5]{Latex[length=3mm,reversed]}}},    postaction={decorate}](O2) .. controls (3.5,-16.5) and (3.5,11) .. (O1);
\node[label=east:$\mathbf{\Psi_{2\pi}}$] at (Prus) {};
\node[label= east:$\mathbf{\Psi_{0}}$] at (Prds) {};
\draw[fill=black]  (Pml) circle (0.051);
\draw [dashed] (O1) -- (Plus);
\draw [dashed] (O1) -- (Pu0);
\coordinate (Om) at (-0.06,-2.6) {};
\draw [ dashed,decoration={markings, mark=at position 0.32 with {\arrow[scale = 1.]{Latex[length=3mm]}}},    postaction={decorate}](Om) arc (20:345:-1 and -4.5);
\end{tikzpicture}}
    \qquad
    \subfloat[]{\begin{tikzpicture}[scale=0.5]
\coordinate (O1) at (0,0);
\draw  (O1) ellipse (6 and 3.6);
\coordinate (t1l) at (-3,1.0);
\coordinate (t1r) at (3,1.0);
\coordinate (t1c1) at  (-4.5,-1.5);
\coordinate (t1c2) at (4.5,-1.5);
\draw (t1l) .. controls  (t1c1)  and (t1c2)  .. (t1r);

\coordinate (t2l) at (-2.7,-0.3) {};
\coordinate (t2r) at (2.7,-0.3) {};
\coordinate (t2c1) at (-4,2) {};
\coordinate (t2c2) at (4,2) {};
\draw (t2l) .. controls  (t2c1)  and (t2c2)  .. (t2r);
\draw[fill=white]  (O1) circle (0.1);
\coordinate (a0) at (0,2) {};
\coordinate (a1) at (2,1.5) {} {};
\draw [] (O1) -- (a0);
\draw [] (O1) -- (a1);
\coordinate (ac1) at (1,2) {} {};
\coordinate (ac2) at (1,2) {} {};

\draw [dashed] (a0) .. controls  (ac1)  and (ac2)  .. (a1);
\coordinate (b0) at (0,1.) {};
\coordinate (b1) at (0.9,0.7) {} {};
\coordinate (bc1) at (0.5,1) {} {} {};
\coordinate (bc2) at (0.5,1) {} {} {};
\coordinate (bc3) at (0,0.5) {} {} {} {};
\draw [-{Latex[length=2mm]}]  (b0) .. controls  (bc1)  and (bc2)  .. (b1);
\coordinate (ac3) at (0.5,0.7) {} {} {};
\node[label=north east:$\mathbf{\psi_{}}$] at (ac3) {};
\coordinate (C1) at (-4.35,-.4);
\draw [dashed] (C1) ellipse (1.55 and 1.4);
\draw [dashed,pattern color=gray!60, pattern = dots] (C1) ellipse (0.75 and 0.65);
\draw [dotted] (C1) ellipse (2.8 and 2.6);

\coordinate (C2a) at (-4.38,0.25) {};
\coordinate (C2b) at (-4.4,1) {};
\coordinate (C3b) at (-5.5,0.5) {};
\coordinate (C3a) at (-5,0) {} {};
\draw [dashed] (C1) -- (C2a);
\draw [-{Latex[length=2mm]}] (C2a) -- (C2b);
\draw [dashed] (C1) -- (C3b);
\draw[fill=black]  (C2a) circle (0.051);
%\draw[fill=black]  (C3a) circle (0.051);
\draw[fill=white]  (C1) circle (0.051);
\draw [-{Latex[length=2mm]}] (C2b) .. controls  (-5.1,0.9)  and (-5.1,0.9)  .. (C3b);
\node[label=south east:$\mathbf{\theta}$] at (C2b) {};
\node[label=north east:$\mathbf{\phi}$] at (C3b) {};
\node at (-7,-2.5) {};
\node[label=south east:$\mathbf{\theta = 2\pi}$] at (-7.5,-2) {};
\node at (-3,-1.5) {};
\node[label=south east:$\mathbf{\theta = \pi}$] at (-5,-1.5) {};
\end{tikzpicture}}
\caption{Homeomorphism of the configuration space of a rigid body with fixed point.}
\label{fig:fig_p181}
\end{figure}
Consider figure $5.2 (a)$. We can stretch like an accordion the cuboid along the $\phi$ axis and bent it so that the planes $\phi=0$ and $\phi=2\pi$ join. We get $(b)$,  a torus with square sections. The dimension $\phi$ is dealt with as a point $P\left( \theta \phi \psi \right)$ in the configuration space  returns to the same point when varying $\phi$ to $\phi+2k\pi$.\\
We can apply the same procedure of stretching and bending for the $psi$ dimension so that the planes $\Psi=0$ and $\Psi=2\pi$ join.
We get $(c)$,  a torus-like object.\\
The only dimension left is $\theta$ which our multi-dimensional crippled mind can't find a way to reshape this pseudo-torus so that when varying $\theta$ we can come back to the same point as started.
$$\blacklozenge$$
\newpage