\setcounter{chapter}{4}
\chapter{Applications to Classical Mechanics}
\pagebreak[4]
\section{p153 - Exercise}
\begin{tcolorbox}
If $\mu^{\alpha}$ are the contravariant components of a unit vector in a surface $S$, show that $\mu^{\alpha}f_{\alpha}$ is the physical component of acceleration in the direction tangent to $S$ defined by $\mu^{\alpha}$.
\end{tcolorbox}
As we are in an Euclidean space we can interpret $a_{mn}\mu^{\alpha}f^{\alpha}$ as $\left|\mu\right|\left|f\right|\cos\theta $ with $\theta$ the angle between the two vectors. As $\left|\mu\right|=1$ we have
\begin{align}
a_{mn}\mu^{\alpha}f^{\alpha}&= \mu^{\alpha}f_{\alpha}\\
&= \left|f\right|\cos\theta 
\end{align}
which is the projection of the vector $f$ on the unit vector $\mu$.
$$\blacklozenge$$
\newpage

\section{p154 - Clarification to 5.226.}
\begin{tcolorbox}
$$\mathbf{\text{5.226.}\spatie v\dv{v}{s}=0,\quad \overline{\kappa}v^2=0}$$
Assuming that the particle is not at rest $v\ne 0$, and therefore $\overline{\kappa}=0$. \textit{\textbf{Since this implies that the curve is a geodesic}...}
\end{tcolorbox}
The assertion in bold is a direct consequence $$\mathbf{\text{2.513.}}\spatie \frac{\delta \dv{x^r}{s}}{\delta s}=0$$ 
As in $ \mathbf{5.233}$ we have $\frac{\delta \lambda^{\alpha}}{\delta s}=\frac{\delta \dv{x^{\alpha}}{s}}{\delta s}=0$, the considered curve follows the geodesic curve.
$$\blacklozenge$$
\newpage

\section{p155 - Exercise}
\begin{tcolorbox}
Show that in relativity the force $4$-vector $X^r$ lies along the first normal of the trajectory in space-time. Express the first curvature in terms of the proper mass $m$ of the particle and the magnitude $X$ of $ X^r$.
\end{tcolorbox}
Let us recall the first Frenet formula $\mathbf{2.705}$ without forgetting that the metric form is not positive-definite, $$\frac{\delta \lambda^r}{\delta s}=\kappa\nu^r,\quad \epsilon_{(1)}\nu_n\nu^n=1$$ As $\mathbf{5.299}$ $$m\frac{\delta \lambda^r}{\delta s}=X^r$$ it is clear that $X^r = m\kappa\nu^r$ and is collinear with the first normal.
\begin{align}
X^r &= m\kappa\nu^r\\
\times \quad a_{mr}X^m\quad\Rightarrow\spatie \underbrace{a_{mr}X^mX^r}_{=\left(X^1\right)^2+\left(X^2\right)^2+\left(X^3\right)^2-\left(X^4\right)^2} &= m\kappa \underbrace{a_{mr}\nu^m\nu^r}_{= \epsilon_{(1)}}
\end{align}
\textbf{$$\Rightarrow\spatie \kappa = \epsilon_{(1)}\frac{\left(X^1\right)^2+\left(X^2\right)^2+\left(X^3\right)^2-\left(X^4\right)^2}{m}
$$}
$$\blacklozenge$$
\newpage


\section{p156 - Clarification}
\begin{tcolorbox}
Interpretation of 
$$\mathbf{5.231.}\spatie M_{rs}=\epsilon_{rsn}M_n=z_rF_s-z_sF_r$$
\end{tcolorbox}
What do the $M_{rs}$ represent?
\begin{figure}[H]

\begin{tikzpicture}[scale=0.75]
\tikzstyle{left-hand-mirror} = [
    draw,
    postaction=decorate, 
    decoration={
        markings,
        mark=between positions 0.015 and 0.98 step 0.1072 with {\draw (0,0)--(60:3pt);}
    }
]  
\coordinate (O) at (0,0);
\coordinate (X) at (-5,-5);
\coordinate (Y) at (10,0);
\coordinate (Z) at (0,10);
\draw [-{Latex[length=3mm]}] (O) -- (X);
\draw [-{Latex[length=3mm]}] (O) -- (Y);
\draw [-{Latex[length=3mm]}] (O) -- (Z);
\node[label=north west:$z_1$] at (X) {};
\node[label=north east:$z_2$] at (Y) {};
\node[label=north west:$z_3$] at (Z) {};
\coordinate (P) at (5,3);
\draw [-{Latex[length=3mm]},ultra thick] (O) -- (P);
\node[label=north west:$P$] at (P) {};
\coordinate (F) at (8,7) {};
\draw [-{Latex[length=3mm]}, ultra thick] (P) -- (F);
\node[above right] at (F) {$\overrightarrow{F}$};
\coordinate (Fp) at (8-5,7-3) {};
\draw [-{Latex[length=3mm]},dashdotted,ultra thick] (O) -- (Fp);
\node[above right,] at (Fp) {$\overrightarrow{F^{'}}$};
\coordinate (Px) at (-2,-2) {};
\coordinate (Py) at (6.8,0) {};
\node[label=north west:$P_1$] at (Px) {};
\node[label=north west:$P_2$] at (Py) {};
\coordinate (Pp) at (5,-2) {};
\draw [dashed] (Pp) -- (P);
\draw [dashed] (Pp) -- (Px);
\draw [dashed] (Pp) -- (Py);
\coordinate (Fpp) at (3,-1) {};
\coordinate (Fx) at (-1,-1) {} {};
\coordinate (Fy) at (4,0) {} {};
\node[label=north west:$F_1$] at (Fx) {};
\node[label=north west:$F_2$] at (Fy) {};
\draw[fill = black]  (Fx) circle (0.1);
\draw[fill = black]  (Fy) circle (0.1);
\draw [dashed] (Fp) -- (Fpp);
\draw [dashed] (Fpp) -- (Fx);
\draw [dashed] (Fpp) -- (Fy);

\coordinate (Fppp) at (6,-1) {} {};
\coordinate (Pppp) at (2,-2) {} {} {};
\node[{anchor=north west }] at (Fppp) {$\overrightarrow{F_1}$};
\node[{anchor=south east }] at (Pppp) {$\overrightarrow{F_2}$};
\draw [-{Latex[length=3mm]}, ultra thick] (O) -- (Px);
\draw [-{Latex[length=3mm]},ultra thick] (O) -- (Py);
\draw [-{Latex[length=3mm]}, ultra thick] (Px) -- (Pppp);
\draw [-{Latex[length=3mm]},ultra thick] (Py) -- (Fppp);
%\node[label=north west:$K$] at (Fppp) {};
%\node[label=north west:$S$] at (Pppp) {};
\draw [dashed] (Fp) -- (Fpp);
\draw [dashed] (Fppp) -- (Fx);
\draw [dashed] (Pppp) -- (Fy);
%\filldraw[ultra thick, gray!10] (Px) -- (Pppp) -- (Fy) -- (O) -- (Px) -- cycle;
%\filldraw[ultra thick,gray!20] (Fx) -- (Fppp) -- (Py) -- (O) -- (Px) -- cycle;
%\draw[ultra thick, gray!80] (Px) -- (Pppp) -- (Fy) -- (O) -- (Px) -- cycle;
%\draw[ultra thick,gray!80] (Fx) -- (Fppp) -- (Py) -- (O) -- (Px) -- cycle;
\coordinate (Vp1) at (0,6) {} {};
\coordinate (Vp2) at (0,9) {} {} {};
\draw [-{Latex[length=3mm]}, ultra thick] (O) -- (Vp1);
\draw [-{Latex[length=3mm]}, ultra thick] (O) -- (Vp2);
\node[{anchor=north west }] at (Vp1) {$\overrightarrow{P_1}\times\overrightarrow{F_2}$};
\node[{anchor=north west }] at (Vp2) {-$\overrightarrow{P_2}\times\overrightarrow{F_1}$};
\draw[fill=white]  (Py) circle (0.1);
\draw[fill=white]  (Px) circle (0.1);
\draw[fill=white]  (P) circle (0.1);
\draw[decoration={markings, mark=at position 0.1 with {\arrow[scale = 1.5]{latex[]}}},
    postaction={decorate}](0,4.3) ellipse (1 and 0.2);
    \draw[decoration={markings, mark=at position 0.1 with {\arrow[scale = 1.5]{latex[reversed]}}},
    postaction={decorate}](0,7.3) ellipse (1 and 0.2);
 \coordinate (Qx) at (3.9,1.7) {} {};
\coordinate (Qy) at (8,3) {} {} {};
\draw [-{Latex[length=3mm]},dotted, ultra thick] (P) -- (Qx);
\draw [-{Latex[length=3mm]},dotted, ultra thick] (P) -- (Qy);
\node[{anchor=north west }] at (Qx) {$\overrightarrow{F_1}$};
\node[{anchor=north west }] at (Qy) {$\overrightarrow{F_2}$};
\end{tikzpicture}
\caption{Interpretation of the tensor moment $M_{12}$}
\label{fig:fig_p156_5320}
\end{figure}
Let's consider a mass point $P$ on which a force $\overrightarrow{F}$ is acting. The force has components $\left(F_x,F_y,F_z\right)$ in the  space $V^{'}_3$ (which is by the way not the space $V_3$ of the considered mass point).\\
Let's investigate the element $M_{12}$ of the \textit{tensor moment}.\\
$P_1F_2\overrightarrow{e_3}$ is the vector product $\overrightarrow{P_1}\times\overrightarrow{F_2}$ and is as such the torque of the component $F_2$ of $\overrightarrow{F}$ acting on the mass point situated at $P_1$. The origin being fixed, $\overrightarrow{F_2}$ tries to move $P_1$, clockwise along the $z_3$ axis. The same is true for the component $\overrightarrow{F_1}$ acting on the mass point situated at $P_2$, and is represented here by the vector $- \overrightarrow{P_2}\times\overrightarrow{F_1}$ ($\overrightarrow{F_1}$ tries to move  $P_2$, counter clockwise along the $z_3$ axis). \\
Hence, $P_1F_2-P_2F_1$ is the net force trying to move the point $P$ along the $z_3$ axis (i.e. in the plane $\parallel$ with the $z_3=0$ plane).
$$\blacklozenge$$
\newpage


\section{p156 - Clarification}
\begin{tcolorbox}
$$\mathbf{5.234.}\spatie \dv{h_r}{t}= M_r$$
\end{tcolorbox}
\begin{align}
h_r &= m\epsilon_{rmn}z_mv_n\\
\Rightarrow \spatie \dv{h_r}{t} &= m\epsilon_{rmn}\dv{z_m}{t} v_n+m\epsilon_{rmn}z_m\dv{v_n}{t}\\
&= m\underbrace{\epsilon_{rmn}v_m v_n}_{=0}+\underbrace{\epsilon_{rmn}z_mF_n}_{=M_r}\\
&=M_r
\end{align}
$$\blacklozenge$$
\newpage



\section{p158-159 - Clarification}
\begin{tcolorbox}
$$\mathbf{5.313.}\spatie \omega_{rs}= -\omega_{sr}$$ From 5.310 and the vector character of $v_r$ and $z_r$ (for transformations which do not change the origin), \textbf{it follows that $\omega_{rs} $ is a Cartesian tensor of second order}.
\end{tcolorbox}
Be 
\begin{align}
v^{}_r = -\omega^{}_{rn}z^{}_n
\end{align}
Considering orthogonal transformation in a flat space $z^{'}_m = A_{mr}z^{}_r+B_m$ with  $B_m=0$ as we consider only transformations which do not change the origin. Differentiation with the parameter $t$ gives 
\begin{align}
v^{'}_m &= A_{mr}v^{}_r\\
&= -\omega^{}_{rn}A_{mr}z^{}_n\\
\end{align}
But $z^{'}_q = A_{qr}z^{}_r\quad\Rightarrow \quad A_{qn}z^{'}_q = A_{qn}A_{qr}z^{}_r\quad\Rightarrow \quad A_{qn}z^{'}_q = z^{}_n$ 
Hence
\begin{align}
v^{'}_m &= -\omega^{}_{rn}A_{mr}z^{}_n\\
&= -\underbrace{\omega^{}_{rn}A_{mr}A_{qn}}_{\overset{\underset{\mathrm{def}}{}}{=}\omega_{mq}^{'}}z^{'}_q\\
v^{'}_m &= -\omega_{mq}^{'}z^{'}_q
\end{align}
$$\blacklozenge$$
\newpage


\section{p159 - Exercise}
\begin{tcolorbox}
Show that if a rigid body rotates about the point $z_r=b_r$ as fixed point, the velociy of a general point of the body is given by $$v_r=-\omega_{rm}\left(z_m-b_m\right)$$
\end{tcolorbox}
By $\mathbf{5.302. }$:
\begin{align}
\left(z^{(1)}_m-z^{(2)}_m\right)\left(dz^{(1)}_m-dz^{(2)}_m\right)=0
\end{align}
At the fixed point we have $z^{(2)}_m=b_m$ and $dz^{(2)}_m=0$, hence
\begin{align}
\left(z^{(1)}_m-b_m\right)\left(dz^{(1)}_m\right)=0\\
\Rightarrow\spatie z^{(1)}_mdz^{(1)}_m =b_mdz^{(1)}_m
\end{align}
As this is true for any point of the rigid mass, expanding (1) and using (3) we get when dividing by $dt$
\begin{align}
\left(z^{(2)}_m-b_m\right)v^{(1)}_m+\left(z^{(1)}_m-b_m\right)v^{(2)}_m=0
\end{align}
Taking twice the partial derivative $\frac{\partial^2}{\partial z^{(1)}_p\partial z^{(1)}_q}$ we get
\begin{align}
\left(z^{(2)}_m-b_m\right)\frac{\partial^2 v_m}{\partial z^{(1)}_p\partial z^{(1)}_q}=0
\end{align}
As this is true for any arbitrary point in the rigid body we get
\begin{align}
\frac{\partial^2 v_m}{\partial z^{(1)}_p\partial z^{(1)}_q}=0\\
\Rightarrow\spatie v_m= K_{mr}z_r+B_m
\end{align}
At the fixed point we have 
\begin{align}
 K_{mr}b_r+B_m =0
\end{align}
Plugging this in (7)
\begin{align}
 v_m= K_{mr}\left(z_r-b_m\right)
\end{align}
Putting $K_{mr}=-\omega_{mr}$ gives us indeed the asked expression.
$$\blacklozenge$$
\newpage


\section{p161 - Clarification}
\begin{tcolorbox}
$$\mathbf{5.325.}\spatie \Omega_{np}\sum \left(mf_nz_p\right)= \Omega_{np}\sum F_nz_p $$
and hence, since $\Omega_{np}$ is arbitrary,
$$\mathbf{5.326.}\spatie \sum m\left(f_nz_p-f_pz_n\right)= \sum \left(F_nz_p-F_pz_n \right)$$
\end{tcolorbox}
To be complete the following step should be inserted

\begin{align}
\Omega_{np}\sum \left(mf_nz_p\right)&= \Omega_{np}\sum F_nz_p \\
\text{As }\Omega_{np}\text{ is skew-symmetric:}\spatie -\Omega_{np}\sum \left(mf_pz_n\right)&= -\Omega_{np}\sum F_pz_n \\
\text{(1)+(2) }\spatie \Omega_{np}\sum m\left(f_nz_p-f_pz_n\right)&= \Omega_{np}\sum\left( F_nz_p -F_pz_n\right)
\end{align}
and hence, since $\Omega_{np}$ is arbitrary,
$$\mathbf{5.326.}\spatie \sum m\left(f_nz_p-f_pz_n\right)= \sum \left(F_nz_p-F_pz_n \right)$$
$$\blacklozenge$$
\newpage

\section{p161 - Clarification}
\begin{tcolorbox}
$$\mathbf{5.329.}\spatie h_{np}=\sum m\left(\omega_{nq} z_q z_p -\omega_{pq} z_qz_n\right)$$
$$= J_{npqr}\omega_{rq}$$
where
$$\mathbf{5.330.}\spatie J_{npqr}= \sum m\left(\delta_{nr}z_qz_p-\delta_{pr}z_nz_q \right)$$
\end{tcolorbox}
\begin{align}
h_{np}&=\sum m\left(\omega_{nq} z_q z_p -\omega_{pq} z_qz_n\right)\\
&=\sum m\left(\omega_{rq}\delta_{rn} z_q z_p -\omega_{rq} \delta_{rp}z_qz_n\right)\\
&=\omega_{rq}\sum m\left(\delta_{rn} z_q z_p -\delta_{rp}z_qz_n\right)\\
&=J_{npqr}\omega_{rq}
\end{align}
$$\blacklozenge$$
\newpage


\section{p166 - Exercise}
\begin{tcolorbox}
Deduce immediately from $\mathbf{5.420.}$ that the Coriolis force is perpendicular to the velocity.
\end{tcolorbox}
\begin{align}
G^{'}_s &= 2m\omega^{'}_{sm}(S^{'},S)v^{'}_m(S^{'})\\
\times v^{'}_s(S^{'})\quad : \spatie G^{'}_s v^{'}_s(S^{'})&=m\left(\omega^{'}_{sm}(S^{'},S)v^{'}_m(S^{'})v^{'}_s(S^{'})+\omega^{'}_{ms}(S^{'},S)v^{'}_m(S^{'})v^{'}_s(S^{'})\right)\\
&=0\quad\text{as }\omega^{'}_{ms}\text{ is skew-symmetric}
\end{align}
$$\blacklozenge$$
\newpage

\section{p166 - Exercise}
\begin{tcolorbox}
Show that if $N=3$ and $\dot{\omega}^{'}_r(S^{'},S) =0$, then the centrifugal force may be written 
$$\mathbf{5.422.}\spatie C^{'}_s = m\omega^{'}_{n}(S^{'},S)\omega^{'}_{n}(S^{'},S)z^{'}_s-m\omega^{'}_{n}(S^{'},S)z^{'}_n\omega^{'}_{s}(S^{'},S)$$
Deduce that $ C^{'}_s$ is coplanar with the vectors $\omega^{'}_{s}(S^{'},S)$ and $z^{'}_n$ and perpendicular to the former.
\end{tcolorbox}
By $\mathbf{5.420.}$ with $\dot{\omega}^{'}_r(S^{'},S) =0$ and using $\mathbf{5.316.} \quad ( \omega^{'}_{rs}=\epsilon_{rsn}\omega^{'}_n)$
\begin{align}
C^{'}_s &= m\omega^{'}_{sm}(S^{'},S)\omega^{'}_{nm}(S^{'},S)z^{'}_n\\
&= m\epsilon_{smk}\omega^{'}_k(S^{'},S)\epsilon_{nmp}\omega^{'}_p(S^{'},S)z^{'}_n\\
&= m\epsilon_{msk}\epsilon_{mnp}\omega^{'}_k(S^{'},S)\omega^{'}_p(S^{'},S)z^{'}_n\\
&= m\left(\delta_{sn}\delta_{kp} -\delta_{sp}\delta_{kn} \right)\omega^{'}_k(S^{'},S)\omega^{'}_p(S^{'},S)z^{'}_n\\
&= m\delta_{sn}\delta_{kp}\omega^{'}_k(S^{'},S)\omega^{'}_p(S^{'},S)z^{'}_n -m\delta_{sp}\delta_{kn} \omega^{'}_k(S^{'},S)\omega^{'}_p(S^{'},S)z^{'}_n\\
&= m\omega^{'}_p(S^{'},S)\omega^{'}_p(S^{'},S)z^{'}_s - m\omega^{'}_n(S^{'},S)\omega^{'}_s(S^{'},S)z^{'}_n
\end{align}
To deduce that $ C^{'}_s$ is coplanar with the vectors $\omega^{'}_{s}(S^{'},S)$ and $z^{'}_n$ we calculate the mixed triple product
\begin{align}
P &= \epsilon_{spr}C^{'}_s \omega^{'}_p(S^{'},S)z^{'}_r\\
&= m\underbrace{\epsilon_{spr} \omega^{'}_n(S^{'},S)\omega^{'}_n(S^{'},S)z^{'}_s \omega^{'}_p(S^{'},S)z^{'}_r}_{=0} -\underbrace{ m\epsilon_{spr}\omega^{'}_n(S^{'},S)\omega^{'}_s(S^{'},S)z^{'}_n \omega^{'}_p(S^{'},S)z^{'}_r}_{=0}\\
=0
\end{align}
Both terms vanish: the first by the presence of the terms $\epsilon_{spr}z^{'}_s z^{'}_r$ which cancel each other and for the second by the terms $\epsilon_{spr}\omega^{'}_s(S^{'},S) \omega^{'}_p(S^{'},S)$. As $P=0$, the three vectors are coplanar.\\
We now calculate the inner product $C^{'}_s\omega^{'}_s(S^{'},S)$
\begin{align}
P&= m\omega^{'}_n(S^{'},S)\omega^{'}_n(S^{'},S)z^{'}_s\omega^{'}_s(S^{'},S) - \underbrace{m\omega^{'}_n(S^{'},S)\omega^{'}_s(S^{'},S)z^{'}_n\omega^{'}_s(S^{'},S)}_{\Leftrightarrow \  m\omega^{'}_n(S^{'},S)\omega^{'}_n(S^{'},S)z^{'}_s\omega^{'}_s(S^{'},S)}\\
&=0
\end{align}
$$\blacklozenge$$
\newpage

\section{p168 - Exercise}
\begin{tcolorbox}
Taking $N=3$, show that $\mathbf{5.424}$ may be reduced to the usual Euler equations:
$$ I_{11}\dv{\omega^{'}_1(S^{'},S)}{t}-\left(I_{22}-I^{'}_{33}\right)\omega^{}_2(S^{'},S)\omega^{'}_3(S^{'},S)=M^{'}_1$$ and two similar equations.
\end{tcolorbox}
$\mathbf{5.424}$:
\begin{align}
M^{'}_{ab}&=J^{'}_{abrq}\dv{\omega^{'}_{rq}(S^{'},S)}{t}+J^{'}_{cdrq}\left(\delta_{ac}\delta_{du}\delta_{bv}+\delta_{bd}\delta_{cu}\delta_{av}\right)\omega^{'}_{rq}(S^{'},S)\omega^{'}_{uv}(S^{'},S) =\\
\times \epsilon_{sab} \text{: }\quad 2M^{'}_{s}&=\epsilon_{sab }J^{'}_{abrq}\dv{\omega^{'}_{rq}(S^{'},S)}{t}+\epsilon_{sab}J^{'}_{cdrq}\left(\delta_{ac}\delta_{du}\delta_{bv}+\delta_{bd}\delta_{cu}\delta_{av}\right)\omega^{'}_{rq}(S^{'},S)\omega^{'}_{uv}(S^{'},S) 
\end{align}
Using $ \omega^{'}_{rq}(S^{'},S)= \epsilon_{rqt}\omega^{'}_{t}(S^{'},S)$ and $I_{st}=\half J^{'}_{abrq}\epsilon_{abs}\epsilon_{rqt}$
\begin{align}
2M^{'}_{s}&=2I_{st}\dv{\omega^{'}_{t}(S^{'},S)}{t}+\epsilon_{sab}\epsilon_{rqi}\epsilon_{uvj}J^{'}_{cdrq}\left(\delta_{ac}\delta_{du}\delta_{bv}+\delta_{bd}\delta_{cu}\delta_{av}\right)\omega^{'}_{i}(S^{'},S)\omega^{'}_{j}(S^{'},S) \\
&=\left\{\begin{array}{l}
2I_{st}\dv{\omega^{'}_{t}(S^{'},S)}{t}\\\\ + \left(\epsilon_{sab}\epsilon_{rqi}\epsilon_{uvj}J^{'}_{cdrq}\delta_{ac}\delta_{du}\delta_{bv}+\epsilon_{sab}\epsilon_{rqi}\epsilon_{uvj}J^{'}_{cdrq}\delta_{bd}\delta_{cu}\delta_{av}\right)\omega^{'}_{i}(S^{'},S)\omega^{'}_{j}(S^{'},S) 
\end{array}\right.\\
&=\left\{\begin{array}{l}
2I_{st}\dv{\omega^{'}_{t}(S^{'},S)}{t}\\\\ + \left(\epsilon_{scb}\epsilon_{rqi}\epsilon_{dbj}J^{'}_{cdrq}+\epsilon_{sad}\epsilon_{rqi}\epsilon_{caj}J^{'}_{cdrq}\right)\omega^{'}_{i}(S^{'},S)\omega^{'}_{j}(S^{'},S) 
\end{array}\right.\\
&=\left\{\begin{array}{l}
2I_{st}\dv{\omega^{'}_{t}(S^{'},S)}{t}\\\\ + \left(\epsilon_{bcs}\epsilon_{bdj}\right)\epsilon_{rqi}J^{'}_{cdrq}\omega^{'}_{i}(S^{'},S)\omega^{'}_{j}(S^{'},S) \\\\+\left(\epsilon_{asd}\epsilon_{acj}\right)\epsilon_{rqi}J^{'}_{cdrq}\omega^{'}_{i}(S^{'},S)\omega^{'}_{j}(S^{'},S) 
\end{array}\right.\\
&=\left\{\begin{array}{l}
2I_{st}\dv{\omega^{'}_{t}(S^{'},S)}{t}\\\\ + \left(\delta_{cd}\delta_{sj}-\delta_{cj}\delta_{sd}\right)\epsilon_{rqi}J^{'}_{cdrq}\omega^{'}_{i}(S^{'},S)\omega^{'}_{j}(S^{'},S) \\\\+\left(\delta_{sc}\delta_{dj}-\delta_{sj}\delta_{dc}\right)\epsilon_{rqi}J^{'}_{cdrq}\omega^{'}_{i}(S^{'},S)\omega^{'}_{j}(S^{'},S) 
\end{array}\right.
\end{align}
\begin{align}
&=\left\{\begin{array}{l}
2I_{st}\dv{\omega^{'}_{t}(S^{'},S)}{t}\\\\ +\epsilon_{rqi}J^{'}_{ccrq}\omega^{'}_{i}(S^{'},S)\omega^{'}_{s}(S^{'},S) \\\\
-\epsilon_{rqi}J^{'}_{jsrq}\omega^{'}_{i}(S^{'},S)\omega^{'}_{j}(S^{'},S) \\\\+\epsilon_{rqi}J^{'}_{sjrq}\omega^{'}_{i}(S^{'},S)\omega^{'}_{j}(S^{'},S) 
\\\\-\epsilon_{rqi}J^{'}_{ccrq}\omega^{'}_{i}(S^{'},S)\omega^{'}_{s}(S^{'},S)
\end{array}\right.
\end{align}
giving
\begin{align}
2M^{'}_{s}&=\left\{\begin{array}{l}
2I_{st}\dv{\omega^{'}_{t}(S^{'},S)}{t} 
\\\\+\epsilon_{rqi}J^{'}_{sjrq}\omega^{'}_{i}(S^{'},S)\omega^{'}_{j}(S^{'},S)
\\\\
-\epsilon_{rqi}J^{'}_{jsrq}\omega^{'}_{i}(S^{'},S)\omega^{'}_{j}(S^{'},S)  
\end{array}\right.
\end{align}
For $s=1$:
\begin{table}[H]
\centering
 \begin{tabular}{||c | | c  | c||} 
 \hline
  & $+\epsilon_{rqi}J^{}_{1jrq}\omega^{}_{i}\omega^{}_{j}$ & $-\epsilon_{rqi}J^{}_{j1rq}\omega^{}_{i}\omega^{}_{j}   $\\ [1.5ex] 
 \hline\hline
 $\epsilon_{123}$ & $+J^{}_{1112}\omega^{}_{3}\omega^{}_{1}+J^{}_{1212}\omega^{}_{3}\omega^{}_{2}+J^{}_{1312}\omega^{}_{3}\omega^{}_{3}$ & $-J^{}_{1112}\omega^{}_{3}\omega^{}_{1}-J^{}_{2112}\omega^{}_{3}\omega^{}_{2}-J^{}_{3112}\omega^{}_{3}\omega^{}_{3}   $\\  
 $\epsilon_{132}$ & $-J^{}_{1113}\omega^{}_{2}\omega^{}_{1}-J^{}_{1213}\omega^{}_{2}\omega^{}_{2}-J^{}_{1313}\omega^{}_{2}\omega^{}_{3}$ & $+J^{}_{1113}\omega^{}_{2}\omega^{}_{1} +J^{}_{2113}\omega^{}_{2}\omega^{}_{2} +J^{}_{3113}\omega^{}_{2}\omega^{}_{3}   $\\
 $\epsilon_{213}$ & $-J^{}_{1121}\omega^{}_{3}\omega^{}_{1}-J^{}_{1221}\omega^{}_{3}\omega^{}_{2}-J^{}_{1321}\omega^{}_{3}\omega^{}_{3}$ & $+J^{}_{1121}\omega^{}_{3}\omega^{}_{1}+J^{}_{2121}\omega^{}_{3}\omega^{}_{2}+J^{}_{3121}\omega^{}_{3}\omega^{}_{3}   $\\
 $\epsilon_{231}$ & $+J^{}_{1123}\omega^{}_{1}\omega^{}_{1}+J^{}_{1223}\omega^{}_{1}\omega^{}_{2}+J^{}_{1323}\omega^{}_{1}\omega^{}_{3}$ & $-J^{}_{1123}\omega^{}_{1}\omega^{}_{1}-J^{}_{2123}\omega^{}_{1}\omega^{}_{2}-J^{}_{3123}\omega^{}_{1}\omega^{}_{3}   $\\
 $\epsilon_{321}$ & $-J^{}_{1132}\omega^{}_{1}\omega^{}_{1}-J^{}_{1232}\omega^{}_{1}\omega^{}_{2}-J^{}_{1332}\omega^{}_{1}\omega^{}_{3}$ & $+J^{}_{1132}\omega^{}_{1}\omega^{}_{1}  +J^{}_{2132}\omega^{}_{1}\omega^{}_{2}  +J^{}_{3132}\omega^{}_{1}\omega^{}_{3}   $\\
 $\epsilon_{312}$ & $+J^{}_{1131}\omega^{}_{2}\omega^{}_{1}+J^{}_{1231}\omega^{}_{2}\omega^{}_{2}+J^{}_{1331}\omega^{}_{2}\omega^{}_{3}$ & $-J^{}_{1131}\omega^{}_{2}\omega^{}_{1} -J^{}_{2131}\omega^{}_{2}\omega^{}_{2} -J^{}_{3131}\omega^{}_{2}\omega^{}_{3}   $\\   [1ex] 
 \hline
 \end{tabular}
\end{table}

Taking into account that $J^{}_{abcd}=0$ for $a\ne c  \wedge b\ne d$

\begin{table}[H]
\centering
 \begin{tabular}{||c | | c  | c||} 
 \hline
  & $+\epsilon_{rqi}J^{}_{1jrq}\omega^{}_{i}\omega^{}_{j}$ & $-\epsilon_{rqi}J^{}_{j1rq}\omega^{}_{i}\omega^{}_{j}   $\\ [1.5ex] 
 \hline\hline
 $\epsilon_{123}$ & $+\cancel{J^{}_{1112}\omega^{}_{3}\omega^{}_{1}}+J^{}_{1212}\omega^{}_{3}\omega^{}_{2}+J^{}_{1312}\omega^{}_{3}\omega^{}_{3}$ & $-\cancel{J^{}_{1112}\omega^{}_{3}\omega^{}_{1}} $\\  
 $\epsilon_{132}$ & $-\cancel{J^{}_{1113}\omega^{}_{2}\omega^{}_{1}}-J^{}_{1213}\omega^{}_{2}\omega^{}_{2}-J^{}_{1313}\omega^{}_{2}\omega^{}_{3}$ & $+\cancel{J^{}_{1113}\omega^{}_{2}\omega^{}_{1} } $\\
 $\epsilon_{213}$ & $-\cancel{J^{}_{1121}\omega^{}_{3}\omega^{}_{1}}$ & $+\cancel{J^{}_{1121}\omega^{}_{3}\omega^{}_{1}}+J^{}_{2121}\omega^{}_{3}\omega^{}_{2}+J^{}_{3121}\omega^{}_{3}\omega^{}_{3}   $\\
 $\epsilon_{231}$ & $+J^{}_{1323}\omega^{}_{1}\omega^{}_{3}$ & $-J^{}_{2123}\omega^{}_{1}\omega^{}_{2}$\\
 $\epsilon_{321}$ & $-J^{}_{1232}\omega^{}_{1}\omega^{}_{2}$ & $+J^{}_{3132}\omega^{}_{1}\omega^{}_{3}   $\\
 $\epsilon_{312}$ & $+\cancel{J^{}_{1131}\omega^{}_{2}\omega^{}_{1}}$ & $-\cancel{J^{}_{1131}\omega^{}_{2}\omega^{}_{1}} -J^{}_{2131}\omega^{}_{2}\omega^{}_{2} -J^{}_{3131}\omega^{}_{2}\omega^{}_{3}   $\\   [1ex] 
 \hline
 \end{tabular}
\end{table}
Opposite sign terms vanish, giving
\begin{table}[H]
\centering
 \begin{tabular}{||c | | c  | c||} 
 \hline
  & $+\epsilon_{rqi}J^{}_{1jrq}\omega^{}_{i}\omega^{}_{j}$ & $-\epsilon_{rqi}J^{}_{j1rq}\omega^{}_{i}\omega^{}_{j}   $\\ [1.5ex] 
 \hline\hline
 $\epsilon_{123}$ & $+J^{}_{1212}\omega^{}_{3}\omega^{}_{2}+J^{}_{1312}\omega^{}_{3}\omega^{}_{3}$ & $ $\\  
 $\epsilon_{132}$ & $-J^{}_{1213}\omega^{}_{2}\omega^{}_{2}-J^{}_{1313}\omega^{}_{2}\omega^{}_{3}$ & $ $\\
 $\epsilon_{213}$ & $ $ & $+J^{}_{2121}\omega^{}_{3}\omega^{}_{2}+J^{}_{3121}\omega^{}_{3}\omega^{}_{3}   $\\
 $\epsilon_{231}$ & $+J^{}_{1323}\omega^{}_{1}\omega^{}_{3}$ & $-J^{}_{2123}\omega^{}_{1}\omega^{}_{2}$\\
 $\epsilon_{321}$ & $-J^{}_{1232}\omega^{}_{1}\omega^{}_{2}$ & $+J^{}_{3132}\omega^{}_{1}\omega^{}_{3}   $\\
 $\epsilon_{312}$ & $ $ & $-J^{}_{2131}\omega^{}_{2}\omega^{}_{2} -J^{}_{3131}\omega^{}_{2}\omega^{}_{3}   $\\   [1ex] 
 \hline
 \end{tabular}
\end{table}
Considering $J_{abcd}= -J_{badc}$
\begin{table}[H]
\centering
 \begin{tabular}{||c | | c  | c||} 
 \hline
  & $+\epsilon_{rqi}J^{}_{1jrq}\omega^{}_{i}\omega^{}_{j}$ & $-\epsilon_{rqi}J^{}_{j1rq}\omega^{}_{i}\omega^{}_{j}   $\\ [1.5ex] 
 \hline\hline
 $\epsilon_{123}$ & $+\cancel{J^{}_{1212}}\omega^{}_{3}\omega^{}_{2}+\cancel{J^{}_{1312}}\omega^{}_{3}\omega^{}_{3}$ & $ $\\  
 $\epsilon_{132}$ & $-\cancel{J^{}_{1213}}\omega^{}_{2}\omega^{}_{2}-\cancel{J^{}_{1313}}\omega^{}_{2}\omega^{}_{3}$ & $ $\\
 $\epsilon_{213}$ & $ $ & $+\cancel{J^{}_{2121}}\omega^{}_{3}\omega^{}_{2}+\cancel{J^{}_{3121}}\omega^{}_{3}\omega^{}_{3}   $\\
 $\epsilon_{231}$ & $+\cancel{J^{}_{1323}}\omega^{}_{1}\omega^{}_{3}$ & $-\cancel{J^{}_{2123}}\omega^{}_{1}\omega^{}_{2}$\\
 $\epsilon_{321}$ & $-\cancel{J^{}_{1232}}\omega^{}_{1}\omega^{}_{2}$ & $+\cancel{J^{}_{3132}}\omega^{}_{1}\omega^{}_{3}   $\\
 $\epsilon_{312}$ & $ $ & $-\cancel{J^{}_{2131}}\omega^{}_{2}\omega^{}_{2} -\cancel{J^{}_{3131}}\omega^{}_{2}\omega^{}_{3}   $\\   [1ex] 
 \hline
 \end{tabular}
\end{table}
??
We get $$ m^{'}_s = I_{st}\dv{\omega^{'}_{t}(S^{'},S)}{t}$$?????\\
\newpage
Let's try another approach. Start with $\mathbf{5.332.}$: $\dv{}{t}\left(I_{st}\omega_{t}\right)= M_s$
\begin{align}
\dv{}{t}\left(I_{st}(S^{'},S)\omega_{t}(S^{'},S)\right)= M_s(S^{'},S)
\end{align}
Cf. $\mathbf{5.408.}$
\begin{align}
\omega^{'}_{u}(S^{'},S)&=A_{uq}\omega^{}_{q}(S^{'},S)\\
\times A_{ut}\quad \rightarrow \spatie A_{ut}\omega^{'}_{u}(S^{'},S)&=A_{ut}A_{uq}\omega^{}_{q}(S^{'},S)\\
&=\omega^{}_{t}(S^{'},S)\\
\spatie \omega^{}_{t}(S^{'},S) &=  A_{ut}\omega^{'}_{u}(S^{'},S)\\
\textbf{(10)}\quad\Rightarrow\spatie M_s(S^{'},S) &=\dv{}{t}\left(I_{st}(S^{'},S)A_{ut}\omega^{'}_{u}(S^{'},S)\right)\\
\times A_{ps}\quad \Rightarrow M^{'}_p(S^{'},S) &=A_{ps}\dv{}{t}\left(I_{st}(S^{'},S)A_{ut}\omega^{'}_{u}(S^{'},S)\right)\\
I_{st}(S^{'},S)&= A_{as}A_{bt}I^{'}_{ab}(S^{'},S)\\
\textbf{(16)}\quad\Rightarrow \spatie M^{'}_p(S^{'},S) &=A_{ps}\dv{}{t}\left(A_{as}A_{bt}I^{'}_{ab}(S^{'},S)A_{ut}\omega^{'}_{u}(S^{'},S)\right)\\
&=A_{ps}\dv{}{t}\left(A_{as}I^{'}_{ak}(S^{'},S)\omega^{'}_{k}(S^{'},S)\right)\\
\end{align}
As we transformed $I_{st}(S^{'},S)$ to a coordinate system fixed to the body we have that the elements of $I^{'}_{ab}(S^{'},S)$ are constants.\\
Hence,
\begin{align}
 M^{'}_p(S^{'},S) &=I^{'}_{ak}S^{'},S)A_{ps}\dv{}{t}\left( A_{as}\omega^{'}_{k}(S^{'},S)\right)\\
 &=I^{'}_{ak}(S^{'},S)A_{ps}\left(\dot{A}_{as}\omega^{'}_{k}(S^{'},S)+A_{as} \dot{\omega}^{'}_{k}(S^{'},S)\right)\\
 &=I^{'}_{ak}(S^{'},S)A_{ps}A_{as} \dot{\omega}^{'}_{k}(S^{'},S)+ I^{'}_{ak}(S^{'},S)A_{ps}\dot{A}_{as}\omega^{'}_{k}(S^{'},S)\\
 &=I^{'}_{pk}(S^{'},S) \dot{\omega}^{'}_{k}(S^{'},S)+ I^{'}_{ak}(S^{'},S)A_{ps}\dot{A}_{as}\omega^{'}_{k}(S^{'},S)\\
\textbf{5.408.}\quad\Rightarrow \spatie  A_{ps}\dot{A}_{as} &= \omega^{'}_{ap}(S^{'},S)\\
\textbf{(24)}\quad\Rightarrow \spatie  M^{'}_p(S^{'},S) &=I^{'}_{pk}(S^{'},S) \dot{\omega}^{'}_{k}(S^{'},S)+ I^{'}_{ak}(S^{'},S)A_{ps}\dot{A}_{as}\omega^{'}_{k}(S^{'},S)\\
\end{align}

$$\blacklozenge$$
\newpage
\section{p169 - Exercise}
\begin{tcolorbox}
Assign convenient generalized coordinates for the three systems $(a), \ (b),\text{ and } (c)$ mentioned at the beginning of this section, and calculate the kinematical metric form in each case
\end{tcolorbox}
\textbf{$(a)$ a particle on a surface $(N=2)$}\\
No need here for fancy general coordinates: the $V_2$ coordinate system in the plane is the metric form of choice.
Indeed $\left|v\right|^2 = a_{mn}v_mv_n$ and for a $V_2$
$$ds^2 =\left(a_{11}\left(v^1\right)^2+2a_{12}v^1v^2+a_{22}\left(v^2\right)^2 \right)dt^2$$ and if the space is Euclidean and the plane smooth, we can choose an orthogonal system where $a_{12}$ will vanish.\\\\  
\textbf{$(b)$ a rigid body which can turn about a fixed point, as in the preceding section $(N=3)$}\\
For a rigid body we can choose a coordinate system $S^{'}$ fixed to the body to describe the geometry of the rigid body. The kinetic energy  referenced to a 'non-moving' (abuse of language) coordinate system $S$ is 
\begin{align}
T = \half\mathbf{\sum}\rho v^{'}_n(S)v^{'}_n(S)\spatie\text{(summation over all masses in the rigid body)}
\end{align}
We know by $\mathbf{5.409}$: $v^{'}_n(S)= v^{'}_n(S^{'})+ \omega^{'}_{mn}(S^{'},S)z^{'}_m$. As the $v^{'}_n(S^{'})$ are fixed, we have $v^{'}_n(S^{'})=0$ giving 
\begin{align}
T = \half\mathbf{\sum}\rho z^{'}_mz^{'}_k\omega^{'}_{mn}(S^{'},S)\omega^{'}_{kn}(S^{'},S)
\end{align}
Note in (2) that we bring $\omega^{'}_{mn}(S^{'},S)$ out of the summation as this expression  is the same for all masses in the body.
\begin{align}
\omega_{mn}(S^{'},S)&= \epsilon_{mnt}\omega^{'}_{t}(S^{'},S)\\
\Rightarrow\quad  
T &= \half\mathbf{\sum}\rho \epsilon_{mnt}\epsilon_{kns}z^{'}_mz^{'}_k\omega^{'}_{t}(S^{'},S)\omega^{'}_{s}(S^{'},S)\\
&= \half\mathbf{\sum}\rho \left(\delta_{mk}\delta_{ts}-\delta_{ms}\delta_{kt}\right)z^{'}_mz^{'}_k\omega^{'}_{t}(S^{'},S)\omega^{'}_{s}(S^{'},S)\\
&= \half\mathbf{\sum}\rho \left(z^{'}_mz^{'}_m\omega^{'}_{t}(S^{'},S)\omega^{'}_{t}(S^{'},S)-z^{'}_sz^{'}_t\omega^{'}_{t}(S^{'},S)\omega^{'}_{s}(S^{'},S)\right)\\
&= \half\mathbf{\sum}\rho \left(\delta_{st}z^{'}_mz^{'}_m\omega^{'}_{s}(S^{'},S)\omega^{'}_{t}(S^{'},S)-z^{'}_sz^{'}_t\omega^{'}_{t}(S^{'},S)\omega^{'}_{s}(S^{'},S)\right)\\
&= \half\mathbf{\sum}\rho \left(\delta_{st}z^{'}_mz^{'}_m-z^{'}_sz^{'}_t\right)\omega^{'}_{s}(S^{'},S)\omega^{'}_{t}(S^{'},S)
\end{align}
By $\mathbf{5.335.}$ we have $ I_{st} = \delta_{st}\mathbf{\sum}\rho z_mz_m-\mathbf{\sum}\rho z_sz_t$ and so (8) can be written as
\begin{align}
T&=\half I_{st}\omega^{'}_{s}(S^{'},S)\omega^{'}_{t}(S^{'},S)
\end{align} 
So we can choose the three angles $\Omega^{'}_{s}(S^{'},S)$  with ($\omega^{'}_{s}(S^{'},S)= \dv{\Omega^{'}_{s}(S^{'},S)}{t}$) as generalized coordinates and define $$ds^2=I_{st}d\Omega^{'}_{s}(S^{'},S)d\Omega^{'}_{t}(S^{'},S)$$ with  $$a_{mn} = I_{mn}$$ having constants as elements.
Some check on consistency of the metric tensor defined by $(14)$:\\
\textbf{Positive definite ?} : Yes, as $T$ is positive by construction.\\
\textbf{Symmetric ?} : Yes, as $a_{mn} = I_{km}$ and $I_{km}$ is symmetric.  \\
\begin{comment}
For a rigid body we can choose a coordinate system fixed to the body to describe the geometry of the rigid body. The kinetic energy is referenced to a 'fixed' coordinate system 
\begin{align}
T = \half\mathbf{\sum}\rho v_nv_n\spatie\text{summation over all masses in the rigid body}
\end{align}
as $z_n$ are fixed in the choose coordinate system it is clear that 
we have to find another way the express the velocity of the masses.

We know $v_n= - \omega_{nm}z_m$. As the $z_m$ are fixed, the only way the kinetic energy can change (and have 'a path' in the general coordinate system) is when $\omega_{nm}$ changes. 
We can write
\begin{align}
dT &=\half\mathbf{\sum}\rho  d\left(v_nv_n\right)\\
&=\half \mathbf{\sum}\rho  d\left(\omega_{nm}\omega_{nk}z_mz_k\right)\\
&=\half \mathbf{\sum}\rho  \left(\omega_{nk}d\omega_{nm}+\omega_{nm}d\omega_{nk}\right)z_mz_k\\
&= \left( \mathbf{\sum}\rho  \omega_{nm}z_mz_k\right)d\omega_{nk}
\end{align}
Note in (5) that we bring $d\omega_{nk}$ out of the summation as $\omega_{nk}$ is the same for all masses in the body.
As $dT$ can be negative we calculate $dT^2$
\begin{align}
dT^2&=\left( \mathbf{\sum}\rho  \omega_{nm}z_mz_k\right)\left( \mathbf{\sum}\rho  \omega_{pq}z_qz_r\right)d\omega_{nk}d\omega_{pr}\end{align}
\begin{align}
\omega_{nm}&= \epsilon_{nma}\omega_a\\
\Rightarrow\quad  
dT^2&=\left( \mathbf{\sum}\rho  \epsilon_{nma}\omega_az_mz_k\right)\left( \mathbf{\sum}\rho  \epsilon_{pqb}\omega_bz_qz_r\right)d\left(\epsilon_{nku}\omega_{u}\right)d\left(\epsilon_{prv}\omega_{v}\right)\\
&=\left( \mathbf{\sum}\rho  \epsilon_{nma}\epsilon_{nku}\omega_az_mz_k\right)\left( \mathbf{\sum}\rho  \epsilon_{pqb}\epsilon_{prv}\omega_bz_qz_r\right)d\omega_{u}\\
&=\left\{\begin{array}{l}\left( \mathbf{\sum}\rho  \left(\delta_{mk}\delta_{au}-\delta_{mu}\delta_{ak}\right)\omega_az_mz_k\right)d\omega_{u}\\ \left( \mathbf{\sum}\rho  \left(\delta_{qr}\delta_{bv}-\delta_{qv}\delta_{rb}\right)\omega_bz_qz_r\right)d\omega_{v}d\omega_{v}\\
\end{array}\right.\\
&=\left( \mathbf{\sum}\rho  \left(\omega_uz_nz_n-\omega_kz_uz_k\right)\right)\left( \mathbf{\sum}\rho  \left(\omega_vz_nz_n-\omega_rz_vz_r\right)\right)d\omega_{u} d\omega_{v}\\
&=\left( \mathbf{\sum}\rho  \left(\delta_{ku}\omega_kz_nz_n-\omega_kz_uz_k\right)\right)\left( \mathbf{\sum}\rho  \left(\delta_{rv}\omega_rz_nz_n-\omega_rz_vz_r\right)\right)d\omega_{u} d\omega_{v}\\
&=\left( \mathbf{\sum}\rho  \left(\delta_{ku}z_nz_n-z_uz_k\right)\right)\left( \mathbf{\sum}\rho  \left(\delta_{rv}z_nz_n-z_vz_r\right)\right)\omega_k\omega_rd\omega_{u} d\omega_{v}
\end{align}
By $\mathbf{5.335.}$ we have $ I_{ku} = \delta_{ku}\mathbf{\sum}\rho z_nz_n-\mathbf{\sum}\rho z_uz_k$ and so (13) can be written as
\begin{align}
dT^2&=I_{ku}I_{rv}\omega_k\omega_rd\omega_{u} d\omega_{v}
\end{align} 
So we can choose the three $\omega_k$ as generalized coordinates and define $$ds^2=a_{mn}d\omega_{m} d\omega_{n}$$ with  $$a_{mn} = I_{km}I_{rn}\omega_k \omega_r$$
The metric tensor $a_{mn}$ contains elements depending on the $\omega_k$ chosen as general coordinates of the system and is a good candidate as metric tensor.
Some check on consistency of the metric tensor defined by $(14)$:\\
\textbf{Positive definite ?} : Yes, as $dT^2$ is positive.\\
\textbf{Symmetric ?} : Yes, as $a_{mn} = I_{km}I_{rn}\omega_k \omega_r = I_{rn}I_{km}\omega_k \omega_r= a_{nm}$\\
\end{comment}
\\\\ \textbf{$(c)$ a chain of six rods smoothly hinged together, with one end fixed and all moving on a smooth plane $(N=6)$}\\

To simplify the notation we will assume that the mass $m_k$ of each rod (with length $L_k$) is concentrated at it's endpoint .\\
First we note that the velocity of a rod is composed of two vectors, one (labelled as $\overline{\nu}_k$) generated by its own rotation relative to the previous rod and the other (labelled as $\overline{v}_{k-1}$) generated by the velocity of the endpoint of the rod to which it is attached (see.fig. 5.2).
\begin{figure}[H]
    \centering
    \subfloat[]{\begin{tikzpicture}

\coordinate (p1) at (-0.5,2.5) {};
\draw[fill=white]  (p1) circle (0.1);
\coordinate (p2) at (-2.5,-2) {} {};
\draw[fill=white]  (p2) circle (0.1);
\coordinate (p12) at (-1.5,0) {} {} {};

\draw []  (p1) -- (p2);
\draw [dashed](-4.3302,-0.6) arc (-140.0002:-80:5);
\coordinate (p3) at (-0.5,-3.5) {} {};
\draw [dashdotted]  (p1) -- (p3);
\coordinate (v1) at (1.5,0.5) {} {} {};
\coordinate (v2) at (0.5,-3.5) {} {} {};
\coordinate (v3) at(-0.5,-4) {} {} {};
\coordinate (v4) at(-0.5+3,-4-1.5) {} {} {};
\draw [-{Latex[length=2mm]},very thick]  (p1) -- (v1);
\draw [-{Latex[length=2mm]},very thick]  (p2) -- (v2);
\draw [-{Latex[length=2mm]},very thick, dashed]  (p2) -- (v3);
\draw [-{Latex[length=2mm]},, dashed]  (p2) -- (v4);
\draw [ dotted]  (v2) -- (v4);
\draw [ dotted]  (v3) -- (v4);
\draw [dashed, decoration={markings, mark=at position 0.52 with {\arrow[scale = 1.]{Latex[length=2mm]}}},    postaction={decorate}](p12) arc [start angle=65,
        end angle=90,
        x radius=-2cm,
        y radius =-2cm];
\node[label=north east:$m_{k-1}$] at (p1) {};
\node[label=south west:$m_{k}$] at (p2) {};
\node[label=north east :$\theta_k$] at (p12) {};
\node[label=north west :$L_k$] at (p12) {};
\node[label=north east :$\overline{v}_{k-1}$] at (v1) {};
\node[label=south :$\overline{v}_{k-1}$] at (v3) {};
\node[label=south :$\overline{v}_{k} {=}\overline{v}_{k-1}+\overline{\nu}_k$] at (v4) {};
\node[label=north east :$\overline{\nu}_k {=} L_k\dot{\theta}_k$] at (v2) {};
\end{tikzpicture}}
\caption{Composition of absolute and relative velocities of a chain of rods}
\label{fig:fig_p169_a}
\end{figure}
If we take Cartesian coordinates it is easy to see that rod (1) will have components $$\left( L_1\dot{\theta}_1\cos\theta,L_1\dot{\theta}_1\sin\theta_1\right)$$
rod (2) $$\left( L_1\dot{\theta}_1\cos\theta_1+ L_2\dot{\theta}_2\cos\theta_2 ,L_1\dot{\theta}_1\sin\theta_1+ L_2\dot{\theta}_2\sin\theta_2 \right)$$
$$\vdots$$
rod (k) $$\left( \sum_{i=1}^{k}L_i\dot{\theta}_i\cos\theta_i , \sum_{i=1}^{k}L_i\dot{\theta}_i\sin\theta_i  \right)$$
and so 
\begin{align}
\left(v^{(k)}\right)^2&= \left( \sum_{i=1}^{k}L_i\dot{\theta}_i\cos\theta_i\right)^2+\left( \sum_{i=1}^{k}L_i\dot{\theta}_i\sin\theta_i  \right)^2\\
&= \sum_{i=1}^{k}\left( L_i\dot{\theta}_i\right)^2+2\sum_{i=1}^{k}\sum_{j=1}^{k-i}\left( L_iL_{i+j}\dot{\theta}_i\dot{\theta}_{i+j}\cos\left(\theta_i -\theta_{i+j}\right) \right)
\end{align}
So the kinetic energy of one rod and the total kinetic energy of the system are
\begin{align}
T^{(k)}&= \half m_k\left[\sum_{i=1}^{k}\left( L_i\dot{\theta}_i\right)^2+2\sum_{i=1}^{k}\sum_{j=1}^{k-i}\left( L_iL_{i+j}\dot{\theta}_i\dot{\theta}_{i+j}\cos\left(\theta_i -\theta_{i+j}\right) \right)\right]\\
T &= \sum_{k=1}^{N}T^{(k)}
\end{align}
For $N=6$ we get\\ \\
\begin{tabular}{||c|l||}
    \cline{1-2}
    rod&$T^{(k)}$ \\
    \hline \hline
    1 & $\half m_1 \left[\left(L_1\dot{\theta}_1\right)^2\right]$\\
    \hline \hline
    2 &$ \half m_2\left[\left(L_1\dot{\theta}_1\right)^2+\left(L_2\dot{\theta}_2\right)^2+2 L_1L_{2}\dot{\theta}_1\dot{\theta}_{2}\cos\left(\theta_1 -\theta_{2}\right) \right]$\\
    \hline \hline
    3 &$ \half m_3\left[\left(L_1\dot{\theta}_1\right)^2+\left(L_2\dot{\theta}_2\right)^2+\left(L_3\dot{\theta}_3\right)^2 +2 L_1L_{2}\dot{\theta}_1\dot{\theta}_{2}\cos\left(\theta_1 -\theta_{2}\right) +2 L_1L_{3}\dot{\theta}_1\dot{\theta}_{3}\cos\left(\theta_1 -\theta_{3}\right) +\dots \right]$\\
    \hline \hline
    4 &$ \half m_4\left[\left(L_1\dot{\theta}_1\right)^2+\left(L_2\dot{\theta}_2\right)^2+\left(L_3\dot{\theta}_3\right)^2+\left(L_4\dot{\theta}_4\right)^2+2 L_1L_{2}\dot{\theta}_1\dot{\theta}_{2}\cos\left(\theta_1 -\theta_{2}\right) +2 L_1L_{3}\dot{\theta}_1\dot{\theta}_{3}\cos\left(\theta_1 -\theta_{3}\right) +\dots \right]$\\
    \hline \hline
    5 &$ \half m_5\left[\left(L_1\dot{\theta}_1\right)^2+\left(L_2\dot{\theta}_2\right)^2+\left(L_3\dot{\theta}_3\right)^2+\left(L_4\dot{\theta}_4\right)^2+\left(L_5\dot{\theta}_5\right)^2+2 L_1L_{2}\dot{\theta}_1\dot{\theta}_{2}\cos\left(\theta_1 -\theta_{2}\right) +\dots \right]$\\
    \hline \hline
    6 &$ \half m_6\left[\left(L_1\dot{\theta}_1\right)^2+\left(L_2\dot{\theta}_2\right)^2+\left(L_3\dot{\theta}_3\right)^2+\left(L_4\dot{\theta}_4\right)^2+\left(L_5\dot{\theta}_5\right)^2+\left(L_6\dot{\theta}_6\right)^2+2 L_1L_{2}\dot{\theta}_1\dot{\theta}_{2}\cos\left(\theta_1 -\theta_{2}\right)  +\dots \right]$\\
    \hline \hline
\end{tabular}
Giving for $T$
\begin{align}
2T=\left\{\begin{array}{l}
\left(m_1+m_2+m_3+m_4+m_5+m_6\right)\left(L_1\dot{\theta}_1\right)^2\\
+\left(m_2+m_3+m_4+m_5+m_6\right)\left(L_2\dot{\theta}_2\right)^2\\
+\left(m_3+m_4+m_5+m_6\right)\left(L_3\dot{\theta}_3\right)^2\\
+\left(m_4+m_5+m_6\right)\left(L_4\dot{\theta}_4\right)^2\\
+\left(m_5+m_6\right)\left(L_5\dot{\theta}_5\right)^2\\
+\left(m_6\right)\left(L_6\dot{\theta}_6\right)^2\\
+2\left(m_2+m_3+m_4+m_5+m_6\right)L_1L_{2}\dot{\theta}_1\dot{\theta}_{2}\cos\left(\theta_1 -\theta_{2}\right)\\
+2\left(m_3+m_4+m_5+m_6\right)L_1L_{3}\dot{\theta}_1\dot{\theta}_{3}\cos\left(\theta_1 -\theta_{3}\right)\\
+2\left(m_3+m_4+m_5+m_6\right)L_2L_{3}\dot{\theta}_2\dot{\theta}_{3}\cos\left(\theta_2 -\theta_{3}\right)\\
+2\left(m_4+m_5+m_6\right)L_1L_{4}\dot{\theta}_1\dot{\theta}_{4}\cos\left(\theta_1 -\theta_{4}\right)\\
+2\left(m_4+m_5+m_6\right)L_2L_{4}\dot{\theta}_2\dot{\theta}_{4}\cos\left(\theta_2 -\theta_{4}\right)\\
+2\left(m_4+m_5+m_6\right)L_3L_{4}\dot{\theta}_3\dot{\theta}_{4}\cos\left(\theta_3 -\theta_{4}\right)\\
+2\left(m_5+m_6\right)L_1L_{5}\dot{\theta}_1\dot{\theta}_{5}\cos\left(\theta_1 -\theta_{5}\right)\\
+2\left(m_5+m_6\right)L_2L_{5}\dot{\theta}_2\dot{\theta}_{5}\cos\left(\theta_2 -\theta_{5}\right)\\
+2\left(m_5+m_6\right)L_3L_{5}\dot{\theta}_3\dot{\theta}_{5}\cos\left(\theta_3 -\theta_{5}\right)\\
+2\left(m_5+m_6\right)L_4L_{5}\dot{\theta}_4\dot{\theta}_{5}\cos\left(\theta_4 -\theta_{5}\right)\\
+2\left(m_6\right)L_1L_{6}\dot{\theta}_1\dot{\theta}_{6}\cos\left(\theta_1 -\theta_{6}\right)\\
+2\left(m_6\right)L_2L_{6}\dot{\theta}_2\dot{\theta}_{6}\cos\left(\theta_2 -\theta_{6}\right)\\
+2\left(m_6\right)L_3L_{6}\dot{\theta}_3\dot{\theta}_{6}\cos\left(\theta_3 -\theta_{6}\right)\\
+2\left(m_6\right)L_4L_{6}\dot{\theta}_4\dot{\theta}_{6}\cos\left(\theta_4 -\theta_{6}\right)\\
+2\left(m_6\right)L_5L_{6}\dot{\theta}_5\dot{\theta}_{6}\cos\left(\theta_5 -\theta_{6}\right)
\end{array}\right.
\end{align}
We define as general coordinates the angles $\theta^i$
and express $ds^2$ as 
$$ds^2= 2Tdt^2$$ 
and see that $ds^2$ is of the required form 
$$ds^2= a_{mn}d\theta^md\theta^n$$
The metric tensor $a_{mn}$ contains elements depending on the $\theta_k$ chosen as general coordinates of the system and is a good candidate as metric tensor.
Some check on consistency of the metric tensor defined by $(8)$:\\
\textbf{Positive definite ?} : Yes, as $T$ is positive by definition\\
\textbf{Symmetric ?} : Yes, as the non-diagonal term $a_{ij}$ contains $\cos\left(\theta_i-\theta_j\right) = \cos\left(\theta_j-\theta_i\right)$ \\
\textbf{Number of elements} : the metric tensor $a_{mn}$ for $N=6$ should contain $6$ diagonal elements and $\frac{6\times 6 -6}{2} = 15$ independent non-diagonal elements. Checking $(8)$, one can find that the numbers yield.


$$\blacklozenge$$
\newpage


\section{p181 and p182 - Clarification Figures 13., 14. and 15.}
\begin{tcolorbox}
There are several ways to perform  a map of the configuration space of a rigid body with fixed point.
\end{tcolorbox}
\begin{figure}[H]
    \centering
    \subfloat[]{\begin{tikzpicture}[scale=0.25]
\coordinate (O) at (0,0);
\node[label=south :$O$] at (O) {};
\coordinate (X) at (-9.5,-8) {} {};
\coordinate (Y) at (17,0) {} {};
\coordinate (Z) at (0,16.5) {} {};

\coordinate (X0) at (-5.35,-4.5) {} {} ;
\coordinate (Y0) at (9,0) {} ;
\coordinate (Z0) at (0,9) {} {} {} ;
\coordinate (XY) at (5,-4.5) {} {} {} ;
\coordinate (YZ) at (9,9) {} {} ;
\coordinate (XZ) at (-5.35,4.5) {} {};
\coordinate (XYZ) at (5,4.5) {} {} {};


\coordinate (HXY1) at (-0.64,-4.5) {} {} {};
\coordinate (HXY2) at (-0.64,4.5) {} {} {};
\coordinate (HZY1) at (4.5,9) {} {} {};
\coordinate (HZY2) at (4.5,0) {} {} {};


\draw [-{Latex[length=2mm]}] (O) -- (X);
\draw [-{Latex[length=2mm]}] (O) -- (Y);
\draw [-{Latex[length=2mm]}] (O) -- (Z);
\node[label=south east:$\theta$] at (X) {};
\node[label=south west:$\phi$] at (Y) {};
\node[label=south east:$\psi$] at (Z) {};
\node[label=north west:$\pi$] at (X0) {};
\node[label=south :$2\pi$] at (Y0) {};
\node[label=north east:$2\pi$] at (Z0) {};
%\node (Sb) [rectangle, minimum width=3cm, minimum height=1cm,draw=black, pattern color=black, pattern = north east lines]{} ;

\draw [] (X0) -- (XY);
\draw [] (Y0) -- (YZ);
\draw [] (Z0) -- (XZ);
\draw [] (X0) -- (XZ);
\draw [] (Y0) -- (XY);
\draw [] (XY) -- (XYZ);
\draw [] (XZ) -- (XYZ);
\draw [] (YZ) -- (XYZ);
\draw [] (YZ) -- (Z0);
\draw [] (O) -- (Z0);
\draw [] (O) -- (X0);
\draw [] (O) -- (Y0);

\draw[fill=white]  (X0) circle (0.1);
\draw[fill=white]  (Y0) circle (0.1);
\draw[fill=white]  (Z0) circle (0.1);

\draw[fill=white]  (XZ) circle (0.1);
\draw[fill=white]  (YZ) circle (0.1);
\draw[fill=white]  (XZ) circle (0.1);
\draw[fill=white]  (O) circle (0.1);
\draw[fill=white]  (XYZ) circle (0.1);

%\draw[pattern color=black, pattern = dots]  (O) rectangle (YZ) node (v0) {};
%\draw[pattern color=black, pattern = dots]  (X0) rectangle (XYZ) node (v1) {};
\coordinate (plane0) at (7.1,2.2) {};
\coordinate (plane1) at (-2.7,2.2) {};
\draw[fill=white]  (plane1) circle (0.1);
\draw[fill=white]  (plane0) circle (0.1);
%\draw[ decoration={markings, mark=at position 0.75 with {\arrow[scale = 1.]{Latex[length=3mm,reversed]}}},    postaction={decorate}](plane0) .. controls (29.5,-2) and (-28,-4) .. (plane1);
\draw[pattern color=black, pattern = dots]  (HXY1) node (v2) {} -- (HXY2) -- (HZY1) -- (HZY2) -- (HZY2) -- cycle;
\draw  (YZ) edge (XY);
\draw  (XYZ) edge (Y0);
\draw  (XZ) edge (O);
\draw  (Z0) edge (X0);
\node[label=north:$\mathbf{\Phi_{2\pi}}$] at (plane0) {};
\node[label=south:$\mathbf{\Phi_{0}}$] at (plane1) {};
\draw [ dashed,decoration={markings, mark=at position 0.55 with {\arrow[scale = 1.]{Latex[length=2mm]}}},    postaction={decorate}](plane1) arc(-70:261:-20cm and -5cm);
\end{tikzpicture}}
	\
    \subfloat[]{
\begin{tikzpicture}[scale=0.5]
\coordinate (O1) at (0,0);
\draw [thick] (O1) ellipse (6 and 0.6);
\coordinate (O2) at (0,-3);
\draw[dashed]  (O2) ellipse (6 and 0.6);
\coordinate (Om) at (0,-1.5);
\coordinate (O1s) at (0,0);
\draw[dashed]  (O1s) ellipse (3 and 0.3);
\coordinate (O2s) at (0,-3);
\draw[dashed]  (O2s) ellipse (3 and 0.3);
\coordinate (Oms) at (0,-1.5);
\draw[pattern color=black, pattern = dots]  (O1) ellipse (3 and 0.3);

%\draw  (O1) ellipse (1 and 0.1);
%\draw[dashed]  (O2) ellipse (1 and 0.1);
\coordinate (Plu) at (-6,0);
\coordinate (Pru) at (6,0);
\coordinate (Pld) at (-6,-3);
\coordinate (Prd) at (6,-3);
\coordinate (Plm) at (-6,-1.5);
\coordinate (Prm) at (6,-1.5);
\coordinate (Plds) at (-3,-3);
\coordinate (Prds) at (3,-3);
\coordinate (Plus) at (-3,0);
\coordinate (Prus) at (3,-0);

\coordinate (Qlu) at (-1,0);
\coordinate (Qru) at (1,0);
\coordinate (Qld) at (-1,-3);
\coordinate (Qrd) at (1,-3);
\draw [-{Latex[length=3mm]}] (Pld) -- (Plu);
\draw [] (Pru) -- (Prd);
\draw [dashed] (Plds) -- (Plus);
\draw [dashed] (Prds) -- (Prus);
%\draw [dashed] (Qlu) -- (Qld);
%\draw [dashed] (Qru) -- (Qrd);
\coordinate (Pu0) at (-1.5,-0.25) {};
\coordinate (Pd0) at (-1.5,-3.25) {};
\coordinate (Pu) at (-3.5,-0.5) {};
\coordinate (Pdl) at (-3.5,-3.5) {} {};
\coordinate (Pdr) at (3.5,-3.5) {} {};
\coordinate (Pmu) at (-3.5,-2) {} {};
\coordinate (Pml) at (-1.5,-1.5) {} {};
\draw[thick]  plot[ smooth,tension=.9] coordinates {(Pld) (Pdl) (Pdr) (Prd)};
\draw [dashed] (Plus) -- (Plu);
\draw[pattern color=gray!60, pattern = dots]  (Pu0) node (v2) {} -- (Pu) -- (Pdl) -- (Pd0) -- (Pu0) -- cycle;
\node[label=west:$\mathbf{\psi}$] at (Plm) {};
\node[label=north east :$\theta$] at (Pmu) {};
\draw  plot[ smooth,tension=.7] coordinates {(-3.5,0) (-3.3,-0.23) (-2.5,-0.35)};
\node[label=west :$\mathbf{\phi}$] at (-3.5,0) {};
\draw [-{Latex[length=1mm]}] (Pml) -- (Pmu);
\draw[fill=white]  (O1) circle (0.1);
\draw[fill=white]  (O2) circle (0.1);
%\draw[ decoration={markings, mark=at position 0.7 with {\arrow[scale = 1.5]{Latex[length=3mm,reversed]}}},    postaction={decorate}](O2) .. controls (3.5,-16.5) and (3.5,11) .. (O1);
\node[label=east:$\mathbf{\Psi_{2\pi}}$] at (Prus) {};
\node[label= east:$\mathbf{\Psi_{0}}$] at (Prds) {};
\draw[fill=black]  (Pml) circle (0.051);
\draw [dashed] (O1) -- (Plus);
\draw [dashed] (O1) -- (Pu0);
\coordinate (Om) at (-0.06,-2.6) {};
\draw [ dashed,decoration={markings, mark=at position 0.32 with {\arrow[scale = 1.]{Latex[length=3mm]}}},    postaction={decorate}](Om) arc (20:345:-1 and -4.5);
\end{tikzpicture}}
    \qquad
    \subfloat[]{\begin{tikzpicture}[scale=0.5]
\coordinate (O1) at (0,0);
\draw  (O1) ellipse (6 and 3.6);
\coordinate (t1l) at (-3,1.0);
\coordinate (t1r) at (3,1.0);
\coordinate (t1c1) at  (-4.5,-1.5);
\coordinate (t1c2) at (4.5,-1.5);
\draw (t1l) .. controls  (t1c1)  and (t1c2)  .. (t1r);

\coordinate (t2l) at (-2.7,-0.3) {};
\coordinate (t2r) at (2.7,-0.3) {};
\coordinate (t2c1) at (-4,2) {};
\coordinate (t2c2) at (4,2) {};
\draw (t2l) .. controls  (t2c1)  and (t2c2)  .. (t2r);
\draw[fill=white]  (O1) circle (0.1);
\coordinate (a0) at (0,2) {};
\coordinate (a1) at (2,1.5) {} {};
\draw [] (O1) -- (a0);
\draw [] (O1) -- (a1);
\coordinate (ac1) at (1,2) {} {};
\coordinate (ac2) at (1,2) {} {};

\draw [dashed] (a0) .. controls  (ac1)  and (ac2)  .. (a1);
\coordinate (b0) at (0,1.) {};
\coordinate (b1) at (0.9,0.7) {} {};
\coordinate (bc1) at (0.5,1) {} {} {};
\coordinate (bc2) at (0.5,1) {} {} {};
\coordinate (bc3) at (0,0.5) {} {} {} {};
\draw [-{Latex[length=2mm]}]  (b0) .. controls  (bc1)  and (bc2)  .. (b1);
\coordinate (ac3) at (0.5,0.7) {} {} {};
\node[label=north east:$\mathbf{\psi_{}}$] at (ac3) {};
\coordinate (C1) at (-4.35,-.4);
\draw [dashed] (C1) ellipse (1.55 and 1.4);
\draw [dashed,pattern color=gray!60, pattern = dots] (C1) ellipse (0.75 and 0.65);
\draw [dotted] (C1) ellipse (2.8 and 2.6);

\coordinate (C2a) at (-4.38,0.25) {};
\coordinate (C2b) at (-4.4,1) {};
\coordinate (C3b) at (-5.5,0.5) {};
\coordinate (C3a) at (-5,0) {} {};
\draw [dashed] (C1) -- (C2a);
\draw [-{Latex[length=2mm]}] (C2a) -- (C2b);
\draw [dashed] (C1) -- (C3b);
\draw[fill=black]  (C2a) circle (0.051);
%\draw[fill=black]  (C3a) circle (0.051);
\draw[fill=white]  (C1) circle (0.051);
\draw [-{Latex[length=2mm]}] (C2b) .. controls  (-5.1,0.9)  and (-5.1,0.9)  .. (C3b);
\node[label=south east:$\mathbf{\theta}$] at (C2b) {};
\node[label=north east:$\mathbf{\phi}$] at (C3b) {};
\node at (-7,-2.5) {};
\node[label=south east:$\mathbf{\theta = 2\pi}$] at (-7.5,-2) {};
\node at (-3,-1.5) {};
\node[label=south east:$\mathbf{\theta = \pi}$] at (-5,-1.5) {};
\end{tikzpicture}}
\caption{Map of the configuration space of a rigid body with fixed point.}
\label{fig:fig_p181}
\end{figure}
Consider figure $5.2 (a)$. We can stretch like an accordion the cuboid along the $\phi$ axis and bent it so that the planes $\phi=0$ and $\phi=2\pi$ join. We get $(b)$,  a torus with square sections. The dimension $\phi$ is dealt with as a point $P\left( \theta, \phi, \psi \right)$ in the configuration space  returns to the same point when varying $\phi$ to $\phi+2k\pi$.\\
We can apply the same procedure of stretching and bending for the $\psi$ dimension so that the planes $\Psi=0$ and $\Psi=2\pi$ join.
We get $(c)$,  a torus-like object.\\
The only dimension left is $\theta$ which our multi-dimensional crippled mind can't find a way to reshape this pseudo-torus so that when varying $\theta$ we can come back to the same point as started.
$$\blacklozenge$$
\newpage

\section{p186 - Exercise 1}
\begin{tcolorbox}
If a vector at the point with coordinates $\left(1,1,1\right)$ in Euclidean $3$-space has components $\left(3,-1,2\right)$, find the contravariant, covariant and physical components in spherical polar coordinates.
\end{tcolorbox}
The tensor $T_n$ to consider is $\left(3,-1,2\right) - \left(1,1,1\right)= \left(2,-2,1\right)$.\\
The Jacobian matrix for the transformation $z^n \rightarrow x^k$, evaluated at the point $\left(1,1,1\right)$ is 
\begin{align}
J_{\left(1,1,1\right)}&={\begin{pmatrix}{\dfrac {x}{r}}&{\dfrac {y}{r}}&{\dfrac {z}{r}}\\\\{\dfrac {xz}{r^{2}{\sqrt {x^{2}+y^{2}}}}}&{\dfrac {yz}{r^{2}{\sqrt {x^{2}+y^{2}}}}}&{\dfrac {-(x^{2}+y^{2})}{r^{2}{\sqrt {x^{2}+y^{2}}}}}\\\\{\dfrac {-y}{x^{2}+y^{2}}}&{\dfrac {x}{x^{2}+y^{2}}}&0\end{pmatrix}}\\
&=\begin{pmatrix}\dfrac {1}{\sqrt{3}}&\dfrac {1}{\sqrt{3}}&\dfrac {1}{\sqrt{3}}\\\\ \dfrac {1}{3\sqrt{2}}& \dfrac {1}{3\sqrt{2}}&-\dfrac {\sqrt{2}}{3}\\\\ -\dfrac {1}{2}&\dfrac {1}{2}&0\end{pmatrix}\\
\Rightarrow \spatie 
\begin{pmatrix}
r\\
\theta\\
\phi
\end{pmatrix}_{T^{'n}}&=\begin{pmatrix}\dfrac {1}{\sqrt{3}}&\dfrac {1}{\sqrt{3}}&\dfrac {1}{\sqrt{3}}\\\\ \dfrac {1}{3\sqrt{2}}& \dfrac {1}{3\sqrt{2}}&-\dfrac {\sqrt{2}}{3}\\\\ -\dfrac {1}{2}&\dfrac {1}{2}&0\end{pmatrix}\begin{pmatrix}
2\\
-2\\
1
\end{pmatrix}\\
&=\begin{pmatrix}
\dfrac {1}{\sqrt{3}}\\
-\dfrac {\sqrt{2}}{3}\\
-2
\end{pmatrix}
\end{align}
We have the metric tensor evaluated at $\left(1,1,1\right)$
\begin{align}
a_{mn} &= \begin{pmatrix}
1&0&0\\\\
0&r^2&0\\\\
0&0&r^2\sin^2\theta\\\\
\end{pmatrix}=\begin{pmatrix}
1&0&0\\\\
0&3&0\\\\
0&0&2\\\\
\end{pmatrix}\\
\Rightarrow \spatie 
\begin{pmatrix}
r\\
\theta\\
\phi
\end{pmatrix}_{T^{'}_n}&=\begin{pmatrix}
1&0&0\\\\
0&3&0\\\\
0&0&2\\\\
\end{pmatrix}\begin{pmatrix}
\dfrac {1}{\sqrt{3}}\\
-\dfrac {\sqrt{2}}{3}\\
-2
\end{pmatrix}\\
&=\begin{pmatrix}
\dfrac {1}{\sqrt{3}}\\
-\sqrt{2}\\
-4
\end{pmatrix}
\end{align}
And the physical components 
\begin{align}
\begin{pmatrix}
r\\
\theta\\
\phi
\end{pmatrix}_{T^{'}_{ph.}}&=\begin{pmatrix}
1&0&0\\\\
0&\frac{1}{\sqrt{3}}&0\\\\
0&0&\frac{1}{\sqrt{2}}\\\\
\end{pmatrix}\begin{pmatrix}
\dfrac {1}{\sqrt{3}}\\
-\sqrt{2}\\
-4
\end{pmatrix}\\
&=\begin{pmatrix}
\dfrac {1}{\sqrt{3}}\\
-\sqrt{\dfrac {{2}}{{3}}}\\
-2\sqrt{2}
\end{pmatrix}
\end{align}
Another way to find the physical components is to project orthogonally the tensor on the unit vectors of a local Cartesian coordinate system, oriented along the unit vectors $\overline{e}_r,\overline{e}_{\theta},\overline{e}_{\phi}$ corresponding to the vector $P \left(1,1,1\right)$ with modulus $\left|P \right|=\sqrt{3}$. 
We have for the tensor $T_n (2,-2,1)$ with modulus $\left|T_n \right|=3$ as component along $\overline{e}_r$:
\begin{align}
\left|T_n \right|\cos \alpha &= \left|T_n \right|\frac{\left<T_n,P  \right>}{\left|T_n \right|\left|P \right|}\\
&= \left|T_n \right|\frac{2-2+1}{\left|T_n \right|\left|P \right|}\\
&= \frac{1}{\sqrt{3}}
\end{align}
For the component along $\overline{e}_{\theta}$ we first have to determine the vector $\overline{e}_{\theta}$. As first equation we have the orthogonality condition with $\overline{e}_r$ and putting $\overline{e}_{\theta} = (a,b,c)$, get $\left<\overline{e}_r,\overline{e}_{\theta}  \right>=  a+b+c=0$. As $\overline{e}_{\theta}$ lies in the plane $(1,1,0)-(0,0,0)-(0,0,1)$ we can put $a=b$ and get $\overline{e}_{\theta} =  \frac{1}{\sqrt{6}}\left(1,1,-2\right)$ and get for the tensor $T_n (2,-2,1)$  as component along $\overline{e}_{\theta}$:
\begin{align}
\left|T_n \right|\cos \beta &= \left|T_n \right|\frac{\left<T_n,\overline{e}_{\theta}  \right>}{\left|T_n \right|}\\
&= \left|T_n \right|\frac{2-2-2}{\left|T_n \right|\sqrt{6}}\\
&= -\frac{\sqrt{2}}{\sqrt{3}}
\end{align}
For the component along $\overline{e}_{\phi}$ we first have to determine the vector $\overline{e}_{\phi}$. As first equation we have the orthogonality condition with the pair $\overline{e}_r,\overline{e}_{\theta}$  and  get $\overline{e}_{\phi} = \overline{e}_r \times \overline{e}_{\theta}  =  \frac{1}{\sqrt{3}\sqrt{6}}\left( -3,3,0\right)= \left( -\frac{1}{\sqrt{2}},\frac{1}{\sqrt{2}},0\right)$.\\
For the tensor $T_n (2,-2,1)$  as component along $\overline{e}_{\phi}$:
\begin{align}
\left|T_n \right|\cos \gamma &= \left|T_n \right|\frac{\left<T_n,\overline{e}_{\phi}  \right>}{\left|T_n \right|}\\
&= \left|T_n \right|\frac{-2-2}{\left|T_n \right|\sqrt{2}}\\
&= -\frac{4}{\sqrt{2}}\\
&= -2\sqrt{2}
\end{align}
giving
\begin{align}
\begin{pmatrix}
r\\
\theta\\
\phi
\end{pmatrix}_{T^{'}_{ph.}}
&=\begin{pmatrix}
\dfrac {1}{\sqrt{3}}\\
-\sqrt{\dfrac {{2}}{{3}}}\\
-2\sqrt{2}
\end{pmatrix}
\end{align}
as in (9).
$$\blacklozenge$$
\newpage



\section{p186 - Exercise 2}
\begin{tcolorbox}
In cylindrical coordinates $\left(r, \phi,z\right)$ in Euclidean $3$-space, a vector field is such that the vector at each point points along the parametric line of $\phi$, in the sense of $\phi$ increasing, and its magnitude is $kr$, where $k$ is a constant. Find the contravariant, covariant and physical components of this vector field.
\end{tcolorbox}
We can work backwards, with the physical components as starting point. Indeed, at a point $P\left(r,\phi,z\right)$ the tensor of this vector field will have $\left(0,kr,0\right)$ as physical components in the cylindrical coordinates $\left( r, \phi,z \right)$ system.\\
We have the metric tensor 
\begin{align}
a_{mn} &= \begin{pmatrix}
1&0&0\\\\
0&r^2&0\\\\
0&0&1\\\\
\end{pmatrix}
\end{align}
Giving
\begin{align}
\left\{\begin{array}{lll}
X_1 = &h_1 X_{1}^{phys.}&=0\\\\
X_2 = &h_2 X_{2}^{phys.}&=kr^2\\\\
X_3 = &h_3 X_{3}^{phys.}&=0\\\\
\end{array}\right.
\end{align}
and 
\begin{align}
\left\{\begin{array}{lll}
X^1 = &\frac {X_{1}^{phys.}}{h_1}&=0\\\\
X^2 = &\frac {X_{2}^{phys.}}{h_2}&=k\\\\
X^3 = &\frac {X_{3}^{phys.}}{h_3}&=0\\\\
\end{array}\right.
\end{align}

$$\blacklozenge$$
\newpage




\section{p186 - Exercise 3}
\begin{tcolorbox}
Find the physical components of velocity and acceleration along the parametric lines of cylindrical coordinates in terms of the  and their derivatives with respect to time.
\end{tcolorbox}
We have the metric tensor 
\begin{align}
a_{mn} &= \begin{pmatrix}
1&0&0\\\\
0&r^2&0\\\\
0&0&1\\\\
\end{pmatrix}
\end{align}
and the contravariant velocities

\begin{align}
\left\{\begin{array}{lll}
v^1= &\dv{r}{t}\\\\
v^2= &\dv{\phi}{t}\\\\
v^3= &\dv{z}{t}\\\\
\end{array}\right.
\end{align}
giving by $v_{K}^{phys.} = h_K v^K$
\begin{align}
\left\{\begin{array}{lll}
v_r= &\dv{r}{t}\\\\
v_{\phi}= &r\dv{\phi}{t}\\\\
v_z= &\dv{z}{t}\\\\
\end{array}\right.
\end{align}
For the acceleration using $f^r=\fdv{v^r}{t} $
and the Christoffel symbols being 
\begin{align}
\left \{ \begin{array}{c}
\Gamma^m_{nk} = 0 \quad\forall\quad (nk) \ne (r, \theta), (\theta, \theta)\\\\
\Gamma^{\theta}_{r\theta} = \frac{1}{r} \quad\text{and}\quad \Gamma^r_{\theta\theta} = -r
\end{array}\right.\
\end{align}
we have
\begin{align}
\left\{\begin{array}{lll}
f^1= &\dv{v^1}{t}-r\underbrace{v^2\dv{x^2}{t}}_{=\left(v^2\right)^2}\\\\
f^2= &\dv{v^2}{t}+\underbrace{\frac{1}{r}v^1\dv{x^2}{t}+\frac{1}{r}v^2\dv{x^21}{t}}_{=\frac{2}{r}v^1 v^2}\\\\
f^3= &\dv{v^3}{t}\\\\
\end{array}\right.
\end{align}
giving by $f_{K}^{phys.} = h_K f^K$
\begin{align}
&\left\{\begin{array}{lll}
f_r= &\dv{v^1}{t}-r\left(v^2\right)^2\\\\
f_{phi}= &r\dv{v^2}{t}+r\frac{2}{r}v^1 v^2\\\\
f_z= &\dv{v^3}{t}\\\\
\end{array}\right.\\
\Rightarrow \spatie &\left\{\begin{array}{lll}
f_r= &\dv[2]{r}{t}-r\left(\dv{\phi}{t}\right)^2\\\\
f_{phi}= &r\dv[2]{\phi}{t}+2\dv{r}{t} \dv{\phi}{t}\\\\
f_z= &\dv[2]{z}{t}\\\\
\end{array}\right.
\end{align}
$$\blacklozenge$$
\newpage



\section{p186 - Exercise 4}
\begin{tcolorbox}
A particle moves on a sphere under the action of gravity. Find the contravariant an covaraiant components of the force, using colatitude and azimuth, and write down the equation of motion.
\end{tcolorbox}
We determine first the physical components of the force.
\begin{figure}[H]

\begin{tikzpicture}[scale=0.5]
\coordinate (O1) at (0,0);
\draw  (O1) circle (6);
%\draw [thick] (O1) ellipse (6 and 0.6);
\coordinate (Om) at (0.0,-6) ;
%\node at (Om){$P$};
\draw [ decoration={markings, mark=at position 0.32 with {\arrow[scale = 1.]{Latex[length=3mm]}}},    postaction={}](Om) arc (90:-90:3 and -6.0);
\draw [ dashed,decoration={markings, mark=at position 0.32 with {\arrow[scale = 1.]{Latex[length=3mm]}}},    postaction={}](Om) arc (90:-90:-3 and -6.0);
\coordinate (Ohl) at (-5.2,3) {} {} {};
\coordinate (Ohr) at (5.2,3) {} {} {};
%\node at (Oh){$P$};
%\draw[fill=white]  (Ohl) circle (0.1);
%\draw[fill=white]  (Ohr) circle (0.1);
\draw [ decoration={markings, mark=at position 0.32 with {\arrow[scale = 1.]{Latex[length=3mm]}}},    postaction={}](Ohl) arc (-10:190:-5.3 and -0.5);
\draw [dashed ,decoration={markings, mark=at position 0.32 with {\arrow[scale = 1.]{Latex[length=3mm]}}},    postaction={}](Ohl) arc (10:190:-5.3 and 0.5);
\coordinate (NPole) at (0,6) {};
\draw[fill=white]  (NPole) circle (0.1);
\coordinate (SPole) at (0,-6) {};
\draw[dashdotted] (NPole) -- (SPole);
\draw[fill=white]  (SPole) circle (0.1);
\coordinate (P) at (2.7,2.5) {};
\draw[fill=white]  (P) circle (0.1);
\draw[fill=white]  (O1) circle (0.1);
\coordinate (etheta) at (3.5,-1) {};
\coordinate (ephi)  at (7,2.5) {};
\draw [-{Latex[length=2mm]}] (P) -- (etheta);
\draw [-{Latex[length=2mm]}] (P) -- (ephi);
\draw[dashed] (O1) -- (P);
\node[label=east:$\mathbf{\overline{1}_{\theta}}$] at (etheta) {};
\node[label=east:$\mathbf{\overline{1}_\phi}$] at (ephi) {};
\node[label=west:$O$] at (O1) {};
\coordinate (O2) at (-0.10,2);
\node[label=south east:$\theta$] at (O2) {};
\draw [dashed, decoration={markings, mark=at position 0.52 with {\arrow[scale = 1.]{Latex[length=2mm]}}},    postaction={decorate}](O2) arc (90:55:3 and 3.0);
\coordinate (F)  at (2.7,0) {};
\draw [-{Latex[length=2mm]}] (P) -- (F);
\node[label=west:$\mathbf{\overline{F}}$] at (F) {};
\coordinate (D0) at (11,-1) {};
\coordinate (DP) at (15.5,4) {};
\node[label=east:$\mathbf{m}$] at (DP) {};
\coordinate (De) at (20.5,0) {};
\coordinate (DF) at (15.5,-3.5) {};
\coordinate (DA) at (11,8.5) {};
\draw [dashed] (D0) -- (DP);
\draw [dashed] (D0) -- (DA);
\draw [-{Latex[length=2mm]}] (DP) -- (De);
\node[label=east:$\mathbf{\overline{1}_{\theta}}$] at (De) {};
\draw [-{Latex[length=2mm]}] (DP) -- (DF);
\node[label=east:$\mathbf{\overline{F}}$] at (DF) {};
\coordinate (thet) at  (11,3) {};
\draw [dashed, decoration={markings, mark=at position 0.52 with {\arrow[scale = 1.]{Latex[length=2mm]}}},    postaction={decorate}](thet) arc (90:35:3 and 3.0);
\node[label=south east:$\theta$] at (thet) {};
\coordinate (thet2) at (15.5,0.5) {};
\draw [dashed, decoration={markings, mark=at position 0.52 with {\arrow[scale = 1.]{Latex[length=2mm]}}},    postaction={decorate}](thet2) arc (90:30:3 and -3.0);
\node[label=south east:$\frac{\pi}{2}-\theta$] at (thet2) {};
\draw[fill=white]  (DP) circle (0.1);
\end{tikzpicture}

\caption{Physical components of the gravitational force tensor acting on a mass $\mathbf{m}$ on a sphere }
\label{fig:fig_p186_Ex2}
\end{figure}
We note first that the unit vector $\overline{1}_{\phi}$ is perpendicular to the place formed by the vectors  $\overline{1}_{\theta},\overline{F}$ and s the force has no components projected on this vector. The vector $\overline{F}$ is parallel with the axis of reference of the sphere with radius $R$ and so the physical components become
\begin{align}
&\left\{\begin{array}{l}
F_{\phi}^{phys}= 0\\\\
F_{\theta}^{phys}= mg\sin{\theta}\\\\
\end{array}\right.\\
\Rightarrow\spatie &\left\{\begin{array}{ll}
F_{\phi}= 0&F_{\phi}= 0\\\\
F^{\theta}= \frac{1}{R}mg\sin{\theta}&F_{\theta}= R m g\sin{\theta}\\\\
\end{array}\right.
\end{align}
We use equation $\mathbf{5.212.}$
\begin{align}
&\left\{\begin{array}{l}
\dv{}{t}\frac{\partial T}{\partial \dot{x}^s} - \frac{\partial T}{\partial x^s} = F_{s}\\\\
T = \half m a_{pq}\dot{x}^p\dot{x}^q, \ \dot{x}^s=\dv{x^s}{t}\\\\
\end{array}\right.
\end{align}
with for our case
\begin{align}
T = \half m R^2\left(\dot{\theta}^2+ \sin^2 \theta \ \dot{\phi}^2 \right)
\end{align}
and get the set of equation of motion (the second column gives the dimensional analysis as a check for consistency)

\begin{align}
&\left\{\begin{array}{lll}
\frac{\ddot{\phi}}{\dot{\phi}}=  -2\cot\theta \  \dot{\theta}&:&\frac{[T]^{-2}}{[T]^{-1}}\cong [T]^{-1}\\\\
\ddot{\theta} - \left(\dot{\phi}\right)^2\sin\theta \cos\theta= \frac{g}{R}\sin{\theta}&:&[T]^{-2}+\left([T]^{-1}\right)^2 \cong \frac{[L][T]^{-2}}{[L]^{}}\\\
\end{array}\right.
\end{align}
Let's check the special case when $\dot{\phi} = 0$.\\
The first equation can be rewritten and gives of course $\phi=C$ while the second equation becomes $$\ddot{\theta} = \frac{g}{R}\sin{\theta}$$
which is similar to the equation of  the simple gravity pendulum.
$$\blacklozenge$$
\newpage

\section{p186 - Exercise 5}
\begin{tcolorbox}
Consider the motion of a particle on a smooth torus under no forces except normal reaction. The geometrical line element may be written $$ ds^2=\left(a-b\cos \theta \right)^2 d{\phi}^2+b^2 d\theta^2$$ where $\phi$ is an azimuthal angle and $\theta$ an angular displacement from the equatorial plane. Show that  the path of a particle satisfies the following two differential equations in which $h$ is a constant 
$$(a)\spatie \left(a-b\cos\theta\right)^2\dv{\phi}{s} = h$$
$$(b)\spatie b^2\left(\dv{\theta}{\phi}\right)^2= \frac{\left(a-b\cos\theta\right)^4}{h^2}-\left(a-b\cos\theta\right)^2$$
\end{tcolorbox}
We use equation $\mathbf{5.212.}$
\begin{align}
&\left\{\begin{array}{l}
\dv{}{t}\frac{\partial T}{\partial \dot{x}^s} - \frac{\partial T}{\partial x^s} = F_{s}\\\\
T = \half m a_{pq}\dot{x}^p\dot{x}^q, \ \dot{x}^s=\dv{x^s}{t}\\\\
\end{array}\right.
\end{align}
with for our case
\begin{align}
T = \half m \left(b^2\dot{\theta}^2+ \left(a-b\cos \theta \right)^2  \ \dot{\phi}^2 \right)
\end{align}giving
\begin{align}
&\left\{\begin{array}{ll}
\frac{\partial T}{\partial \dot{\phi}}= m \left(a-b\cos \theta \right)^2   \dot{\phi}&\frac{\partial T}{\partial {\phi}}= 0\\\\
\frac{\partial T}{\partial \dot{\theta}}=  mb^2\dot{\theta}&\frac{\partial T}{\partial {\theta}}=  mb\left(a-b\cos \theta \right)\dot{\phi}^2\sin \theta\\\\
\end{array}\right.\\
\Rightarrow\spatie &\left\{\begin{array}{l}
\left(a-b\cos \theta \right)^2   \ddot{\phi}+2b\left(a-b\cos \theta \right)\dot{\theta} \dot{\phi}\sin \theta  =0\\\\
b^2\ddot{\theta}-b\left(a-b\cos \theta \right)\dot{\phi}^2\sin \theta=0\\\\
\end{array}\right.\\
\Rightarrow\spatie &\left\{\begin{array}{l}
\left(a-b\cos \theta \right)   \ddot{\phi}=-2b\dot{\theta} \dot{\phi}\sin \theta  \\\\
b^2\ddot{\theta}-b\left(a-b\cos \theta \right)\dot{\phi}^2\sin \theta=0\\\\
\end{array}\right.
\end{align}
In the first equation, put $y \equiv \dot{\phi}$ giving for the first equation:
\begin{align}
\frac{dy}{y}&=-2b \frac{\sin \theta d\theta} {\left(a-b\cos \theta \right)}  \\
\Leftrightarrow\spatie\frac{dy}{y}&=-2 \frac{d\left(a-b\cos \theta \right)} {\left(a-b\cos \theta \right)}\\
\Rightarrow \spatie \log y&=-2 \log \left(a-b\cos \theta \right)+\log C^{}\\
\Rightarrow \spatie \dot{\phi}&= C \left(a-b\cos \theta \right)^{-2}
\end{align}
Note that $\dot{\phi}$ is a time derivative. But as we are on a geodesic, $\mathbf{5.226.}$ stands and so $v$ is constant as $\dv{v}{s} =0$. Using $v = \dv{s}{t}$, (9) can be written as 
\begin{align}
\left(a-b\cos \theta \right)^{2}\dv{\phi}{t}&= C \\
\Leftrightarrow\spatie\left(a-b\cos \theta \right)^{2}\dv{\phi}{s}\underbrace{\dv{s}{t}}_{=v}&= C \\
\Leftrightarrow\spatie\left(a-b\cos \theta \right)^{2}\dv{\phi}{s}&= h \quad \text{with } h=\frac{C}{v}
\end{align}
We don't use the second equation in (5) but the line element equation instead
\begin{align}
ds^2&=\left(a-b\cos \theta \right)^2 d{\phi}^2+b^2 d\theta^2\\
\Rightarrow\spatie \left(\dv{s}{\phi}\right)^2 &= \left(a-b\cos \theta \right)^2 +b^2 \left(\dv{\theta}{\phi}\right)^2\\
\Rightarrow\spatie b^2 \left(\dv{\theta}{\phi}\right)^2 &= \left(\dv{\phi}{s}\right)^{-2 }- \left(a-b\cos \theta \right)^2 \\
\left(12\right)\quad \text{: }\spatie b^2 \left(\dv{\theta}{\phi}\right)^2 &= \frac{\left(a-b\cos \theta \right)^4}{h^2}- \left(a-b\cos \theta \right)^2
\end{align}
$$\blacklozenge$$
\newpage
