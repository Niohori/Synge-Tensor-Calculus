\setcounter{chapter}{4}
\chapter{Applications to Classical Mechanics}
\pagebreak[4]
\section{p153 - Exercise}
\begin{tcolorbox}
If $\mu^{\alpha}$ are the contravariant components of a unit vector in a surface $S$, show that $\mu^{\alpha}f_{\alpha}$ is the physical component of acceleration in the direction tangent to $S$ defined by $\mu^{\alpha}$.
\end{tcolorbox}
As we are in an Euclidean space we can interpret $a_{mn}\mu^{\alpha}f^{\alpha}$ as $\left|\mu\right|\left|f\right|\cos\theta $ with $\theta$ the angle between the two vectors. As $\left|\mu\right|=1$ we have
\begin{align}
a_{mn}\mu^{\alpha}f^{\alpha}&= \mu^{\alpha}f_{\alpha}\\
&= \left|f\right|\cos\theta 
\end{align}
which is the projection of the vector $f$ on the unit vector $\mu$.
$$\blacklozenge$$
\newpage

\section{p154 - Clarification to 5.226.}
\begin{tcolorbox}
$$\mathbf{\text{5.226.}\spatie v\dv{v}{s}=0,\quad \overline{\kappa}v^2=0}$$
Assuming that the particle is not at rest $v\ne 0$, and therefore $\overline{\kappa}=0$. \textit{\textbf{Since this implies that the curve is a geodesic}...}
\end{tcolorbox}
The assertion in bold is a direct consequence $$\mathbf{\text{2.513.}}\spatie \frac{\delta \dv{x^r}{s}}{\delta s}=0$$ 
As in $ \mathbf{5.233}$ we have $\frac{\delta \lambda^{\alpha}}{\delta s}=\frac{\delta \dv{x^{\alpha}}{s}}{\delta s}=0$, the considered curve follows the geodesic curve.
$$\blacklozenge$$
\newpage

\section{p155 - Exercise}
\begin{tcolorbox}
Show that in relativity the force $4$-vector $X^r$ lies along the first normal of the trajectory in space-time. Express the first curvature in terms of the proper mass $m$ of the particle and the magnitude $X$ of $ X^r$.
\end{tcolorbox}
Let us recall the first Frenet formula $\mathbf{2.705}$ without forgetting that the metric form is not positive-definite, $$\frac{\delta \lambda^r}{\delta s}=\kappa\nu^r,\quad \epsilon_{(1)}\nu_n\nu^n=1$$ As $\mathbf{5.299}$ $$m\frac{\delta \lambda^r}{\delta s}=X^r$$ it is clear that $X^r = m\kappa\nu^r$ and is collinear with the first normal.
\begin{align}
X^r &= m\kappa\nu^r\\
\times \quad a_{mr}X^m\quad\Rightarrow\spatie \underbrace{a_{mr}X^mX^r}_{=\left(X^1\right)^2+\left(X^2\right)^2+\left(X^3\right)^2-\left(X^4\right)^2} &= m\kappa \underbrace{a_{mr}\nu^m\nu^r}_{= \epsilon_{(1)}}
\end{align}
\textbf{$$\Rightarrow\spatie \kappa = \epsilon_{(1)}\frac{\left(X^1\right)^2+\left(X^2\right)^2+\left(X^3\right)^2-\left(X^4\right)^2}{m}
$$}
$$\blacklozenge$$
\newpage


\section{p156 - Clarification}
\begin{tcolorbox}
Interpretation of 
$$\mathbf{5.231.}\spatie M_{rs}=\epsilon_{rsn}M_n=z_rF_s-z_sF_r$$
\end{tcolorbox}
What do the $M_{rs}$ represent?
\begin{figure}[H]

\begin{tikzpicture}[scale=0.75]
\tikzstyle{left-hand-mirror} = [
    draw,
    postaction=decorate, 
    decoration={
        markings,
        mark=between positions 0.015 and 0.98 step 0.1072 with {\draw (0,0)--(60:3pt);}
    }
]  
\coordinate (O) at (0,0);
\coordinate (X) at (-5,-5);
\coordinate (Y) at (10,0);
\coordinate (Z) at (0,10);
\draw [-{Latex[length=3mm]}] (O) -- (X);
\draw [-{Latex[length=3mm]}] (O) -- (Y);
\draw [-{Latex[length=3mm]}] (O) -- (Z);
\node[label=north west:$z_1$] at (X) {};
\node[label=north east:$z_2$] at (Y) {};
\node[label=north west:$z_3$] at (Z) {};
\coordinate (P) at (5,3);
\draw [-{Latex[length=3mm]},ultra thick] (O) -- (P);
\node[label=north west:$P$] at (P) {};
\coordinate (F) at (8,7) {};
\draw [-{Latex[length=3mm]}, ultra thick] (P) -- (F);
\node[above right] at (F) {$\overrightarrow{F}$};
\coordinate (Fp) at (8-5,7-3) {};
\draw [-{Latex[length=3mm]},dashdotted,ultra thick] (O) -- (Fp);
\node[above right,] at (Fp) {$\overrightarrow{F^{'}}$};
\coordinate (Px) at (-2,-2) {};
\coordinate (Py) at (6.8,0) {};
\node[label=north west:$P_1$] at (Px) {};
\node[label=north west:$P_2$] at (Py) {};
\coordinate (Pp) at (5,-2) {};
\draw [dashed] (Pp) -- (P);
\draw [dashed] (Pp) -- (Px);
\draw [dashed] (Pp) -- (Py);
\coordinate (Fpp) at (3,-1) {};
\coordinate (Fx) at (-1,-1) {} {};
\coordinate (Fy) at (4,0) {} {};
\node[label=north west:$F_1$] at (Fx) {};
\node[label=north west:$F_2$] at (Fy) {};
\draw[fill = black]  (Fx) circle (0.1);
\draw[fill = black]  (Fy) circle (0.1);
\draw [dashed] (Fp) -- (Fpp);
\draw [dashed] (Fpp) -- (Fx);
\draw [dashed] (Fpp) -- (Fy);

\coordinate (Fppp) at (6,-1) {} {};
\coordinate (Pppp) at (2,-2) {} {} {};
\node[{anchor=north west }] at (Fppp) {$\overrightarrow{F_1}$};
\node[{anchor=south east }] at (Pppp) {$\overrightarrow{F_2}$};
\draw [-{Latex[length=3mm]}, ultra thick] (O) -- (Px);
\draw [-{Latex[length=3mm]},ultra thick] (O) -- (Py);
\draw [-{Latex[length=3mm]}, ultra thick] (Px) -- (Pppp);
\draw [-{Latex[length=3mm]},ultra thick] (Py) -- (Fppp);
%\node[label=north west:$K$] at (Fppp) {};
%\node[label=north west:$S$] at (Pppp) {};
\draw [dashed] (Fp) -- (Fpp);
\draw [dashed] (Fppp) -- (Fx);
\draw [dashed] (Pppp) -- (Fy);
%\filldraw[ultra thick, gray!10] (Px) -- (Pppp) -- (Fy) -- (O) -- (Px) -- cycle;
%\filldraw[ultra thick,gray!20] (Fx) -- (Fppp) -- (Py) -- (O) -- (Px) -- cycle;
%\draw[ultra thick, gray!80] (Px) -- (Pppp) -- (Fy) -- (O) -- (Px) -- cycle;
%\draw[ultra thick,gray!80] (Fx) -- (Fppp) -- (Py) -- (O) -- (Px) -- cycle;
\coordinate (Vp1) at (0,6) {} {};
\coordinate (Vp2) at (0,9) {} {} {};
\draw [-{Latex[length=3mm]}, ultra thick] (O) -- (Vp1);
\draw [-{Latex[length=3mm]}, ultra thick] (O) -- (Vp2);
\node[{anchor=north west }] at (Vp1) {$\overrightarrow{P_1}\times\overrightarrow{F_2}$};
\node[{anchor=north west }] at (Vp2) {-$\overrightarrow{P_2}\times\overrightarrow{F_1}$};
\draw[fill=white]  (Py) circle (0.1);
\draw[fill=white]  (Px) circle (0.1);
\draw[fill=white]  (P) circle (0.1);
\draw[decoration={markings, mark=at position 0.1 with {\arrow[scale = 1.5]{latex[]}}},
    postaction={decorate}](0,4.3) ellipse (1 and 0.2);
    \draw[decoration={markings, mark=at position 0.1 with {\arrow[scale = 1.5]{latex[reversed]}}},
    postaction={decorate}](0,7.3) ellipse (1 and 0.2);
 \coordinate (Qx) at (3.9,1.7) {} {};
\coordinate (Qy) at (8,3) {} {} {};
\draw [-{Latex[length=3mm]},dotted, ultra thick] (P) -- (Qx);
\draw [-{Latex[length=3mm]},dotted, ultra thick] (P) -- (Qy);
\node[{anchor=north west }] at (Qx) {$\overrightarrow{F_1}$};
\node[{anchor=north west }] at (Qy) {$\overrightarrow{F_2}$};
\end{tikzpicture}
\caption{Interpretation of the tensor moment $M_{12}$}
\label{fig:fig_p156_5320}
\end{figure}
Let's consider a mass point $P$ on which a force $\overrightarrow{F}$ is acting. The force has components $\left(F_x,F_y,F_z\right)$ in the  space $V^{'}_3$ (which is by the way not the space $V_3$ of the considered mass point).\\
Let's investigate the element $M_{12}$ of the \textit{tensor moment}.\\
$P_1F_2\overrightarrow{e_3}$ is the vector product $\overrightarrow{P_1}\times\overrightarrow{F_2}$ and is as such the torque of the component $F_2$ of $\overrightarrow{F}$ acting on the mass point situated at $P_1$. The origin being fixed, $\overrightarrow{F_2}$ tries to move $P_1$, clockwise along the $z_3$ axis. The same is true for the component $\overrightarrow{F_1}$ acting on the mass point situated at $P_2$, and is represented here by the vector $- \overrightarrow{P_2}\times\overrightarrow{F_1}$ ($\overrightarrow{F_1}$ tries to move  $P_2$, counter clockwise along the $z_3$ axis). \\
Hence, $P_1F_2-P_2F_1$ is the net force trying to move the point $P$ along the $z_3$ axis (i.e. in the plane $\parallel$ with the $z_3=0$ plane).
$$\blacklozenge$$
\newpage


\section{p156 - Clarification}
\begin{tcolorbox}
$$\mathbf{5.234.}\spatie \dv{h_r}{t}= M_r$$
\end{tcolorbox}
\begin{align}
h_r &= m\epsilon_{rmn}z_mv_n\\
\Rightarrow \spatie \dv{h_r}{t} &= m\epsilon_{rmn}\dv{z_m}{t} v_n+m\epsilon_{rmn}z_m\dv{v_n}{t}\\
&= m\underbrace{\epsilon_{rmn}v_m v_n}_{=0}+\underbrace{\epsilon_{rmn}z_mF_n}_{=M_r}\\
&=M_r
\end{align}
$$\blacklozenge$$
\newpage



\section{p158-159 - Clarification}
\begin{tcolorbox}
$$\mathbf{5.313.}\spatie \omega_{rs}= -\omega_{sr}$$ From 5.310 and the vector character of $v_r$ and $z_r$ (for transformations which do not change the origin), \textbf{it follows that $\omega_{rs} $ is a Cartesian tensor of second order}.
\end{tcolorbox}
Be 
\begin{align}
v^{}_r = -\omega^{}_{rn}z^{}_n
\end{align}
Considering orthogonal transformation in a flat space $z^{'}_m = A_{mr}z^{}_r+B_m$ with  $B_m=0$ as we consider only transformations which do not change the origin. Differentiation with the parameter $t$ gives 
\begin{align}
v^{'}_m &= A_{mr}v^{}_r\\
&= -\omega^{}_{rn}A_{mr}z^{}_n\\
\end{align}
But $z^{'}_q = A_{qr}z^{}_r\quad\Rightarrow \quad A_{qn}z^{'}_q = A_{qn}A_{qr}z^{}_r\quad\Rightarrow \quad A_{qn}z^{'}_q = z^{}_n$ 
Hence
\begin{align}
v^{'}_m &= -\omega^{}_{rn}A_{mr}z^{}_n\\
&= -\underbrace{\omega^{}_{rn}A_{mr}A_{qn}}_{\overset{\underset{\mathrm{def}}{}}{=}\omega_{mq}^{'}}z^{'}_q\\
v^{'}_m &= -\omega_{mq}^{'}z^{'}_q
\end{align}
$$\blacklozenge$$
\newpage


\section{p159 - Exercise}
\begin{tcolorbox}
Show that if a rigid body rotates about the point $z_r=b_r$ as fixed point, the velociy of a general point of the body is given by $$v_r=-\omega_{rm}\left(z_m-b_m\right)$$
\end{tcolorbox}
By $\mathbf{5.302. }$:
\begin{align}
\left(z^{(1)}_m-z^{(2)}_m\right)\left(dz^{(1)}_m-dz^{(2)}_m\right)=0
\end{align}
At the fixed point we have $z^{(2)}_m=b_m$ and $dz^{(2)}_m=0$, hence
\begin{align}
\left(z^{(1)}_m-b_m\right)\left(dz^{(1)}_m\right)=0\\
\Rightarrow\spatie z^{(1)}_mdz^{(1)}_m =b_mdz^{(1)}_m
\end{align}
As this is true for any point of the rigid mass, expanding (1) and using (3) we get when dividing by $dt$
\begin{align}
\left(z^{(2)}_m-b_m\right)v^{(1)}_m+\left(z^{(1)}_m-b_m\right)v^{(2)}_m=0
\end{align}
Taking twice the partial derivative $\frac{\partial^2}{\partial z^{(1)}_p\partial z^{(1)}_q}$ we get
\begin{align}
\left(z^{(2)}_m-b_m\right)\frac{\partial^2 v_m}{\partial z^{(1)}_p\partial z^{(1)}_q}=0
\end{align}
As this is true for any arbitrary point in the rigid body we get
\begin{align}
\frac{\partial^2 v_m}{\partial z^{(1)}_p\partial z^{(1)}_q}=0\\
\Rightarrow\spatie v_m= K_{mr}z_r+B_m
\end{align}
At the fixed point we have 
\begin{align}
 K_{mr}b_r+B_m =0
\end{align}
Plugging this in (7)
\begin{align}
 v_m= K_{mr}\left(z_r-b_m\right)
\end{align}
Putting $K_{mr}=-\omega_{mr}$ gives us indeed the asked expression.
$$\blacklozenge$$
\newpage


\section{p161 - Clarification}
\begin{tcolorbox}
$$\mathbf{5.325.}\spatie \Omega_{np}\sum \left(mf_nz_p\right)= \Omega_{np}\sum F_nz_p $$
and hence, since $\Omega_{np}$ is arbitrary,
$$\mathbf{5.326.}\spatie \sum m\left(f_nz_p-f_pz_n\right)= \sum \left(F_nz_p-F_pz_n \right)$$
\end{tcolorbox}
To be complete the following step should be inserted

\begin{align}
\Omega_{np}\sum \left(mf_nz_p\right)&= \Omega_{np}\sum F_nz_p \\
\text{As }\Omega_{np}\text{ is skew-symmetric:}\spatie -\Omega_{np}\sum \left(mf_pz_n\right)&= -\Omega_{np}\sum F_pz_n \\
\text{(1)+(2) }\spatie \Omega_{np}\sum m\left(f_nz_p-f_pz_n\right)&= \Omega_{np}\sum\left( F_nz_p -F_pz_n\right)
\end{align}
and hence, since $\Omega_{np}$ is arbitrary,
$$\mathbf{5.326.}\spatie \sum m\left(f_nz_p-f_pz_n\right)= \sum \left(F_nz_p-F_pz_n \right)$$
$$\blacklozenge$$
\newpage

\section{p161 - Clarification}
\begin{tcolorbox}
$$\mathbf{5.329.}\spatie h_{np}=\sum m\left(\omega_{nq} z_q z_p -\omega_{pq} z_qz_n\right)$$
$$= J_{npqr}\omega_{rq}$$
where
$$\mathbf{5.330.}\spatie J_{npqr}= \sum m\left(\delta_{nr}z_qz_p-\delta_{pr}z_nz_q \right)$$
\end{tcolorbox}
\begin{align}
h_{np}&=\sum m\left(\omega_{nq} z_q z_p -\omega_{pq} z_qz_n\right)\\
&=\sum m\left(\omega_{rq}\delta_{rn} z_q z_p -\omega_{rq} \delta_{rp}z_qz_n\right)\\
&=\omega_{rq}\sum m\left(\delta_{rn} z_q z_p -\delta_{rp}z_qz_n\right)\\
&=J_{npqr}\omega_{rq}
\end{align}
$$\blacklozenge$$
\newpage


\section{p166 - Exercise}
\begin{tcolorbox}
Deduce immediately from $\mathbf{5.420.}$ that the Coriolis force is perpendicular to the velocity.
\end{tcolorbox}
\begin{align}
G^{'}_s &= 2m\omega^{'}_{sm}(S^{'},S)v^{'}_m(S^{'})\\
\times v^{'}_s(S^{'})\quad : \spatie G^{'}_s v^{'}_s(S^{'})&=m\left(\omega^{'}_{sm}(S^{'},S)v^{'}_m(S^{'})v^{'}_s(S^{'})+\omega^{'}_{ms}(S^{'},S)v^{'}_m(S^{'})v^{'}_s(S^{'})\right)\\
&=0\quad\text{as }\omega^{'}_{ms}\text{ is skew-symmetric}
\end{align}
$$\blacklozenge$$
\newpage

\section{p166 - Exercise}
\begin{tcolorbox}
Show that if $N=3$ and $\dot{\omega}^{'}_r(S^{'},S) =0$, then the centrifugal force may be written 
$$\mathbf{5.422.}\spatie C^{'}_s = m\omega^{'}_{n}(S^{'},S)\omega^{'}_{n}(S^{'},S)z^{'}_s-m\omega^{'}_{n}(S^{'},S)z^{'}_n\omega^{'}_{s}(S^{'},S)$$
Deduce that $ C^{'}_s$ is coplanar with the vectors $\omega^{'}_{s}(S^{'},S)$ and $z^{'}_n$ and perpendicular to the former.
\end{tcolorbox}
By $\mathbf{5.420.}$ with $\dot{\omega}^{'}_r(S^{'},S) =0$ and using $\mathbf{5.316.} \quad ( \omega^{'}_{rs}=\epsilon_{rsn}\omega^{'}_n)$
\begin{align}
C^{'}_s &= m\omega^{'}_{sm}(S^{'},S)\omega^{'}_{nm}(S^{'},S)z^{'}_n\\
&= m\epsilon_{smk}\omega^{'}_k(S^{'},S)\epsilon_{nmp}\omega^{'}_p(S^{'},S)z^{'}_n\\
&= m\epsilon_{msk}\epsilon_{mnp}\omega^{'}_k(S^{'},S)\omega^{'}_p(S^{'},S)z^{'}_n\\
&= m\left(\delta_{sn}\delta_{kp} -\delta_{sp}\delta_{kn} \right)\omega^{'}_k(S^{'},S)\omega^{'}_p(S^{'},S)z^{'}_n\\
&= m\delta_{sn}\delta_{kp}\omega^{'}_k(S^{'},S)\omega^{'}_p(S^{'},S)z^{'}_n -m\delta_{sp}\delta_{kn} \omega^{'}_k(S^{'},S)\omega^{'}_p(S^{'},S)z^{'}_n\\
&= m\omega^{'}_p(S^{'},S)\omega^{'}_p(S^{'},S)z^{'}_s - m\omega^{'}_n(S^{'},S)\omega^{'}_s(S^{'},S)z^{'}_n
\end{align}
To deduce that $ C^{'}_s$ is coplanar with the vectors $\omega^{'}_{s}(S^{'},S)$ and $z^{'}_n$ we calculate the mixed triple product
\begin{align}
P &= \epsilon_{spr}C^{'}_s \omega^{'}_p(S^{'},S)z^{'}_r\\
&= m\underbrace{\epsilon_{spr} \omega^{'}_n(S^{'},S)\omega^{'}_n(S^{'},S)z^{'}_s \omega^{'}_p(S^{'},S)z^{'}_r}_{=0} -\underbrace{ m\epsilon_{spr}\omega^{'}_n(S^{'},S)\omega^{'}_s(S^{'},S)z^{'}_n \omega^{'}_p(S^{'},S)z^{'}_r}_{=0}\\
=0
\end{align}
Both terms vanish: the first by the presence of the terms $\epsilon_{spr}z^{'}_s z^{'}_r$ which cancel each other and for the second by the terms $\epsilon_{spr}\omega^{'}_s(S^{'},S) \omega^{'}_p(S^{'},S)$. As $P=0$, the three vectors are coplanar.\\
We now calculate the inner product $C^{'}_s\omega^{'}_s(S^{'},S)$
\begin{align}
P&= m\omega^{'}_n(S^{'},S)\omega^{'}_n(S^{'},S)z^{'}_s\omega^{'}_s(S^{'},S) - \underbrace{m\omega^{'}_n(S^{'},S)\omega^{'}_s(S^{'},S)z^{'}_n\omega^{'}_s(S^{'},S)}_{\Leftrightarrow \  m\omega^{'}_n(S^{'},S)\omega^{'}_n(S^{'},S)z^{'}_s\omega^{'}_s(S^{'},S)}\\
&=0
\end{align}
$$\blacklozenge$$
\newpage

\section{p168 - Exercise}
\begin{tcolorbox}
Taking $N=3$, show that $\mathbf{5.424}$ may be reduced to the usual Euler equations:
$$ I_{11}\dv{\omega^{'}_1(S^{'},S)}{t}-\left(I_{22}-I^{'}_{33}\right)\omega^{}_2(S^{'},S)\omega^{'}_3(S^{'},S)=M^{'}_1$$ and two similar equations.
\end{tcolorbox}
We first begin with an approach which leads to nothing. I probably made a reasoning error. I give here the whole calculation as this was interesting and alo to, later, find my mistake. After this buggy solution, I will give a second version, which works.
$\mathbf{5.424}$:
\begin{align}
M^{'}_{ab}&=J^{'}_{abrq}\dv{\omega^{'}_{rq}(S^{'},S)}{t}+J^{'}_{cdrq}\left(\delta_{ac}\delta_{du}\delta_{bv}+\delta_{bd}\delta_{cu}\delta_{av}\right)\omega^{'}_{rq}(S^{'},S)\omega^{'}_{uv}(S^{'},S) =\\
\times \epsilon_{sab} \text{: }\quad 2M^{'}_{s}&=\epsilon_{sab }J^{'}_{abrq}\dv{\omega^{'}_{rq}(S^{'},S)}{t}+\epsilon_{sab}J^{'}_{cdrq}\left(\delta_{ac}\delta_{du}\delta_{bv}+\delta_{bd}\delta_{cu}\delta_{av}\right)\omega^{'}_{rq}(S^{'},S)\omega^{'}_{uv}(S^{'},S) 
\end{align}
Using $ \omega^{'}_{rq}(S^{'},S)= \epsilon_{rqt}\omega^{'}_{t}(S^{'},S)$ and $I_{st}=\half J^{'}_{abrq}\epsilon_{abs}\epsilon_{rqt}$
\begin{align}
2M^{'}_{s}&=2I_{st}\dv{\omega^{'}_{t}(S^{'},S)}{t}+\epsilon_{sab}\epsilon_{rqi}\epsilon_{uvj}J^{'}_{cdrq}\left(\delta_{ac}\delta_{du}\delta_{bv}+\delta_{bd}\delta_{cu}\delta_{av}\right)\omega^{'}_{i}(S^{'},S)\omega^{'}_{j}(S^{'},S) \\
&=\left\{\begin{array}{l}
2I_{st}\dv{\omega^{'}_{t}(S^{'},S)}{t}\\\\ + \left(\epsilon_{sab}\epsilon_{rqi}\epsilon_{uvj}J^{'}_{cdrq}\delta_{ac}\delta_{du}\delta_{bv}+\epsilon_{sab}\epsilon_{rqi}\epsilon_{uvj}J^{'}_{cdrq}\delta_{bd}\delta_{cu}\delta_{av}\right)\omega^{'}_{i}(S^{'},S)\omega^{'}_{j}(S^{'},S) 
\end{array}\right.\\
&=\left\{\begin{array}{l}
2I_{st}\dv{\omega^{'}_{t}(S^{'},S)}{t}\\\\ + \left(\epsilon_{scb}\epsilon_{rqi}\epsilon_{dbj}J^{'}_{cdrq}+\epsilon_{sad}\epsilon_{rqi}\epsilon_{caj}J^{'}_{cdrq}\right)\omega^{'}_{i}(S^{'},S)\omega^{'}_{j}(S^{'},S) 
\end{array}\right.\\
&=\left\{\begin{array}{l}
2I_{st}\dv{\omega^{'}_{t}(S^{'},S)}{t}\\\\ + \left(\epsilon_{bcs}\epsilon_{bdj}\right)\epsilon_{rqi}J^{'}_{cdrq}\omega^{'}_{i}(S^{'},S)\omega^{'}_{j}(S^{'},S) \\\\+\left(\epsilon_{asd}\epsilon_{acj}\right)\epsilon_{rqi}J^{'}_{cdrq}\omega^{'}_{i}(S^{'},S)\omega^{'}_{j}(S^{'},S) 
\end{array}\right.\\
&=\left\{\begin{array}{l}
2I_{st}\dv{\omega^{'}_{t}(S^{'},S)}{t}\\\\ + \left(\delta_{cd}\delta_{sj}-\delta_{cj}\delta_{sd}\right)\epsilon_{rqi}J^{'}_{cdrq}\omega^{'}_{i}(S^{'},S)\omega^{'}_{j}(S^{'},S) \\\\+\left(\delta_{sc}\delta_{dj}-\delta_{sj}\delta_{dc}\right)\epsilon_{rqi}J^{'}_{cdrq}\omega^{'}_{i}(S^{'},S)\omega^{'}_{j}(S^{'},S) 
\end{array}\right.
\end{align}
\begin{align}
&=\left\{\begin{array}{l}
2I_{st}\dv{\omega^{'}_{t}(S^{'},S)}{t}\\\\ +\epsilon_{rqi}J^{'}_{ccrq}\omega^{'}_{i}(S^{'},S)\omega^{'}_{s}(S^{'},S) \\\\
-\epsilon_{rqi}J^{'}_{jsrq}\omega^{'}_{i}(S^{'},S)\omega^{'}_{j}(S^{'},S) \\\\+\epsilon_{rqi}J^{'}_{sjrq}\omega^{'}_{i}(S^{'},S)\omega^{'}_{j}(S^{'},S) 
\\\\-\epsilon_{rqi}J^{'}_{ccrq}\omega^{'}_{i}(S^{'},S)\omega^{'}_{s}(S^{'},S)
\end{array}\right.
\end{align}
giving
\begin{align}
2M^{'}_{s}&=\left\{\begin{array}{l}
2I_{st}\dv{\omega^{'}_{t}(S^{'},S)}{t} 
\\\\+\epsilon_{rqi}J^{'}_{sjrq}\omega^{'}_{i}(S^{'},S)\omega^{'}_{j}(S^{'},S)
\\\\
-\epsilon_{rqi}J^{'}_{jsrq}\omega^{'}_{i}(S^{'},S)\omega^{'}_{j}(S^{'},S)  
\end{array}\right.
\end{align}
For $s=1$:
\begin{table}[H]
\centering
 \begin{tabular}{||c | | c  | c||} 
 \hline
  & $+\epsilon_{rqi}J^{}_{1jrq}\omega^{}_{i}\omega^{}_{j}$ & $-\epsilon_{rqi}J^{}_{j1rq}\omega^{}_{i}\omega^{}_{j}   $\\ [1.5ex] 
 \hline\hline
 $\epsilon_{123}$ & $+J^{}_{1112}\omega^{}_{3}\omega^{}_{1}+J^{}_{1212}\omega^{}_{3}\omega^{}_{2}+J^{}_{1312}\omega^{}_{3}\omega^{}_{3}$ & $-J^{}_{1112}\omega^{}_{3}\omega^{}_{1}-J^{}_{2112}\omega^{}_{3}\omega^{}_{2}-J^{}_{3112}\omega^{}_{3}\omega^{}_{3}   $\\  
 $\epsilon_{132}$ & $-J^{}_{1113}\omega^{}_{2}\omega^{}_{1}-J^{}_{1213}\omega^{}_{2}\omega^{}_{2}-J^{}_{1313}\omega^{}_{2}\omega^{}_{3}$ & $+J^{}_{1113}\omega^{}_{2}\omega^{}_{1} +J^{}_{2113}\omega^{}_{2}\omega^{}_{2} +J^{}_{3113}\omega^{}_{2}\omega^{}_{3}   $\\
 $\epsilon_{213}$ & $-J^{}_{1121}\omega^{}_{3}\omega^{}_{1}-J^{}_{1221}\omega^{}_{3}\omega^{}_{2}-J^{}_{1321}\omega^{}_{3}\omega^{}_{3}$ & $+J^{}_{1121}\omega^{}_{3}\omega^{}_{1}+J^{}_{2121}\omega^{}_{3}\omega^{}_{2}+J^{}_{3121}\omega^{}_{3}\omega^{}_{3}   $\\
 $\epsilon_{231}$ & $+J^{}_{1123}\omega^{}_{1}\omega^{}_{1}+J^{}_{1223}\omega^{}_{1}\omega^{}_{2}+J^{}_{1323}\omega^{}_{1}\omega^{}_{3}$ & $-J^{}_{1123}\omega^{}_{1}\omega^{}_{1}-J^{}_{2123}\omega^{}_{1}\omega^{}_{2}-J^{}_{3123}\omega^{}_{1}\omega^{}_{3}   $\\
 $\epsilon_{321}$ & $-J^{}_{1132}\omega^{}_{1}\omega^{}_{1}-J^{}_{1232}\omega^{}_{1}\omega^{}_{2}-J^{}_{1332}\omega^{}_{1}\omega^{}_{3}$ & $+J^{}_{1132}\omega^{}_{1}\omega^{}_{1}  +J^{}_{2132}\omega^{}_{1}\omega^{}_{2}  +J^{}_{3132}\omega^{}_{1}\omega^{}_{3}   $\\
 $\epsilon_{312}$ & $+J^{}_{1131}\omega^{}_{2}\omega^{}_{1}+J^{}_{1231}\omega^{}_{2}\omega^{}_{2}+J^{}_{1331}\omega^{}_{2}\omega^{}_{3}$ & $-J^{}_{1131}\omega^{}_{2}\omega^{}_{1} -J^{}_{2131}\omega^{}_{2}\omega^{}_{2} -J^{}_{3131}\omega^{}_{2}\omega^{}_{3}   $\\   [1ex] 
 \hline
 \end{tabular}
\end{table}

Taking into account that $J^{}_{abcd}=0$ for $a\ne c  \wedge b\ne d$

\begin{table}[H]
\centering
 \begin{tabular}{||c | | c  | c||} 
 \hline
  & $+\epsilon_{rqi}J^{}_{1jrq}\omega^{}_{i}\omega^{}_{j}$ & $-\epsilon_{rqi}J^{}_{j1rq}\omega^{}_{i}\omega^{}_{j}   $\\ [1.5ex] 
 \hline\hline
 $\epsilon_{123}$ & $+\cancel{J^{}_{1112}\omega^{}_{3}\omega^{}_{1}}+J^{}_{1212}\omega^{}_{3}\omega^{}_{2}+J^{}_{1312}\omega^{}_{3}\omega^{}_{3}$ & $-\cancel{J^{}_{1112}\omega^{}_{3}\omega^{}_{1}} $\\  
 $\epsilon_{132}$ & $-\cancel{J^{}_{1113}\omega^{}_{2}\omega^{}_{1}}-J^{}_{1213}\omega^{}_{2}\omega^{}_{2}-J^{}_{1313}\omega^{}_{2}\omega^{}_{3}$ & $+\cancel{J^{}_{1113}\omega^{}_{2}\omega^{}_{1} } $\\
 $\epsilon_{213}$ & $-\cancel{J^{}_{1121}\omega^{}_{3}\omega^{}_{1}}$ & $+\cancel{J^{}_{1121}\omega^{}_{3}\omega^{}_{1}}+J^{}_{2121}\omega^{}_{3}\omega^{}_{2}+J^{}_{3121}\omega^{}_{3}\omega^{}_{3}   $\\
 $\epsilon_{231}$ & $+J^{}_{1323}\omega^{}_{1}\omega^{}_{3}$ & $-J^{}_{2123}\omega^{}_{1}\omega^{}_{2}$\\
 $\epsilon_{321}$ & $-J^{}_{1232}\omega^{}_{1}\omega^{}_{2}$ & $+J^{}_{3132}\omega^{}_{1}\omega^{}_{3}   $\\
 $\epsilon_{312}$ & $+\cancel{J^{}_{1131}\omega^{}_{2}\omega^{}_{1}}$ & $-\cancel{J^{}_{1131}\omega^{}_{2}\omega^{}_{1}} -J^{}_{2131}\omega^{}_{2}\omega^{}_{2} -J^{}_{3131}\omega^{}_{2}\omega^{}_{3}   $\\   [1ex] 
 \hline
 \end{tabular}
\end{table}
Opposite sign terms vanish, giving
\begin{table}[H]
\centering
 \begin{tabular}{||c | | c  | c||} 
 \hline
  & $+\epsilon_{rqi}J^{}_{1jrq}\omega^{}_{i}\omega^{}_{j}$ & $-\epsilon_{rqi}J^{}_{j1rq}\omega^{}_{i}\omega^{}_{j}   $\\ [1.5ex] 
 \hline\hline
 $\epsilon_{123}$ & $+J^{}_{1212}\omega^{}_{3}\omega^{}_{2}+J^{}_{1312}\omega^{}_{3}\omega^{}_{3}$ & $ $\\  
 $\epsilon_{132}$ & $-J^{}_{1213}\omega^{}_{2}\omega^{}_{2}-J^{}_{1313}\omega^{}_{2}\omega^{}_{3}$ & $ $\\
 $\epsilon_{213}$ & $ $ & $+J^{}_{2121}\omega^{}_{3}\omega^{}_{2}+J^{}_{3121}\omega^{}_{3}\omega^{}_{3}   $\\
 $\epsilon_{231}$ & $+J^{}_{1323}\omega^{}_{1}\omega^{}_{3}$ & $-J^{}_{2123}\omega^{}_{1}\omega^{}_{2}$\\
 $\epsilon_{321}$ & $-J^{}_{1232}\omega^{}_{1}\omega^{}_{2}$ & $+J^{}_{3132}\omega^{}_{1}\omega^{}_{3}   $\\
 $\epsilon_{312}$ & $ $ & $-J^{}_{2131}\omega^{}_{2}\omega^{}_{2} -J^{}_{3131}\omega^{}_{2}\omega^{}_{3}   $\\   [1ex] 
 \hline
 \end{tabular}
\end{table}
Considering $J_{abcd}= -J_{badc}$
\begin{table}[H]
\centering
 \begin{tabular}{||c | | c  | c||} 
 \hline
  & $+\epsilon_{rqi}J^{}_{1jrq}\omega^{}_{i}\omega^{}_{j}$ & $-\epsilon_{rqi}J^{}_{j1rq}\omega^{}_{i}\omega^{}_{j}   $\\ [1.5ex] 
 \hline\hline
 $\epsilon_{123}$ & $+\cancel{J^{}_{1212}}\omega^{}_{3}\omega^{}_{2}+\cancel{J^{}_{1312}}\omega^{}_{3}\omega^{}_{3}$ & $ $\\  
 $\epsilon_{132}$ & $-\cancel{J^{}_{1213}}\omega^{}_{2}\omega^{}_{2}-\cancel{J^{}_{1313}}\omega^{}_{2}\omega^{}_{3}$ & $ $\\
 $\epsilon_{213}$ & $ $ & $+\cancel{J^{}_{2121}}\omega^{}_{3}\omega^{}_{2}+\cancel{J^{}_{3121}}\omega^{}_{3}\omega^{}_{3}   $\\
 $\epsilon_{231}$ & $+\cancel{J^{}_{1323}}\omega^{}_{1}\omega^{}_{3}$ & $-\cancel{J^{}_{2123}}\omega^{}_{1}\omega^{}_{2}$\\
 $\epsilon_{321}$ & $-\cancel{J^{}_{1232}}\omega^{}_{1}\omega^{}_{2}$ & $+\cancel{J^{}_{3132}}\omega^{}_{1}\omega^{}_{3}   $\\
 $\epsilon_{312}$ & $ $ & $-\cancel{J^{}_{2131}}\omega^{}_{2}\omega^{}_{2} -\cancel{J^{}_{3131}}\omega^{}_{2}\omega^{}_{3}   $\\   [1ex] 
 \hline
 \end{tabular}
\end{table}
??
We get $$ m^{'}_s = I_{st}\dv{\omega^{'}_{t}(S^{'},S)}{t}$$?????\\
$$\lozenge$$\\
\newpage
Let's try another approach. Start with $\mathbf{5.332.}$: $\dv{}{t}\left(I_{st}\omega_{t}\right)= M_s$
\begin{align}
\dv{}{t}\left(I_{st}(S^{'},S)\omega_{t}(S^{'},S)\right)= M_s(S^{'},S)
\end{align}
Cf. $\mathbf{5.408.}$
\begin{align}
\omega^{'}_{u}(S^{'},S)&=A_{uq}\omega^{}_{q}(S^{'},S)\\
\times A_{ut}\quad \rightarrow \spatie A_{ut}\omega^{'}_{u}(S^{'},S)&=A_{ut}A_{uq}\omega^{}_{q}(S^{'},S)\\
&=\omega^{}_{t}(S^{'},S)\\
\spatie \omega^{}_{t}(S^{'},S) &=  A_{ut}\omega^{'}_{u}(S^{'},S)\\
\textbf{(10)}\quad\Rightarrow\spatie M_s(S^{'},S) &=\dv{}{t}\left(I_{st}(S^{'},S)A_{ut}\omega^{'}_{u}(S^{'},S)\right)\\
\times A_{ps}\quad \Rightarrow M^{'}_p(S^{'},S) &=A_{ps}\dv{}{t}\left(I_{st}(S^{'},S)A_{ut}\omega^{'}_{u}(S^{'},S)\right)\\
I_{st}(S^{'},S)&= A_{as}A_{bt}I^{'}_{ab}(S^{'},S)\\
\textbf{(16)}\quad\Rightarrow \spatie M^{'}_p(S^{'},S) &=A_{ps}\dv{}{t}\left(A_{as}A_{bt}I^{'}_{ab}(S^{'},S)A_{ut}\omega^{'}_{u}(S^{'},S)\right)\\
&=A_{ps}\dv{}{t}\left(A_{as}I^{'}_{ak}(S^{'},S)\omega^{'}_{k}(S^{'},S)\right)
\end{align}
As we transformed $I_{st}(S^{'},S)$ to a coordinate system fixed to the body we have that the elements of $I^{'}_{ab}(S^{'},S)$ are constants.\\
Hence,
\begin{align}
 M^{'}_p(S^{'},S) &=I^{'}_{ak}S^{'},S)A_{ps}\dv{}{t}\left( A_{as}\omega^{'}_{k}(S^{'},S)\right)\\
 &=I^{'}_{ak}(S^{'},S)A_{ps}\left(\dot{A}_{as}\omega^{'}_{k}(S^{'},S)+A_{as} \dot{\omega}^{'}_{k}(S^{'},S)\right)\\
 &=I^{'}_{ak}(S^{'},S)A_{ps}A_{as} \dot{\omega}^{'}_{k}(S^{'},S)+ I^{'}_{ak}(S^{'},S)A_{ps}\dot{A}_{as}\omega^{'}_{k}(S^{'},S)\\
 &=I^{'}_{pk}(S^{'},S) \dot{\omega}^{'}_{k}(S^{'},S)+ I^{'}_{ak}(S^{'},S)A_{ps}\dot{A}_{as}\omega^{'}_{k}(S^{'},S)\\
\textbf{5.408.}\quad\Rightarrow \spatie  A_{ps}\dot{A}_{as} &= \omega^{'}_{ap}(S^{'},S)\\
\textbf{(23)}\quad\Rightarrow \spatie  M^{'}_p(S^{'},S) &=I^{'}_{pk}(S^{'},S) \dot{\omega}^{'}_{k}(S^{'},S)+ I^{'}_{ak}(S^{'},S)\omega^{'}_{ap}(S^{'},S)\omega^{'}_{k}(S^{'},S)
\end{align}
Let's now calculate the last expression for $p=1$
\begin{align}
M^{'}_1(S^{'},S) &=I^{'}_{1k}(S^{'},S) \dot{\omega}^{'}_{k}(S^{'},S)+ I^{'}_{ak}(S^{'},S)\omega^{'}_{a1}(S^{'},S)\omega^{'}_{k}(S^{'},S)
\end{align}
As we want an arbitrary, fixed to the body of course, coordinate system, it is possible to chose one so that the $I^{'}_{kj}(S^{'},S) = 0 $ for $k\neq j$ i.e. $I^{'}_{kj}(S^{'},S) $ is diagonal. This is possible because $ I^{'}_{kj}(S^{'},S) $ is symmetric (the finite-dimensional spectral theorem says that any symmetric matrix whose entries are real can be diagonalized by an orthogonal matrix).\\
We get, noticing that $\omega^{'}_{ab}(S^{'},S)$ is skew-symmetric and hence $\omega^{'}_{11}(S^{'},S) = 0$ :
\begin{align}
M^{'}_1(S^{'},S) &= I^{'}_{11}(S^{'},S) \dot{\omega}^{'}_{1}(S^{'},S)+ I^{'}_{22}(S^{'},S)\omega^{'}_{21}(S^{'},S)\omega^{'}_{2}(S^{'},S)+ I^{'}_{33}(S^{'},S)\omega^{'}_{31}(S^{'},S)\omega^{'}_{3}(S^{'},S)
\end{align}
Using $\mathbf{5.317}$:  $\omega^{'}_{21}(S^{'},S)=-\omega^{'}_{3}(S^{'},S)$ and $\omega^{'}_{31}(S^{'},S)=\omega^{'}_{2}(S^{'},S)$  we get the asked expression 
\begin{align}
M^{'}_1(S^{'},S) &= I^{'}_{11}(S^{'},S) \dot{\omega}^{'}_{1}(S^{'},S)-\left( I^{'}_{22}(S^{'},S)- I^{'}_{33}(S^{'},S)\right)\omega^{'}_{2}(S^{'},S)\omega^{'}_{3}(S^{'},S)
\end{align}
$$\blacklozenge$$
\newpage


\section{p169 - Exercise}
\begin{tcolorbox}
Assign convenient generalized coordinates for the three systems $(a), \ (b),\text{ and } (c)$ mentioned at the beginning of this section, and calculate the kinematical metric form in each case
\end{tcolorbox}
\textbf{$(a)$ a particle on a surface $(N=2)$}\\
No need here for fancy general coordinates: the $V_2$ coordinate system in the plane is the metric form of choice.
Indeed $\left|v\right|^2 = a_{mn}v_mv_n$ and for a $V_2$
$$ds^2 =\left(a_{11}\left(v^1\right)^2+2a_{12}v^1v^2+a_{22}\left(v^2\right)^2 \right)dt^2$$ and if the space is Euclidean and the plane smooth, we can choose an orthogonal system where $a_{12}$ will vanish.\\\\  
\textbf{$(b)$ a rigid body which can turn about a fixed point, as in the preceding section $(N=3)$}\\
For a rigid body we can choose a coordinate system $S^{'}$ fixed to the body to describe the geometry of the rigid body. The kinetic energy  referenced to a 'non-moving' (abuse of language) coordinate system $S$ is 
\begin{align}
T = \half\mathbf{\sum}\rho v^{'}_n(S)v^{'}_n(S)\spatie\text{(summation over all masses in the rigid body)}
\end{align}
We know by $\mathbf{5.409}$: $v^{'}_n(S)= v^{'}_n(S^{'})+ \omega^{'}_{mn}(S^{'},S)z^{'}_m$. As the $v^{'}_n(S^{'})$ are fixed, we have $v^{'}_n(S^{'})=0$ giving 
\begin{align}
T = \half\mathbf{\sum}\rho z^{'}_mz^{'}_k\omega^{'}_{mn}(S^{'},S)\omega^{'}_{kn}(S^{'},S)
\end{align}
Note in (2) that we bring $\omega^{'}_{mn}(S^{'},S)$ out of the summation as this expression  is the same for all masses in the body.
\begin{align}
\omega_{mn}(S^{'},S)&= \epsilon_{mnt}\omega^{'}_{t}(S^{'},S)\\
\Rightarrow\quad  
T &= \half\mathbf{\sum}\rho \epsilon_{mnt}\epsilon_{kns}z^{'}_mz^{'}_k\omega^{'}_{t}(S^{'},S)\omega^{'}_{s}(S^{'},S)\\
&= \half\mathbf{\sum}\rho \left(\delta_{mk}\delta_{ts}-\delta_{ms}\delta_{kt}\right)z^{'}_mz^{'}_k\omega^{'}_{t}(S^{'},S)\omega^{'}_{s}(S^{'},S)\\
&= \half\mathbf{\sum}\rho \left(z^{'}_mz^{'}_m\omega^{'}_{t}(S^{'},S)\omega^{'}_{t}(S^{'},S)-z^{'}_sz^{'}_t\omega^{'}_{t}(S^{'},S)\omega^{'}_{s}(S^{'},S)\right)\\
&= \half\mathbf{\sum}\rho \left(\delta_{st}z^{'}_mz^{'}_m\omega^{'}_{s}(S^{'},S)\omega^{'}_{t}(S^{'},S)-z^{'}_sz^{'}_t\omega^{'}_{t}(S^{'},S)\omega^{'}_{s}(S^{'},S)\right)\\
&= \half\mathbf{\sum}\rho \left(\delta_{st}z^{'}_mz^{'}_m-z^{'}_sz^{'}_t\right)\omega^{'}_{s}(S^{'},S)\omega^{'}_{t}(S^{'},S)
\end{align}
By $\mathbf{5.335.}$ we have $ I_{st} = \delta_{st}\mathbf{\sum}\rho z_mz_m-\mathbf{\sum}\rho z_sz_t$ and so (8) can be written as
\begin{align}
T&=\half I_{st}\omega^{'}_{s}(S^{'},S)\omega^{'}_{t}(S^{'},S)
\end{align} 
So we can choose the three angles $\Omega^{'}_{s}(S^{'},S)$  with ($\omega^{'}_{s}(S^{'},S)= \dv{\Omega^{'}_{s}(S^{'},S)}{t}$) as generalized coordinates and define $$ds^2=I_{st}d\Omega^{'}_{s}(S^{'},S)d\Omega^{'}_{t}(S^{'},S)$$ with  $$a_{mn} = I_{mn}$$ having constants as elements.
Some check on consistency of the metric tensor defined by $(14)$:\\
\textbf{Positive definite ?} : Yes, as $T$ is positive by construction.\\
\textbf{Symmetric ?} : Yes, as $a_{mn} = I_{km}$ and $I_{km}$ is symmetric.  \\
\begin{comment}
For a rigid body we can choose a coordinate system fixed to the body to describe the geometry of the rigid body. The kinetic energy is referenced to a 'fixed' coordinate system 
\begin{align}
T = \half\mathbf{\sum}\rho v_nv_n\spatie\text{summation over all masses in the rigid body}
\end{align}
as $z_n$ are fixed in the choose coordinate system it is clear that 
we have to find another way the express the velocity of the masses.

We know $v_n= - \omega_{nm}z_m$. As the $z_m$ are fixed, the only way the kinetic energy can change (and have 'a path' in the general coordinate system) is when $\omega_{nm}$ changes. 
We can write
\begin{align}
dT &=\half\mathbf{\sum}\rho  d\left(v_nv_n\right)\\
&=\half \mathbf{\sum}\rho  d\left(\omega_{nm}\omega_{nk}z_mz_k\right)\\
&=\half \mathbf{\sum}\rho  \left(\omega_{nk}d\omega_{nm}+\omega_{nm}d\omega_{nk}\right)z_mz_k\\
&= \left( \mathbf{\sum}\rho  \omega_{nm}z_mz_k\right)d\omega_{nk}
\end{align}
Note in (5) that we bring $d\omega_{nk}$ out of the summation as $\omega_{nk}$ is the same for all masses in the body.
As $dT$ can be negative we calculate $dT^2$
\begin{align}
dT^2&=\left( \mathbf{\sum}\rho  \omega_{nm}z_mz_k\right)\left( \mathbf{\sum}\rho  \omega_{pq}z_qz_r\right)d\omega_{nk}d\omega_{pr}\end{align}
\begin{align}
\omega_{nm}&= \epsilon_{nma}\omega_a\\
\Rightarrow\quad  
dT^2&=\left( \mathbf{\sum}\rho  \epsilon_{nma}\omega_az_mz_k\right)\left( \mathbf{\sum}\rho  \epsilon_{pqb}\omega_bz_qz_r\right)d\left(\epsilon_{nku}\omega_{u}\right)d\left(\epsilon_{prv}\omega_{v}\right)\\
&=\left( \mathbf{\sum}\rho  \epsilon_{nma}\epsilon_{nku}\omega_az_mz_k\right)\left( \mathbf{\sum}\rho  \epsilon_{pqb}\epsilon_{prv}\omega_bz_qz_r\right)d\omega_{u}\\
&=\left\{\begin{array}{l}\left( \mathbf{\sum}\rho  \left(\delta_{mk}\delta_{au}-\delta_{mu}\delta_{ak}\right)\omega_az_mz_k\right)d\omega_{u}\\ \left( \mathbf{\sum}\rho  \left(\delta_{qr}\delta_{bv}-\delta_{qv}\delta_{rb}\right)\omega_bz_qz_r\right)d\omega_{v}d\omega_{v}\\
\end{array}\right.\\
&=\left( \mathbf{\sum}\rho  \left(\omega_uz_nz_n-\omega_kz_uz_k\right)\right)\left( \mathbf{\sum}\rho  \left(\omega_vz_nz_n-\omega_rz_vz_r\right)\right)d\omega_{u} d\omega_{v}\\
&=\left( \mathbf{\sum}\rho  \left(\delta_{ku}\omega_kz_nz_n-\omega_kz_uz_k\right)\right)\left( \mathbf{\sum}\rho  \left(\delta_{rv}\omega_rz_nz_n-\omega_rz_vz_r\right)\right)d\omega_{u} d\omega_{v}\\
&=\left( \mathbf{\sum}\rho  \left(\delta_{ku}z_nz_n-z_uz_k\right)\right)\left( \mathbf{\sum}\rho  \left(\delta_{rv}z_nz_n-z_vz_r\right)\right)\omega_k\omega_rd\omega_{u} d\omega_{v}
\end{align}
By $\mathbf{5.335.}$ we have $ I_{ku} = \delta_{ku}\mathbf{\sum}\rho z_nz_n-\mathbf{\sum}\rho z_uz_k$ and so (13) can be written as
\begin{align}
dT^2&=I_{ku}I_{rv}\omega_k\omega_rd\omega_{u} d\omega_{v}
\end{align} 
So we can choose the three $\omega_k$ as generalized coordinates and define $$ds^2=a_{mn}d\omega_{m} d\omega_{n}$$ with  $$a_{mn} = I_{km}I_{rn}\omega_k \omega_r$$
The metric tensor $a_{mn}$ contains elements depending on the $\omega_k$ chosen as general coordinates of the system and is a good candidate as metric tensor.
Some check on consistency of the metric tensor defined by $(14)$:\\
\textbf{Positive definite ?} : Yes, as $dT^2$ is positive.\\
\textbf{Symmetric ?} : Yes, as $a_{mn} = I_{km}I_{rn}\omega_k \omega_r = I_{rn}I_{km}\omega_k \omega_r= a_{nm}$\\
\end{comment}
\\\\ \textbf{$(c)$ a chain of six rods smoothly hinged together, with one end fixed and all moving on a smooth plane $(N=6)$}\\

To simplify the notation we will assume that the mass $m_k$ of each rod (with length $L_k$) is concentrated at it's endpoint .\\
First we note that the velocity of a rod is composed of two vectors, one (labelled as $\overline{\nu}_k$) generated by its own rotation relative to the previous rod and the other (labelled as $\overline{v}_{k-1}$) generated by the velocity of the endpoint of the rod to which it is attached (see.fig. 5.2).
\begin{figure}[H]
    \centering
    \subfloat[]{\begin{tikzpicture}

\coordinate (p1) at (-0.5,2.5) {};
\draw[fill=white]  (p1) circle (0.1);
\coordinate (p2) at (-2.5,-2) {} {};
\draw[fill=white]  (p2) circle (0.1);
\coordinate (p12) at (-1.5,0) {} {} {};

\draw []  (p1) -- (p2);
\draw [dashed](-4.3302,-0.6) arc (-140.0002:-80:5);
\coordinate (p3) at (-0.5,-3.5) {} {};
\draw [dashdotted]  (p1) -- (p3);
\coordinate (v1) at (1.5,0.5) {} {} {};
\coordinate (v2) at (0.5,-3.5) {} {} {};
\coordinate (v3) at(-0.5,-4) {} {} {};
\coordinate (v4) at(-0.5+3,-4-1.5) {} {} {};
\draw [-{Latex[length=2mm]},very thick]  (p1) -- (v1);
\draw [-{Latex[length=2mm]},very thick]  (p2) -- (v2);
\draw [-{Latex[length=2mm]},very thick, dashed]  (p2) -- (v3);
\draw [-{Latex[length=2mm]},, dashed]  (p2) -- (v4);
\draw [ dotted]  (v2) -- (v4);
\draw [ dotted]  (v3) -- (v4);
\draw [dashed, decoration={markings, mark=at position 0.52 with {\arrow[scale = 1.]{Latex[length=2mm]}}},    postaction={decorate}](p12) arc [start angle=65,
        end angle=90,
        x radius=-2cm,
        y radius =-2cm];
\node[label=north east:$m_{k-1}$] at (p1) {};
\node[label=south west:$m_{k}$] at (p2) {};
\node[label=north east :$\theta_k$] at (p12) {};
\node[label=north west :$L_k$] at (p12) {};
\node[label=north east :$\overline{v}_{k-1}$] at (v1) {};
\node[label=south :$\overline{v}_{k-1}$] at (v3) {};
\node[label=south :$\overline{v}_{k} {=}\overline{v}_{k-1}+\overline{\nu}_k$] at (v4) {};
\node[label=north east :$\overline{\nu}_k {=} L_k\dot{\theta}_k$] at (v2) {};
\end{tikzpicture}}
\caption{Composition of absolute and relative velocities of a chain of rods}
\label{fig:fig_p169_a}
\end{figure}
If we take Cartesian coordinates it is easy to see that rod (1) will have components $$\left( L_1\dot{\theta}_1\cos\theta,L_1\dot{\theta}_1\sin\theta_1\right)$$
rod (2) $$\left( L_1\dot{\theta}_1\cos\theta_1+ L_2\dot{\theta}_2\cos\theta_2 ,L_1\dot{\theta}_1\sin\theta_1+ L_2\dot{\theta}_2\sin\theta_2 \right)$$
$$\vdots$$
rod (k) $$\left( \sum_{i=1}^{k}L_i\dot{\theta}_i\cos\theta_i , \sum_{i=1}^{k}L_i\dot{\theta}_i\sin\theta_i  \right)$$
and so 
\begin{align}
\left(v^{(k)}\right)^2&= \left( \sum_{i=1}^{k}L_i\dot{\theta}_i\cos\theta_i\right)^2+\left( \sum_{i=1}^{k}L_i\dot{\theta}_i\sin\theta_i  \right)^2\\
&= \sum_{i=1}^{k}\left( L_i\dot{\theta}_i\right)^2+2\sum_{i=1}^{k}\sum_{j=1}^{k-i}\left( L_iL_{i+j}\dot{\theta}_i\dot{\theta}_{i+j}\cos\left(\theta_i -\theta_{i+j}\right) \right)
\end{align}
So the kinetic energy of one rod and the total kinetic energy of the system are
\begin{align}
T^{(k)}&= \half m_k\left[\sum_{i=1}^{k}\left( L_i\dot{\theta}_i\right)^2+2\sum_{i=1}^{k}\sum_{j=1}^{k-i}\left( L_iL_{i+j}\dot{\theta}_i\dot{\theta}_{i+j}\cos\left(\theta_i -\theta_{i+j}\right) \right)\right]\\
T &= \sum_{k=1}^{N}T^{(k)}
\end{align}
For $N=6$ we get\\ \\
\begin{tabular}{||c|l||}
    \cline{1-2}
    rod&$T^{(k)}$ \\
    \hline \hline
    1 & $\half m_1 \left[\left(L_1\dot{\theta}_1\right)^2\right]$\\
    \hline \hline
    2 &$ \half m_2\left[\left(L_1\dot{\theta}_1\right)^2+\left(L_2\dot{\theta}_2\right)^2+2 L_1L_{2}\dot{\theta}_1\dot{\theta}_{2}\cos\left(\theta_1 -\theta_{2}\right) \right]$\\
    \hline \hline
    3 &$ \half m_3\left[\left(L_1\dot{\theta}_1\right)^2+\left(L_2\dot{\theta}_2\right)^2+\left(L_3\dot{\theta}_3\right)^2 +2 L_1L_{2}\dot{\theta}_1\dot{\theta}_{2}\cos\left(\theta_1 -\theta_{2}\right) +2 L_1L_{3}\dot{\theta}_1\dot{\theta}_{3}\cos\left(\theta_1 -\theta_{3}\right) +\dots \right]$\\
    \hline \hline
    4 &$ \half m_4\left[\left(L_1\dot{\theta}_1\right)^2+\left(L_2\dot{\theta}_2\right)^2+\left(L_3\dot{\theta}_3\right)^2+\left(L_4\dot{\theta}_4\right)^2+2 L_1L_{2}\dot{\theta}_1\dot{\theta}_{2}\cos\left(\theta_1 -\theta_{2}\right) +2 L_1L_{3}\dot{\theta}_1\dot{\theta}_{3}\cos\left(\theta_1 -\theta_{3}\right) +\dots \right]$\\
    \hline \hline
    5 &$ \half m_5\left[\left(L_1\dot{\theta}_1\right)^2+\left(L_2\dot{\theta}_2\right)^2+\left(L_3\dot{\theta}_3\right)^2+\left(L_4\dot{\theta}_4\right)^2+\left(L_5\dot{\theta}_5\right)^2+2 L_1L_{2}\dot{\theta}_1\dot{\theta}_{2}\cos\left(\theta_1 -\theta_{2}\right) +\dots \right]$\\
    \hline \hline
    6 &$ \half m_6\left[\left(L_1\dot{\theta}_1\right)^2+\left(L_2\dot{\theta}_2\right)^2+\left(L_3\dot{\theta}_3\right)^2+\left(L_4\dot{\theta}_4\right)^2+\left(L_5\dot{\theta}_5\right)^2+\left(L_6\dot{\theta}_6\right)^2+2 L_1L_{2}\dot{\theta}_1\dot{\theta}_{2}\cos\left(\theta_1 -\theta_{2}\right)  +\dots \right]$\\
    \hline \hline
\end{tabular}
Giving for $T$
\begin{align}
2T=\left\{\begin{array}{l}
\left(m_1+m_2+m_3+m_4+m_5+m_6\right)\left(L_1\dot{\theta}_1\right)^2\\
+\left(m_2+m_3+m_4+m_5+m_6\right)\left(L_2\dot{\theta}_2\right)^2\\
+\left(m_3+m_4+m_5+m_6\right)\left(L_3\dot{\theta}_3\right)^2\\
+\left(m_4+m_5+m_6\right)\left(L_4\dot{\theta}_4\right)^2\\
+\left(m_5+m_6\right)\left(L_5\dot{\theta}_5\right)^2\\
+\left(m_6\right)\left(L_6\dot{\theta}_6\right)^2\\
+2\left(m_2+m_3+m_4+m_5+m_6\right)L_1L_{2}\dot{\theta}_1\dot{\theta}_{2}\cos\left(\theta_1 -\theta_{2}\right)\\
+2\left(m_3+m_4+m_5+m_6\right)L_1L_{3}\dot{\theta}_1\dot{\theta}_{3}\cos\left(\theta_1 -\theta_{3}\right)\\
+2\left(m_3+m_4+m_5+m_6\right)L_2L_{3}\dot{\theta}_2\dot{\theta}_{3}\cos\left(\theta_2 -\theta_{3}\right)\\
+2\left(m_4+m_5+m_6\right)L_1L_{4}\dot{\theta}_1\dot{\theta}_{4}\cos\left(\theta_1 -\theta_{4}\right)\\
+2\left(m_4+m_5+m_6\right)L_2L_{4}\dot{\theta}_2\dot{\theta}_{4}\cos\left(\theta_2 -\theta_{4}\right)\\
+2\left(m_4+m_5+m_6\right)L_3L_{4}\dot{\theta}_3\dot{\theta}_{4}\cos\left(\theta_3 -\theta_{4}\right)\\
+2\left(m_5+m_6\right)L_1L_{5}\dot{\theta}_1\dot{\theta}_{5}\cos\left(\theta_1 -\theta_{5}\right)\\
+2\left(m_5+m_6\right)L_2L_{5}\dot{\theta}_2\dot{\theta}_{5}\cos\left(\theta_2 -\theta_{5}\right)\\
+2\left(m_5+m_6\right)L_3L_{5}\dot{\theta}_3\dot{\theta}_{5}\cos\left(\theta_3 -\theta_{5}\right)\\
+2\left(m_5+m_6\right)L_4L_{5}\dot{\theta}_4\dot{\theta}_{5}\cos\left(\theta_4 -\theta_{5}\right)\\
+2\left(m_6\right)L_1L_{6}\dot{\theta}_1\dot{\theta}_{6}\cos\left(\theta_1 -\theta_{6}\right)\\
+2\left(m_6\right)L_2L_{6}\dot{\theta}_2\dot{\theta}_{6}\cos\left(\theta_2 -\theta_{6}\right)\\
+2\left(m_6\right)L_3L_{6}\dot{\theta}_3\dot{\theta}_{6}\cos\left(\theta_3 -\theta_{6}\right)\\
+2\left(m_6\right)L_4L_{6}\dot{\theta}_4\dot{\theta}_{6}\cos\left(\theta_4 -\theta_{6}\right)\\
+2\left(m_6\right)L_5L_{6}\dot{\theta}_5\dot{\theta}_{6}\cos\left(\theta_5 -\theta_{6}\right)
\end{array}\right.
\end{align}
We define as general coordinates the angles $\theta^i$
and express $ds^2$ as 
$$ds^2= 2Tdt^2$$ 
and see that $ds^2$ is of the required form 
$$ds^2= a_{mn}d\theta^md\theta^n$$
The metric tensor $a_{mn}$ contains elements depending on the $\theta_k$ chosen as general coordinates of the system and is a good candidate as metric tensor.
Some check on consistency of the metric tensor defined by $(8)$:\\
\textbf{Positive definite ?} : Yes, as $T$ is positive by definition\\
\textbf{Symmetric ?} : Yes, as the non-diagonal term $a_{ij}$ contains $\cos\left(\theta_i-\theta_j\right) = \cos\left(\theta_j-\theta_i\right)$ \\
\textbf{Number of elements} : the metric tensor $a_{mn}$ for $N=6$ should contain $6$ diagonal elements and $\frac{6\times 6 -6}{2} = 15$ independent non-diagonal elements. Checking $(8)$, one can find that the numbers yield.
$$\blacklozenge$$
\newpage



\section{p174 - Exercise}
\begin{tcolorbox}
Establish the general result
$$v\dv{v}{s}=X_r\lambda^r, \quad \kappa v^2=X_r\nu^r$$
Deduce that, if no forces at on the system, the trajectory is a geodesic in configuration space and the magnitude of the velocity is constant.
\end{tcolorbox}
In configuration space $f_r=X_r$. Hence by $\mathbf{5,515}$

\begin{align}
X^r&=v\dv{v}{s}\lambda^r+\kappa v^2\nu^r\\
\Rightarrow \spatie X^r \lambda_r &=X_r\lambda^r = v\dv{v}{s} \quad \text{as } \lambda^r \perp \nu^r\\
\text{and} \spatie X^r \nu_r &=X_r\nu^r = \kappa v^2\quad \text{as } \lambda^r \perp \nu^r\\
\end{align}
The trajectory is a geodesic if $\kappa = 0$ which is the case as $X_r=0$ and $$v\dv{v}{s} =0 \Rightarrow \dv{v}{s} =0 \Rightarrow v= C^{t}$$
$$\blacklozenge$$
\newpage


\section{p174 - Clarification}
\begin{tcolorbox}
It is easy to see thet the lines of force are the orthogonal trajectories of the equipotential surface $V=C^{t}$
\end{tcolorbox}
Consider a curve given by $x^r=x^r\left(u\right)$. \\Along that line we have $V= V\left(x^r\left(u\right)\right)$. Take $u=s$ as parameter and let's impose that $V\left(s\right)= C^{t}$.\\ We have
$\dv{V}{s} = \frac{\partial{V}}{\partial{x^r}}\dv{x^r}{s} =\frac{\partial{V}}{\partial{x^r}}\lambda^r =0$ with $\lambda^r=\dv{x^r}{s}$ the tangent vector along that curve.\\ But $X_r = \frac{\partial{V}}{\partial{x^r}}$. \\
So, $X_r\lambda^r=0$ and as $X_r$ is collinear with $dx^r$ (the infinitesimal line element of the line of force) we have $dx_n \lambda^n = a_{mn}dx^m \lambda^n=0$ proving the perpendicularity of both curves.
$$\blacklozenge$$
\newpage

\section{p176 - Exercise}
\begin{tcolorbox}
For a spherical pendulum, show that the lines of force are geodesics on the sphere on which the particle is constrained to move. What does the theorem stated above tell us in this case?
\end{tcolorbox}
For the spherical pendulum we have the following situation
\begin{figure}[H]

\begin{tikzpicture}[scale=0.5]
\coordinate (O1) at (0,0);
\draw  (O1) circle (6);
%\draw [thick] (O1) ellipse (6 and 0.6);
\coordinate (Om) at (0.0,-6) ;
%\node at (Om){$P$};
\draw [ decoration={markings, mark=at position 0.32 with {\arrow[scale = 1.]{Latex[length=3mm]}}},    postaction={}](Om) arc (90:-90:3 and -6.0);
\draw [ dashed,decoration={markings, mark=at position 0.32 with {\arrow[scale = 1.]{Latex[length=3mm]}}},    postaction={}](Om) arc (90:-90:-3 and -6.0);
\coordinate (Ohl) at (-5.2,3) {} {} {};
\coordinate (Ohr) at (5.2,3) {} {} {};
%\node at (Oh){$P$};
%\draw[fill=white]  (Ohl) circle (0.1);
%\draw[fill=white]  (Ohr) circle (0.1);
\draw [ decoration={markings, mark=at position 0.32 with {\arrow[scale = 1.]{Latex[length=3mm]}}},    postaction={}](Ohl) arc (-10:190:-5.3 and -0.5);
\draw [dashed ,decoration={markings, mark=at position 0.32 with {\arrow[scale = 1.]{Latex[length=3mm]}}},    postaction={}](Ohl) arc (10:190:-5.3 and 0.5);
\coordinate (NPole) at (0,6) {};
\draw[fill=white]  (NPole) circle (0.1);
\coordinate (SPole) at (0,-6) {};
\draw[dashdotted] (NPole) -- (SPole);
\draw[fill=white]  (SPole) circle (0.1);
\coordinate (P) at (2.7,2.5) {};
\draw[fill=white]  (P) circle (0.1);
\draw[fill=white]  (O1) circle (0.1);
\coordinate (etheta) at (3.5,-1) {};
\coordinate (ephi)  at (7,2.5) {};
\draw [-{Latex[length=2mm]}] (P) -- (etheta);
\draw [-{Latex[length=2mm]}] (P) -- (ephi);
\draw[dashed] (O1) -- (P);
\node[label=east:$\mathbf{\overline{1}_{\theta}}$] at (etheta) {};
\node[label=east:$\mathbf{\overline{1}_\phi}$] at (ephi) {};
\node[label=west:$O$] at (O1) {};
\coordinate (O2) at (-0.10,2);
\node[label=south east:$\theta$] at (O2) {};
\draw [dashed, decoration={markings, mark=at position 0.52 with {\arrow[scale = 1.]{Latex[length=2mm]}}},    postaction={decorate}](O2) arc (90:55:3 and 3.0);
\coordinate (F)  at (2.7,0) {};
\draw [-{Latex[length=2mm]}] (P) -- (F);
\node[label=west:$\mathbf{\overline{F}}$] at (F) {};
\coordinate (D0) at (11,-1) {};
\coordinate (DP) at (15.5,4) {};
\node[label=east:$\mathbf{m}$] at (DP) {};
\coordinate (De) at (20.5,0) {};
\coordinate (DF) at (15.5,-3.5) {};
\coordinate (DA) at (11,8.5) {};
\draw [dashed] (D0) -- (DP);
\draw [dashed] (D0) -- (DA);
\draw [-{Latex[length=2mm]}] (DP) -- (De);
\node[label=east:$\mathbf{\overline{1}_{\theta}}$] at (De) {};
\draw [-{Latex[length=2mm]}] (DP) -- (DF);
\node[label=east:$\mathbf{\overline{F}}$] at (DF) {};
\coordinate (thet) at  (11,3) {};
\draw [dashed, decoration={markings, mark=at position 0.52 with {\arrow[scale = 1.]{Latex[length=2mm]}}},    postaction={decorate}](thet) arc (90:35:3 and 3.0);
\node[label=south east:$\theta$] at (thet) {};
\coordinate (thet2) at (15.5,0.5) {};
\draw [dashed, decoration={markings, mark=at position 0.52 with {\arrow[scale = 1.]{Latex[length=2mm]}}},    postaction={decorate}](thet2) arc (90:30:3 and -3.0);
\node[label=south east:$\frac{\pi}{2}-\theta$] at (thet2) {};
\draw[fill=white]  (DP) circle (0.1);
\end{tikzpicture}

\caption{Physical components of the gravitational force tensor acting on a mass $\mathbf{m}$ on a sphere }
\label{fig:fig_p176_Ex1}
\end{figure}
From the figure it is clear that the only component of the gravitational force acting on the mass is restricted along the $\overline{1}_{\theta}$ vector which, with varying $\theta$ lays along a great circle of the sphere which is a geodesic. Hnce the lines of force are great circle on the sphere. \\
For the theorem stated this means that as a mass is launched along a great circle, it will stay on this great circle.
$$\blacklozenge$$
\newpage



\section{p176 - Exercise (PARTLY SOLVED}
\begin{tcolorbox}
A system starts from rest at a configuration $O$. Prove that the trajectory at $O$ is tangent to the line of force through $O$, and that the first curvature of the trajectory is one-third of the first curvature of the line of force.
\end{tcolorbox}
From $\mathbf{5.533}$ we have 
\begin{align}
v\dv{v}{s}= X_r\lambda^r, \quad \kappa v^2=X_r\nu^r
\end{align}
From the second expression we have as $v=0$ at $O$ that $X_r\nu^r=0$, meaning that $X_r$ is perpendicular to $\nu^r$.
Also by $\mathbf{5.516}$
\begin{align}
f^r = \dv{v}{t}\lambda^r + \kappa v^2 \nu^r
\end{align}
we know that the acceleration lies in the elementary two-space containing the tangent and the first normal to the trajectory implying by the previous result that $X_r$ and $\lambda^r$ are collinear.
Note that from $(1)$ we can not conclude (because $v=0$) from the first expression  that $X_r\lambda^r=0$ . Indeed, $v\dv{v}{s}$ is a derived expression form of $\dv{v}{t}$. As $\dv{v}{t}$ is not necessarily $0$ (otherwise the system would for ever  stay on the configuration at $O$ meaning that $ds=0$, making the expression $v\dv{v}{s}$ meaningless.)
\\\\
SECOND PART?


$$\blacklozenge$$
\newpage


\section{p181 and p182 - Clarification Figures 13., 14. and 15.}
\begin{tcolorbox}
There are several ways to perform  a map of the configuration space of a rigid body with fixed point.
\end{tcolorbox}
\begin{figure}[H]
    \centering
    \subfloat[]{\begin{tikzpicture}[scale=0.25]
\coordinate (O) at (0,0);
\node[label=south :$O$] at (O) {};
\coordinate (X) at (-9.5,-8) {} {};
\coordinate (Y) at (17,0) {} {};
\coordinate (Z) at (0,16.5) {} {};

\coordinate (X0) at (-5.35,-4.5) {} {} ;
\coordinate (Y0) at (9,0) {} ;
\coordinate (Z0) at (0,9) {} {} {} ;
\coordinate (XY) at (5,-4.5) {} {} {} ;
\coordinate (YZ) at (9,9) {} {} ;
\coordinate (XZ) at (-5.35,4.5) {} {};
\coordinate (XYZ) at (5,4.5) {} {} {};


\coordinate (HXY1) at (-0.64,-4.5) {} {} {};
\coordinate (HXY2) at (-0.64,4.5) {} {} {};
\coordinate (HZY1) at (4.5,9) {} {} {};
\coordinate (HZY2) at (4.5,0) {} {} {};


\draw [-{Latex[length=2mm]}] (O) -- (X);
\draw [-{Latex[length=2mm]}] (O) -- (Y);
\draw [-{Latex[length=2mm]}] (O) -- (Z);
\node[label=south east:$\theta$] at (X) {};
\node[label=south west:$\phi$] at (Y) {};
\node[label=south east:$\psi$] at (Z) {};
\node[label=north west:$\pi$] at (X0) {};
\node[label=south :$2\pi$] at (Y0) {};
\node[label=north east:$2\pi$] at (Z0) {};
%\node (Sb) [rectangle, minimum width=3cm, minimum height=1cm,draw=black, pattern color=black, pattern = north east lines]{} ;

\draw [] (X0) -- (XY);
\draw [] (Y0) -- (YZ);
\draw [] (Z0) -- (XZ);
\draw [] (X0) -- (XZ);
\draw [] (Y0) -- (XY);
\draw [] (XY) -- (XYZ);
\draw [] (XZ) -- (XYZ);
\draw [] (YZ) -- (XYZ);
\draw [] (YZ) -- (Z0);
\draw [] (O) -- (Z0);
\draw [] (O) -- (X0);
\draw [] (O) -- (Y0);

\draw[fill=white]  (X0) circle (0.1);
\draw[fill=white]  (Y0) circle (0.1);
\draw[fill=white]  (Z0) circle (0.1);

\draw[fill=white]  (XZ) circle (0.1);
\draw[fill=white]  (YZ) circle (0.1);
\draw[fill=white]  (XZ) circle (0.1);
\draw[fill=white]  (O) circle (0.1);
\draw[fill=white]  (XYZ) circle (0.1);

%\draw[pattern color=black, pattern = dots]  (O) rectangle (YZ) node (v0) {};
%\draw[pattern color=black, pattern = dots]  (X0) rectangle (XYZ) node (v1) {};
\coordinate (plane0) at (7.1,2.2) {};
\coordinate (plane1) at (-2.7,2.2) {};
\draw[fill=white]  (plane1) circle (0.1);
\draw[fill=white]  (plane0) circle (0.1);
%\draw[ decoration={markings, mark=at position 0.75 with {\arrow[scale = 1.]{Latex[length=3mm,reversed]}}},    postaction={decorate}](plane0) .. controls (29.5,-2) and (-28,-4) .. (plane1);
\draw[pattern color=black, pattern = dots]  (HXY1) node (v2) {} -- (HXY2) -- (HZY1) -- (HZY2) -- (HZY2) -- cycle;
\draw  (YZ) edge (XY);
\draw  (XYZ) edge (Y0);
\draw  (XZ) edge (O);
\draw  (Z0) edge (X0);
\node[label=north:$\mathbf{\Phi_{2\pi}}$] at (plane0) {};
\node[label=south:$\mathbf{\Phi_{0}}$] at (plane1) {};
\draw [ dashed,decoration={markings, mark=at position 0.55 with {\arrow[scale = 1.]{Latex[length=2mm]}}},    postaction={decorate}](plane1) arc(-70:261:-20cm and -5cm);
\end{tikzpicture}}
	\
    \subfloat[]{
\begin{tikzpicture}[scale=0.5]
\coordinate (O1) at (0,0);
\draw [thick] (O1) ellipse (6 and 0.6);
\coordinate (O2) at (0,-3);
\draw[dashed]  (O2) ellipse (6 and 0.6);
\coordinate (Om) at (0,-1.5);
\coordinate (O1s) at (0,0);
\draw[dashed]  (O1s) ellipse (3 and 0.3);
\coordinate (O2s) at (0,-3);
\draw[dashed]  (O2s) ellipse (3 and 0.3);
\coordinate (Oms) at (0,-1.5);
\draw[pattern color=black, pattern = dots]  (O1) ellipse (3 and 0.3);

%\draw  (O1) ellipse (1 and 0.1);
%\draw[dashed]  (O2) ellipse (1 and 0.1);
\coordinate (Plu) at (-6,0);
\coordinate (Pru) at (6,0);
\coordinate (Pld) at (-6,-3);
\coordinate (Prd) at (6,-3);
\coordinate (Plm) at (-6,-1.5);
\coordinate (Prm) at (6,-1.5);
\coordinate (Plds) at (-3,-3);
\coordinate (Prds) at (3,-3);
\coordinate (Plus) at (-3,0);
\coordinate (Prus) at (3,-0);

\coordinate (Qlu) at (-1,0);
\coordinate (Qru) at (1,0);
\coordinate (Qld) at (-1,-3);
\coordinate (Qrd) at (1,-3);
\draw [-{Latex[length=3mm]}] (Pld) -- (Plu);
\draw [] (Pru) -- (Prd);
\draw [dashed] (Plds) -- (Plus);
\draw [dashed] (Prds) -- (Prus);
%\draw [dashed] (Qlu) -- (Qld);
%\draw [dashed] (Qru) -- (Qrd);
\coordinate (Pu0) at (-1.5,-0.25) {};
\coordinate (Pd0) at (-1.5,-3.25) {};
\coordinate (Pu) at (-3.5,-0.5) {};
\coordinate (Pdl) at (-3.5,-3.5) {} {};
\coordinate (Pdr) at (3.5,-3.5) {} {};
\coordinate (Pmu) at (-3.5,-2) {} {};
\coordinate (Pml) at (-1.5,-1.5) {} {};
\draw[thick]  plot[ smooth,tension=.9] coordinates {(Pld) (Pdl) (Pdr) (Prd)};
\draw [dashed] (Plus) -- (Plu);
\draw[pattern color=gray!60, pattern = dots]  (Pu0) node (v2) {} -- (Pu) -- (Pdl) -- (Pd0) -- (Pu0) -- cycle;
\node[label=west:$\mathbf{\psi}$] at (Plm) {};
\node[label=north east :$\theta$] at (Pmu) {};
\draw  plot[ smooth,tension=.7] coordinates {(-3.5,0) (-3.3,-0.23) (-2.5,-0.35)};
\node[label=west :$\mathbf{\phi}$] at (-3.5,0) {};
\draw [-{Latex[length=1mm]}] (Pml) -- (Pmu);
\draw[fill=white]  (O1) circle (0.1);
\draw[fill=white]  (O2) circle (0.1);
%\draw[ decoration={markings, mark=at position 0.7 with {\arrow[scale = 1.5]{Latex[length=3mm,reversed]}}},    postaction={decorate}](O2) .. controls (3.5,-16.5) and (3.5,11) .. (O1);
\node[label=east:$\mathbf{\Psi_{2\pi}}$] at (Prus) {};
\node[label= east:$\mathbf{\Psi_{0}}$] at (Prds) {};
\draw[fill=black]  (Pml) circle (0.051);
\draw [dashed] (O1) -- (Plus);
\draw [dashed] (O1) -- (Pu0);
\coordinate (Om) at (-0.06,-2.6) {};
\draw [ dashed,decoration={markings, mark=at position 0.32 with {\arrow[scale = 1.]{Latex[length=3mm]}}},    postaction={decorate}](Om) arc (20:345:-1 and -4.5);
\end{tikzpicture}}
    \qquad
    \subfloat[]{\begin{tikzpicture}[scale=0.5]
\coordinate (O1) at (0,0);
\draw  (O1) ellipse (6 and 3.6);
\coordinate (t1l) at (-3,1.0);
\coordinate (t1r) at (3,1.0);
\coordinate (t1c1) at  (-4.5,-1.5);
\coordinate (t1c2) at (4.5,-1.5);
\draw (t1l) .. controls  (t1c1)  and (t1c2)  .. (t1r);

\coordinate (t2l) at (-2.7,-0.3) {};
\coordinate (t2r) at (2.7,-0.3) {};
\coordinate (t2c1) at (-4,2) {};
\coordinate (t2c2) at (4,2) {};
\draw (t2l) .. controls  (t2c1)  and (t2c2)  .. (t2r);
\draw[fill=white]  (O1) circle (0.1);
\coordinate (a0) at (0,2) {};
\coordinate (a1) at (2,1.5) {} {};
\draw [] (O1) -- (a0);
\draw [] (O1) -- (a1);
\coordinate (ac1) at (1,2) {} {};
\coordinate (ac2) at (1,2) {} {};

\draw [dashed] (a0) .. controls  (ac1)  and (ac2)  .. (a1);
\coordinate (b0) at (0,1.) {};
\coordinate (b1) at (0.9,0.7) {} {};
\coordinate (bc1) at (0.5,1) {} {} {};
\coordinate (bc2) at (0.5,1) {} {} {};
\coordinate (bc3) at (0,0.5) {} {} {} {};
\draw [-{Latex[length=2mm]}]  (b0) .. controls  (bc1)  and (bc2)  .. (b1);
\coordinate (ac3) at (0.5,0.7) {} {} {};
\node[label=north east:$\mathbf{\psi_{}}$] at (ac3) {};
\coordinate (C1) at (-4.35,-.4);
\draw [dashed] (C1) ellipse (1.55 and 1.4);
\draw [dashed,pattern color=gray!60, pattern = dots] (C1) ellipse (0.75 and 0.65);
\draw [dotted] (C1) ellipse (2.8 and 2.6);

\coordinate (C2a) at (-4.38,0.25) {};
\coordinate (C2b) at (-4.4,1) {};
\coordinate (C3b) at (-5.5,0.5) {};
\coordinate (C3a) at (-5,0) {} {};
\draw [dashed] (C1) -- (C2a);
\draw [-{Latex[length=2mm]}] (C2a) -- (C2b);
\draw [dashed] (C1) -- (C3b);
\draw[fill=black]  (C2a) circle (0.051);
%\draw[fill=black]  (C3a) circle (0.051);
\draw[fill=white]  (C1) circle (0.051);
\draw [-{Latex[length=2mm]}] (C2b) .. controls  (-5.1,0.9)  and (-5.1,0.9)  .. (C3b);
\node[label=south east:$\mathbf{\theta}$] at (C2b) {};
\node[label=north east:$\mathbf{\phi}$] at (C3b) {};
\node at (-7,-2.5) {};
\node[label=south east:$\mathbf{\theta = 2\pi}$] at (-7.5,-2) {};
\node at (-3,-1.5) {};
\node[label=south east:$\mathbf{\theta = \pi}$] at (-5,-1.5) {};
\end{tikzpicture}}
\caption{Map of the configuration space of a rigid body with fixed point.}
\label{fig:fig_p181}
\end{figure}
Consider figure $5.2 (a)$. We can stretch like an accordion the cuboid along the $\phi$ axis and bent it so that the planes $\phi=0$ and $\phi=2\pi$ join. We get $(b)$,  a torus with square sections. The dimension $\phi$ is dealt with as a point $P\left( \theta, \phi, \psi \right)$ in the configuration space  returns to the same point when varying $\phi$ to $\phi+2k\pi$.\\
We can apply the same procedure of stretching and bending for the $\psi$ dimension so that the planes $\Psi=0$ and $\Psi=2\pi$ join.
We get $(c)$,  a torus-like object.\\
The only dimension left is $\theta$ which our multi-dimensional crippled mind can't find a way to reshape this pseudo-torus so that when varying $\theta$ we can come back to the same point as started.
$$\blacklozenge$$
\newpage




\section{p183 - Clarification for 5.561}
\begin{tcolorbox}
The kinetic energy is 
$$\mathbf{5.561.}\spatie T=\half I\left(\dot{\theta}^2+\dot{\phi}^2+\dot{\psi}^2+2\dot{\phi}\dot{\psi}\cos{\theta} \right)$$ 
\end{tcolorbox}
We first determine the general form of the kinetic energy for a rigid body rotating around a fixed point.
From $\mathbf{5,310}$ we have
\begin{align}
v_r&= - \omega_{rm}z_m= -\epsilon_{rst}\omega_s z_t\\
T&= \half\sum m v_r v_r\\
\Rightarrow\spatie T&= \half\sum m \epsilon_{rst}\omega_s z_t\epsilon_{ruv}\omega_u z_v\\
&=\half \sum m \left(\delta_{su}\delta_{tv}\omega_s \omega_u z_tz_v - \delta_{sv}\delta_{tu}\omega_s \omega_u z_tz_v  \right)
\end{align}
For the case $N=3$ we get from $(4)$:
\begin{align}
T&= \half\sum m  \left[ \omega_1^2 \left( z_2^2+z_3^2 \right)+\omega_2^2\left( z_1^2+z_3^2 \right)+\omega_3^2 \left( z_1^2+z_2^2 \right)
-2\omega_1 \omega_2 z_1 z_2 -2\omega_1 \omega_3 z_1 z_3 -2\omega_2 \omega_3 z_2 z_3 \right]
\end{align}
Using the result from $\mathbf{5.336}$ this can be written as 
\begin{align}
T&= \half \left[ I_{11}\omega_1^2 +I_{22}\omega_2^2+I_{33}\omega_3^2 
+2I_{12}\omega_1 \omega_2+2I_{13}\omega_1 \omega_3 +2I_{23}\omega_2 \omega_3  \right]
\end{align}
Considering that the matrix $I_{ij}$ is symmetric, one can always find an appropriate basis so that the matrix becomes diagonal. Hence $(6)$ can be simplified to
\begin{align}
T&= \half \left[ I_{11}\omega_1^2 +I_{22}\omega_2^2+I_{33}\omega_3^2 
 \right]
\end{align}
Of course the $\omega_i$ in $(7)$ are not the Euler angles and we have to express the $\omega_i$ as functions of the Euler angles.

\begin{figure}[H]
\begin{center}
\tdplotsetmaincoords{70}{130}
\begin{tikzpicture}[tdplot_main_coords,scale=5]
		%%Styles
		\tikzstyle{init} = [black];			% observers base
		\tikzstyle{prec} = [blue]	;		% second rotation
		\tikzstyle{nuta} = [red]	;		% second rotation
		\tikzstyle{rotp} = [brown];		% thrid rotation
		\tikzstyle{base} = [thick,-Latex]	;% basis
		\tikzstyle{angle} = [thick,-latex]	;%angles
		\tikzstyle{circle} = [thin,dashed]	;%circles
 
		%Parameters	
		\def\epsi{15};	%  precession angle
		\def\etheta{25};	% nutation angle
		\def\ethetatwo{25};	% nutation angle helping
		\def\ephi{15};	% rotation angle
		\def\rang{0.7};	% radius arc angles
 
		%% 
		% observers basis
		\coordinate (O) at (0,0,0);
		\draw[base,init] (O) -- (1.,0,0) node[anchor=north east]{$\overrightarrow{x}$};
		\draw[dashed,thick,init] (O) -- (-1,0,0) node[anchor=north east]{$$};
		\draw[base,init] (O) -- (0,1,0) node[anchor=north east]{$\overrightarrow{y}$};
		\draw[base,init] (O) -- (0,0,1) node[anchor=north west]{$\overrightarrow{z} $};
		\coordinate (Z0) at (0,0,1.4);
		%\draw[tdplot_rotated_coords,blue,, dashed ] (0,-1.2,0) --(0,-1,0) node[anchor=west]{$$};
		\draw[blue,thick, dashed ] (0,0,1) --(Z0) node[anchor=west]{$$};

		
		
		% Precession
		\tdplotsetrotatedcoords{\epsi}{0}{0};
		\draw[tdplot_rotated_coords,angle,prec] (O) --(1,0,0) node[anchor=north west]{$\overrightarrow{u}$};
		\draw[tdplot_rotated_coords,angle,blue,ultra  thick] (O) --(0,1,0) node[anchor=north west,]{$\overrightarrow{N}$};
		\draw[tdplot_rotated_coords,thick,prec,,ultra  thick] (O) --(0,-1,0) node[anchor=west]{$$};
		%\tdplotdrawarc[tdplot_rotated_coords,circle,prec]{(0,0,0)}{1}{0}{360}{}{};
		\tdplotdrawarc[tdplot_rotated_coords,ultra thick,,prec,dashed,-Latex]{(0,0,1.25)}{0.15}{-350}{10}{}{};
		\coordinate (psi) at (0,0.1,1.4);
		\draw[](psi) node[{anchor= west},blue] {$1^{st}\rightarrow \dot{\psi}$};
		%\draw plot [mark=*, mark size=0.10] coordinates{(0,0,0.95)}; 
		\tdplotdrawarc[tdplot_rotated_coords,black,ultra thick]{(0,0,0)}{1}{-90}{90}{}{}	;
		\tdplotdrawarc[tdplot_rotated_coords,dashed,black,ultra thick]{(0,0,0)}{1}{90}{270}{}{}	;		
		\tdplotdrawarc[tdplot_rotated_coords,angle,prec]{(0,0,0)}{\rang}{90-\epsi}{90}{anchor=north east,prec}{$\psi$};	
		\tdplotdrawarc[tdplot_rotated_coords,angle,prec]{(0,0,0)}{\rang}{-\epsi}{0}{anchor=north east,prec}{$\psi$};	
		\draw[tdplot_rotated_coords,red,thick, dashed ] (0,1.4,0) --(0,1,0) node[anchor=west]{$$};
		\draw[tdplot_rotated_coords,red,thick,, dashed ] (0,-1.2,0) --(0,-1,0) node[anchor=west]{$$};
		
		% Nutation
		\tdplotsetrotatedcoords{\epsi}{\ethetatwo}{0};
		\tdplotdrawarc[tdplot_rotated_coords,rotp,ultra thick]{(0,0,0)}{1}{90}{270}{}{};
		\draw[tdplot_rotated_coords,,nuta] (O) --(-1,0,0) node[anchor=north east]{$$};
		\tdplotsetrotatedcoords{\epsi}{\etheta}{0};
		\draw[tdplot_rotated_coords,base,nuta] (O) --(1,0,0) node[anchor=north east]{$\overrightarrow{x}_0$};
		\draw[tdplot_rotated_coords,base,nuta] (O) --(0,0,1) node[anchor=west]{$\overrightarrow{z}_1$};
		\tdplotsetrotatedthetaplanecoords{0};
		\tdplotdrawarc[tdplot_rotated_coords,circle,dashdotted,red]{(0,0,0)}{1}{0}{360}{}{}	;	
		\tdplotdrawarc[tdplot_rotated_coords,angle,nuta]{(0,0,0)}{\rang}{90-\etheta}{90}{anchor=south west,nuta}{$\theta$};		
		\tdplotdrawarc[tdplot_rotated_coords,angle,nuta]{(0,0,0)}{\rang}{-\etheta}{0}{anchor=north,nuta}{$\theta$};
		\tdplotdrawarc[tdplot_rotated_coords,ultra thick,nuta,dashed,-Latex]{(0,0,1.3)}{0.15}{-340}{80}{}{};		
 		\coordinate (theta) at (0,1.6,0.4);
		\draw[](theta) node[{anchor= west},red] {$2^{nd}\rightarrow\dot{\theta}$};
		%\draw plot [mark=*, mark size=0.510] coordinates{(0,0.95,0.2)}; 
		
		
		% Spin
		\tdplotsetrotatedcoords{\epsi}{\etheta}{\ephi};
		\draw[tdplot_rotated_coords,base,rotp] (O) --(1,0,0) node[anchor=north]{$\overrightarrow{x}_1$};
		\draw[tdplot_rotated_coords,base,rotp] (O) --(0,1,0) node[anchor=west]{$\overrightarrow{y}_1$};
		\tdplotdrawarc[tdplot_rotated_coords,ultra thick,dashed,rotp]{(0,0,0)}{1}{-105}{-87}{}{};	
		\tdplotdrawarc[tdplot_rotated_coords,ultra thick,rotp]{(0,0,0)}{1}{-87}{90}{}{};		
		\tdplotdrawarc[tdplot_rotated_coords,angle,rotp]{(0,0,0)}{\rang}{90-\ephi}{90}{anchor=west,rotp}{$\varphi$}	;	
		\tdplotdrawarc[tdplot_rotated_coords,angle,rotp]{(0,0,0)}{\rang}{-\ephi}{0}{anchor=north,rotp}{$\varphi$};	
		\tdplotdrawarc[tdplot_rotated_coords,ultra thick,,rotp,dashed,-Latex]{(0,0,1.43)}{0.15}{0}{360}{}{};		
 		\coordinate (phi) at (0,-0.9,0.9);
		\draw[](phi) node[{anchor=west},rotp] {$3^{rd}\rightarrow\dot{\varphi}$};
		\draw[tdplot_rotated_coords,brown,thick,, dashed ] (0,0,1) --(0,0,1.64) node[anchor=west]{$$};
		% Base
		\tdplotsetrotatedcoords{0}{0}{0};
		\draw[tdplot_rotated_coords,gray,  fill=white,,thick](0,0,0) circle[radius=0.4pt];
		\draw[tdplot_rotated_coords,gray,  fill=white,,thick](0,0,1.25) circle[radius=0.4pt];
		% Precession
		\tdplotsetrotatedcoords{\epsi}{0}{0};
		\draw[tdplot_rotated_coords,gray,  fill=white,,thick](0,-1,0) circle[radius=0.4pt];
		\draw[tdplot_rotated_coords,gray,  fill=white,,thick](0,1,0) circle[radius=0.4pt];
		\draw[tdplot_rotated_coords,gray,  fill=white,,thick](0,1.3,0) circle[radius=0.4pt];
		% Nutation
		\tdplotsetrotatedcoords{\epsi}{\ethetatwo}{0};
		\draw[tdplot_rotated_coords,gray, fill=white, thick](-1,0,0) circle[radius=0.4pt];
		\draw[tdplot_rotated_coords,gray, fill=white, thick](0,0,1.430) circle[radius=0.4pt];
		\end{tikzpicture};
\caption{Euler angles}
\label{fig:fig_p183_5.561}
\end{center}
\end{figure}

Consider the Euler angles as in figure 5.5.The resulting angular velocity of the rigid body can be expressed as 
\begin{align}
\overline{\omega}&= \dot{\psi}\overline{z}+\dot{\theta}\overline{N}+\dot{\phi}\overline{z}_1
\end{align}
The projection of $\overline{\omega}$ on the basis $\overline{x}_1,\overline{y}_1,\overline{z}_1$ (which we choose fixed to the rigid body) will then coincide with the $\omega_i$.\\
We determine the components of $\overline{z},\overline{N},\overline{z}_1$ with $\overline{x}_1,\overline{y}_1,\overline{z}_1$ as basis.\\
We have
\begin{align}
& \left\{\begin{array}{l}
\overline{N}=\cos{\phi} \ \overline{y}_1+\sin{\phi} \ \overline{x}_1\\
\overline{z}= \cos{\theta} \ \overline{z}_1-\sin{\theta}\ \overline{x}_0\\
\overline{x}_0=\cos{\phi} \ \overline{x}_1-\sin{\phi} \ \overline{y}_1
\end{array}\right.\\
\Rightarrow \spatie &\left\{\begin{array}{l}
\overline{N}=\cos{\phi} \ \overline{y}_1+\sin{\phi} \ \overline{x}_1\\
\overline{z}= \cos{\theta} \ \overline{z}_1-\sin{\theta}\ \cos{\phi} \ \overline{x}_1 +\sin{\theta}\ \sin{\phi} \ \overline{y}_1
\end{array}\right.
\end{align}
Hence,
\begin{align}
 \overline{\omega}&= \dot{\psi}\cos{\theta} \ \overline{z}_1-\dot{\psi}\sin{\theta}\ \cos{\phi} \ \overline{x}_1 +\dot{\psi}\sin{\theta}\ \sin{\phi} \ \overline{y}_1+\dot{\theta}\cos{\phi} \ \overline{y}_1+\dot{\theta}\sin{\phi} \ \overline{x}_1+\dot{\phi}\overline{z}_1
\end{align}
giving
\begin{align}
\left\{\begin{array}{l}
\omega_1= \dot{\theta}\sin{\phi} -\dot{\psi}\sin{\theta}\ \cos{\phi}\\
\omega_2= \dot{\psi}\sin{\theta}\ \sin{\phi}+\dot{\theta}\cos{\phi}\\
\omega_3= \dot{\psi}\cos{\theta} +\dot{\phi}\\
\end{array}\right.
\end{align}
In the case  considered $I_{11}=I_{22}=I_{33}=I$. Plugging $(12)$ in $(7)$ gives indeed
$$T=\half I\left(\dot{\theta}^2+\dot{\phi}^2+\dot{\psi}^2+2\dot{\phi}\dot{\psi}\cos{\theta} \right)$$


$$\blacklozenge$$
\newpage

\section{p186 - Exercise 1}
\begin{tcolorbox}
If a vector at the point with coordinates $\left(1,1,1\right)$ in Euclidean $3$-space has components $\left(3,-1,2\right)$, find the contravariant, covariant and physical components in spherical polar coordinates.
\end{tcolorbox}
The tensor $T_n$ to consider is $\left(3,-1,2\right) - \left(1,1,1\right)= \left(2,-2,1\right)$.\\
The Jacobian matrix for the transformation $z^n \rightarrow x^k$, evaluated at the point $\left(1,1,1\right)$ is 
\begin{align}
J_{\left(1,1,1\right)}&={\begin{pmatrix}{\dfrac {x}{r}}&{\dfrac {y}{r}}&{\dfrac {z}{r}}\\\\{\dfrac {xz}{r^{2}{\sqrt {x^{2}+y^{2}}}}}&{\dfrac {yz}{r^{2}{\sqrt {x^{2}+y^{2}}}}}&{\dfrac {-(x^{2}+y^{2})}{r^{2}{\sqrt {x^{2}+y^{2}}}}}\\\\{\dfrac {-y}{x^{2}+y^{2}}}&{\dfrac {x}{x^{2}+y^{2}}}&0\end{pmatrix}}\\
&=\begin{pmatrix}\dfrac {1}{\sqrt{3}}&\dfrac {1}{\sqrt{3}}&\dfrac {1}{\sqrt{3}}\\\\ \dfrac {1}{3\sqrt{2}}& \dfrac {1}{3\sqrt{2}}&-\dfrac {\sqrt{2}}{3}\\\\ -\dfrac {1}{2}&\dfrac {1}{2}&0\end{pmatrix}\\
\Rightarrow \spatie 
\begin{pmatrix}
r\\
\theta\\
\phi
\end{pmatrix}_{T^{'n}}&=\begin{pmatrix}\dfrac {1}{\sqrt{3}}&\dfrac {1}{\sqrt{3}}&\dfrac {1}{\sqrt{3}}\\\\ \dfrac {1}{3\sqrt{2}}& \dfrac {1}{3\sqrt{2}}&-\dfrac {\sqrt{2}}{3}\\\\ -\dfrac {1}{2}&\dfrac {1}{2}&0\end{pmatrix}\begin{pmatrix}
2\\
-2\\
1
\end{pmatrix}\\
&=\begin{pmatrix}
\dfrac {1}{\sqrt{3}}\\
-\dfrac {\sqrt{2}}{3}\\
-2
\end{pmatrix}
\end{align}
We have the metric tensor evaluated at $\left(1,1,1\right)$
\begin{align}
a_{mn} &= \begin{pmatrix}
1&0&0\\\\
0&r^2&0\\\\
0&0&r^2\sin^2\theta\\\\
\end{pmatrix}=\begin{pmatrix}
1&0&0\\\\
0&3&0\\\\
0&0&2\\\\
\end{pmatrix}\\
\Rightarrow \spatie 
\begin{pmatrix}
r\\
\theta\\
\phi
\end{pmatrix}_{T^{'}_n}&=\begin{pmatrix}
1&0&0\\\\
0&3&0\\\\
0&0&2\\\\
\end{pmatrix}\begin{pmatrix}
\dfrac {1}{\sqrt{3}}\\
-\dfrac {\sqrt{2}}{3}\\
-2
\end{pmatrix}\\
&=\begin{pmatrix}
\dfrac {1}{\sqrt{3}}\\
-\sqrt{2}\\
-4
\end{pmatrix}
\end{align}
And the physical components 
\begin{align}
\begin{pmatrix}
r\\
\theta\\
\phi
\end{pmatrix}_{T^{'}_{ph.}}&=\begin{pmatrix}
1&0&0\\\\
0&\frac{1}{\sqrt{3}}&0\\\\
0&0&\frac{1}{\sqrt{2}}\\\\
\end{pmatrix}\begin{pmatrix}
\dfrac {1}{\sqrt{3}}\\
-\sqrt{2}\\
-4
\end{pmatrix}\\
&=\begin{pmatrix}
\dfrac {1}{\sqrt{3}}\\
-\sqrt{\dfrac {{2}}{{3}}}\\
-2\sqrt{2}
\end{pmatrix}
\end{align}
Another way to find the physical components is to project orthogonally the tensor on the unit vectors of a local Cartesian coordinate system, oriented along the unit vectors $\overline{e}_r,\overline{e}_{\theta},\overline{e}_{\phi}$ corresponding to the vector $P \left(1,1,1\right)$ with modulus $\left|P \right|=\sqrt{3}$. 
We have for the tensor $T_n (2,-2,1)$ with modulus $\left|T_n \right|=3$ as component along $\overline{e}_r$:
\begin{align}
\left|T_n \right|\cos \alpha &= \left|T_n \right|\frac{\left<T_n,P  \right>}{\left|T_n \right|\left|P \right|}\\
&= \left|T_n \right|\frac{2-2+1}{\left|T_n \right|\left|P \right|}\\
&= \frac{1}{\sqrt{3}}
\end{align}
For the component along $\overline{e}_{\theta}$ we first have to determine the vector $\overline{e}_{\theta}$. As first equation we have the orthogonality condition with $\overline{e}_r$ and putting $\overline{e}_{\theta} = (a,b,c)$, get $\left<\overline{e}_r,\overline{e}_{\theta}  \right>=  a+b+c=0$. As $\overline{e}_{\theta}$ lies in the plane $(1,1,0)-(0,0,0)-(0,0,1)$ we can put $a=b$ and get $\overline{e}_{\theta} =  \frac{1}{\sqrt{6}}\left(1,1,-2\right)$ and get for the tensor $T_n (2,-2,1)$  as component along $\overline{e}_{\theta}$:
\begin{align}
\left|T_n \right|\cos \beta &= \left|T_n \right|\frac{\left<T_n,\overline{e}_{\theta}  \right>}{\left|T_n \right|}\\
&= \left|T_n \right|\frac{2-2-2}{\left|T_n \right|\sqrt{6}}\\
&= -\frac{\sqrt{2}}{\sqrt{3}}
\end{align}
For the component along $\overline{e}_{\phi}$ we first have to determine the vector $\overline{e}_{\phi}$. As first equation we have the orthogonality condition with the pair $\overline{e}_r,\overline{e}_{\theta}$  and  get $\overline{e}_{\phi} = \overline{e}_r \times \overline{e}_{\theta}  =  \frac{1}{\sqrt{3}\sqrt{6}}\left( -3,3,0\right)= \left( -\frac{1}{\sqrt{2}},\frac{1}{\sqrt{2}},0\right)$.\\
For the tensor $T_n (2,-2,1)$  as component along $\overline{e}_{\phi}$:
\begin{align}
\left|T_n \right|\cos \gamma &= \left|T_n \right|\frac{\left<T_n,\overline{e}_{\phi}  \right>}{\left|T_n \right|}\\
&= \left|T_n \right|\frac{-2-2}{\left|T_n \right|\sqrt{2}}\\
&= -\frac{4}{\sqrt{2}}\\
&= -2\sqrt{2}
\end{align}
giving
\begin{align}
\begin{pmatrix}
r\\
\theta\\
\phi
\end{pmatrix}_{T^{'}_{ph.}}
&=\begin{pmatrix}
\dfrac {1}{\sqrt{3}}\\
-\sqrt{\dfrac {{2}}{{3}}}\\
-2\sqrt{2}
\end{pmatrix}
\end{align}
as in (9).
$$\blacklozenge$$
\newpage



\section{p186 - Exercise 2}
\begin{tcolorbox}
In cylindrical coordinates $\left(r, \phi,z\right)$ in Euclidean $3$-space, a vector field is such that the vector at each point points along the parametric line of $\phi$, in the sense of $\phi$ increasing, and its magnitude is $kr$, where $k$ is a constant. Find the contravariant, covariant and physical components of this vector field.
\end{tcolorbox}
We can work backwards, with the physical components as starting point. Indeed, at a point $P\left(r,\phi,z\right)$ the tensor of this vector field will have $\left(0,kr,0\right)$ as physical components in the cylindrical coordinates $\left( r, \phi,z \right)$ system.\\
We have the metric tensor 
\begin{align}
a_{mn} &= \begin{pmatrix}
1&0&0\\\\
0&r^2&0\\\\
0&0&1\\\\
\end{pmatrix}
\end{align}
Giving
\begin{align}
\left\{\begin{array}{lll}
X_1 = &h_1 X_{1}^{phys.}&=0\\\\
X_2 = &h_2 X_{2}^{phys.}&=kr^2\\\\
X_3 = &h_3 X_{3}^{phys.}&=0\\\\
\end{array}\right.
\end{align}
and 
\begin{align}
\left\{\begin{array}{lll}
X^1 = &\frac {X_{1}^{phys.}}{h_1}&=0\\\\
X^2 = &\frac {X_{2}^{phys.}}{h_2}&=k\\\\
X^3 = &\frac {X_{3}^{phys.}}{h_3}&=0\\\\
\end{array}\right.
\end{align}

$$\blacklozenge$$
\newpage




\section{p186 - Exercise 3}
\begin{tcolorbox}
Find the physical components of velocity and acceleration along the parametric lines of cylindrical coordinates in terms of the  and their derivatives with respect to time.
\end{tcolorbox}
We have the metric tensor 
\begin{align}
a_{mn} &= \begin{pmatrix}
1&0&0\\\\
0&r^2&0\\\\
0&0&1\\\\
\end{pmatrix}
\end{align}
and the contravariant velocities

\begin{align}
\left\{\begin{array}{lll}
v^1= &\dv{r}{t}\\\\
v^2= &\dv{\phi}{t}\\\\
v^3= &\dv{z}{t}\\\\
\end{array}\right.
\end{align}
giving by $v_{K}^{phys.} = h_K v^K$
\begin{align}
\left\{\begin{array}{lll}
v_r= &\dv{r}{t}\\\\
v_{\phi}= &r\dv{\phi}{t}\\\\
v_z= &\dv{z}{t}\\\\
\end{array}\right.
\end{align}
For the acceleration using $f^r=\fdv{v^r}{t} $
and the Christoffel symbols being 
\begin{align}
\left \{ \begin{array}{c}
\Gamma^m_{nk} = 0 \quad\forall\quad (nk) \ne (r, \theta), (\theta, \theta)\\\\
\Gamma^{\theta}_{r\theta} = \frac{1}{r} \quad\text{and}\quad \Gamma^r_{\theta\theta} = -r
\end{array}\right.\
\end{align}
we have
\begin{align}
\left\{\begin{array}{lll}
f^1= &\dv{v^1}{t}-r\underbrace{v^2\dv{x^2}{t}}_{=\left(v^2\right)^2}\\\\
f^2= &\dv{v^2}{t}+\underbrace{\frac{1}{r}v^1\dv{x^2}{t}+\frac{1}{r}v^2\dv{x^21}{t}}_{=\frac{2}{r}v^1 v^2}\\\\
f^3= &\dv{v^3}{t}\\\\
\end{array}\right.
\end{align}
giving by $f_{K}^{phys.} = h_K f^K$
\begin{align}
&\left\{\begin{array}{lll}
f_r= &\dv{v^1}{t}-r\left(v^2\right)^2\\\\
f_{phi}= &r\dv{v^2}{t}+r\frac{2}{r}v^1 v^2\\\\
f_z= &\dv{v^3}{t}\\\\
\end{array}\right.\\
\Rightarrow \spatie &\left\{\begin{array}{lll}
f_r= &\dv[2]{r}{t}-r\left(\dv{\phi}{t}\right)^2\\\\
f_{phi}= &r\dv[2]{\phi}{t}+2\dv{r}{t} \dv{\phi}{t}\\\\
f_z= &\dv[2]{z}{t}\\\\
\end{array}\right.
\end{align}
$$\blacklozenge$$
\newpage



\section{p186 - Exercise 4}
\begin{tcolorbox}
A particle moves on a sphere under the action of gravity. Find the contravariant an covaraiant components of the force, using colatitude and azimuth, and write down the equation of motion.
\end{tcolorbox}
We determine first the physical components of the force.
\begin{figure}[H]

\begin{tikzpicture}[scale=0.5]
\coordinate (O1) at (0,0);
\draw  (O1) circle (6);
%\draw [thick] (O1) ellipse (6 and 0.6);
\coordinate (Om) at (0.0,-6) ;
%\node at (Om){$P$};
\draw [ decoration={markings, mark=at position 0.32 with {\arrow[scale = 1.]{Latex[length=3mm]}}},    postaction={}](Om) arc (90:-90:3 and -6.0);
\draw [ dashed,decoration={markings, mark=at position 0.32 with {\arrow[scale = 1.]{Latex[length=3mm]}}},    postaction={}](Om) arc (90:-90:-3 and -6.0);
\coordinate (Ohl) at (-5.2,3) {} {} {};
\coordinate (Ohr) at (5.2,3) {} {} {};
%\node at (Oh){$P$};
%\draw[fill=white]  (Ohl) circle (0.1);
%\draw[fill=white]  (Ohr) circle (0.1);
\draw [ decoration={markings, mark=at position 0.32 with {\arrow[scale = 1.]{Latex[length=3mm]}}},    postaction={}](Ohl) arc (-10:190:-5.3 and -0.5);
\draw [dashed ,decoration={markings, mark=at position 0.32 with {\arrow[scale = 1.]{Latex[length=3mm]}}},    postaction={}](Ohl) arc (10:190:-5.3 and 0.5);
\coordinate (NPole) at (0,6) {};
\draw[fill=white]  (NPole) circle (0.1);
\coordinate (SPole) at (0,-6) {};
\draw[dashdotted] (NPole) -- (SPole);
\draw[fill=white]  (SPole) circle (0.1);
\coordinate (P) at (2.7,2.5) {};
\draw[fill=white]  (P) circle (0.1);
\draw[fill=white]  (O1) circle (0.1);
\coordinate (etheta) at (3.5,-1) {};
\coordinate (ephi)  at (7,2.5) {};
\draw [-{Latex[length=2mm]}] (P) -- (etheta);
\draw [-{Latex[length=2mm]}] (P) -- (ephi);
\draw[dashed] (O1) -- (P);
\node[label=east:$\mathbf{\overline{1}_{\theta}}$] at (etheta) {};
\node[label=east:$\mathbf{\overline{1}_\phi}$] at (ephi) {};
\node[label=west:$O$] at (O1) {};
\coordinate (O2) at (-0.10,2);
\node[label=south east:$\theta$] at (O2) {};
\draw [dashed, decoration={markings, mark=at position 0.52 with {\arrow[scale = 1.]{Latex[length=2mm]}}},    postaction={decorate}](O2) arc (90:55:3 and 3.0);
\coordinate (F)  at (2.7,0) {};
\draw [-{Latex[length=2mm]}] (P) -- (F);
\node[label=west:$\mathbf{\overline{F}}$] at (F) {};
\coordinate (D0) at (11,-1) {};
\coordinate (DP) at (15.5,4) {};
\node[label=east:$\mathbf{m}$] at (DP) {};
\coordinate (De) at (20.5,0) {};
\coordinate (DF) at (15.5,-3.5) {};
\coordinate (DA) at (11,8.5) {};
\draw [dashed] (D0) -- (DP);
\draw [dashed] (D0) -- (DA);
\draw [-{Latex[length=2mm]}] (DP) -- (De);
\node[label=east:$\mathbf{\overline{1}_{\theta}}$] at (De) {};
\draw [-{Latex[length=2mm]}] (DP) -- (DF);
\node[label=east:$\mathbf{\overline{F}}$] at (DF) {};
\coordinate (thet) at  (11,3) {};
\draw [dashed, decoration={markings, mark=at position 0.52 with {\arrow[scale = 1.]{Latex[length=2mm]}}},    postaction={decorate}](thet) arc (90:35:3 and 3.0);
\node[label=south east:$\theta$] at (thet) {};
\coordinate (thet2) at (15.5,0.5) {};
\draw [dashed, decoration={markings, mark=at position 0.52 with {\arrow[scale = 1.]{Latex[length=2mm]}}},    postaction={decorate}](thet2) arc (90:30:3 and -3.0);
\node[label=south east:$\frac{\pi}{2}-\theta$] at (thet2) {};
\draw[fill=white]  (DP) circle (0.1);
\end{tikzpicture}

\caption{Physical components of the gravitational force tensor acting on a mass $\mathbf{m}$ on a sphere }
\label{fig:fig_p186_Ex2}
\end{figure}
We note first that the unit vector $\overline{1}_{\phi}$ is perpendicular to the place formed by the vectors  $\overline{1}_{\theta},\overline{F}$ and s the force has no components projected on this vector. The vector $\overline{F}$ is parallel with the axis of reference of the sphere with radius $R$ and so the physical components become
\begin{align}
&\left\{\begin{array}{l}
F_{\phi}^{phys}= 0\\\\
F_{\theta}^{phys}= mg\sin{\theta}\\\\
\end{array}\right.\\
\Rightarrow\spatie &\left\{\begin{array}{ll}
F_{\phi}= 0&F_{\phi}= 0\\\\
F^{\theta}= \frac{1}{R}mg\sin{\theta}&F_{\theta}= R m g\sin{\theta}\\\\
\end{array}\right.
\end{align}
We use equation $\mathbf{5.212.}$
\begin{align}
&\left\{\begin{array}{l}
\dv{}{t}\frac{\partial T}{\partial \dot{x}^s} - \frac{\partial T}{\partial x^s} = F_{s}\\\\
T = \half m a_{pq}\dot{x}^p\dot{x}^q, \ \dot{x}^s=\dv{x^s}{t}\\\\
\end{array}\right.
\end{align}
with for our case
\begin{align}
T = \half m R^2\left(\dot{\theta}^2+ \sin^2 \theta \ \dot{\phi}^2 \right)
\end{align}
and get the set of equation of motion (the second column gives the dimensional analysis as a check for consistency)

\begin{align}
&\left\{\begin{array}{lll}
\frac{\ddot{\phi}}{\dot{\phi}}=  -2\cot\theta \  \dot{\theta}&:&\frac{[T]^{-2}}{[T]^{-1}}\cong [T]^{-1}\\\\
\ddot{\theta} - \left(\dot{\phi}\right)^2\sin\theta \cos\theta= \frac{g}{R}\sin{\theta}&:&[T]^{-2}+\left([T]^{-1}\right)^2 \cong \frac{[L][T]^{-2}}{[L]^{}}\\\
\end{array}\right.
\end{align}
Let's check the special case when $\dot{\phi} = 0$.\\
The first equation can be rewritten and gives of course $\phi=C$ while the second equation becomes $$\ddot{\theta} = \frac{g}{R}\sin{\theta}$$
which is similar to the equation of  the simple gravity pendulum.
$$\blacklozenge$$
\newpage

\section{p186 - Exercise 5}
\begin{tcolorbox}
Consider the motion of a particle on a smooth torus under no forces except normal reaction. The geometrical line element may be written $$ ds^2=\left(a-b\cos \theta \right)^2 d{\phi}^2+b^2 d\theta^2$$ where $\phi$ is an azimuthal angle and $\theta$ an angular displacement from the equatorial plane. Show that  the path of a particle satisfies the following two differential equations in which $h$ is a constant 
$$(a)\spatie \left(a-b\cos\theta\right)^2\dv{\phi}{s} = h$$
$$(b)\spatie b^2\left(\dv{\theta}{\phi}\right)^2= \frac{\left(a-b\cos\theta\right)^4}{h^2}-\left(a-b\cos\theta\right)^2$$
\end{tcolorbox}
We use equation $\mathbf{5.212.}$ and  $\mathbf{5.212.}$
\begin{align}
&\left\{\begin{array}{l}
\dv{}{t}\frac{\partial T}{\partial \dot{x}^s} - \frac{\partial T}{\partial x^s} = F_{s}\\\\
T = \half m a_{pq}\dot{x}^p\dot{x}^q, \ \dot{x}^s=\dv{x^s}{t}\\\\
\end{array}\right.
\end{align}
with for our case
\begin{align}
T = \half m \left(b^2\dot{\theta}^2+ \left(a-b\cos \theta \right)^2  \ \dot{\phi}^2 \right)
\end{align}
\begin{align}
&\left\{\begin{array}{ll}
\frac{\partial T}{\partial \dot{\phi}}= m \left(a-b\cos \theta \right)^2   \dot{\phi}&\frac{\partial T}{\partial {\phi}}= 0\\\\
\frac{\partial T}{\partial \dot{\theta}}=  mb^2\dot{\theta}&\frac{\partial T}{\partial {\theta}}=  mb\left(a-b\cos \theta \right)\dot{\phi}^2\sin \theta\\\\
\end{array}\right.
\end{align}giving
\begin{align} 
&\left\{\begin{array}{l}
\left(a-b\cos \theta \right)^2   \ddot{\phi}+2b\left(a-b\cos \theta \right)\dot{\theta} \dot{\phi}\sin \theta  =0\\\\
b^2\ddot{\theta}-b\left(a-b\cos \theta \right)\dot{\phi}^2\sin \theta=0\\\\
\end{array}\right.\\
\Rightarrow\spatie &\left\{\begin{array}{l}
\left(a-b\cos \theta \right)   \ddot{\phi}=-2b\dot{\theta} \dot{\phi}\sin \theta  \\\\
b^2\ddot{\theta}-b\left(a-b\cos \theta \right)\dot{\phi}^2\sin \theta=0\\\\
\end{array}\right.
\end{align}
In the first equation, put $y \equiv \dot{\phi}$ giving for the first equation:
\begin{align}
\frac{dy}{y}&=-2b \frac{\sin \theta d\theta} {\left(a-b\cos \theta \right)}  \\
\Leftrightarrow\spatie\frac{dy}{y}&=-2 \frac{d\left(a-b\cos \theta \right)} {\left(a-b\cos \theta \right)}\\
\Rightarrow \spatie \log y&=-2 \log \left(a-b\cos \theta \right)+\log C^{}\\
\Rightarrow \spatie \dot{\phi}&= C \left(a-b\cos \theta \right)^{-2}
\end{align}
Note that $\dot{\phi}$ is a time derivative. But as we are on a geodesic, $\mathbf{5.226.}$ stands and so $v$ is constant as $\dv{v}{s} =0$. Using $v = \dv{s}{t}$, (9) can be written as 
\begin{align}
\left(a-b\cos \theta \right)^{2}\dv{\phi}{t}&= C \\
\Leftrightarrow\spatie\left(a-b\cos \theta \right)^{2}\dv{\phi}{s}\underbrace{\dv{s}{t}}_{=v}&= C \\
\Leftrightarrow\spatie\left(a-b\cos \theta \right)^{2}\dv{\phi}{s}&= h \quad \text{with } h=\frac{C}{v}
\end{align}
Next, we don't use the second equation in (5) but the line element equation instead
\begin{align}
ds^2&=\left(a-b\cos \theta \right)^2 d{\phi}^2+b^2 d\theta^2\\
\Rightarrow\spatie \left(\dv{s}{\phi}\right)^2 &= \left(a-b\cos \theta \right)^2 +b^2 \left(\dv{\theta}{\phi}\right)^2\\
\Rightarrow\spatie b^2 \left(\dv{\theta}{\phi}\right)^2 &= \left(\dv{\phi}{s}\right)^{-2 }- \left(a-b\cos \theta \right)^2 \\
\left(12\right)\quad \text{: }\spatie b^2 \left(\dv{\theta}{\phi}\right)^2 &= \frac{\left(a-b\cos \theta \right)^4}{h^2}- \left(a-b\cos \theta \right)^2
\end{align}
$$\blacklozenge$$
\newpage



\section{p186 - Exercise 6}
\begin{tcolorbox}
Consider the motion of a particle under gravity on the smooth torus o the previous problem, the equatorial plane of the torus being horizontal. Taking the mass of the particle to unity, so that $V= bg\sin{\theta}$, show that the path of the paticle satisfies the following two differential equations. 
$$(a)\spatie \left(E-V\right)\left(a-b\cos\theta\right)^2\dv{\phi}{s} = h$$
$$(b)\spatie b^2\left(\dv{\theta}{\phi}\right)^2=  \left(E-V\right)\frac{\left(a-b\cos\theta\right)^4}{h^2}-\left(a-b\cos\theta\right)^2$$
where $E$ is the total energy,  $h$ is a constant and $d\sigma$ is the action line element.
\end{tcolorbox}
The line of reasoning is quite the same as problem $(5)$.
We use equation $\mathbf{5.212.}$ and  $\mathbf{5.212.}$
\begin{align}
&\left\{\begin{array}{l}
\dv{}{t}\frac{\partial T}{\partial \dot{x}^s} - \frac{\partial T}{\partial x^s} = F_{s}\\\\
T = \half m a_{pq}\dot{x}^p\dot{x}^q, \ \dot{x}^s=\dv{x^s}{t}\\\\
\end{array}\right.
\end{align}
with for our case
\begin{align}
T = \half m \left(b^2\dot{\theta}^2+ \left(a-b\cos \theta \right)^2  \ \dot{\phi}^2 \right)
\end{align}
\begin{align}
&\left\{\begin{array}{ll}
\frac{\partial T}{\partial \dot{\phi}}= m \left(a-b\cos \theta \right)^2   \dot{\phi}&\frac{\partial T}{\partial {\phi}}= 0\\\\
\frac{\partial T}{\partial \dot{\theta}}=  mb^2\dot{\theta}&\frac{\partial T}{\partial {\theta}}=  mb\left(a-b\cos \theta \right)\dot{\phi}^2\sin \theta\\\\
\end{array}\right.
\end{align}giving (as $F_{\phi} = -\partial_{\phi} V = 0$ and $F_{\theta} = -\partial_{\theta} V = -bg\cos{\theta}$)
\begin{align} 
&\left\{\begin{array}{l}
\left(a-b\cos \theta \right)^2   \ddot{\phi}+2b\left(a-b\cos \theta \right)\dot{\theta} \dot{\phi}\sin \theta  =0\\\\
b^2\ddot{\theta}-b\left(a-b\cos \theta \right)\dot{\phi}^2\sin \theta=-bg\cos{\theta}\\\\
\end{array}\right.\\
\Rightarrow\spatie &\left\{\begin{array}{l}
\left(a-b\cos \theta \right)   \ddot{\phi}=-2b\dot{\theta} \dot{\phi}\sin \theta  \\\\
b^2\ddot{\theta}-b\left(a-b\cos \theta \right)\dot{\phi}^2\sin \theta=-bg\cos{\theta}\\\\
\end{array}\right.
\end{align}
In the first equation, put $y \equiv \dot{\phi}$ giving for the first equation:
\begin{align}
\frac{dy}{y}&=-2b \frac{\sin \theta d\theta} {\left(a-b\cos \theta \right)}  \\
\Leftrightarrow\spatie\frac{dy}{y}&=-2 \frac{d\left(a-b\cos \theta \right)} {\left(a-b\cos \theta \right)}\\
\Rightarrow \spatie \log y&=-2 \log \left(a-b\cos \theta \right)+\log C^{}\\
\Rightarrow \spatie \dot{\phi}&= C \left(a-b\cos \theta \right)^{-2}
\end{align}
Note that $\dot{\phi}$ is a time derivative. Using  $\dv{s}{t}=v = \sqrt{2T} = \sqrt{2}\sqrt{E-V}$, (9) can be written as 
\begin{align}
\left(a-b\cos \theta \right)^{2}\dv{\phi}{t}&= C \\
\Leftrightarrow\spatie\left(a-b\cos \theta \right)^{2}\dv{\phi}{\sigma}\underbrace{\dv{\sigma}{s}}_{= \sqrt{E-V}}\underbrace{\dv{s}{t}}_{=\sqrt{2}\sqrt{E-V}}&= C \\
\Leftrightarrow\spatie \left(E-V\right)\left(a-b\cos \theta \right)^{2}\dv{\phi}{\sigma}(E-V)&= h
\end{align}
with $h=\frac{C}{\sqrt{2}}$.\\
Next, we don't use the second equation in (5) but the line element equation instead
\begin{align}
ds^2&=\left(a-b\cos \theta \right)^2 d{\phi}^2+b^2 d\theta^2\\
\Rightarrow\spatie \left(\dv{s}{\phi}\right)^2 &= \left(a-b\cos \theta \right)^2 +b^2 \left(\dv{\theta}{\phi}\right)^2\\
\Rightarrow\spatie b^2 \left(\dv{\theta}{\phi}\right)^2 &= \left(\dv{\phi}{s}\right)^{-2 }- \left(a-b\cos \theta \right)^2 \\
\Leftrightarrow\spatie b^2 \left(\dv{\theta}{\phi}\right)^2 &= \left(\dv{\phi}{\sigma}\right)^{-2 }\left(\dv{\sigma}{s}\right)^{-2 }- \left(a-b\cos \theta \right)^2 \\
\left(12\right)\quad \text{: }\spatie b^2 \left(\dv{\theta}{\phi}\right)^2 &= \left( E-V \right)^2\frac{1}{\left(E-V\right)}\frac{\left(a-b\cos \theta \right)^4}{h^2}- \left(a-b\cos \theta \right)^2\\
\Rightarrow\spatie b^2 \left(\dv{\theta}{\phi}\right)^2 &= \left( E-V \right)\frac{\left(a-b\cos \theta \right)^4}{h^2}- \left(a-b\cos \theta \right)^2
\end{align}
$$\blacklozenge$$
\newpage



\section{p187 - Exercise 7}
\begin{tcolorbox}
A dynamical system consists of a thin straight smooth tube which can rotate in a horizontal plan about one end $O$, together with a bead$B$ inside the tube connected to $O$ by a spring. Taking as coordinates $r=OB$ and $\theta = $ angle of rotation of the tube about $O$, the potential energy $V$ is a function of $r$ only. Show that in configuration space , all the lines of force are geodesics for the kinematical line element.
\end{tcolorbox}
Well understanding the question is of course paramount:
\begin{itemize}
  \item  The tube mentioned plays only a functional role to hold the spring "stiff" along the line $OB$ as its mass can be neglected.It  will play  no further role in the dynamics of the system. 
  \item Nothing is said that the system contains any force that keeps the angular velocity at a constant speed $\omega$. 
\end{itemize}  
That being clarified, one can expect that the system will behave as a harmonic oscillator along the line  $OB$ and that ,given an initial rotational momentum, the angular momentum will be a constant during the trajectory of the bead.
This means that the bead will oscillate along $OB$ but as the angular momentum is a constant and  given $m \omega r^2 =C$($m$ = mass of the bead), the  instant radial speed will vary.\\
The only conservative force acting on the bead will be that of the spring and will be $V=\half k \left( r-r_0 \right)^2$, $r_0$ being the point along $OB$ where the spring is not stretched. The generalized forces are $F_r = - k\left( r-r_0 \right)$ and $F_{\theta}= 0$ meaning the lines of force are straight lines pointing to the origin $O$.\\
About the geodesics. Clearly the instantaneous velocity of the bead is $\overrightarrow{v} = \dot{r}\overrightarrow{1}_r+ \dot{\theta}r\overrightarrow{1}_{\theta}$ giving as kinetic energy $T=\half\left(\dot{r}^2+\dot{\theta}^2 r^2\right)$ giving as kinematic line element
$$ds^2= 2Tdt = dr^2+r^2d\theta^2$$
Referring to $\mathbf{3.101}$, the configuration space is flat and the geodesics are straight lines. As the line forces are straight lines towards the origin $O$, these line of force are also geodesics in the configuration space equipped with the kinematical line element.

$$\blacklozenge$$
\newpage


\section{p187 - Exercise 8}
\begin{tcolorbox}
Show that if a line of force is a geodesic         for the kinematical line element, it is also a geodesic for the action line element.
\end{tcolorbox}
From $\mathbf{5.516}$ and  $\mathbf{5.529}$ we have
\begin{align}
X^r = v\dv{v}{s}\lambda^r +\kappa v^2\nu^r
\end{align}
As the line of force is a geodesic, we can start with a velocity tangent to the line of force, ensuring that the trajectory of the dynamical system will lie on the geodesic line of force (see page 175) and thus $\kappa=0$ for the trajectory. Hence,
\begin{align}
X^r = v\dv{v}{s}\lambda^r
\end{align}
expressing now the function of the action line element $d \sigma = \sqrt{E-V}ds $ we have 
\begin{align}
X^r &= v\dv{v}{s}\lambda^r\\
&= v\dv{v}{s}\dv{x^r}{\sigma}\dv{\sigma}{s}\\
&= v\dv{v}{s}\dv{x^r}{\sigma}\sqrt{E-V}\\
&= \sqrt{E-V}v\dv{v}{s}\lambda^{'r}
\end{align}
As stated page 177, this dynamical system will describe in configuration space a geodesic for the action metric, meaning that $\lambda^{'r}$ is tangent to this geodesic and that $X^r$, being collinear with $\lambda^{'r}$ (with the factor $\sqrt{E-V}v\dv{v}{s}$), is also tangent to this geodesic. Hence, this  line of force is also a geodesic for the action line element.
$$\blacklozenge$$
\newpage

\section{p187 - Exercise 9}
\begin{tcolorbox}
Using the methods of Chapter II and $\mathbf{5.532}$, show that the trajectories of a dynamical system with kinetic energy $T$ and potential energy $V$ satisfy the variational equation $$\delta \int_{t_1}^{t_2}\left(T-V\right)dt = 0$$
\end{tcolorbox}
Let's start with a function $L$ defined by 
\begin{align}
dL=(T-V)du\\
\end{align}
As in figure $2$ page 38 we will make $L$ a function of two parameters, $u$ and $v$ , the latter defining a family of curves between the begin  point $u_1$ and the endpoint $u_2$.
\begin{align}
L&= L(u,v)
\end{align}
with 
\begin{align}
(T-V)(u_1,v)=(T-V)_1 \quad (T-V)(u_2,v)=(T-V)_2 \quad \forall v  
\end{align}
We will try to minimize (with respect to $v$) the following functional
\begin{align}
L &= = \int_{u_1}^{u_2}(T-V)(u,v)du
\end{align}
It's derivative with respect to $v$
\begin{align}
\dv{L}{v} &= \int_{u_1}^{u_2}\frac{\partial (T-V)(u,v)}{\partial v}du
\end{align}
We express $(T-V)(u,v)$ as a function of the generalized coordinates $x^r$ and their derivatives. Then,
\begin{align}
\frac{\partial (T-V)(u,v)}{\partial v}&= \frac{\partial (T-V)(u,v)}{\partial \dot{x^r}}\frac{\partial \dot{x^r}}{\partial v}+\frac{\partial (T-V)(u,v)}{\partial x^r}\frac{\partial x^r}{\partial v}
\end{align}
where $\dot{x^r}= \frac{\partial x^r}{\partial u}$.\\
We have 
\begin{align}
\frac{\partial \dot{x^r}}{\partial v} &= \frac{\partial}{\partial v} \frac{\partial x^r}{\partial u}= \frac{\partial}{ \partial u} \frac{\partial x^r}{\partial v}
\end{align}
So, 
\begin{align}
\frac{\partial (T-V)(u,v)}{\partial v}&= \frac{\partial (T-V)(u,v)}{\partial \dot{x^r}}\frac{\partial}{ \partial u} \frac{\partial x^r}{\partial v}+\frac{\partial (T-V)(u,v)}{\partial x^r}\frac{\partial x^r}{\partial v}
\end{align}
Consider the expression
\begin{align}
\int_{u_1}^{u_2}d(AB) &= \int_{u_1}^{u_2}Ad(B) + \int_{u_1}^{u_2}Bd(A)\\
\Rightarrow \spatie \int_{u_1}^{u_2}Ad(B)&=\int_{u_1}^{u_2}d(AB) -\int_{u_1}^{u_2}Bd(A)
\end{align}
Put $B= \frac{\partial x^r}{\partial v}$ and $A=  \frac{\partial (T-V)(u,v)}{\partial \dot{x^r}}$ and putting this inside (9) and (6):
\begin{align}
\dv{L}{v} &= \int_{u_1}^{u_2}d\left(\frac{\partial (T-V)(u,v)}{\partial \dot{x^r}}\frac{\partial x^r}{\partial v} \right)-\int_{u_1}^{u_2}\frac{\partial x^r}{\partial v} d \left(\frac{ \partial (T-V)(u,v)}{\partial \dot{x^r}} \right)+\int_{u_1}^{u_2}\frac{\partial (T-V)(u,v)}{\partial x^r}\frac{\partial x^r}{\partial v}du\\
&=\left.\frac{\partial (T-V)(u,v)}{\partial \dot{x^r}}\frac{\partial x^r}{\partial v} \right|_{u_1}^{u_2}-\left[\int_{u_1}^{u_2}\left(\frac{\partial}{\partial u} \left(\frac{ \partial (T-V)(u,v)}{\partial \dot{x^r}} \right)-\frac{\partial (T-V)(u,v)}{\partial x^r}\right)\frac{\partial x^r}{\partial v} du \right]
\end{align}
We express now the results in term of infinitesimals. A change in "length" $\delta L$ when we pas from a curve $v$ to a curve $v+dv$ is
\begin{align}
\delta L &= \dv{L}{v}\delta v\\
&= \left.\frac{\partial (T-V)(u,v)}{\partial \dot{x^r}}\frac{\partial x^r}{\partial v} \delta v\right|_{u_1}^{u_2}-\int_{u_1}^{u_2}\left(\frac{\partial}{\partial u} \left(\frac{ \partial (T-V)(u,v)}{\partial \dot{x^r}} \right)-\frac{\partial (T-V)(u,v)}{\partial x^r}\right)\frac{\partial x^r}{\partial v}\delta v du\\
&= \left.\frac{\partial (T-V)(u,v)}{\partial \dot{x^r}}\delta x^r\right|_{u_1}^{u_2}-\int_{u_1}^{u_2}\left(\frac{\partial}{\partial u} \left(\frac{ \partial (T-V)(u,v)}{\partial \dot{x^r}} \right)-\frac{\partial (T-V)(u,v)}{\partial x^r}\right)\delta x^r du
\end{align}
The first term vanish as at the endpoints  the $\delta x^r$ are zero and hence we get
\begin{align}
\delta L &= -\int_{u_1}^{u_2}\left(\frac{\partial}{\partial u} \left(\frac{ \partial (T-V)(u,v)}{\partial \dot{x^r}} \right)-\frac{\partial (T-V)(u,v)}{\partial x^r}\right)\delta x^r du
\end{align}
As the $\delta x^r$ are arbitrary, we must have for $\delta L =0$
\begin{align}
\frac{\partial}{\partial u} \left(\frac{ \partial (T-V)(u)}{\partial \dot{x^r}} \right)-\frac{\partial (T-V)(u)}{\partial x^r}&=0
\end{align}
This is the same equation as $\mathbf{5.532}$ which describe the motion of a system with a conservative force. 
$$\blacklozenge$$
\newpage

\section{p188 - Exercise 10}
\begin{tcolorbox}
Using the definition $\mathbf{5.5335}$ for $I_{rs}$, prove that if $X_r$ is any non-zero vector, then $I_{rs}X_rX_s \geq 0$, and that the equality occurs only if all particles of the system are distributed on a single line.
\end{tcolorbox}
By  $\mathbf{5.335}$
\begin{align}
I_{rs}&= \delta_{rs}\sum m z_qz_q - \sum m z_r z_s
\end{align}
Multiplying by $X_rX_s$:

\begin{align}
I_{rs}X_rX_s&= \underbrace{X_r X_s \delta_{rs}}_{=X_rX_r}\sum m z_qz_q - \sum m \underbrace{z_rX_r}_{=\left|z\right|_{(m)}\left|X\right|\cos{\theta}_m }\underbrace{ z_sX_s}_{=\left|z\right|_{(m)}\left|X\right|\cos{\theta}_m }
\end{align}
with $\theta_m$ the angle between the vector $X_r$ and the position vector $z_m$ of a particle.
\begin{align}
I_{rs}X_rX_s &=\left|X\right|^2\sum m \left|z\right|_{(m)}^2 - \left|X\right|^2\sum m\left|z\right|_{(m)}^2\cos^2{\theta}_m \\
&=\left|X\right|^2\sum m \left|z\right|_{(m)}^2\left(1 - \cos^2{\theta}_m \right)
\end{align}
As we have $\left(1 - \cos^2{\theta}_m \right) \in [0,1]$ it is clear that $I_{rs}X_rX_s  \geq0$ and that it only will be zero when $\theta_m = 0 \quad \forall m$ which means that all position vectors are collinear wit $X_r$ and are on a line.
$$\blacklozenge$$
\newpage

\section{p188 - Exercise 11}
\begin{tcolorbox}
Let $Oz_1z_2z_3$ and $O^{'}z^{'}_1z^{'}_2z^{'}_3$ be two sets of Cartesian axes parallel to one another. Consider a mass distribution and let $I_{rs}, I^{'}_{rs}$be its moment of inertia tensors calculated for these two axes in accordance with $\mathbf{5.335}$. Writing $ I^{'}_{rs}=I_{rs}+K_{rs}$, evaluate $K_{rs}$.
\end{tcolorbox}
By  $\mathbf{5.335}$
\begin{align}
I_{rs}&= \delta_{rs}\sum m z_qz_q - \sum m z_r z_s
\end{align}
As the axes of both coordinate systems are parallel, we can write
\begin{align}
z^{'}_q = z_q + b_q
\end{align}
which gives for (1):
\begin{align}
I_{rs}^{'}&= \delta_{rs}\sum m \left(z_q + b_q\right)\left(z_q + b_q\right) - \sum m \left(z_r + b_r\right) \left(z_s + b_s\right)\\
&=\left\{\begin{array}{l}\delta_{rs}\sum m z_q z_q  - \sum m z_r z_s\\ +\delta_{rs}\sum m b_qz_q - \sum m  b_rz_s \\+\delta_{rs}\sum m b_qz_q - \sum m  b_sz_r \\+\delta_{rs}\sum m  b_qb_q- \sum mb_rb_s\end{array}\right.\\
&=\left\{\begin{array}{l}I_{rs}\\ +\delta_{rs}\sum m b_qz_q - \sum m  b_rz_s \\+\delta_{rs}\sum m b_qz_q - \sum m  b_sz_r \\+\delta_{rs}\sum m  b_qb_q- \sum mb_rb_s\end{array}\right.\\
\end{align}
The last term $\delta_{rs}\sum m  b_qb_q- \sum mb_rb_s$ can be interpreted as a  moment of inertia tensor for a single  virtual mass $M=\sum m$ situated at the point $b_q$ seen from the axes $Oz_1z_2z_3$. Let's denote it with $\tilde{I}_{rs}= \sum m \left(\delta_{rs} b_qb_q- b_rb_s\right)$.\\
The other two terms can also be seen as a rigid body of particles distributed in a plane perpendicular to one of the axis i.e. all particles are transported perpendicularity to a plane. We note that $\delta_{rs}\sum m b_qz_q - \sum m  b_rz_s = \delta_{rs}\sum m b_qz_q - \sum m  b_sz_r$. This follows immediately from the symmetric character of $I_{rs}^{'},I_{rs}^{},\tilde{I}_{rs}$. \\ Denoting $\overline{I}_{rs}= \delta_{rs}\sum m b_qz_q - \sum m  b_rz_s +\delta_{rs}\sum m b_qz_q - \sum m  b_sz_r$ giving
$$ K_{rs} = I_{rs}+\overline{I}_{rs}+\tilde{I}_{rs}$$
$$\blacklozenge$$
\newpage

\section{p188 - Exercise 12}
\begin{tcolorbox}
A rigid body is turning about a fixed point. Referred to right-handed axes $Oz_1z_2z_3$, its angular velocity tensor has components
$$\omega_{23}=1, \quad \omega_{31}=2, \quad\omega_{12}=3$$
If we refer the same motion to the axis  $O^{'}z^{'}_1z^{'}_2z^{'}_3$ , such that the axis $O^{'}z^{'}_1$ is $O^{}z^{'}_1$ reversed, while $z_2z_3$ coincide with $O^{'}z^{'}_2z^{'}_3$, what are the $\omega^{'}_{rs}$ and $\omega^{'}_{rs}$?
\end{tcolorbox}

We use the following identities
\begin{align}
\left\{\begin{array}{ll}
 \mathbf{5.312}& \omega_{rm}=-\omega_{mr}\\
\mathbf{5.316}& \omega_{rs}=\epsilon_{rsn}\omega_{n}\\
\mathbf{5.317}& \omega_{1}= \omega_{23}\quad\omega_{2}= \omega_{31}\quad\omega_{3}= \omega_{12}
\end{array}\right.
\end{align}
The angular velocity tensor is
\begin{align}
\Omega =\left( \begin{matrix}
0&3&-2\\
-3&0&1\\
2&-1&0
\end{matrix}\right)
\end{align}
giving by $\mathbf{5.317}$
\begin{align}
 \omega_{1}= \omega_{23}\quad\omega_{2}= \omega_{31}\quad\omega_{3}= \omega_{12}
 \end{align}
 From pure geometrical consideration we can conclude that
 \begin{align}
 \omega_{1}^{'}= -\omega_{1}^{}\quad\omega_{2}^{'}= \omega_{2}^{}\quad\omega_{3}^{'}= \omega_{3}^{}
 \end{align}
 
\begin{figure}[H]
    \centering
    \subfloat[]{\begin{tikzpicture}[scale=0.8]
\coordinate (O) at (0,0);
\coordinate (z1) at (-3,-3);
\coordinate (z2) at (5,0);
\coordinate (z3) at (0,5);
\coordinate (Op) at (10,0);
\coordinate (z2p) at (7,-3);
\coordinate (z1p) at (15,0);
\coordinate (z3p) at (10,5);

\draw [-{Latex[length=2mm]},, ]  (O) -- (z1);
\node[label=south east:$z_1$] at (z1) {};
\draw [-{Latex[length=2mm]},]  (O) -- (z2);
\node[label=south east:$z_2$] at (z2) {};
\draw [-{Latex[length=2mm]} ]  (O) -- (z3);
\node[label=south east:$z_3$] at (z3) {};
\draw [-{Latex[length=2mm]}, ]  (Op) -- (z1p);
\node[label=south east:$z_1^{'}$] at (z1p) {};
\draw [-{Latex[length=2mm]},, ]  (Op) -- (z2p);
\node[label=south east:$z_2^{'}$] at (z2p) {};
\draw [-{Latex[length=2mm]}, ]  (Op) -- (z3p);
\node[label=south east:$z_3^{'}$] at (z3p) {};

\coordinate (O1a) at (-1,0);
\coordinate (O1b) at (-2,-1);
\draw [-{Latex[length=2mm]},very thick, ]  (O1a) -- (O1b);
\node[label=south west:$\omega_1$] at (O1b) {};
\coordinate (O2a) at (1,1);
\coordinate (O2b) at (3,1);
\draw [-{Latex[length=2mm]},very thick, ]  (O2a) -- (O2b);
\node[label=south east:$\omega_2$] at (O2b) {};
\coordinate (O3a) at (-1,1);
\coordinate (O3b) at (-1,3);
\draw [-{Latex[length=2mm]},very thick, ]  (O3a) -- (O3b);
\node[label=south east:$\omega_3$] at (O3b) {};

\coordinate (O1bp) at (11,1);
\coordinate (O1ap) at (13,1);
\draw [-{Latex[length=2mm]},very thick, ]  (O1ap) -- (O1bp);
\node[label=south east:$\omega_1^{'}$] at (O1bp) {};
\coordinate (O2ap) at (9,0);
\coordinate (O2bp) at (8,-1);
\draw [-{Latex[length=2mm]},very thick, ]  (O2ap) -- (O2bp);
\node[label=south west:$\omega_2^{'}$] at (O2bp) {};
\coordinate (O3ap) at (9,1);
\coordinate (O3bp) at (9,3);
\draw [-{Latex[length=2mm]},very thick, ]  (O3ap) -- (O3bp);
\node[label=south east:$\omega_3^{'}$] at (O3bp) {};
\end{tikzpicture}}
\caption{Angular velocity vectors in mirrored axis}
\label{fig:fig_p169_a}
\end{figure}
Indeed, the $\omega_i$ can be considered as vectors, objects independent from the chosen coordinate system. Reversing the direction of the first axis, will for the observer looking along the positive direction, look as if the $\omega_1$ is reversed.
We now use $\omega_{rs}=\epsilon_{rsn}\omega_{n}$ but here we have to be careful with $\epsilon_{rsn}$ when using the equation in the transformed coordinate system.

Looking at $\mathbf{4.312} \quad \epsilon_{stu}^{'} = \epsilon_{mnr}^{} \frac{\partial z_m}{\partial z_s^{'}}\frac{\partial z_n}{\partial z_t^{'}}\frac{\partial z_r^{}}{\partial z_u^{'}}$ and noting that $\frac{\partial z_1}{\partial z_1^{'}}=-1$ and $1$ or $0$ for the others, we have
$\epsilon_{stu}^{'} = -\epsilon_{mnr}^{}$.
Now with $\mathbf{5.316}$ we get
\begin{align}
 \omega_{rs}^{'}=-\epsilon_{rsn}\omega_{n}^{'}
 \end{align}
 giving
 
 \begin{align}
 \omega_{12}^{'}= -\omega_{12}^{}\quad\omega_{13}^{'}= -\omega_{13}^{}\quad\omega_{23}^{'}= \omega_{23}^{}
 \end{align}
 Giving
 \begin{align}
\Omega^{'} =\left( \begin{matrix}
0&-3&2\\
3&0&1\\
-2&-1&0
\end{matrix}\right)
\end{align}
 
$$\blacklozenge$$
\newpage

\section{p188 - Exercise 13}
\begin{tcolorbox}
Consider three rigid bodies, $S,S^{'}, S^{"}$, turning about a common point. If all angular velocities are referred to common axes, show that the angular velocity tensors of $S^{"}$ relative to $S$ is the sum of the angular velocity tensors of $S^{'}$ relative to $S$ and of  $S^{"}$ relative to $S^{'}$.
\end{tcolorbox} 

Consider the following three transformation from one axes system to another 
\begin{align}
\left\{\begin{array}{llll}
z^{'}_r= A_{rm}z^{}_m&z^{}_r= A_{mr}z^{'}_m& A_{mp} A_{mq}=\delta_{pq}&A_{pm} A_{qm}=\delta_{pq}\\
z^{"}_r= B_{rm}z^{'}_m&z^{'}_r=B_{mr}z^{"}_m& B_{mp} B_{mq}=\delta_{pq}&B_{pm} B_{qm}=\delta_{pq}\\
z^{"}_r= C_{rm}z^{}_m&z^{}_r=C_{mr}z^{"}_m& C_{mp} C_{mq}=\delta_{pq}&C_{pm} C_{qm}=\delta_{pq}
\end{array}\right.
\end{align}
We then have,
\begin{align}
\left\{\begin{array}{l}
\omega^{'}_{pq}\left(S^{'},S^{}\right)= -A_{pm}\dot{A_{qm}}\\
\omega^{"}_{pq}\left(S^{"},S^{'}\right)= -B_{pm}\dot{B_{qm}}\\
\omega^{"}_{pq}\left(S^{"},S^{}\right)= -C_{pm}\dot{C_{qm}}
\end{array}\right.
 \end{align}
 From (1) we see that 
 \begin{align}
 C_{rq}= B_{rm}A_{mq}
 \end{align}
 And thus 
 \begin{align}
 \omega^{"}_{pq}\left(S^{"},S^{}\right)&= -B_{pk}A_{km}\dot{\left(B_{qn}A_{nm}\right)}\\
\Rightarrow\spatie &= \underbrace{-A_{km}\dot{A_{nm}}}_{= \omega^{'}_{kn}\left(S^{'},S^{}\right)}B_{pk}B_{qn}-\underbrace{A_{km}A_{nm}}_{=\delta_{kn}}B_{pk}\dot{B_{qn}}\\
\Rightarrow\spatie &=  \omega^{'}_{kn}\left(S^{'},S^{}\right)B_{pk}B_{qn}-\underbrace{B_{pn}\dot{B_{qn}}}_{=-\omega^{"}_{pq}\left(S^{"},S^{'}\right)}
 \end{align}
 The fist term of the right side expression is a bilinear map of the tensor $\omega^{'}_{kn}\left(S^{'},S^{}\right)$ from the reference axis $S^{'}$ to $S^{"}$. Hence we get 
\begin{align}
\omega^{"}_{pq}\left(S^{"},S^{}\right)=\omega^{"}_{pq}\left(S^{"},S^{'}\right)+\omega^{"}_{pq}\left(S^{'},S^{}\right)
\end{align}
$$\blacklozenge$$

\newpage
\section{p188 - Exercise 14}
\begin{tcolorbox}
A freely moving particle is observed from a platform which rotates with angular velocity $\omega_r = n\delta_{r3}$, where $n$ is constant, relative to a Newtonian frame $S$ in which $z_r$ are rectangular Cartesians. Use $\mathbf{5.421}$ to find the equations of motion relative to $S^{'}$ in terms of coordinates  $z^{'}_r$ in $S^{'}$, such that the axis of $z^{'}_3$ coincides permanently with the axis of $z_3$.
\end{tcolorbox} 
$\mathbf{5.421}$ gives (where the equation is expressed in term of the $z^{'}_r$
\begin{align}
\left\{\begin{array}{l}
mf_s = F^{'}_s + C^{'}_s + G^{'}_s\\
C^{'}_s= m\left[\dot{\omega}^{'}_{sn}\left(S^{'},S\right) + \omega^{'}_{sm}\left(S^{'},S\right)\omega^{'}_{nm}\left(S^{'},S\right)\right]z^{'}_n\\
C^{'}_s = 2m\omega^{'}_{sm}v^{'}_m\left(S^{'}\right)
\end{array}\right.
\end{align}
We note the particle is free, so $F^{'}_s=0$ and the angular velocity is a constant, so $\dot{\omega}^{'}_{sn}\left(S^{'},S\right)=0$, and the equation simplify to 
\begin{align}
\left\{\begin{array}{l}
f^{'}_s = K^{'}_s + J^{'}_s\\
K^{'}_s= \left[ \omega^{'}_{sm}\left(S^{'},S\right)\omega^{'}_{nm}\left(S^{'},S\right)\right]z^{'}_n\\
J^{'}_s = 2\omega^{'}_{sm}v^{'}_m\left(S^{'}\right)
\end{array}\right.
\end{align}
As $\omega_s= n\delta_{s3}$ and by the requirement that the that the axis of $z^{'}_3$ coincides permanently with the axis of $z_3$, it is not hard to see that
\begin{align}
\left\{\begin{array}{llll}
\omega^{}_{12}\left(S^{'},S\right) = n\\
\omega^{'}_{12}\left(S^{'},S\right) = n\\
\omega^{}_{12}\left(S^{},S^{'}\right) = -n\\
\omega^{'}_{12}\left(S^{},S^{'}\right) = -n
\end{array}\right.
\end{align}
while all other elements vanish.\\
We get 
\begin{align}
&\left\{\begin{array}{l}
K^{'}_1 = \omega^{'}_{12}\left(S^{'},S\right)\omega^{'}_{12}\left(S^{'},S\right)z^{'}_1 = n^2 z_1^{'}\\
K^{'}_1 = \omega^{'}_{21}\left(S^{'},S\right)\omega^{'}_{21}\left(S^{'},S\right)z^{'}_1 = n^2 z_1^{'}\\
K^{'}_3 = 0
\end{array}\right.\\
&\left\{\begin{array}{l}
J^{'}_1= 2\omega^{'}_{12}\left(S^{'},S\right)v^{'}_2\left(S^{'}\right)=2 n v_2^{'}\left(S^{'}\right)\\
J^{'}_2= 2\omega^{'}_{21}\left(S^{'},S\right)v^{'}_1\left(S^{'}\right)=-2 n v_1^{'}\left(S^{'}\right)\\
J^{'}_3 = 0
\end{array}\right.
\end{align}
and get as equations of motion
\begin{align}
\left\{\begin{array}{l}
f^{'}_1= n^2z_1^{'} +2nv^{'}_2\left(S^{'}\right)\\
f^{'}_2= n^2z_1^{'} -2nv^{'}_1\left(S^{'}\right)\\
f^{'}_3= 0
\end{array}\right.
\end{align}
$$\blacklozenge$$
\newpage

\section{p188 - Exercise 15}
\begin{tcolorbox}
If the tensor $I_{st}$ is defined by $\mathbf{5.335}$ for N dimensions, and $J_{nprq}$ is defined by $\mathbf{5.330}$, establish the following relations:
$$J_{nprq}= \left(N-1\right)^{-1}I_{ss}\left(\delta_{nr}\delta_{pq}- \delta_{nq}\delta_{pr}\right) - \delta_{nr}I_{pq}+\delta_{pr}I_{nq}$$
$$J_{nppq}=I_{ss}$$
$$I_{nq}= \left(N-1\right)^{-1}\left(J_{nprq}-\delta_{nq}J_{nprq}\right) $$
\end{tcolorbox} 
$\mathbf{5.421}$ and $\mathbf{5.421}$:
\begin{align}
\left\{\begin{array}{l}
I_{st}= \delta_{st}\sum m z_q z_q - \sum m z_s z_t\\
J_{nprq} = \sum m \left( \delta_{nr} z_p z_q - \delta_{pr} z_n z_q \right)
\end{array}\right.
\end{align}
The first equation can be expressed as $ \sum m z_p z_q=  \delta_{pq}\sum m z_k z_k-I_{pq}$ and $ \sum m z_n z_q=  \delta_{st}\sum m z_k z_k-I_{nq}$ 

giving 
\begin{align}
J_{nprq}& =  \delta_{nr} \delta_{pq}\sum m z_k z_k-\delta_{nr}I_{pq} - \delta_{pr} \delta_{st}\sum m z_k z_k+\delta_{pr}I_{nq} \\
&= \sum m z_k z_k\left( \delta_{nr} \delta_{pq}- \delta_{nr}I_{pq}\right)-\delta_{nr}I_{pq} +\delta_{pr}I_{nq}
\end{align}
Now, consider the expressions 
\begin{align}
\left\{\begin{array}{l}
I_{11}= \sum m z_q z_q - \sum m z_1 z_1\\
I_{11}= \sum m z_q z_q - \sum m z_1 z_1\\
\vdots\\
I_{NN} = \sum m z_q z_q - \sum m z_N z_N\\
\end{array}\right.
\end{align}
Summin up these $N$ expressions we have
\begin{align}
I_{ss} &= N\left( \sum m z_q z_q \right) - \sum m z_q z_q\\
&= \left( N-1  \right)\sum m z_q z_q \\
\Rightarrow\spatie \sum m z_q z_q &= I_{ss} \left( N-1  \right)^{-1}
\end{align}
Plugging this in $(3)$ we get
\begin{align}
J_{nprq}
&= I_{ss} \left( N-1  \right)^{-1}\left( \delta_{nr} \delta_{pq}- \delta_{nr}I_{pq}\right)-\delta_{nr}I_{pq} +\delta_{pr}I_{nq}
\end{align}
$$\blacklozenge$$
\newpage


\section{p188 - Exercise 16}
\begin{tcolorbox}
The motion of a dynamical system is represented by a curve in configuration-space. Using the kinematical line element, express the curvature as a function of its total energy $E$, and deduce that as $E$ tends to infinity, the trajectorry tends to become a geodesic . Illustrate by considering a particle moving under gravity on a smooth sphere.
\end{tcolorbox} 
$\mathbf{5.421}$ and $\mathbf{5.421}$:
\begin{align}
\left\{\begin{array}{l}
I_{st}= \delta_{st}\sum m z_q z_q - \sum m z_s z_t\\
J_{nprq} = \sum m \left( \delta_{nr} z_p z_q - \delta_{pr} z_n z_q \right)
\end{array}\right.
\end{align}

$$\blacklozenge$$
\newpage

\section{p189 - Exercise 17}
\begin{tcolorbox}
A particle moves on a smooth sphere under action of gravity. Using he action line element, calculate the Gaussian curvature of configuration-space as a function of total energy $E$ and height $z$ above the centre of the sphere. Show that if the total energy is not sufficient to raise the particle to the top of the sphere, but only to a level $z=h$, then the Gaussian curvature tends to infinity as $z$ approaches $h$ from below.
\end{tcolorbox} 
$\mathbf{5.421}$ and $\mathbf{5.421}$:
\begin{align}
\left\{\begin{array}{l}
I_{st}= \delta_{st}\sum m z_q z_q - \sum m z_s z_t\\
J_{nprq} = \sum m \left( \delta_{nr} z_p z_q - \delta_{pr} z_n z_q \right)
\end{array}\right.
\end{align}

$$\blacklozenge$$
\newpage


\section{p189 - Exercise 18}
\begin{tcolorbox}
Show that the equations of motion of a rigid body with a fixed point may be written in either of the forms\\
$(a)\spatie \dot{h}^{'}_r+\omega^{'}_{mr}\left(S^{'},S^{}\right)h^{'}_m = M^{'}_{rs},\\
(b)\spatie  \spatie \dot{h}^{'}_r-K^{'}_{rmn}h^{'}_m h^{'}_n= M^{'}_{rs},$\\
where $h^{'}_r$ are the components on $z^{'}$-axes (moving with the body) of angular moment as given in $\mathbf{5.338}$ and $K^{'}_{rmn}$ is a certain moment of inertia tensor. Evaluate the components  $K^{'}_{rmn}$ in terms of the moments and products of inertia.
\end{tcolorbox} 
$\mathbf{5.421}$ and $\mathbf{5.421}$:
\begin{align}
\left\{\begin{array}{l}
I_{st}= \delta_{st}\sum m z_q z_q - \sum m z_s z_t\\
J_{nprq} = \sum m \left( \delta_{nr} z_p z_q - \delta_{pr} z_n z_q \right)
\end{array}\right.
\end{align}

$$\blacklozenge$$
\newpage


\section{p189 - Exercise 19}
\begin{tcolorbox}
A rigid body turns about a fixed point $O$in a flat space of $N$ dimensions. prove that if $N$ is odd, there exists at any instant a line $OP$ of particles instantaneously at rest, but that, if $N$ is even, no point other than $O$ is, in general, instantaneously at rest. Show that if $N=4$, there are are points oher than $O$ instantaneously at rest if, and only if,
$$\omega_{23}\omega_{14}+\omega_{31}\omega_{24}+\omega_{12}\omega_{34}=0$$
\end{tcolorbox} 
Suppose $N$ is odd. Let's define the following vector
\begin{align}
\omega_{i_1}= \epsilon_{i_{1} i_{2} \dots i_{N}}\prod_{k=1}^{\frac{N-1}{2}}\omega_{i_{2k} i_{2k+1}}
\end{align}
and a line
\begin{align}
z_{i_1} = \theta \omega_{i_1}\quad (\theta \in \mathbb{R})
\end{align}
First we note that $\omega_{i_1}$ (and hence $z_{i_1}$) is not a null-vector:\\
Let's consider in $(1)$ the terms consisting of the permutation of the sequence of pairs $$\left\{\left( i_2,i_3 \right),\left( i_4,i_5 \right),\left( i_6,i_7\right),\dots ,\left( i_{N-1},i_N \right)\right\}$$ This sequence contains $\frac{N-1}{2}$ pairs and so can be arranged in $\frac{N-1}{2}!$ ways. As for each pair we have two valid possibilities e.g. $\left( i_2,i_3 \right)$ and $\left( i_3,i_2 \right)$ and as a sequence contains $\frac{N-1}{2}$ pairs, we will have for a given order of pairs $2^{\frac{N-1}{2}}$ possibilities. So in $(1)$ there will be $2^{\frac{N-1}{2}}\frac{N-1}{2}!$ terms consisting of the permutation of the sequence of pairs $\left\{\left( i_2,i_3 \right),\left( i_4,i_5 \right),\left( i_6,i_7\right),\dots ,\left( i_{N-1},i_N \right)\right\}$.\\
Without loss of generality, suppose that $\epsilon_{i_{1} i_{2} \dots i_{N}}$ is positive and also all $\omega_{i_{2k} i_{2k+1}}$ are positive.
Let's first consider a permutation of two pairs in the sequence $\left\{\left( i_2,i_3 \right),\left( i_4,i_5 \right),\left( i_6,i_7\right),\dots ,\left( i_{N-1},i_N \right)\right\}$. Obviously, this does not change the product of the $\omega_{i_{2k} i_{2k+1}}$. Also $\epsilon_{i_{1} i_{2} \dots i_{N}}$ will hold it's initial sign as the considered permutation needs two permutation of indices.\\
Next consider a permutation in one of the pairs of the sequence. Obviously $\epsilon_{i_{1} i_{2} \dots i_{N}}$ will change sign but also the picked $\omega_{i_{2k} i_{2k+1}}$(skew-symmetric). \\
Conclusion, all $2^{\frac{N-1}{2}}\frac{N-1}{2}!$ terms can be reduced to the sum of $2^{\frac{N-1}{2}}\frac{N-1}{2}!$ of a same quantity and the  $\omega_{i_1}$ will not trivially be zero.\\
Let's consider now $\mathbf{5.310}$
\begin{align}
v_p &= - \omega_{pi_{1}}z_{i_{1}}\\
\text{(2):}\spatie v_p &= -\theta \epsilon_{i_{1} i_{2} \dots i_{N}}\omega_{pi_{1}}\prod_{k=1}^{\frac{N-1}{2}}\omega_{i_{2k} i_{2k+1}}
\end{align}
On the right side, for having a non-zero term, we need that $p\ne i_{1}$ ( $\omega_{st}$ skew-symmetric ). This leaves us with only $N-1$ possible choices in the indices but as $\epsilon_{i_{1} i_{2} \dots i_{N}}$ needs $N$ mutual different indices it is obvious that each term in $(4)$ will have a $\epsilon_{i_{1} i_{2} \dots i_{N}}= 0$ \\
Conclusion, all $v_p$are zero and hence the defined line in $(2)$ is an instantaneous line of rotation.
\\
What about $N$is even?\\
Obviously we can't construct a vector the way we did for the odd case.
$$\blacklozenge$$
\newpage 



\section{p189 - Exercise 20}
\begin{tcolorbox}
The equations $\mathbf{5.329}$ do not determine $J_{nprq}$ uniquely. Why? As an alternative to $\mathbf{5.330}$, we can require $J_{npqr}$ to be skew-symmetric in the last two suffixes. Show that this defines $J_{nprq}$ uniquely as follows:
$$J_{nprq}= \half\sum m \left(\delta_{nr}z_pz_q +\delta_{pq}z_nz_r-\delta_{nq}z_pz_r-\delta_{pr}z_nz_q \right)$$
Prove that$J_{nprq}$, as defined here, has the same symmetries as the covariant curvature tensor (cf. $\mathbf{3.115}$, $\mathbf{3.116}$) and that, for $N=3$, we have
$$ I_{st} = \half\epsilon_{snp}\epsilon_{trq}J_{nprq}, \quad  J_{nprq}=  \half\epsilon_{snp}\epsilon_{trq} I_{st}$$
\end{tcolorbox} 
The equations $\mathbf{5.329}$ , $h_{np}=J_{nprq}\omega_{rq}$ do not determine $J_{nprq}$ uniquely because $\omega_{rq}$ is skew symmetric, so all elements at the positions $J_{np(rr)}$ can be chosen arbitrarily and still comply with the equation.\\
Consider now the expression
\begin{align}
J^{'}_{nprq}&= \half\left(  J^{}_{nprq}-J^{'}_{npqr}\right)\\
\Rightarrow \spatie h^{'}_{np}&=J^{'}_{nprq}\omega_{rq}\\
&=  \half\left(  J^{}_{nprq}\omega_{rq}-J^{}_{npqr}\omega_{rq}\right)\\
&= \half\left(  J^{}_{nprq}\omega_{rq}+J^{}_{npqr}\omega_{qr}\right)\\
&= \half\left(  J^{}_{nprq}\omega_{rq}+J^{}_{nprq}\omega_{rq}\right)\\
&=h_{np}
\end{align}
So this expression $J_{nprq}= \half\sum m \left(\delta_{nr}z_pz_q +\delta_{pq}z_nz_r-\delta_{nq}z_pz_r-\delta_{pr}z_nz_q \right)$ still describes the dynamical system and we note that this expression is skew-symmetric in the two last suffixes:
\begin{align}
J_{np(rr)}&= \half\sum m \left(\delta_{nr}z_pz_r +\delta_{pr}z_nz_r-\delta_{nr}z_pz_r-\delta_{pr}z_nz_r \right)=0\\
J_{npqr}&= \half\sum m \left(\delta_{nq}z_pz_r +\delta_{pr}z_nz_q-\delta_{nr}z_pz_q-\delta_{pq}z_nz_r \right)\\
&=-\half\sum m \left(-\delta_{nq}z_pz_r -\delta_{pr}z_nz_q+\delta_{nr}z_pz_q+\delta_{pq}z_nz_r \right)\\
&= - J_{nprq}
\end{align}

\textbf{Symmetries to prove:}
\begin{align}
\left\{\begin{array}{l}
J_{nprq} = -J_{pnrq}, \quad J_{nprq} = -J_{npqr},\quad J_{nprq} = J_{rqnp}\\
J_{nprq}+J_{nrqp}+J_{nqpr}=0
\end{array}\right.
\end{align}
The second identity of $(11)$ is already proven as $J_{nprq}$ is skew-symmetric in the last two suffixes. For the rest:
\begin{align}
J_{nprq}&= \half\sum m \left(\delta_{nr}z_pz_q +\delta_{pq}z_nz_r-\delta_{nq}z_pz_r-\delta_{pr}z_nz_q \right)\\
&=- \half\sum m \left(-\delta_{nr}z_pz_q -\delta_{pq}z_nz_r+\delta_{nq}z_pz_r+\delta_{pr}z_nz_q \right)\\
&= J_{pnrq}
\end{align}
and
\begin{align}
J_{nprq}&= \half\sum m \left(\delta_{nr}z_pz_q +\delta_{pq}z_nz_r-\delta_{nq}z_pz_r-\delta_{pr}z_nz_q \right)\\
J_{rqnp}&= \half\sum m \left(\delta_{rn}z_qz_p +\delta_{qp}z_rz_n-\delta_{rp}z_qz_n-\delta_{qn}z_rz_p \right)\\
&= J_{pnrq}
\end{align}
and
\begin{align}
J_{nprq}+J_{nrqp}+J_{nqpr}&=\left\{\begin{array}{l}
+\half\sum m \left(\delta_{nr}z_pz_q +\delta_{pq}z_nz_r-\delta_{nq}z_pz_r-\delta_{pr}z_nz_q \right)\\
+\half\sum m \left(\delta_{nq}z_pz_r +\delta_{rp}z_nz_q-\delta_{np}z_qz_r-\delta_{rq}z_nz_p \right)\\
+\half\sum m \left(\delta_{np}z_rz_q +\delta_{qr}z_nz_p-\delta_{nr}z_qz_p-\delta_{qp}z_nz_r \right)
\end{array}\right.
&=0
\end{align}
For the last part:
From $\mathbf{5.33}$,  ($J^{'}_{nprq}$ being a not necessarily skew-symmetric tensor)
\begin{align}
I_{st} = \half J^{'}_{nprq} \epsilon_{rqt}\epsilon_{snp}
\end{align}
or 
\begin{align}
I_{st} &= \half J^{'}_{npqr} \epsilon_{qrt}\epsilon_{snp}\\
&= -\half J{'}_{npqr} \epsilon_{rqt}\epsilon_{snp}
\end{align}
Adding $(19)$ and $(21)$ gives
\begin{align}
2I_{st} &= \underbrace{\half \left(J^{'}_{npqr} - J{'}_{npqr} \right)}_{= J^{}_{npqr}}\epsilon_{rqt}\epsilon_{snp}\\
\Rightarrow\spatie I_{st} &=\half J^{}_{npqr}\epsilon_{rqt}\epsilon_{snp}\
\end{align}
And
\begin{align}
I_{st}\epsilon_{trq}\epsilon_{snp} &= \half J^{'}_{kjuv} \underbrace{\epsilon_{tuv}\epsilon_{trq}}_{= \delta_{ur}\delta_{vq}-\delta_{uq}\delta_{vr}}\underbrace{\epsilon_{skj}\epsilon_{snp}}_{= \delta_{kn}\delta_{jp}-\delta_{kp}\delta_{jn}}\\
&= -\half J{'}_{npqr} \epsilon_{rqt}\epsilon_{snp}
\end{align}
expending the right product we get
\begin{align}
I_{st}\epsilon_{trq}\epsilon_{snp} &= \half \left( J{}_{nprq} +J{}_{pnqr}-J{}_{npqr}-J{}_{pnrq}\right)
\end{align}
And considering the symmetries described previously we get
\begin{align}
I_{st}\epsilon_{trq}\epsilon_{snp} &= \half 4 J{}_{nprq}\\
&= 2 J{}_{nprq}\\
\Rightarrow \spatie J{}_{nprq}&= \half I_{st}\epsilon_{trq}\epsilon_{snp} 
\end{align}
$$\blacklozenge$$
\newpage