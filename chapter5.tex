\setcounter{chapter}{4}
\chapter{Applications to Classical Mechanics}
\pagebreak[4]
\section{p153 - Exercise}
\begin{tcolorbox}
If $\mu^{\alpha}$ are the contravariant components of a unit vector in a surface $S$, show that $\mu^{\alpha}f_{\alpha}$ is the physical component of acceleration in the direction tangent to $S$ defined by $\mu^{\alpha}$.
\end{tcolorbox}
As we are in an Euclidean space we can interpret $a_{mn}\mu^{\alpha}f^{\alpha}$ as $\left|\mu\right|\left|f\right|\cos\theta $ with $\theta$ the angle between the two vectors. As $\left|\mu\right|=1$ we have
\begin{align}
a_{mn}\mu^{\alpha}f^{\alpha}&= \mu^{\alpha}f_{\alpha}\\
&= \left|f\right|\cos\theta 
\end{align}
which is the projection of the vector $f$ on the unit vector $\mu$.
$$\blacklozenge$$
\newpage

\section{p154 - Clarification to 5.226.}
\begin{tcolorbox}
$$\mathbf{\text{5.226.}\spatie v\dv{v}{s}=0,\quad \overline{\kappa}v^2=0}$$
Assuming that the particle is not at rest $v\ne 0$, and therefore $\overline{\kappa}=0$. \textit{\textbf{Since this implies that the curve is a geodesic}...}
\end{tcolorbox}
The assertion in bold is a direct consequence $$\mathbf{\text{2.513.}}\spatie \frac{\delta \dv{x^r}{s}}{\delta s}=0$$ 
As in $ \mathbf{5.233}$ we have $\frac{\delta \lambda^{\alpha}}{\delta s}=\frac{\delta \dv{x^{\alpha}}{s}}{\delta s}=0$, the considered curve follows the geodesic curve.
$$\blacklozenge$$
\newpage

\section{p155 - Exercise}
\begin{tcolorbox}
Show that in relativity the force $4$-vector $X^r$ lies along the first normal of the trajectory in space-time. Express the first curvature in terms of the proper mass $m$ of the particle and the magnitude $X$ of $ X^r$.
\end{tcolorbox}
Let us recall the first Frenet formula $\mathbf{2.705}$ without forgetting that the metric form is not positive-definite, $$\frac{\delta \lambda^r}{\delta s}=\kappa\nu^r,\quad \epsilon_{(1)}\nu_n\nu^n=1$$ As $\mathbf{5.299}$ $$m\frac{\delta \lambda^r}{\delta s}=X^r$$ it is clear that $X^r = m\kappa\nu^r$ and is collinear with the first normal.
\begin{align}
X^r &= m\kappa\nu^r\\
\times \quad a_{mr}X^m\quad\Rightarrow\spatie \underbrace{a_{mr}X^mX^r}_{=\left(X^1\right)^2+\left(X^2\right)^2+\left(X^3\right)^2-\left(X^4\right)^2} &= m\kappa \underbrace{a_{mr}\nu^m\nu^r}_{= \epsilon_{(1)}}
\end{align}
\textbf{$$\Rightarrow\spatie \kappa = \epsilon_{(1)}\frac{\left(X^1\right)^2+\left(X^2\right)^2+\left(X^3\right)^2-\left(X^4\right)^2}{m}
$$}
$$\blacklozenge$$
\newpage


\section{p156 - Clarification}
\begin{tcolorbox}
Interpretation of 
$$\mathbf{5.231.}\spatie M_{rs}=\epsilon_{rsn}M_n=z_rF_s-z_sF_r$$
\end{tcolorbox}
What do the $M_{rs}$ represent?
\begin{figure}[H]

\begin{tikzpicture}[scale=0.75]
\tikzstyle{left-hand-mirror} = [
    draw,
    postaction=decorate, 
    decoration={
        markings,
        mark=between positions 0.015 and 0.98 step 0.1072 with {\draw (0,0)--(60:3pt);}
    }
]  
\coordinate (O) at (0,0);
\coordinate (X) at (-5,-5);
\coordinate (Y) at (10,0);
\coordinate (Z) at (0,10);
\draw [-{Latex[length=3mm]}] (O) -- (X);
\draw [-{Latex[length=3mm]}] (O) -- (Y);
\draw [-{Latex[length=3mm]}] (O) -- (Z);
\node[label=north west:$z_1$] at (X) {};
\node[label=north east:$z_2$] at (Y) {};
\node[label=north west:$z_3$] at (Z) {};
\coordinate (P) at (5,3);
\draw [-{Latex[length=3mm]},ultra thick] (O) -- (P);
\node[label=north west:$P$] at (P) {};
\coordinate (F) at (8,7) {};
\draw [-{Latex[length=3mm]}, ultra thick] (P) -- (F);
\node[above right] at (F) {$\overrightarrow{F}$};
\coordinate (Fp) at (8-5,7-3) {};
\draw [-{Latex[length=3mm]},dashdotted,ultra thick] (O) -- (Fp);
\node[above right,] at (Fp) {$\overrightarrow{F^{'}}$};
\coordinate (Px) at (-2,-2) {};
\coordinate (Py) at (6.8,0) {};
\node[label=north west:$P_1$] at (Px) {};
\node[label=north west:$P_2$] at (Py) {};
\coordinate (Pp) at (5,-2) {};
\draw [dashed] (Pp) -- (P);
\draw [dashed] (Pp) -- (Px);
\draw [dashed] (Pp) -- (Py);
\coordinate (Fpp) at (3,-1) {};
\coordinate (Fx) at (-1,-1) {} {};
\coordinate (Fy) at (4,0) {} {};
\node[label=north west:$F_1$] at (Fx) {};
\node[label=north west:$F_2$] at (Fy) {};
\draw[fill = black]  (Fx) circle (0.1);
\draw[fill = black]  (Fy) circle (0.1);
\draw [dashed] (Fp) -- (Fpp);
\draw [dashed] (Fpp) -- (Fx);
\draw [dashed] (Fpp) -- (Fy);

\coordinate (Fppp) at (6,-1) {} {};
\coordinate (Pppp) at (2,-2) {} {} {};
\node[{anchor=north west }] at (Fppp) {$\overrightarrow{F_1}$};
\node[{anchor=south east }] at (Pppp) {$\overrightarrow{F_2}$};
\draw [-{Latex[length=3mm]}, ultra thick] (O) -- (Px);
\draw [-{Latex[length=3mm]},ultra thick] (O) -- (Py);
\draw [-{Latex[length=3mm]}, ultra thick] (Px) -- (Pppp);
\draw [-{Latex[length=3mm]},ultra thick] (Py) -- (Fppp);
%\node[label=north west:$K$] at (Fppp) {};
%\node[label=north west:$S$] at (Pppp) {};
\draw [dashed] (Fp) -- (Fpp);
\draw [dashed] (Fppp) -- (Fx);
\draw [dashed] (Pppp) -- (Fy);
%\filldraw[ultra thick, gray!10] (Px) -- (Pppp) -- (Fy) -- (O) -- (Px) -- cycle;
%\filldraw[ultra thick,gray!20] (Fx) -- (Fppp) -- (Py) -- (O) -- (Px) -- cycle;
%\draw[ultra thick, gray!80] (Px) -- (Pppp) -- (Fy) -- (O) -- (Px) -- cycle;
%\draw[ultra thick,gray!80] (Fx) -- (Fppp) -- (Py) -- (O) -- (Px) -- cycle;
\coordinate (Vp1) at (0,6) {} {};
\coordinate (Vp2) at (0,9) {} {} {};
\draw [-{Latex[length=3mm]}, ultra thick] (O) -- (Vp1);
\draw [-{Latex[length=3mm]}, ultra thick] (O) -- (Vp2);
\node[{anchor=north west }] at (Vp1) {$\overrightarrow{P_1}\times\overrightarrow{F_2}$};
\node[{anchor=north west }] at (Vp2) {-$\overrightarrow{P_2}\times\overrightarrow{F_1}$};
\draw[fill=white]  (Py) circle (0.1);
\draw[fill=white]  (Px) circle (0.1);
\draw[fill=white]  (P) circle (0.1);
\draw[decoration={markings, mark=at position 0.1 with {\arrow[scale = 1.5]{latex[]}}},
    postaction={decorate}](0,4.3) ellipse (1 and 0.2);
    \draw[decoration={markings, mark=at position 0.1 with {\arrow[scale = 1.5]{latex[reversed]}}},
    postaction={decorate}](0,7.3) ellipse (1 and 0.2);
 \coordinate (Qx) at (3.9,1.7) {} {};
\coordinate (Qy) at (8,3) {} {} {};
\draw [-{Latex[length=3mm]},dotted, ultra thick] (P) -- (Qx);
\draw [-{Latex[length=3mm]},dotted, ultra thick] (P) -- (Qy);
\node[{anchor=north west }] at (Qx) {$\overrightarrow{F_1}$};
\node[{anchor=north west }] at (Qy) {$\overrightarrow{F_2}$};
\end{tikzpicture}
\caption{Interpretation of the tensor moment $M_{12}$}
\label{fig:fig_p156_5320}
\end{figure}
Let's consider a mass point $P$ on which a force $\overrightarrow{F}$ is acting. The force has components $\left(F_x,F_y,F_z\right)$ in the  space $V^{'}_3$ (which is by the way not the space $V_3$ of the considered mass point).\\
Let's investigate the element $M_{12}$ of the \textit{tensor moment}.\\
$P_1F_2\overrightarrow{e_3}$ is the vector product $\overrightarrow{P_1}\times\overrightarrow{F_2}$ and is as such the torque of the component $F_2$ of $\overrightarrow{F}$ acting on the mass point situated at $P_1$. The origin being fixed, $\overrightarrow{F_2}$ tries to move $P_1$, clockwise along the $z_3$ axis. The same is true for the component $\overrightarrow{F_1}$ acting on the mass point situated at $P_2$, and is represented here by the vector $- \overrightarrow{P_2}\times\overrightarrow{F_1}$ ($\overrightarrow{F_1}$ tries to move  $P_2$, counter clockwise along the $z_3$ axis). \\
Hence, $P_1F_2-P_2F_1$ is the net force trying to move the point $P$ along the $z_3$ axis (i.e. in the plane $\parallel$ with the $z_3=0$ plane).
$$\blacklozenge$$
\newpage


\section{p156 - Clarification}
\begin{tcolorbox}
$$\mathbf{5.234.}\spatie \dv{h_r}{t}= M_r$$
\end{tcolorbox}
\begin{align}
h_r &= m\epsilon_{rmn}z_mv_n\\
\Rightarrow \spatie \dv{h_r}{t} &= m\epsilon_{rmn}\dv{z_m}{t} v_n+m\epsilon_{rmn}z_m\dv{v_n}{t}\\
&= m\underbrace{\epsilon_{rmn}v_m v_n}_{=0}+\underbrace{\epsilon_{rmn}z_mF_n}_{=M_r}\\
&=M_r
\end{align}
$$\blacklozenge$$
\newpage



\section{p158-159 - Clarification}
\begin{tcolorbox}
$$\mathbf{5.313.}\spatie \omega_{rs}= -\omega_{sr}$$ From 5.310 and the vector character of $v_r$ and $z_r$ (for transformations which do not change the origin), \textbf{it follows that $\omega_{rs} $ is a Cartesian tensor of second order}.
\end{tcolorbox}
Be 
\begin{align}
v^{}_r = -\omega^{}_{rn}z^{}_n
\end{align}
Considering orthogonal transformation in a flat space $z^{'}_m = A_{mr}z^{}_r+B_m$ with  $B_m=0$ as we consider only transformations which do not change the origin. Differentiation with the parameter $t$ gives 
\begin{align}
v^{'}_m &= A_{mr}v^{}_r\\
&= -\omega^{}_{rn}A_{mr}z^{}_n\\
\end{align}
But $z^{'}_q = A_{qr}z^{}_r\quad\Rightarrow \quad A_{qn}z^{'}_q = A_{qn}A_{qr}z^{}_r\quad\Rightarrow \quad A_{qn}z^{'}_q = z^{}_n$ 
Hence
\begin{align}
v^{'}_m &= -\omega^{}_{rn}A_{mr}z^{}_n\\
&= -\underbrace{\omega^{}_{rn}A_{mr}A_{qn}}_{\overset{\underset{\mathrm{def}}{}}{=}\omega_{mq}^{'}}z^{'}_q\\
v^{'}_m &= -\omega_{mq}^{'}z^{'}_q
\end{align}
$$\blacklozenge$$
\newpage


\section{p159 - Exercise}
\begin{tcolorbox}
Show that if a rigid body rotates about the point $z_r=b_r$ as fixed point, the velociy of a general point of the body is given by $$v_r=-\omega_{rm}\left(z_m-b_m\right)$$
\end{tcolorbox}
By $\mathbf{5.302. }$:
\begin{align}
\left(z^{(1)}_m-z^{(2)}_m\right)\left(dz^{(1)}_m-dz^{(2)}_m\right)=0
\end{align}
At the fixed point we have $z^{(2)}_m=b_m$ and $dz^{(2)}_m=0$, hence
\begin{align}
\left(z^{(1)}_m-b_m\right)\left(dz^{(1)}_m\right)=0\\
\Rightarrow\spatie z^{(1)}_mdz^{(1)}_m =b_mdz^{(1)}_m
\end{align}
As this is true for any point of the rigid mass, expanding (1) and using (3) we get when dividing by $dt$
\begin{align}
\left(z^{(2)}_m-b_m\right)v^{(1)}_m+\left(z^{(1)}_m-b_m\right)v^{(2)}_m=0
\end{align}
Taking twice the partial derivative $\frac{\partial^2}{\partial z^{(1)}_p\partial z^{(1)}_q}$ we get
\begin{align}
\left(z^{(2)}_m-b_m\right)\frac{\partial^2 v_m}{\partial z^{(1)}_p\partial z^{(1)}_q}=0
\end{align}
As this is true for any arbitrary point in the rigid body we get
\begin{align}
\frac{\partial^2 v_m}{\partial z^{(1)}_p\partial z^{(1)}_q}=0\\
\Rightarrow\spatie v_m= K_{mr}z_r+B_m
\end{align}
At the fixed point we have 
\begin{align}
 K_{mr}b_r+B_m =0
\end{align}
Plugging this in (7)
\begin{align}
 v_m= K_{mr}\left(z_r-b_m\right)
\end{align}
Putting $K_{mr}=-\omega_{mr}$ gives us indeed the asked expression.
$$\blacklozenge$$
\newpage


\section{p161 - Clarification}
\begin{tcolorbox}
$$\mathbf{5.325.}\spatie \Omega_{np}\sum \left(mf_nz_p\right)= \Omega_{np}\sum F_nz_p $$
and hence, since $\Omega_{np}$ is arbitrary,
$$\mathbf{5.326.}\spatie \sum m\left(f_nz_p-f_pz_n\right)= \sum \left(F_nz_p-F_pz_n \right)$$
\end{tcolorbox}
To be complete the following step should be inserted

\begin{align}
\Omega_{np}\sum \left(mf_nz_p\right)&= \Omega_{np}\sum F_nz_p \\
\text{As }\Omega_{np}\text{ is skew-symmetric:}\spatie -\Omega_{np}\sum \left(mf_pz_n\right)&= -\Omega_{np}\sum F_pz_n \\
\text{(1)+(2) }\spatie \Omega_{np}\sum m\left(f_nz_p-f_pz_n\right)&= \Omega_{np}\sum\left( F_nz_p -F_pz_n\right)
\end{align}
and hence, since $\Omega_{np}$ is arbitrary,
$$\mathbf{5.326.}\spatie \sum m\left(f_nz_p-f_pz_n\right)= \sum \left(F_nz_p-F_pz_n \right)$$
$$\blacklozenge$$
\newpage

\section{p161 - Clarification}
\begin{tcolorbox}
$$\mathbf{5.329.}\spatie h_{np}=\sum m\left(\omega_{nq} z_q z_p -\omega_{pq} z_qz_n\right)$$
$$= J_{npqr}\omega_{rq}$$
where
$$\mathbf{5.330.}\spatie J_{npqr}= \sum m\left(\delta_{nr}z_qz_p-\delta_{pr}z_nz_q \right)$$
\end{tcolorbox}
\begin{align}
h_{np}&=\sum m\left(\omega_{nq} z_q z_p -\omega_{pq} z_qz_n\right)\\
&=\sum m\left(\omega_{rq}\delta_{rn} z_q z_p -\omega_{rq} \delta_{rp}z_qz_n\right)\\
&=\omega_{rq}\sum m\left(\delta_{rn} z_q z_p -\delta_{rp}z_qz_n\right)\\
&=J_{npqr}\omega_{rq}
\end{align}
$$\blacklozenge$$
\newpage