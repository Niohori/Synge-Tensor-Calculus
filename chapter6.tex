\setcounter{chapter}{4}
\chapter{Applications to Hydrodynamics, Elasticity, and Electromagnetic radiation}
\pagebreak[4]
\section{p191 - Exercise}
\begin{tcolorbox}
A fluid rotates as a rigid body about the axis of $z_3$ with variable angular velocity $\omega(t)$. Write out explicitly the three Lagrangian equations $\mathbf{6.101}$ and the three Eulerian equations $\mathbf{6.103}$.
\end{tcolorbox}
The motion described reduces to a motion in a $V_2$ plane with $z_3$ a constant for a definite particle.\\\\\\
\textbf{Lagrangian}\\

A particular particle with starting coordinates $\left(z^{(*)}_1,z^{(*)}_2, z^{(*)}_3\right)$ will describe a  circle with radius $\sqrt{z^{(*)2}_1+z^{(*)2}_2}$ in the  plane $V_2$ parallel with the $1,2$ axes.
Taking axis $1$ as reference to determine the instantaneous angle $\theta$ of the vertex $OP$ (origin and particle) we get
\begin{align}
\left\{\begin{array}{l}
z_1 = \sqrt{z^{(*)2}_1+z^{(*)2}_2}\cos\left(\theta(t) + \phi_0\right)\\
z_2 = \sqrt{z^{(*)2}_1+z^{(*)2}_2}\sin\left(\theta(t) + \phi_0\right)\\
z_3 = z^{(*)}_3
\end{array}\right.
\end{align}
with 
\begin{align}
\phi_0 & = \arctan\frac{z^{(*)}_2}{z^{(*)}_1}
\end{align}
Note that $\omega(t)$ is not a constant, so
\begin{align}
\theta(t)&= \int_{0}^{t}\omega(\tau)d\tau
\end{align}
and get 
\begin{align}
\left\{\begin{array}{l}
z_1 = \sqrt{z^{(*)2}_1+z^{(*)2}_2}\cos\left(\int_{0}^{t}\omega(\tau)d\tau + \phi_0\right)\\\\
z_2 = \sqrt{z^{(*)2}_1+z^{(*)2}_2}\sin\left(\int_{0}^{t}\omega(\tau)d\tau + \phi_0\right)\\\\
z_3 = z^{(*)}_3\\\\
\phi_0 = \arctan\frac{z^{(*)}_2}{z^{(*)}_1}
\end{array}\right.
\end{align}
$$\lozenge$$\\
\newpage
\textbf{Eulerian}\\
The equations get simplified and reduce to a motion of a particle on a circle.
\begin{figure}[H]
\centering
\tikzstyle{left-hand-mirror} = [
    draw,
    postaction=decorate, 
    decoration={
        markings,
        mark=between positions 0.015 and 0.98 step 0.1072 with {\draw (0,0)--(60:3pt);}
    }
]   

\begin{tikzpicture}[scale=0.5]
\coordinate (O1) at (0,0);
\draw  (O1) circle (6);
\coordinate (Om) at (0.0,-6) ;
\coordinate  {} {} {};
\coordinate  {} {} {};
\coordinate (NPole) at (0,8.5) {} {};
\coordinate (SPole) at (0,-7.5) {} {};
\draw[-{Latex[length=2mm]}] (SPole) -- (NPole);
\coordinate (P) at (4.2426,4.2426) {} {};
\draw[fill=white]  (P) circle (0.1);
\draw[fill=white]  (O1) circle (0.05);
\draw[dashed] (O1) -- (P);
%\node[label=west:$O$] at (O1) {};
\node[label=east:$P\left(x \text{, } y\right)$] at (P) {};
\coordinate (O2) at (1.5,0) {};
\node[label=north east:$\theta\text{=} \arctan\frac{y}{x}$] at (O2) {};
\draw [dashed, decoration={markings, mark=at position 0.7 with {\arrow[scale = 1.]{Latex[length=2mm]}}},    postaction={decorate}](O2) arc (0:30:2 and 2.0);
\coordinate (F) at (1.5,7) {} {} {};
\draw [-{Latex[length=2mm]}, ] (P) -- (F);
\node[label=east:$\mathbf{\overline{v}}$] at (F) {};
\coordinate (Fp) at (-3,3) {} {} {} {};
\draw [-{Latex[length=2mm]},dashed] (O1) -- (Fp);
\node[label=west:$\mathbf{\overline{v}}$] at (Fp) {};
\tkzMarkRightAngle[size=0.43](F,P,O1);
\tkzMarkRightAngle[size=0.43, ](P,O1,Fp);
\coordinate (x1)at (-7,0) {};
\coordinate (X) at (8.5,0) {} {};
\draw [-{Latex[length=2mm]}] (x1) -- (X);
\node[label=south east:$x$] at (X) {};
\node[label=south east:$y$] at (NPole) {};
\coordinate (xx)at (-3,0) {};
\draw[fill=white]  (xx) circle (0.1);
\coordinate (yy)at (0,3) {};
\draw[fill=white]  (yy) circle (0.1);
\draw [dashed, ] (xx) -- (Fp);
\draw [dashed ] (yy) -- (Fp);
\node[label=south :$-sin\left(\theta\right)$] at (xx) {};
\node[label= east:$cos\left(\theta\right)$] at (yy) {};
\end{tikzpicture}
\caption{Eulerian viewpoint of a spinning fluid}
\label{fig:fig_p191_Exa}
\end{figure}

\begin{align}
\left\{\begin{array}{l}
v_1 = \sqrt{z^{2}_1+z^{2}_2}\omega(t)\sin\left(\arctan\frac{z^{}_2}{z^{}_1}\right)\\\\
v_2 = \sqrt{z^{2}_1+z^{2}_2}\omega(t)\cos\left(\arctan\frac{z^{}_2}{z^{}_1}\right)\\\\
v_3 = 0
\end{array}\right.
\end{align}
$$\blacklozenge$$
\newpage

\section{p191 - Exercise}
\begin{tcolorbox}
Compute the components of acceleration for the motion described in the preceding exercise.
\end{tcolorbox}
We have
\begin{align}
\left\{\begin{array}{l}
v_1 = \sqrt{z^{2}_1+z^{2}_2}\omega(t)\sin\left(\arctan\frac{z^{}_2}{z^{}_1}\right)\\\\
v_2 = \sqrt{z^{2}_1+z^{2}_2}\omega(t)\cos\left(\arctan\frac{z^{}_2}{z^{}_1}\right)\\\\
v_3 = 0
\end{array}\right.
\end{align}
and 
\begin{align}
f_r &= \partial_tv_r + v_{r,s}v_s
\end{align}
\begin{align}
&\left\{\begin{array}{l}
\partial_t v_1 = \sqrt{z^{2}_1+z^{2}_2}\dot{\omega}(t)\sin\left(\arctan\frac{z^{}_2}{z^{}_1}\right)\\\\
\partial_t v_2 = \sqrt{z^{2}_1+z^{2}_2}\dot{\omega}(t)\cos\left(\arctan\frac{z^{}_2}{z^{}_1}\right)\\\\
v_{1,1}= \omega(t)\left[\frac{z_1}{\sqrt{z^{2}_1+z^{2}_2}}\sin\left(\arctan\frac{z^{}_2}{z^{}_1}\right) -\sqrt{z^{2}_1+z^{2}_2}\cos\left(\arctan\frac{z^{}_2}{z^{}_1}\right)\frac{z_2}{z_1^2}\frac{1}{1+\frac{z_2^2}{z_1^2}}\right]\\\\
v_{1,2}= \omega(t)\left[\frac{z_2}{\sqrt{z^{2}_1+z^{2}_2}}\sin\left(\arctan\frac{z^{}_2}{z^{}_1}\right) +\sqrt{z^{2}_1+z^{2}_2}\cos\left(\arctan\frac{z^{}_2}{z^{}_1}\right)\frac{1}{z^{}_1}\frac{1}{1+\frac{z_2^2}{z_1^2}}\right]\\\\
v_{2,1}= \omega(t)\left[\frac{z_1}{\sqrt{z^{2}_1+z^{2}_2}}\cos\left(\arctan\frac{z^{}_2}{z^{}_1}\right) +\sqrt{z^{2}_1+z^{2}_2}\sin\left(\arctan\frac{z^{}_2}{z^{}_1}\right)\frac{z_2}{z_1^2}\frac{1}{1+\frac{z_2^2}{z_1^2}}\right]\\\\
v_{2,2}= \omega(t)\left[\frac{z_2}{\sqrt{z^{2}_1+z^{2}_2}}\cos\left(\arctan\frac{z^{}_2}{z^{}_1}\right) -\sqrt{z^{2}_1+z^{2}_2}\sin\left(\arctan\frac{z^{}_2}{z^{}_1}\right)\frac{1}{z^{}_1}\frac{1}{1+\frac{z_2^2}{z_1^2}}\right]\\\\
\end{array}\right.
\end{align}
\begin{align}
&=\left\{\begin{array}{l}
\partial_t v_1 = \sqrt{z^{2}_1+z^{2}_2}\dot{\omega}(t)\sin\left(\arctan\frac{z^{}_2}{z^{}_1}\right)\\\\
\partial_t v_2 = \sqrt{z^{2}_1+z^{2}_2}\dot{\omega}(t)\cos\left(\arctan\frac{z^{}_2}{z^{}_1}\right)\\\\
v_{1,1}= \frac{\omega(t)}{\sqrt{z^{2}_1+z^{2}_2}}\left[z_1\sin\left(\arctan\frac{z^{}_2}{z^{}_1}\right) -z_2\cos\left(\arctan\frac{z^{}_2}{z^{}_1}\right)\right]\\\\
v_{1,2}=  \frac{\omega(t)}{\sqrt{z^{2}_1+z^{2}_2}}\left[z_2\sin\left(\arctan\frac{z^{}_2}{z^{}_1}\right) +z^{}_1\cos\left(\arctan\frac{z^{}_2}{z^{}_1}\right)\right]\\\\
v_{2,1}=  \frac{\omega(t)}{\sqrt{z^{2}_1+z^{2}_2}}\left[z_1\cos\left(\arctan\frac{z^{}_2}{z^{}_1}\right) +z_2\sin\left(\arctan\frac{z^{}_2}{z^{}_1}\right)\right]\\\\
v_{2,2}=  \frac{\omega(t)}{\sqrt{z^{2}_1+z^{2}_2}}\left[z_2\cos\left(\arctan\frac{z^{}_2}{z^{}_1}\right) -z^{}_1\sin\left(\arctan\frac{z^{}_2}{z^{}_1}\right)\right]\\\\
\end{array}\right.
\end{align}
and get
\begin{align}
v_{1,s}v_s &= \left\{\begin{array}{l}
\sqrt{z^{2}_1+z^{2}_2}\omega(t)\sin\left(\arctan\frac{z^{}_2}{z^{}_1}\right)\frac{\omega(t)}{\sqrt{z^{2}_1+z^{2}_2}}\left[z_1\sin\left(\arctan\frac{z^{}_2}{z^{}_1}\right) -z_2\cos\left(\arctan\frac{z^{}_2}{z^{}_1}\right)\right]\\\\
+\sqrt{z^{2}_1+z^{2}_2}\omega(t)\cos\left(\arctan\frac{z^{}_2}{z^{}_1}\right)\frac{\omega(t)}{\sqrt{z^{2}_1+z^{2}_2}}\left[z_2\sin\left(\arctan\frac{z^{}_2}{z^{}_1}\right) +z^{}_1\cos\left(\arctan\frac{z^{}_2}{z^{}_1}\right)\right]
\end{array}\right.\\
&=z_1\omega^2(t)\\
\end{align}
\begin{align}
v_{2,s}v_s &= \left\{\begin{array}{l}
\sqrt{z^{2}_1+z^{2}_2}\omega(t)\sin\left(\arctan\frac{z^{}_2}{z^{}_1}\right)\frac{\omega(t)}{\sqrt{z^{2}_1+z^{2}_2}}\left[z_1\cos\left(\arctan\frac{z^{}_2}{z^{}_1}\right) +z_2\sin\left(\arctan\frac{z^{}_2}{z^{}_1}\right)\right]\\\\
+\sqrt{z^{2}_1+z^{2}_2}\omega(t)\cos\left(\arctan\frac{z^{}_2}{z^{}_1}\right)\frac{\omega(t)}{\sqrt{z^{2}_1+z^{2}_2}}\left[z_2\cos\left(\arctan\frac{z^{}_2}{z^{}_1}\right) -z^{}_1\sin\left(\arctan\frac{z^{}_2}{z^{}_1}\right)\right]\\\\
\end{array}\right.\\
&= z_2\omega^2(t)\\
\end{align}
giving with the second derivative term
\begin{align}
&=\left\{\begin{array}{l}
f_1 = \dot{\omega}(t)\sqrt{z^{2}_1+z^{2}_2}\sin\left(\arctan\frac{z^{}_2}{z^{}_1}\right)+z_1\omega^2(t)\\\\\\
f_2 = \dot{\omega}(t)\sqrt{z^{2}_1+z^{2}_2}\cos\left(\arctan\frac{z^{}_2}{z^{}_1}\right)+z_2\omega^2(t)\\\\
f_3=0
\end{array}\right.
\end{align}
$$\blacklozenge$$
\newpage
