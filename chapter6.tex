\setcounter{chapter}{4}
\chapter{Applications to Hydrodynamics, Elasticity, and Electromagnetic radiation}
\pagebreak[4]
\section{p191 - Exercise}
\begin{tcolorbox}
A fluid rotates as a rigid body about the axis of $z_3$ with variable angular velocity $\omega(t)$. Write out explicitly the three Lagrangian equations $\mathbf{6.101}$ and the three Eulerian equations $\mathbf{6.103}$.
\end{tcolorbox}
The motion described reduces to a motion in a $V_2$ plane with $z_3$ a constant for a definite particle.\\\\\\
\textbf{Lagrangian}\\

A particular particle with starting coordinates $\left(z^{(*)}_1,z^{(*)}_2, z^{(*)}_3\right)$ will describe a  circle with radius $\sqrt{z^{(*)2}_1+z^{(*)2}_2}$ in the  plane $V_2$ parallel with the $1,2$ axes.
Taking axis $1$ as reference to determine the instantaneous angle $\theta$ of the vertex $OP$ (origin and particle) we get
\begin{align}
\left\{\begin{array}{l}
z_1 = \sqrt{z^{(*)2}_1+z^{(*)2}_2}\cos\left(\theta(t) + \phi_0\right)\\
z_2 = \sqrt{z^{(*)2}_1+z^{(*)2}_2}\sin\left(\theta(t) + \phi_0\right)\\
z_3 = z^{(*)}_3
\end{array}\right.
\end{align}
with 
\begin{align}
\phi_0 & = \arctan\frac{z^{(*)}_2}{z^{(*)}_1}
\end{align}
Note that $\omega(t)$ is not a constant, so
\begin{align}
\theta(t)&= \int_{0}^{t}\omega(\tau)d\tau
\end{align}
and get 
\begin{align}
\left\{\begin{array}{l}
z_1 = \sqrt{z^{(*)2}_1+z^{(*)2}_2}\cos\left(\int_{0}^{t}\omega(\tau)d\tau + \phi_0\right)\\\\
z_2 = \sqrt{z^{(*)2}_1+z^{(*)2}_2}\sin\left(\int_{0}^{t}\omega(\tau)d\tau + \phi_0\right)\\\\
z_3 = z^{(*)}_3\\\\
\phi_0 = \arctan\frac{z^{(*)}_2}{z^{(*)}_1}
\end{array}\right.
\end{align}
$$\lozenge$$\\
\newpage
\textbf{Eulerian}\\
The equations get simplified and reduce to a motion of a particle on a circle.
\begin{figure}[H]
\centering
\tikzstyle{left-hand-mirror} = [
    draw,
    postaction=decorate, 
    decoration={
        markings,
        mark=between positions 0.015 and 0.98 step 0.1072 with {\draw (0,0)--(60:3pt);}
    }
]   

\begin{tikzpicture}[scale=0.5]
\coordinate (O1) at (0,0);
\draw  (O1) circle (6);
\coordinate (Om) at (0.0,-6) ;
\coordinate  {} {} {};
\coordinate  {} {} {};
\coordinate (NPole) at (0,8.5) {} {};
\coordinate (SPole) at (0,-7.5) {} {};
\draw[-{Latex[length=2mm]}] (SPole) -- (NPole);
\coordinate (P) at (4.2426,4.2426) {} {};
\draw[fill=white]  (P) circle (0.1);
\draw[fill=white]  (O1) circle (0.05);
\draw[dashed] (O1) -- (P);
%\node[label=west:$O$] at (O1) {};
\node[label=east:$P\left(x \text{, } y\right)$] at (P) {};
\coordinate (O2) at (1.5,0) {};
\node[label=north east:$\theta\text{=} \arctan\frac{y}{x}$] at (O2) {};
\draw [dashed, decoration={markings, mark=at position 0.7 with {\arrow[scale = 1.]{Latex[length=2mm]}}},    postaction={decorate}](O2) arc (0:30:2 and 2.0);
\coordinate (F) at (1.5,7) {} {} {};
\draw [-{Latex[length=2mm]}, ] (P) -- (F);
\node[label=east:$\mathbf{\overline{v}}$] at (F) {};
\coordinate (Fp) at (-3,3) {} {} {} {};
\draw [-{Latex[length=2mm]},dashed] (O1) -- (Fp);
\node[label=west:$\mathbf{\overline{v}}$] at (Fp) {};
\tkzMarkRightAngle[size=0.43](F,P,O1);
\tkzMarkRightAngle[size=0.43, ](P,O1,Fp);
\coordinate (x1)at (-7,0) {};
\coordinate (X) at (8.5,0) {} {};
\draw [-{Latex[length=2mm]}] (x1) -- (X);
\node[label=south east:$x$] at (X) {};
\node[label=south east:$y$] at (NPole) {};
\coordinate (xx)at (-3,0) {};
\draw[fill=white]  (xx) circle (0.1);
\coordinate (yy)at (0,3) {};
\draw[fill=white]  (yy) circle (0.1);
\draw [dashed, ] (xx) -- (Fp);
\draw [dashed ] (yy) -- (Fp);
\node[label=south :$-sin\left(\theta\right)$] at (xx) {};
\node[label= east:$cos\left(\theta\right)$] at (yy) {};
\end{tikzpicture}
\caption{Eulerian viewpoint of a spinning fluid}
\label{fig:fig_p191_Exa}
\end{figure}

\begin{align}
\left\{\begin{array}{l}
v_1 = -\sqrt{z^{2}_1+z^{2}_2}\omega(t)\sin\left(\arctan\frac{z^{}_2}{z^{}_1}\right)\\\\
v_2 = \sqrt{z^{2}_1+z^{2}_2}\omega(t)\cos\left(\arctan\frac{z^{}_2}{z^{}_1}\right)\\\\
v_3 = 0
\end{array}\right.
\end{align}
$$\blacklozenge$$
\newpage

\section{p191 - Exercise}
\begin{tcolorbox}
Compute the components of acceleration for the motion described in the preceding exercise.
\end{tcolorbox}
We have
\begin{align}
\left\{\begin{array}{l}
v_1 = -\sqrt{z^{2}_1+z^{2}_2}\omega(t)\sin\left(\arctan\frac{z^{}_2}{z^{}_1}\right)\\\\
v_2 = \sqrt{z^{2}_1+z^{2}_2}\omega(t)\cos\left(\arctan\frac{z^{}_2}{z^{}_1}\right)\\\\
v_3 = 0
\end{array}\right.
\end{align}
and 
\begin{align}
f_r &= \partial_tv_r + v_{r,s}v_s
\end{align}
\begin{align}
&\left\{\begin{array}{l}
\partial_t v_1 = -\sqrt{z^{2}_1+z^{2}_2}\dot{\omega}(t)\sin\left(\arctan\frac{z^{}_2}{z^{}_1}\right)\\\\
\partial_t v_2 = \sqrt{z^{2}_1+z^{2}_2}\dot{\omega}(t)\cos\left(\arctan\frac{z^{}_2}{z^{}_1}\right)\\\\
v_{1,1}= -\omega(t)\left[\frac{z_1}{\sqrt{z^{2}_1+z^{2}_2}}\sin\left(\arctan\frac{z^{}_2}{z^{}_1}\right) -\sqrt{z^{2}_1+z^{2}_2}\cos\left(\arctan\frac{z^{}_2}{z^{}_1}\right)\frac{z_2}{z_1^2}\frac{1}{1+\frac{z_2^2}{z_1^2}}\right]\\\\
v_{1,2}= -\omega(t)\left[\frac{z_2}{\sqrt{z^{2}_1+z^{2}_2}}\sin\left(\arctan\frac{z^{}_2}{z^{}_1}\right) +\sqrt{z^{2}_1+z^{2}_2}\cos\left(\arctan\frac{z^{}_2}{z^{}_1}\right)\frac{1}{z^{}_1}\frac{1}{1+\frac{z_2^2}{z_1^2}}\right]\\\\
v_{2,1}= \omega(t)\left[\frac{z_1}{\sqrt{z^{2}_1+z^{2}_2}}\cos\left(\arctan\frac{z^{}_2}{z^{}_1}\right) +\sqrt{z^{2}_1+z^{2}_2}\sin\left(\arctan\frac{z^{}_2}{z^{}_1}\right)\frac{z_2}{z_1^2}\frac{1}{1+\frac{z_2^2}{z_1^2}}\right]\\\\
v_{2,2}= \omega(t)\left[\frac{z_2}{\sqrt{z^{2}_1+z^{2}_2}}\cos\left(\arctan\frac{z^{}_2}{z^{}_1}\right) -\sqrt{z^{2}_1+z^{2}_2}\sin\left(\arctan\frac{z^{}_2}{z^{}_1}\right)\frac{1}{z^{}_1}\frac{1}{1+\frac{z_2^2}{z_1^2}}\right]\\\\
\end{array}\right.
\end{align}
\begin{align}
&=\left\{\begin{array}{l}
\partial_t v_1 = -\sqrt{z^{2}_1+z^{2}_2}\dot{\omega}(t)\sin\left(\arctan\frac{z^{}_2}{z^{}_1}\right)\\\\
\partial_t v_2 = \sqrt{z^{2}_1+z^{2}_2}\dot{\omega}(t)\cos\left(\arctan\frac{z^{}_2}{z^{}_1}\right)\\\\
v_{1,1}= -\frac{\omega(t)}{\sqrt{z^{2}_1+z^{2}_2}}\left[z_1\sin\left(\arctan\frac{z^{}_2}{z^{}_1}\right) -z_2\cos\left(\arctan\frac{z^{}_2}{z^{}_1}\right)\right]\\\\
v_{1,2}=  -\frac{\omega(t)}{\sqrt{z^{2}_1+z^{2}_2}}\left[z_2\sin\left(\arctan\frac{z^{}_2}{z^{}_1}\right) +z^{}_1\cos\left(\arctan\frac{z^{}_2}{z^{}_1}\right)\right]\\\\
v_{2,1}=  \frac{\omega(t)}{\sqrt{z^{2}_1+z^{2}_2}}\left[z_1\cos\left(\arctan\frac{z^{}_2}{z^{}_1}\right) +z_2\sin\left(\arctan\frac{z^{}_2}{z^{}_1}\right)\right]\\\\
v_{2,2}=  \frac{\omega(t)}{\sqrt{z^{2}_1+z^{2}_2}}\left[z_2\cos\left(\arctan\frac{z^{}_2}{z^{}_1}\right) -z^{}_1\sin\left(\arctan\frac{z^{}_2}{z^{}_1}\right)\right]\\\\
\end{array}\right.
\end{align}
and get
\begin{align}
v_{1,s}v_s &= \left\{\begin{array}{l}
-\sqrt{z^{2}_1+z^{2}_2}\omega(t)\sin\left(\arctan\frac{z^{}_2}{z^{}_1}\right)\frac{\omega(t)}{\sqrt{z^{2}_1+z^{2}_2}}\left[z_1\sin\left(\arctan\frac{z^{}_2}{z^{}_1}\right) -z_2\cos\left(\arctan\frac{z^{}_2}{z^{}_1}\right)\right]\\\\
-\sqrt{z^{2}_1+z^{2}_2}\omega(t)\cos\left(\arctan\frac{z^{}_2}{z^{}_1}\right)\frac{\omega(t)}{\sqrt{z^{2}_1+z^{2}_2}}\left[z_2\sin\left(\arctan\frac{z^{}_2}{z^{}_1}\right) +z^{}_1\cos\left(\arctan\frac{z^{}_2}{z^{}_1}\right)\right]
\end{array}\right.\\
&=-z_1\omega^2(t)
\end{align}
\begin{align}
v_{2,s}v_s &= \left\{\begin{array}{l}
\sqrt{z^{2}_1+z^{2}_2}\omega(t)\sin\left(\arctan\frac{z^{}_2}{z^{}_1}\right)\frac{\omega(t)}{\sqrt{z^{2}_1+z^{2}_2}}\left[z_1\cos\left(\arctan\frac{z^{}_2}{z^{}_1}\right) +z_2\sin\left(\arctan\frac{z^{}_2}{z^{}_1}\right)\right]\\\\
+\sqrt{z^{2}_1+z^{2}_2}\omega(t)\cos\left(\arctan\frac{z^{}_2}{z^{}_1}\right)\frac{\omega(t)}{\sqrt{z^{2}_1+z^{2}_2}}\left[z_2\cos\left(\arctan\frac{z^{}_2}{z^{}_1}\right) -z^{}_1\sin\left(\arctan\frac{z^{}_2}{z^{}_1}\right)\right]\\\\
\end{array}\right.\\
&= z_2\omega^2(t)
\end{align}
giving with the second derivative term
\begin{align}
&=\left\{\begin{array}{l}
f_1 = -\dot{\omega}(t)\sqrt{z^{2}_1+z^{2}_2}\sin\left(\arctan\frac{z^{}_2}{z^{}_1}\right)-z_1\omega^2(t)\\\\\\
f_2 = \dot{\omega}(t)\sqrt{z^{2}_1+z^{2}_2}\cos\left(\arctan\frac{z^{}_2}{z^{}_1}\right)+z_2\omega^2(t)\\\\
f_3=0
\end{array}\right.
\end{align}
$$\blacklozenge$$
\newpage


\section{p193 - Exercise}
\begin{tcolorbox}
Verify that the operator $\frac{\partial}{\partial t}$ does not alter tensor character.
\end{tcolorbox}
Be $X^r$ and $Y^r$, two tensors so that $I=X_rY^r$ is an invariant. Obviously, $\frac{\partial I}{\partial t}$ will also be invariant and 
\begin{align}
\frac{\partial I}{\partial t}= \frac{\partial X_r}{\partial t}Y^r+X_r\frac{\partial Y^r}{\partial t}
\end{align}
Meaning that the right side is a sum of two invariants, from which we conclude (see page 20, $\mathbf{1.607}$) that $\frac{\partial X_r}{\partial t}$ and $\frac{\partial Y^r}{\partial t}$ are tensors.
$$\blacklozenge$$
\newpage

\section{p193 - Clarification to 6.112}
\begin{tcolorbox}
$$\mathbf{6.112.}\spatie \int Fn_rdS = \int F_{,r}dV$$
\end{tcolorbox}
Green's theorem is generally presented in the form
$$\int\overline{F}.\overline{n}dS = \int \overline{\nabla}. \overline{F}dV$$or
$$\int F_rn_rdS = \int F_{r,r}dV$$
We can define $$\overline{F} = F\overline{1}_r$$
 $\overline{F}.\overline{n}$  will then become $Fn_r$ while $\overline{\nabla}. \overline{F}$ will become $\partial_r F$, giving the expression $\mathbf{6.112.}$.
 $$\blacklozenge$$
\newpage



\section{p196 - Exercise}
\begin{tcolorbox}
Write out $\mathbf{6.126}$ and $\mathbf{6.127b}$ explicitly for spherical polar coordinates.
\end{tcolorbox}
For spherical polar coordinates we have 
\begin{align}
(v^r) = \left(\begin{matrix}\dot{r}\\r\dot{\theta}\\r\sin\theta\dot{\phi} \end{matrix}\right)
\end{align}
and (see $\mathbf{2.546}$ page 58):
\begin{align}
v^r_{|r} &= \frac{1}{r^2}\partial_r\left(r^2v^1\right)+\frac{1}{\sin\theta}\partial_{\theta}\left(\sin\theta v^2\right)+\partial_{\phi} v^3\\
&=  \frac{1}{r^2}\left(2rv^1+r^2\partial_rv^1\right)+\frac{1}{\sin\theta}\left(v^2\cos\theta + \sin\theta\partial_{\theta}v^2\right)\\
&= \frac{2}{r}\dot{r}+r\dot{\theta}\cot \theta 
\end{align}
and 
\begin{align}
&\partial_t \rho+ \left(\rho v^r\right)_{|r} =0\\
\Leftrightarrow\spatie &\partial_t\rho+ \rho_{|r}v^r+\rho v^r_{|r} =0\\
\Leftrightarrow\spatie &\dv{\rho}{t}+\rho \left(\frac{2}{r}\dot{r}+r\dot{\theta}\cot \theta \right) =0
\end{align}
 $$\blacklozenge$$
\newpage


\section{p198 - Exercise}
\begin{tcolorbox}
If $\epsilon^{rmn}$ is defined in precisely the same way as $\epsilon_{rmn}$, prove that $$ \epsilon^{'uvw}= J\epsilon^{rmn}\frac{\partial x^{'u}}{\partial x^r}\frac{\partial x^{'v}}{\partial x^m}\frac{\partial x^{'w}}{\partial x^n}$$
\end{tcolorbox}
We follow the pretty same line of reasoning as for $\epsilon_{rst}$. Going from $x^{'r}$ to $x^{s}$ we have (expanding the determinant of the inverse Jacobian along the rows instead of the columns):
\begin{align}
J^{-1}=\left|\frac{\partial x^{'p}}{\partial x^{q}}\right|&= \epsilon^{rmn}\frac{\partial x^{'1}}{\partial x^r}\frac{\partial x^{'2}}{\partial x^m}\frac{\partial x^{'3}}{\partial x^n}\\
\times \epsilon^{uvw}\spatie J^{-1}\epsilon^{uvw}&= \epsilon^{rmn}\frac{\partial x^{'u}}{\partial x^r}\frac{\partial x^{'v}}{\partial x^m}\frac{\partial x^{'w}}{\partial x^n}\\
\times J\spatie\epsilon^{uvw}&= J\epsilon^{rmn}\frac{\partial x^{'u}}{\partial x^r}\frac{\partial x^{'v}}{\partial x^m}\frac{\partial x^{'w}}{\partial x^n}\\ 
\end{align}
 $$\blacklozenge$$
\newpage



\section{p198 - Exercise}
\begin{tcolorbox}
Prove that  $\frac{\epsilon^{rmn}}{\sqrt{a}}$ is an (absolute) contravariant tensor of the third order.
\end{tcolorbox}
\begin{align} 
\sqrt{a^{'}} &= J\sqrt{a}\\
\epsilon^{'uvw}&= J\epsilon^{rmn}\frac{\partial x^{'u}}{\partial x^r}\frac{\partial x^{'v}}{\partial x^m}\frac{\partial x^{'w}}{\partial x^n}\\
\text{(1)  in (2)} \spatie \epsilon^{'uvw}&= \frac{\sqrt{a^{'}}}{\sqrt{a^{}}}\epsilon^{rmn}\frac{\partial x^{'u}}{\partial x^r}\frac{\partial x^{'v}}{\partial x^m}\frac{\partial x^{'w}}{\partial x^n}\\
\Rightarrow \spatie \frac{\epsilon^{'uvw}}{\sqrt{a^{'}}}&= \frac{\epsilon^{rmn}}{\sqrt{a}}\frac{\partial x^{'u}}{\partial x^r}\frac{\partial x^{'v}}{\partial x^m}\frac{\partial x^{'w}}{\partial x^n}
\end{align}
which is the required transformation rule for a "normal" (absolute) tensor.
 $$\blacklozenge$$
\newpage


\section{p199 - Clarification to pressure invariance to direction of the surface element.}
\begin{tcolorbox}
Pressure is independent of the direction
\end{tcolorbox}
\begin{figure}[H]%
    \centering
    \subfloat[]{\tdplotsetmaincoords{80}{160}
\begin{tikzpicture}[tdplot_main_coords, >=Latex]
\tikzmath{\aax=3.84;\bby=6;\ccz=4;\a = (\bby+\bby)/4;\d=-8.5;\e=4;};
\aax, \bby,\ccz;
\coordinate (O) at (0,0,0);
\coordinate (A) at (0,0,\ccz);
\coordinate (B) at (0,\bby,0);
\coordinate (C) at (\aax,0,0);
\coordinate (x) at (1.5*\aax,0,0);
\coordinate (y) at (0,1.5*\bby,0);
\coordinate (z) at (0,0,1.5*\ccz);

\coordinate (xy) at (\aax/3,\bby/3,0);
\draw [fill=white] (xy) circle (2pt) node[above right] (n1) {$$};
\coordinate (xz) at (\aax/3,0,\ccz/3);
\draw [fill=white] (xz) circle (2pt) node[above right] (n1) {$$};
\coordinate (yz) at (0,\bby/3,\ccz/3);
\draw [fill=white] (yz) circle (2pt) node[above right] (n1) {$$};
\coordinate (xyz) at (\aax/3,\bby/3,\ccz/3);
\draw [fill=gray] (xyz) circle (2pt) node[above right] (n1) {$$};

\coordinate (nxy) at (\aax/3,\bby/3,-2);
\coordinate (nxyO) at (\aax/3,\bby/3,-0.2);
\draw[dashed, thick](xy)--(nxyO);
\draw[-Latex,very thick](nxyO)--(nxy);
\coordinate (nxz) at (\aax/3,-5,\ccz/3);
\coordinate (nxzO) at (\aax/3,-2.5,\ccz/3);
\draw[dashed,very thick](xz)--(nxzO);
\draw[-Latex,very thick](nxzO)--(nxz);
\coordinate (nyz) at (-2,\bby/3,\ccz/3);
\coordinate (nyzO) at (-0.5,\bby/3,\ccz/3);
\draw[dashed,very thick](yz)--(nyzO);
\draw[-Latex,very thick](nyzO)--(nyz);

\coordinate (nxyz) at (3*\aax/3,3*\bby/3,3*\ccz/3);
\draw[-Latex, very thick](xyz)--(nxyz);
\node[anchor = south east] at (nxy){$dS_z$};
\node[anchor = south east] at (nxz){$dS_y$};
\node[anchor = south east] at (nyz){$dS_x$};
\node[anchor = south east] at (nxyz){$dS_t$};

\coordinate (yzp) at (3*\aax/3,\bby/3,\ccz/3);
\draw[-Latex,](xyz)--(yzp);
\node[anchor = south east] at (yzp){$$};

\coordinate (xyp) at (\aax/3,\bby/3,3.1*\ccz/3);
\draw[-Latex,](xyz)--(xyp);
\node[anchor = south east] at (xyp){$$};

\coordinate (xzp) at (\aax/3,3*\bby/3,\ccz/3);
\draw[-Latex, ](xyz)--(xzp);
\node[anchor = north west] at (xzp){$$};

\coordinate (pz) at (\aax/3,\bby/3,3.1*\ccz/3);
\draw[ dashed](nxyz)--(pz);
\draw [fill=white] (pz) circle (1pt)node[above left] (pz) {$\gamma_z$};

\coordinate (pxy) at (3.*\aax/3,3*\bby/3,\ccz/3);
\draw[ dashed](nxyz)--(pxy);
\draw[ dashed](xyz)--(pxy);
\draw [fill=white] (pxy) circle (1pt) {};

\coordinate (px) at (3*\aax/3,\bby/3,\ccz/3);
\draw[ dashed](pxy)--(px);
\draw [fill=white] (px) circle (1pt) node[above left] (px) {$\gamma_x$};
\coordinate (py) at (\aax/3,3*\bby/3,\ccz/3);
\draw[ dashed](pxy)--(py);
\draw [fill=white] (py) circle (1pt)node[above right] (py) {$\gamma_y$};


\draw[-Latex](O)--(x);
\node[anchor = south east] at (x){x};
\draw[-Latex](O)--(y);
\node[anchor = south east] at (y){y};
\draw[-Latex](O)--(z);
\node[anchor = south east] at (z){z};

\draw[very thick,](A)--(B)--(C)--cycle;

\draw[dashed, very thick](O)--(B);
\draw[dashed, very thick](O)--(C);
\draw[dashed,very thick](O)--(A);

\node[anchor = south west] at (A){A};
\node[anchor = north east] at (O){o};
\node[anchor = north east] at (C){C};
\node[anchor = south west ] at (B){B};

\end{tikzpicture}    }
    \quad
        \subfloat[]{\tdplotsetmaincoords{80}{160}
\begin{tikzpicture}[scale = 0.8,tdplot_main_coords, >=Latex]
\tikzmath{\aax=3.84;\bby=6;\ccz=4;\a = (\bby+\bby)/4;\d=-8.5;\e=4;};
\aax, \bby,\ccz;
\coordinate (O) at (0,0,0);
\coordinate (A) at (0,0,\ccz);
\coordinate (B) at (0,\bby,0);
\coordinate (C) at (\aax,0,0);
\coordinate (x) at (1.5*\aax,0,0);
\coordinate (y) at (0,1.5*\bby,0);
\coordinate (z) at (0,0,1.5*\ccz);

\coordinate (xyz) at (\aax/3,\bby/3,\ccz/3);
\draw [fill=gray] (xyz) circle (2pt) node[above right] (n1) {$$};


\coordinate (nxyz) at (3*\aax/3,3*\bby/3,3*\ccz/3);
\draw[-Latex, very thick](xyz)--(nxyz);
\node[anchor = south east] at (nxyz){$dS_t$};

\coordinate (yzp) at (3*\aax/3,\bby/3,\ccz/3);
\draw[-Latex,](xyz)--(yzp);
\node[anchor = south east] at (yzp){$$};

\coordinate (xyp) at (\aax/3,\bby/3,3.1*\ccz/3);
\draw[-Latex,](xyz)--(xyp);
\node[anchor = south east] at (xyp){$$};

\coordinate (xzp) at (\aax/3,3*\bby/3,\ccz/3);
\draw[-Latex, ](xyz)--(xzp);
\node[anchor = north west] at (xzp){$$};
\node[anchor = north east] at (xyz){$q_{t}$};

\coordinate (pz) at (\aax/3,\bby/3,3.1*\ccz/3);

\draw [fill=white] (pz) circle (1pt)node[above left] (pz) {$\gamma_z$};

\coordinate (pxy) at (3.*\aax/3,3*\bby/3,\ccz/3);


\coordinate (px) at (3*\aax/3,\bby/3,\ccz/3);

\draw [fill=white] (px) circle (1pt) node[above left] (px) {$\gamma_x$};
\coordinate (py) at (\aax/3,3*\bby/3,\ccz/3);

\draw [fill=white] (py) circle (1pt)node[above right] (py) {$\gamma_y$};


\draw[-Latex](O)--(x);
\node[anchor = south east] at (x){x};
\draw[-Latex](O)--(y);
\node[anchor = south east] at (y){y};
\draw[-Latex](O)--(z);
\node[anchor = south east] at (z){z};

\draw[,](A)--(B)--(C)--cycle;
\draw[](A)--(O)--(C)--cycle;
\draw[](A)--(O)--(B)--cycle;
\draw[](O)--(B)--(C)--cycle;

\node[anchor = south west] at (A){A};
\node[anchor = south west] at (O){o};
\node[anchor = north east] at (C){C};
\node[anchor = south west ] at (B){B};
%\draw[-Latex, ultra thick](xyz)--(A);
\coordinate (Ht) at (\aax/2,\bby/2,0);
\draw[dashed,ultra thick](A)--(Ht);
\draw [fill=white] (Ht) circle (1pt)node[below left] (Ht) {$h$};
\draw[dashed,ultra thick](O)--(Ht);

\end{tikzpicture}    }
    \quad
        \subfloat[]{\begin{tikzpicture}[scale = 0.8]
\coordinate (O) at (0,0);
\node[anchor = north east ] at (O){$h$};
\coordinate (ds) at (-2,3);
\draw[-Latex,thick] (O)--(ds);
\node[anchor = south west ] at (ds){$dS_t$};
\coordinate (A) at (4,3.5) {};
\draw[-Latex,thick] (O)--(A);
\node[anchor = south west ] at (A){$A$};
\coordinate (gamma) at (0,4.5) {};
\draw[-Latex,thick] (O)--(gamma);
\node[anchor = south west ] at (gamma){$\gamma_z$};
 \draw pic[draw,,angle radius=3cm,"$\frac{\pi}{2}-\alpha$"] {angle=A--O--gamma};
  \draw pic[draw,,angle radius=2cm,"$\alpha$"] {angle=gamma--O--ds};
  \coordinate (P) at (5,0);
  \draw[dashed] (O)--(P);
  \draw pic[draw,,angle radius=2cm,"$\alpha$"] {angle=P--O--A};
\end{tikzpicture}
}
    \quad
\caption{Pressure on a trirectangular tetrahedron}
\label{fig:fig_p199}
\end{figure}
To see that the pressure is independent of the direction of the surface element on which we measure it, let's consider a trirectangular tetrahedron $OABC$ as  depicted in figure $5.2(a)$. Let's define $P_x,P_y,P_z, P_t$ the pressure measured on the $4$ surfaces with normal vectors $dS_x,dS_y,dS_z,dS_t$. Let's neglect second order terms due to acceleration and external forces. For the forces along axis $z$ (the same reasoning is valid for the two others) we will have $P_zdS_z= P_tdS_t\gamma_z$ where $\gamma_z$ is the cosine of the angle formed by  normal on $dS_t$ and the z-axis.\\
Let's investigate the relationship between $dS_t$ and $dS_x,dS_y,dS_z$.\\
 Be $hA$ the line element lying in the plane $ABC$ (see figure $5.2(b)$) and $hO$ the line element lying in the plane $OBC$. As the area of a triangle $=\half\times base \times perpendicular \ height$ we get $dS_z = \half |hO||BC|$. But $|hO|= |hA|\cos\alpha$  (see figure $5.2(c)$) and so $dS_z = \half |BC||hA|\cos\alpha = dS_t\gamma_z$
 and get 
\begin{align}
&P_zdS_z= P_tdS_t\gamma_z\\ 
\Rightarrow \spatie & P_z dS_t\gamma_z= P_tdS_t\gamma_z\\ 
\Rightarrow \spatie & P_z = P_t
\end{align}

 $$\blacklozenge$$
\newpage
\section{p201 - Exercise}
\begin{tcolorbox}
Write down the contravariant form of $\mathbf{6.147}$
$$\mathbf{6.147} \spatie \partial_t v_r + v_s v_{r|s} = X_r - \rho^{-1} p_{,r}$$
\end{tcolorbox}
\begin{align}
\mathbf{6.147} \spatie \partial_t v_r + v_s v_{r|s} &= X_r - \rho^{-1} p_{,r}\\
\times a^{mr}\spatie \partial_t a^{mr} v_r + v_s a^{mr}v_{r|s} &= a^{mr}X_r - \rho^{-1} a^{mr}p_{,r}
\end{align}
By $\mathbf{2.527}$ page 53 we  have $a^{rs}_{|t}=0$ and thus 
\begin{align}
v^{r}_{|s}= \left(a^{mr}v_{r}\right)_{|s} &= \left(a^{mr}\right)_{|n}v_{r}+a^{mr}v_{r|s}= a^{mr}v_{r|s} \\
(2) \Rightarrow \spatie \partial_t  v^m + v_s v^{m}_{|s} &= X^m - \rho^{-1} a^{mr}p_{,r}\\
(2) \Rightarrow \spatie \partial_t  v^m + v_s v^{m}_{|s} &= X^m - \rho^{-1} p^{'}_{,r}
\end{align}
Note that $p_{,r}$ in $(1)$ and $(5)$ are not the same vector function as $p_{,r}$ can be written as $\overline{\nabla}p$ which is coordinate system  dependent.
 $$\blacklozenge$$
\newpage

\section{p202 - Exercise}
\begin{tcolorbox}
Verify by means of  $\mathbf{3.204}$ that $\mathbf{6.157}$ and $\mathbf{6.156}$  are the same equation.
\end{tcolorbox}
\begin{align}
\left\{\begin{array}{lll}
\mathbf{3.204}&\spatie&\Gamma^{n}_{rn} = \half\partial_r\log a = \partial_r\log \sqrt{a}\\
\mathbf{6.157}&\spatie&\left(\sqrt{a}a^{mn}\phi_{,m}\right)_{,n}=0\\
\mathbf{6.156}&\spatie&a^{mn}\phi_{|mn}=0
\end{array}\right.
\end{align}
Considering also
\begin{align}
\left\{\begin{array}{l}
a^{mn}_{|k}=0\\
T^m_{|n}= \partial_n T_m+\Gamma^{m}_{kn}T_k
\end{array}\right.
\end{align}
So $\mathbf{6.156}$ can written as
\begin{align}
& a^{mn}\phi_{|mn}=0\\
\Leftrightarrow\spatie & \left(a^{mn}\phi_{|m}\right)_{|n}=0\\
T^n = a^{mn}\phi_{|m}\spatie\Rightarrow\spatie &\partial_n T^n+\Gamma^{n}_{kn}T^k=0\\
\Rightarrow\spatie &\partial_n \left(a^{mn}\phi_{|m}\right)+\Gamma^{n}_{kn}a^{pk}\phi_{|p}=0\\
\Leftrightarrow\spatie &\partial_n \left(a^{mn}\right) \phi_{,m}+a^{mn}\phi_{,mn}+\Gamma^{n}_{kn}a^{pk}\phi_{,p}=0\\
\text{(2)}\quad \Rightarrow\spatie &\partial_n \left(a^{mn}\right) \phi_{,m}+a^{mn}\phi_{,mn}+\partial_k\log \sqrt{a}a^{pk}\phi_{,p}=0\\
\Leftrightarrow\spatie &\left(a^{mn}_{,n}\right) \phi_{,m}+a^{mn}\phi_{,mn}+\frac{1}{\sqrt{a}}\left(\sqrt{a}a^{pk}\right)_{,k}\phi_{,p}=0\\
\Leftrightarrow\spatie &\sqrt{a} \phi_{,m}\left(a^{mn}\right)_{,n}+\sqrt{a}a^{mn}\phi_{,mn}+a^{mn}\phi_{,m}\left(\sqrt{a}\right)_{,n}=0\\
\Leftrightarrow\spatie &\left(\sqrt{a}a^{mn}\phi_{,m}\right)_{,n}=0
\end{align}
 $$\blacklozenge$$
\newpage


\section{p205 - Exercise}
\begin{tcolorbox}
Show that a small strain is a rigid body displacement if, and only if, $e_{rs}=0$. In the case of finite strain, deduce form $\mathbf{6.206}$ the conditions which must be satisfied by the partial derivatives of the displacement in order that it may b a rigid body displacement.
\end{tcolorbox}
\textbf{Suppose we deal with  a rigid body.} Then the position of two particles of the rigid body are given by

\begin{align}
\left\{\begin{array}{l}
p_r=z_r+u_r(z)\\
p^{'}_r=z^{'}_r+u_r(z^{'})\\
\end{array}\right.
\end{align}
and
\begin{align}
\left\{\begin{array}{l}
L_0= z_r-z^{'}_r \\
L_1 = z^{'}_r+u_r(z^{'})-z_r-u_r(z)  
\end{array}\right.
\end{align}
A rigid body means $L_1=L_0$, giving $u_r(z^{'})=u_r(z^{})$ i.e $u_r(z^{})$ is a constant and thus $u_{r,s}(z^{})=0$.\\
As $e_{rs}=\half\left(u_{r,s}(z^{})+u_{s,r}(z^{})\right)$ we get $e_{rs}=0$\\\\
\textbf{Suppose now that $e_{rs}=0$}\\
We have $e= e_{rs}\lambda_r\lambda_s = 0$ and $e= u_{r,s}(z^{})\lambda_r\lambda_s $
As the $\lambda_r$ are arbitrary, in the sense that we are free to choose whatever curve to approach the initial point, we conclude that $u_{r,s}(z^{})=0$ . So$u_{r}(z^{})$ is a constant, meaning that the mutual distance between two arbitrary points, do not change. The body is a rigid body.
 $$\lozenge$$
 For a finite strain we have $\mathbf{6.206}$:
 \begin{align}
 \lim \frac{L_1^2-L_0^2}{L_0^2}=2 u_{r,s}(z^{})\lambda_r\lambda_s + u_{m,r}(z^{})u_{m,s}(z^{})\lambda_r\lambda_s
 \end{align}
 This limit is $0$ and so we get as condition
  \begin{align}
 \left(2 u_{r,s}(z^{}) + u_{m,r}(z^{})u_{m,s}(z^{})\right)\lambda_r\lambda_s=0
 \end{align}
 As the $\lambda_r$ are arbitrary, in the sense that we are free to choose whatever curve to approach the initial point, we conclude that $2 u_{r,s}(z^{}) + u_{m,r}(z^{})u_{m,s}(z^{})$ must be zero.
 $$2 u_{r,s}(z^{}) + u_{m,r}(z^{})u_{m,s}(z^{})=0$$
 $$\blacklozenge$$
\newpage

\section{p207 - Clarification}
\begin{tcolorbox}
Then, clearly, since the the volume of the tetrahedron is less than $a^3$,\\\\
$\mathbf{6.217}\spatie \displaystyle \lim_{a \to 0} \frac{1}{a^2} \dv{M_r}{t}=0,\quad \displaystyle \lim_{a \to 0} \frac{1}{a^2} \int X_rdV=0$
$$\vdots$$
But  $\displaystyle \lim_{a \to 0} \frac{S}{a^2}$ is not zero,...
\end{tcolorbox}
First, note that the volume of  a trirectangular tetrahedron is $V=\frac{1}{6}abc$ with $a,b,c$ the bases of the 3 rectangular triangles (see clarification for page 199), so if $a\ge b,c$ we have  $V< a^3$.\\
There is no assurance that $\displaystyle \lim_{a \to 0} \frac{1}{a^3} \dv{M_r}{t}=0$. Indeed, consider $\mathbf{5.334}$. For a continuous medium, this equation can be written as
\begin{align}
I_{st} &= \int_{V}\rho\epsilon_{ptq}\epsilon_{psn}z_sz_tdV
\end{align} 
If $V$ goes to zero, the quantities under the integral can be approximated by constant values and hence, the dynamics of the tetrahedron are govern by
\begin{align}
&\displaystyle \lim_{V \to 0}I_{st} = \rho\epsilon_{ptq}\epsilon_{psn}z_sz_t V\\
\mathbf{5.332}\text{:}\spatie &\dv{I_{st}\omega_t}{t}= M_s\\
\Rightarrow\spatie &\lim_{V \to 0} \dv{M_s}{t} = \lim_{V \to 0}V\dv{\left(\rho\epsilon_{ptq}\epsilon_{psn}z_sz_t\omega_t\right)}{t} \\
\Rightarrow\spatie &\lim_{a \to 0} \frac{1}{a^3}\dv{M_s}{t} = \lim_{a \to 0}\frac{1}{a^3}V\dv{\left(\rho\epsilon_{ptq}\epsilon_{psn}z_sz_t\omega_t\right)}{t} \\
\Rightarrow\spatie &\lim_{a \to 0} \frac{1}{a^3}\dv{M_s}{t} < \lim_{a \to 0}\frac{1}{a^3}a^3\dv{\left(\rho\epsilon_{ptq}\epsilon_{psn}z_sz_t\omega_t\right)}{t} \\
\Rightarrow\spatie &\lim_{a \to 0} \frac{1}{a^3}\dv{M_s}{t} < \lim_{a \to 0}\dv{\left(\rho\epsilon_{ptq}\epsilon_{psn}z_sz_t\omega_t\right)}{t}
\end{align}
but there is no reason to admit that $\displaystyle \lim_{V \to 0}\dv{\left(\rho\epsilon_{ptq}\epsilon_{psn}z_sz_t\omega_t\right)}{t}=0$. \\
On the other hand, replacing $a_3$ with $a^2$ in $(5)$ gives 
\begin{align}
&\lim_{a \to 0} \frac{1}{a^2}\dv{M_s}{t} = \lim_{a \to 0}\frac{1}{a^2}V\dv{\left(\rho\epsilon_{ptq}\epsilon_{psn}z_sz_t\omega_t\right)}{t} \\
\Rightarrow\spatie &\lim_{a \to 0} \frac{1}{a^2}\dv{M_s}{t} < \lim_{a \to 0}\frac{1}{a^2}a^3\dv{\left(\rho\epsilon_{ptq}\epsilon_{psn}z_sz_t\omega_t\right)}{t} \\
\Rightarrow\spatie &\lim_{a \to 0} \frac{1}{a^2}\dv{M_s}{t} < \lim_{a \to 0}a\dv{\left(\rho\epsilon_{ptq}\epsilon_{psn}z_sz_t\omega_t\right)}{t}\\
\Rightarrow\spatie &\lim_{a \to 0} \frac{1}{a^2}\dv{M_s}{t} =0
\end{align}
For $ \displaystyle \lim_{a \to 0} \frac{1}{a^2} \int X_rdV=0$ the reasoning is even simpler as for a volume going to zero , we can consider $X_r$ as constant and thus 
\begin{align}
 \displaystyle \lim_{a \to 0} \frac{1}{a^2} \int X_rdV=\displaystyle \lim_{a \to 0} \frac{1}{a^2}  X_r \int dV < \displaystyle \lim_{a \to 0} \frac{1}{a^2}  X_r a^3 = 0
\end{align}
$$\lozenge$$
\textbf{But  $\displaystyle \lim_{a \to 0} \frac{S}{a^2}$ is not zero,...}\\\\
Be $S_t$ the area of the "sloped" triangle in the tetrahedron. Then, the total area of the tetrahedron is:
\begin{align}
&S= \half\left(ab + bc + ac \right) + S_t\\
&S_t > \half ab\text{,   }S_t > \half bc\text{,   }S_t > \half ac\\
\Rightarrow \spatie & S>  \half\left(ab + bc + ac \right) + \half ab\\
\Rightarrow \spatie & \frac{S}{a^2}>  \half\left( \frac{bc}{a^2} + \frac{c}{a} \right) + b
\end{align}
If we shrink the tetrahedron uniformly and put $a=\epsilon a_0, b=\epsilon b_0, c=\epsilon c_0$ then $(16)$ can be written as 
\begin{align}
\lim_{\epsilon \to 0}\frac{S}{a^2}>   \half\left( \frac{b_0c_0}{a_0^2} + \frac{c_0}{a_0} \right) + b_0\lim_{\epsilon \to 0}\epsilon 
\end{align}
which is indeed not zero.
 $$\blacklozenge$$
\newpage

\section{p208 - Exercise}
\begin{tcolorbox}
Show that the stress across a plane $z_1 = \text{const.}$ has the components $E_{11}, \ E_{21}, \ E_{31}$. What are the components across planes $z_2 = \text{const.}$ and $z_3 = \text{const.}$?
\end{tcolorbox}
\begin{align}
\mathbf{6.223} \spatie T_r = E_{rs}n_s
\end{align}
So for the the stress across a plane $z_1 = \text{const.}$, we have $n_1=1, \ n_2=0, \ n_3 = 0$ and so 
\begin{align}
T_r\left(z_1 = \text{const.}\right) = \left(\begin{matrix}E_{11}\\E_{21}\\E_{31}\end{matrix}\right)
\end{align}
For the the stress across a plane $z_2 = \text{const.}$, we have $n_1=0, \ n_2=1, \ n_3 = 0$ and for the the stress across a plane $z_3 = \text{const.}$, we have $n_1=0, \ n_2=0, \ n_3 = 1$ and so
\begin{align}
&T_r\left(z_2 = \text{const.}\right) = \left(\begin{matrix}E_{12}\\E_{22}\\E_{32}\end{matrix}\right)\\
&T_r\left(z_3 = \text{const.}\right) = \left(\begin{matrix}E_{13}\\E_{23}\\E_{33}\end{matrix}\right)\
\end{align}
 $$\blacklozenge$$
\newpage