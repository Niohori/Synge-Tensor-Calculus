\setcounter{chapter}{4}
\chapter{Applications to Hydrodynamics, Elasticity, and Electromagnetic radiation}
\pagebreak[4]
\section{p191 - Exercise}
\begin{tcolorbox}
A fluid rotates as a rigid body about the axis of $z_3$ with variable angular velocity $\omega(t)$. Write out explicitly the three Lagrangian equations $\mathbf{6.101}$ and the three Eulerian equations $\mathbf{6.103}$.
\end{tcolorbox}
The motion described reduces to a motion in a $V_2$ plane with $z_3$ a constant for a definite particle.\\\\\\
\textbf{Lagrangian}\\

A particular particle with starting coordinates $\left(z^{(*)}_1,z^{(*)}_2, z^{(*)}_3\right)$ will describe a  circle with radius $\sqrt{z^{(*)2}_1+z^{(*)2}_2}$ in the  plane $V_2$ parallel with the $1,2$ axes.
Taking axis $1$ as reference to determine the instantaneous angle $\theta$ of the vertex $OP$ (origin and particle) we get
\begin{align}
\left\{\begin{array}{l}
z_1 = \sqrt{z^{(*)2}_1+z^{(*)2}_2}\cos\left(\theta(t) + \phi_0\right)\\
z_2 = \sqrt{z^{(*)2}_1+z^{(*)2}_2}\sin\left(\theta(t) + \phi_0\right)\\
z_3 = z^{(*)}_3
\end{array}\right.
\end{align}
with 
\begin{align}
\phi_0 & = \arctan\frac{z^{(*)}_2}{z^{(*)}_1}
\end{align}
Note that $\omega(t)$ is not a constant, so
\begin{align}
\theta(t)&= \int_{0}^{t}\omega(\tau)d\tau
\end{align}
and get 
\begin{align}
\left\{\begin{array}{l}
z_1 = \sqrt{z^{(*)2}_1+z^{(*)2}_2}\cos\left(\int_{0}^{t}\omega(\tau)d\tau + \phi_0\right)\\\\
z_2 = \sqrt{z^{(*)2}_1+z^{(*)2}_2}\sin\left(\int_{0}^{t}\omega(\tau)d\tau + \phi_0\right)\\\\
z_3 = z^{(*)}_3\\\\
\phi_0 = \arctan\frac{z^{(*)}_2}{z^{(*)}_1}
\end{array}\right.
\end{align}
$$\lozenge$$\\
\newpage
\textbf{Eulerian}\\
The equations get simplified and reduce to a motion of a particle on a circle.
\begin{figure}[H]
\centering
\tikzstyle{left-hand-mirror} = [
    draw,
    postaction=decorate, 
    decoration={
        markings,
        mark=between positions 0.015 and 0.98 step 0.1072 with {\draw (0,0)--(60:3pt);}
    }
]   

\begin{tikzpicture}[scale=0.5]
\coordinate (O1) at (0,0);
\draw  (O1) circle (6);
\coordinate (Om) at (0.0,-6) ;
\coordinate  {} {} {};
\coordinate  {} {} {};
\coordinate (NPole) at (0,8.5) {} {};
%\draw[fill=white]  (NPole) circle (0.1);
\coordinate (SPole) at (0,-7.5) {} {};
\draw[-{Latex[length=2mm]}] (SPole) -- (NPole);
%\draw[fill=white]  (SPole) circle (0.1);
\coordinate (P) at (4.5,4) {} {};
\draw[fill=white]  (P) circle (0.1);
\draw[fill=white]  (O1) circle (0.1);
\draw[dashed] (O1) -- (P);
%\node[label=west:$O$] at (O1) {};
\node[label=east:$P\left(x \text{, } y\right)$] at (P) {};
\coordinate (O2) at (3,0) {};
\node[label=north east:$\theta\text{=} \arctan\frac{y}{x}$] at (O2) {};
\draw [dashed, decoration={markings, mark=at position 0.52 with {\arrow[scale = 1.]{Latex[length=2mm]}}},    postaction={decorate}](O2) arc (0:42:3 and 3.0);
\coordinate (F) at (1,7) {} {};
\draw [-{Latex[length=2mm]}] (P) -- (F);
\node[label=east:$\mathbf{\overline{v}}$] at (F) {};

\coordinate (x1)at (-7,0) {};
\coordinate (X) at (8.5,0) {} {};
\draw [-{Latex[length=2mm]}] (x1) -- (X);
\node[label=south east:$x$] at (X) {};
\node[label=south east:$y$] at (NPole) {};
\coordinate (C1) at (4,4.5) {};
\coordinate (C2) at (2.5,2.5) {};
\tkzMarkRightAngle[size=0.3](C1,P,C2);
\end{tikzpicture}

\caption{Eulerian viewpoint of a spinning fluid}
\label{fig:fig_p191_Exa}
\end{figure}

\begin{align}
\left\{\begin{array}{l}
v_1 = -\sqrt{z^{2}_1+z^{2}_2}\omega(t)\sin\left(\arctan\frac{z^{}_2}{z^{}_1}\right)\\\\
v_2 = \sqrt{z^{2}_1+z^{2}_2}\omega(t)\cos\left(\arctan\frac{z^{}_2}{z^{}_1}\right)\\\\
v_3 = 0
\end{array}\right.
\end{align}
$$\blacklozenge$$
\newpage

\section{p191 - Exercise}
\begin{tcolorbox}
Compute the components of acceleration for the motion described in the preceding exercise.
\end{tcolorbox}
We have
\begin{align}
\left\{\begin{array}{l}
v_1 = -\sqrt{z^{2}_1+z^{2}_2}\omega(t)\sin\left(\arctan\frac{z^{}_2}{z^{}_1}\right)\\\\
v_2 = \sqrt{z^{2}_1+z^{2}_2}\omega(t)\cos\left(\arctan\frac{z^{}_2}{z^{}_1}\right)\\\\
v_3 = 0
\end{array}\right.
\end{align}
and 
\begin{align}
f_r &= \partial_tv_r + v_{r,s}v_s
\end{align}
\begin{align}
&\left\{\begin{array}{l}
\partial_t v_1 = -\sqrt{z^{2}_1+z^{2}_2}\dot{\omega}(t)\sin\left(\arctan\frac{z^{}_2}{z^{}_1}\right)\\\\
\partial_t v_2 = \sqrt{z^{2}_1+z^{2}_2}\dot{\omega}(t)\cos\left(\arctan\frac{z^{}_2}{z^{}_1}\right)\\\\
v_{1,1}= -\omega(t)\left[\frac{z_1}{\sqrt{z^{2}_1+z^{2}_2}}\sin\left(\arctan\frac{z^{}_2}{z^{}_1}\right) -\sqrt{z^{2}_1+z^{2}_2}\cos\left(\arctan\frac{z^{}_2}{z^{}_1}\right)\frac{z_2}{z_1^2}\frac{1}{1+\frac{z_2^2}{z_1^2}}\right]\\\\
v_{1,2}= -\omega(t)\left[\frac{z_2}{\sqrt{z^{2}_1+z^{2}_2}}\sin\left(\arctan\frac{z^{}_2}{z^{}_1}\right) +\sqrt{z^{2}_1+z^{2}_2}\cos\left(\arctan\frac{z^{}_2}{z^{}_1}\right)\frac{1}{z^{}_1}\frac{1}{1+\frac{z_2^2}{z_1^2}}\right]\\\\
v_{2,1}= \omega(t)\left[\frac{z_1}{\sqrt{z^{2}_1+z^{2}_2}}\cos\left(\arctan\frac{z^{}_2}{z^{}_1}\right) +\sqrt{z^{2}_1+z^{2}_2}\sin\left(\arctan\frac{z^{}_2}{z^{}_1}\right)\frac{z_2}{z_1^2}\frac{1}{1+\frac{z_2^2}{z_1^2}}\right]\\\\
v_{2,2}= \omega(t)\left[\frac{z_2}{\sqrt{z^{2}_1+z^{2}_2}}\cos\left(\arctan\frac{z^{}_2}{z^{}_1}\right) -\sqrt{z^{2}_1+z^{2}_2}\sin\left(\arctan\frac{z^{}_2}{z^{}_1}\right)\frac{1}{z^{}_1}\frac{1}{1+\frac{z_2^2}{z_1^2}}\right]\\\\
\end{array}\right.
\end{align}
\begin{align}
&=\left\{\begin{array}{l}
\partial_t v_1 = -\sqrt{z^{2}_1+z^{2}_2}\dot{\omega}(t)\sin\left(\arctan\frac{z^{}_2}{z^{}_1}\right)\\\\
\partial_t v_2 = \sqrt{z^{2}_1+z^{2}_2}\dot{\omega}(t)\cos\left(\arctan\frac{z^{}_2}{z^{}_1}\right)\\\\
v_{1,1}= -\frac{\omega(t)}{\sqrt{z^{2}_1+z^{2}_2}}\left[z_1\sin\left(\arctan\frac{z^{}_2}{z^{}_1}\right) -z_2\cos\left(\arctan\frac{z^{}_2}{z^{}_1}\right)\right]\\\\
v_{1,2}=  -\frac{\omega(t)}{\sqrt{z^{2}_1+z^{2}_2}}\left[z_2\sin\left(\arctan\frac{z^{}_2}{z^{}_1}\right) +z^{}_1\cos\left(\arctan\frac{z^{}_2}{z^{}_1}\right)\right]\\\\
v_{2,1}=  \frac{\omega(t)}{\sqrt{z^{2}_1+z^{2}_2}}\left[z_1\cos\left(\arctan\frac{z^{}_2}{z^{}_1}\right) +z_2\sin\left(\arctan\frac{z^{}_2}{z^{}_1}\right)\right]\\\\
v_{2,2}=  \frac{\omega(t)}{\sqrt{z^{2}_1+z^{2}_2}}\left[z_2\cos\left(\arctan\frac{z^{}_2}{z^{}_1}\right) -z^{}_1\sin\left(\arctan\frac{z^{}_2}{z^{}_1}\right)\right]\\\\
\end{array}\right.
\end{align}
and get
\begin{align}
v_{1,s}v_s &= \left\{\begin{array}{l}
-\sqrt{z^{2}_1+z^{2}_2}\omega(t)\sin\left(\arctan\frac{z^{}_2}{z^{}_1}\right)\frac{\omega(t)}{\sqrt{z^{2}_1+z^{2}_2}}\left[z_1\sin\left(\arctan\frac{z^{}_2}{z^{}_1}\right) -z_2\cos\left(\arctan\frac{z^{}_2}{z^{}_1}\right)\right]\\\\
-\sqrt{z^{2}_1+z^{2}_2}\omega(t)\cos\left(\arctan\frac{z^{}_2}{z^{}_1}\right)\frac{\omega(t)}{\sqrt{z^{2}_1+z^{2}_2}}\left[z_2\sin\left(\arctan\frac{z^{}_2}{z^{}_1}\right) +z^{}_1\cos\left(\arctan\frac{z^{}_2}{z^{}_1}\right)\right]
\end{array}\right.\\
&=-z_1\omega^2(t)
\end{align}
\begin{align}
v_{2,s}v_s &= \left\{\begin{array}{l}
\sqrt{z^{2}_1+z^{2}_2}\omega(t)\sin\left(\arctan\frac{z^{}_2}{z^{}_1}\right)\frac{\omega(t)}{\sqrt{z^{2}_1+z^{2}_2}}\left[z_1\cos\left(\arctan\frac{z^{}_2}{z^{}_1}\right) +z_2\sin\left(\arctan\frac{z^{}_2}{z^{}_1}\right)\right]\\\\
+\sqrt{z^{2}_1+z^{2}_2}\omega(t)\cos\left(\arctan\frac{z^{}_2}{z^{}_1}\right)\frac{\omega(t)}{\sqrt{z^{2}_1+z^{2}_2}}\left[z_2\cos\left(\arctan\frac{z^{}_2}{z^{}_1}\right) -z^{}_1\sin\left(\arctan\frac{z^{}_2}{z^{}_1}\right)\right]\\\\
\end{array}\right.\\
&= z_2\omega^2(t)
\end{align}
giving with the second derivative term
\begin{align}
&=\left\{\begin{array}{l}
f_1 = -\dot{\omega}(t)\sqrt{z^{2}_1+z^{2}_2}\sin\left(\arctan\frac{z^{}_2}{z^{}_1}\right)-z_1\omega^2(t)\\\\\\
f_2 = \dot{\omega}(t)\sqrt{z^{2}_1+z^{2}_2}\cos\left(\arctan\frac{z^{}_2}{z^{}_1}\right)+z_2\omega^2(t)\\\\
f_3=0
\end{array}\right.
\end{align}
$$\blacklozenge$$
\newpage


\section{p193 - Exercise}
\begin{tcolorbox}
Verify that the operator $\frac{\partial}{\partial t}$ does not alter tensor character.
\end{tcolorbox}
Be $X^r$ and $Y^r$, two tensors so that $I=X_rY^r$ is an invariant. Obviously, $\frac{\partial I}{\partial t}$ will also be invariant and 
\begin{align}
\frac{\partial I}{\partial t}= \frac{\partial X_r}{\partial t}Y^r+X_r\frac{\partial Y^r}{\partial t}
\end{align}
Meaning that the right side is a sum of two invariants, from which we conclude (see page 20, $\mathbf{1.607}$) that $\frac{\partial X_r}{\partial t}$ and $\frac{\partial Y^r}{\partial t}$ are tensors.
$$\blacklozenge$$
\newpage

\section{p193 - Clarification to 6.112}
\begin{tcolorbox}
$$\mathbf{6.112.}\spatie \int Fn_rdS = \int F_{,r}dV$$
\end{tcolorbox}
Green's theorem is generally presented in the form
$$\int\overline{F}.\overline{n}dS = \int \overline{\nabla}. \overline{F}dV$$or
$$\int F_rn_rdS = \int F_{r,r}dV$$
We can define $$\overline{F} = F\overline{1}_r$$
 $\overline{F}.\overline{n}$  will then become $Fn_r$ while $\overline{\nabla}. \overline{F}$ will become $\partial_r F$, giving the expression $\mathbf{6.112.}$.
 $$\blacklozenge$$
\newpage



\section{p196 - Exercise}
\begin{tcolorbox}
Write out $\mathbf{6.126}$ and $\mathbf{6.127b}$ explicitly for spherical polar coordinates.
\end{tcolorbox}
For spherical polar coordinates we have 
\begin{align}
(v^r) = \left(\begin{matrix}\dot{r}\\r\dot{\theta}\\r\sin\theta\dot{\phi} \end{matrix}\right)
\end{align}
and (see $\mathbf{2.546}$ page 58):
\begin{align}
v^r_{|r} &= \frac{1}{r^2}\partial_r\left(r^2v^1\right)+\frac{1}{\sin\theta}\partial_{\theta}\left(\sin\theta v^2\right)+\partial_{\phi} v^3\\
&=  \frac{1}{r^2}\left(2rv^1+r^2\partial_rv^1\right)+\frac{1}{\sin\theta}\left(v^2\cos\theta + \sin\theta\partial_{\theta}v^2\right)\\
&= \frac{2}{r}\dot{r}+r\dot{\theta}\cot \theta 
\end{align}
and 
\begin{align}
&\partial_t \rho+ \left(\rho v^r\right)_{|r} =0\\
\Leftrightarrow\spatie &\partial_t\rho+ \rho_{|r}v^r+\rho v^r_{|r} =0\\
\Leftrightarrow\spatie &\dv{\rho}{t}+\rho \left(\frac{2}{r}\dot{r}+r\dot{\theta}\cot \theta \right) =0
\end{align}
 $$\blacklozenge$$
\newpage


\section{p198 - Exercise}
\begin{tcolorbox}
If $\epsilon^{rmn}$ is defined in precisely the same way as $\epsilon_{rmn}$, prove that $$ \epsilon^{'uvw}= J\epsilon^{rmn}\frac{\partial x^{'u}}{\partial x^r}\frac{\partial x^{'v}}{\partial x^m}\frac{\partial x^{'w}}{\partial x^n}$$
\end{tcolorbox}
We follow the pretty same line of reasoning as for $\epsilon_{rst}$. Going from $x^{'r}$ to $x^{s}$ we have (expanding the determinant of the inverse Jacobian along the rows instead of the columns):
\begin{align}
J^{-1}=\left|\frac{\partial x^{'p}}{\partial x^{q}}\right|&= \epsilon^{rmn}\frac{\partial x^{'1}}{\partial x^r}\frac{\partial x^{'2}}{\partial x^m}\frac{\partial x^{'3}}{\partial x^n}\\
\times \epsilon^{uvw}\spatie J^{-1}\epsilon^{uvw}&= \epsilon^{rmn}\frac{\partial x^{'u}}{\partial x^r}\frac{\partial x^{'v}}{\partial x^m}\frac{\partial x^{'w}}{\partial x^n}\\
\times J\spatie\epsilon^{uvw}&= J\epsilon^{rmn}\frac{\partial x^{'u}}{\partial x^r}\frac{\partial x^{'v}}{\partial x^m}\frac{\partial x^{'w}}{\partial x^n}\\ 
\end{align}
 $$\blacklozenge$$
\newpage



\section{p198 - Exercise}
\begin{tcolorbox}
Prove that  $\frac{\epsilon^{rmn}}{\sqrt{a}}$ is an (absolute) contravariant tensor of the third order.
\end{tcolorbox}
\begin{align} 
\sqrt{a^{'}} &= J\sqrt{a}\\
\epsilon^{'uvw}&= J\epsilon^{rmn}\frac{\partial x^{'u}}{\partial x^r}\frac{\partial x^{'v}}{\partial x^m}\frac{\partial x^{'w}}{\partial x^n}\\
\text{(1)  in (2)} \spatie \epsilon^{'uvw}&= \frac{\sqrt{a^{'}}}{\sqrt{a^{}}}\epsilon^{rmn}\frac{\partial x^{'u}}{\partial x^r}\frac{\partial x^{'v}}{\partial x^m}\frac{\partial x^{'w}}{\partial x^n}\\
\Rightarrow \spatie \frac{\epsilon^{'uvw}}{\sqrt{a^{'}}}&= \frac{\epsilon^{rmn}}{\sqrt{a}}\frac{\partial x^{'u}}{\partial x^r}\frac{\partial x^{'v}}{\partial x^m}\frac{\partial x^{'w}}{\partial x^n}
\end{align}
which is the required transformation rule for a "normal" (absolute) tensor.
 $$\blacklozenge$$
\newpage


\section{p199 - Clarification to pressure anisotropy}
\begin{tcolorbox}
Pressure is independent of the direction
\end{tcolorbox}
\begin{figure}[H]%
    \centering
    \subfloat[]{\tdplotsetmaincoords{50}{220}
\begin{tikzpicture}[tdplot_main_coords, >=Latex]
\tikzmath{\aax=3;\bby=4;\ccz=3;\a = (\bby+\bby)/4;};
\aax, \bby,\ccz;
\coordinate (a0) at (0,0,0);
\coordinate (a5) at (\aax,0,0);
\coordinate (a3) at (0,\bby,0);
\coordinate (a1) at (0,0,\ccz);
\coordinate (a4) at (\aax,\bby,0);
\coordinate (a6) at  (\aax,0,\ccz);
\coordinate (a2) at  (0,\bby,\ccz);
\coordinate (a7) at  (\aax,\bby,\ccz);
\coordinate (A) at (0,0,0);
\coordinate (B) at (0,0,\ccz);
\coordinate (C) at (\aax,\bby,\ccz);
\coordinate (D) at (\aax,\bby,0);


\draw[dashed,ultra thick,fill= gray!10](A)--(B)--(C)--(D)--cycle;
\draw[](a0)--(a1);
\draw[](a0)--(a5);
\draw[](a0)--(a3);
\draw[](a5)--(a6)node[midway,thick,left] {$c$};
\draw[](a5)--(a4)node[midway,thick,below] {$b$};
\draw[](a6)--(a1)node[midway,very thick,above] {$a$};;
\draw[](a4)--(a3);
\draw[](a7)--(a4);
\draw[](a7)--(a6);
\draw[](a7)--(a2);
\draw[](a2)--(a3);
\draw[](a2)--(a1);
\end{tikzpicture}     }
	\qquad
    \subfloat[]{\tdplotsetmaincoords{60}{160}
\begin{tikzpicture}[tdplot_main_coords, >=Latex]
\tikzmath{\aax=3;\bby=4;\ccz=3;\a = (\bby+\bby)/4;};
\aax, \bby,\ccz;
\coordinate (a0) at (0,0,0);
\coordinate (a5) at (\aax,0,0);
\coordinate (a3) at (0,\bby,0);
\coordinate (a1) at (0,0,\ccz);
\coordinate (a4) at (\aax,\bby,0);
\coordinate (a6) at  (\aax,0,\ccz);
\coordinate (a2) at  (0,\bby,\ccz);
\coordinate (a7) at  (\aax,\bby,\ccz);
\coordinate (A) at (0,0,0);
\coordinate (B) at (0,0,\ccz);
\coordinate (C) at (\aax,\bby,\ccz);
\coordinate (D) at (\aax,\bby,0);
\draw[very thick,fill = gray!10](D)--(a6)--(a0)--(D);
\draw[dashed,very thin](a0)--(a1);
\draw[very thick](a0)--(a5);
\draw[dashed,very thin](a0)--(D);
\draw[very thick](a5)--(a6)node[midway,thick,left] {$c$};;
\draw[very thick](a5)--(D)node[midway,thick,below] {$b$};;
\draw[dashed,very thick](a6)--(a7);
\draw[dashed,very thick](a6)--(a1)node[midway,very thick,above] {$a$};;
\draw[dashed,very thick](a7)--(a4);
\draw[dashed,very thick](A)--(B);
\draw[dashed,very thick](B)--(C);
\draw[dashed,very thick](C)--(D);
\draw[dashed,very thick](D)--(A);

\draw[very thick](a6)--(a0);
\node[anchor = south east] at (a6){A};
\node[anchor = south east] at (a5){O};
\node[anchor = north east] at (D){C};
\node[anchor = south west ] at (a0){B};

\end{tikzpicture}    }
\caption{Volume of a tetrahedron}
\label{fig:fig_p126_1a}
\end{figure}
First remark that the volume of a rectangular tetrahedron $OABC$, as described in figure $5.2(b)$, is 
$$ V= \frac{1}{4}abc$$ 
This can easily verified by slicing, as described in figure $5.2$, a cuboid with sides $a,b,c$ through 2 planes.
 $$\blacklozenge$$
\newpage
\section{p201 - Exercise}
\begin{tcolorbox}
Write down the contravariant form of $\mathbf{6.147}$
$$\mathbf{6.147} \spatie \partial_t v_r + v_s v_{r|s} = X_r - \rho^{-1} p_{,r}$$
\end{tcolorbox}
\begin{align}
\mathbf{6.147} \spatie \partial_t v_r + v_s v_{r|s} &= X_r - \rho^{-1} p_{,r}\\
\times a^{mr}\spatie \partial_t a^{mr} v_r + v_s a^{mr}v_{r|s} &= a^{mr}X_r - \rho^{-1} a^{mr}p_{,r}
\end{align}
By $\mathbf{2.527}$ page 53 we  have $a^{rs}_{|t}=0$ and thus 
\begin{align}
v^{r}_{|s}= \left(a^{mr}v_{r}\right)_{|s} &= \left(a^{mr}\right)_{|n}v_{r}+a^{mr}v_{r|s}= a^{mr}v_{r|s} \\
(2) \Rightarrow \spatie \partial_t  v^m + v_s v^{m}_{|s} &= X^m - \rho^{-1} a^{mr}p_{,r}\\
(2) \Rightarrow \spatie \partial_t  v^m + v_s v^{m}_{|s} &= X^m - \rho^{-1} p^{'}_{,r}
\end{align}
Note that $p_{,r}$ in $(1)$ and $(5)$ are not the same vector function as $p_{,r}$ can be written as $\overline{\nabla}p$ which is coordinate system  dependent.
 $$\blacklozenge$$
\newpage

\section{p202 - Exercise}
\begin{tcolorbox}
Verify by means of  $\mathbf{3.204}$ that $\mathbf{6.157}$ and $\mathbf{6.156}$  are the same equation.
\end{tcolorbox}
\begin{align}
\left\{\begin{array}{lll}
\mathbf{3.204}&\spatie&\Gamma^{n}_{rn} = \half\partial_r\log a = \partial_r\log \sqrt{a}\\
\mathbf{6.157}&\spatie&\left(\sqrt{a}a^{mn}\phi_{,m}\right)_{,n}=0\\
\mathbf{6.156}&\spatie&a^{mn}\phi_{|mn}=0
\end{array}\right.
\end{align}
Considering also
\begin{align}
\left\{\begin{array}{l}
a^{mn}_{|k}=0\\
T^m_{|n}= \partial_n T_m+\Gamma^{m}_{kn}T_k
\end{array}\right.
\end{align}
So $\mathbf{6.156}$ can written as
\begin{align}
& a^{mn}\phi_{|mn}=0\\
\Leftrightarrow\spatie & \left(a^{mn}\phi_{|m}\right)_{|n}=0\\
T^n = a^{mn}\phi_{|m}\spatie\Rightarrow\spatie &\partial_n T^n+\Gamma^{n}_{kn}T^k=0\\
\Rightarrow\spatie &\partial_n \left(a^{mn}\phi_{|m}\right)+\Gamma^{n}_{kn}a^{pk}\phi_{|p}=0\\
\Leftrightarrow\spatie &\partial_n \left(a^{mn}\right) \phi_{,m}+a^{mn}\phi_{,mn}+\Gamma^{n}_{kn}a^{pk}\phi_{,p}=0\\
\text{(2)}\quad \Rightarrow\spatie &\partial_n \left(a^{mn}\right) \phi_{,m}+a^{mn}\phi_{,mn}+\partial_k\log \sqrt{a}a^{pk}\phi_{,p}=0\\
\Leftrightarrow\spatie &\left(a^{mn}_{,n}\right) \phi_{,m}+a^{mn}\phi_{,mn}+\frac{1}{\sqrt{a}}\left(\sqrt{a}a^{pk}\right)_{,k}\phi_{,p}=0\\
\Leftrightarrow\spatie &\sqrt{a} \phi_{,m}\left(a^{mn}\right)_{,n}+\sqrt{a}a^{mn}\phi_{,mn}+a^{mn}\phi_{,m}\left(\sqrt{a}\right)_{,n}=0\\
\Leftrightarrow\spatie &\left(\sqrt{a}a^{mn}\phi_{,m}\right)_{,n}=0
\end{align}
 $$\blacklozenge$$
\newpage