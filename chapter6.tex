\setcounter{chapter}{4}
\chapter{Applications to Hydrodynamics, Elasticity, and Electromagnetic radiation}
\pagebreak[4]
\section{p191 - Exercise}
\begin{tcolorbox}
A fluid rotates as a rigid body about the axis of $z_3$ with variable angular velocity $\omega(t)$. Write out explicitly the three Lagrangian equations $\mathbf{6.101}$ and the three Eulerian equations $\mathbf{6.103}$.
\end{tcolorbox}
The motion described reduces to a motion in a $V_2$ plane with $z_3$ a constant for a definite particle.\\\\\\
\textbf{Lagrangian}\\

A particular particle with starting coordinates $\left(z^{(*)}_1,z^{(*)}_2, z^{(*)}_3\right)$ will describe a  circle with radius $\sqrt{z^{(*)2}_1+z^{(*)2}_2}$ in the  plane $V_2$ parallel with the $1,2$ axes.
Taking axis $1$ as reference to determine the instantaneous angle $\theta$ of the vertex $OP$ (origin and particle) we get
\begin{align}
\left\{\begin{array}{l}
z_1 = \sqrt{z^{(*)2}_1+z^{(*)2}_2}\cos\left(\theta(t) + \phi_0\right)\\
z_2 = \sqrt{z^{(*)2}_1+z^{(*)2}_2}\sin\left(\theta(t) + \phi_0\right)\\
z_3 = z^{(*)}_3
\end{array}\right.
\end{align}
with 
\begin{align}
\phi_0 & = \arctan\frac{z^{(*)}_2}{z^{(*)}_1}
\end{align}
Note that $\omega(t)$ is not a constant, so
\begin{align}
\theta(t)&= \int_{0}^{t}\omega(\tau)d\tau
\end{align}
and get 
\begin{align}
\left\{\begin{array}{l}
z_1 = \sqrt{z^{(*)2}_1+z^{(*)2}_2}\cos\left(\int_{0}^{t}\omega(\tau)d\tau + \phi_0\right)\\\\
z_2 = \sqrt{z^{(*)2}_1+z^{(*)2}_2}\sin\left(\int_{0}^{t}\omega(\tau)d\tau + \phi_0\right)\\\\
z_3 = z^{(*)}_3\\\\
\phi_0 = \arctan\frac{z^{(*)}_2}{z^{(*)}_1}
\end{array}\right.
\end{align}
$$\lozenge$$\\
\newpage
\textbf{Eulerian}\\
The equations get simplified and reduce to a motion of a particle on a circle.
\begin{figure}[H]
\centering
\tikzstyle{left-hand-mirror} = [
    draw,
    postaction=decorate, 
    decoration={
        markings,
        mark=between positions 0.015 and 0.98 step 0.1072 with {\draw (0,0)--(60:3pt);}
    }
]   

\begin{tikzpicture}[scale=0.5]
\coordinate (O1) at (0,0);
\draw  (O1) circle (6);
\coordinate (Om) at (0.0,-6) ;
\coordinate  {} {} {};
\coordinate  {} {} {};
\coordinate (NPole) at (0,8.5) {} {};
\coordinate (SPole) at (0,-7.5) {} {};
\draw[-{Latex[length=2mm]}] (SPole) -- (NPole);
\coordinate (P) at (4.2426,4.2426) {} {};
\draw[fill=white]  (P) circle (0.1);
\draw[fill=white]  (O1) circle (0.05);
\draw[dashed] (O1) -- (P);
%\node[label=west:$O$] at (O1) {};
\node[label=east:$P\left(x \text{, } y\right)$] at (P) {};
\coordinate (O2) at (1.5,0) {};
\node[label=north east:$\theta\text{=} \arctan\frac{y}{x}$] at (O2) {};
\draw [dashed, decoration={markings, mark=at position 0.7 with {\arrow[scale = 1.]{Latex[length=2mm]}}},    postaction={decorate}](O2) arc (0:30:2 and 2.0);
\coordinate (F) at (1.5,7) {} {} {};
\draw [-{Latex[length=2mm]}, ] (P) -- (F);
\node[label=east:$\mathbf{\overline{v}}$] at (F) {};
\coordinate (Fp) at (-3,3) {} {} {} {};
\draw [-{Latex[length=2mm]},dashed] (O1) -- (Fp);
\node[label=west:$\mathbf{\overline{v}}$] at (Fp) {};
\tkzMarkRightAngle[size=0.43](F,P,O1);
\tkzMarkRightAngle[size=0.43, ](P,O1,Fp);
\coordinate (x1)at (-7,0) {};
\coordinate (X) at (8.5,0) {} {};
\draw [-{Latex[length=2mm]}] (x1) -- (X);
\node[label=south east:$x$] at (X) {};
\node[label=south east:$y$] at (NPole) {};
\coordinate (xx)at (-3,0) {};
\draw[fill=white]  (xx) circle (0.1);
\coordinate (yy)at (0,3) {};
\draw[fill=white]  (yy) circle (0.1);
\draw [dashed, ] (xx) -- (Fp);
\draw [dashed ] (yy) -- (Fp);
\node[label=south :$-sin\left(\theta\right)$] at (xx) {};
\node[label= east:$cos\left(\theta\right)$] at (yy) {};
\end{tikzpicture}
\caption{Eulerian viewpoint of a spinning fluid}
\label{fig:fig_p191_Exa}
\end{figure}

\begin{align}
\left\{\begin{array}{l}
v_1 = -\sqrt{z^{2}_1+z^{2}_2}\omega(t)\sin\left(\arctan\frac{z^{}_2}{z^{}_1}\right)\\\\
v_2 = \sqrt{z^{2}_1+z^{2}_2}\omega(t)\cos\left(\arctan\frac{z^{}_2}{z^{}_1}\right)\\\\
v_3 = 0
\end{array}\right.
\end{align}
$$\blacklozenge$$
\newpage

\section{p191 - Exercise}
\begin{tcolorbox}
Compute the components of acceleration for the motion described in the preceding exercise.
\end{tcolorbox}
We have
\begin{align}
\left\{\begin{array}{l}
v_1 = -\sqrt{z^{2}_1+z^{2}_2}\omega(t)\sin\left(\arctan\frac{z^{}_2}{z^{}_1}\right)\\\\
v_2 = \sqrt{z^{2}_1+z^{2}_2}\omega(t)\cos\left(\arctan\frac{z^{}_2}{z^{}_1}\right)\\\\
v_3 = 0
\end{array}\right.
\end{align}
and 
\begin{align}
f_r &= \partial_tv_r + v_{r,s}v_s
\end{align}
\begin{align}
&\left\{\begin{array}{l}
\partial_t v_1 = -\sqrt{z^{2}_1+z^{2}_2}\dot{\omega}(t)\sin\left(\arctan\frac{z^{}_2}{z^{}_1}\right)\\\\
\partial_t v_2 = \sqrt{z^{2}_1+z^{2}_2}\dot{\omega}(t)\cos\left(\arctan\frac{z^{}_2}{z^{}_1}\right)\\\\
v_{1,1}= -\omega(t)\left[\frac{z_1}{\sqrt{z^{2}_1+z^{2}_2}}\sin\left(\arctan\frac{z^{}_2}{z^{}_1}\right) -\sqrt{z^{2}_1+z^{2}_2}\cos\left(\arctan\frac{z^{}_2}{z^{}_1}\right)\frac{z_2}{z_1^2}\frac{1}{1+\frac{z_2^2}{z_1^2}}\right]\\\\
v_{1,2}= -\omega(t)\left[\frac{z_2}{\sqrt{z^{2}_1+z^{2}_2}}\sin\left(\arctan\frac{z^{}_2}{z^{}_1}\right) +\sqrt{z^{2}_1+z^{2}_2}\cos\left(\arctan\frac{z^{}_2}{z^{}_1}\right)\frac{1}{z^{}_1}\frac{1}{1+\frac{z_2^2}{z_1^2}}\right]\\\\
v_{2,1}= \omega(t)\left[\frac{z_1}{\sqrt{z^{2}_1+z^{2}_2}}\cos\left(\arctan\frac{z^{}_2}{z^{}_1}\right) +\sqrt{z^{2}_1+z^{2}_2}\sin\left(\arctan\frac{z^{}_2}{z^{}_1}\right)\frac{z_2}{z_1^2}\frac{1}{1+\frac{z_2^2}{z_1^2}}\right]\\\\
v_{2,2}= \omega(t)\left[\frac{z_2}{\sqrt{z^{2}_1+z^{2}_2}}\cos\left(\arctan\frac{z^{}_2}{z^{}_1}\right) -\sqrt{z^{2}_1+z^{2}_2}\sin\left(\arctan\frac{z^{}_2}{z^{}_1}\right)\frac{1}{z^{}_1}\frac{1}{1+\frac{z_2^2}{z_1^2}}\right]\\\\
\end{array}\right.
\end{align}
\begin{align}
&=\left\{\begin{array}{l}
\partial_t v_1 = -\sqrt{z^{2}_1+z^{2}_2}\dot{\omega}(t)\sin\left(\arctan\frac{z^{}_2}{z^{}_1}\right)\\\\
\partial_t v_2 = \sqrt{z^{2}_1+z^{2}_2}\dot{\omega}(t)\cos\left(\arctan\frac{z^{}_2}{z^{}_1}\right)\\\\
v_{1,1}= -\frac{\omega(t)}{\sqrt{z^{2}_1+z^{2}_2}}\left[z_1\sin\left(\arctan\frac{z^{}_2}{z^{}_1}\right) -z_2\cos\left(\arctan\frac{z^{}_2}{z^{}_1}\right)\right]\\\\
v_{1,2}=  -\frac{\omega(t)}{\sqrt{z^{2}_1+z^{2}_2}}\left[z_2\sin\left(\arctan\frac{z^{}_2}{z^{}_1}\right) +z^{}_1\cos\left(\arctan\frac{z^{}_2}{z^{}_1}\right)\right]\\\\
v_{2,1}=  \frac{\omega(t)}{\sqrt{z^{2}_1+z^{2}_2}}\left[z_1\cos\left(\arctan\frac{z^{}_2}{z^{}_1}\right) +z_2\sin\left(\arctan\frac{z^{}_2}{z^{}_1}\right)\right]\\\\
v_{2,2}=  \frac{\omega(t)}{\sqrt{z^{2}_1+z^{2}_2}}\left[z_2\cos\left(\arctan\frac{z^{}_2}{z^{}_1}\right) -z^{}_1\sin\left(\arctan\frac{z^{}_2}{z^{}_1}\right)\right]\\\\
\end{array}\right.
\end{align}
and get
\begin{align}
v_{1,s}v_s &= \left\{\begin{array}{l}
-\sqrt{z^{2}_1+z^{2}_2}\omega(t)\sin\left(\arctan\frac{z^{}_2}{z^{}_1}\right)\frac{\omega(t)}{\sqrt{z^{2}_1+z^{2}_2}}\left[z_1\sin\left(\arctan\frac{z^{}_2}{z^{}_1}\right) -z_2\cos\left(\arctan\frac{z^{}_2}{z^{}_1}\right)\right]\\\\
-\sqrt{z^{2}_1+z^{2}_2}\omega(t)\cos\left(\arctan\frac{z^{}_2}{z^{}_1}\right)\frac{\omega(t)}{\sqrt{z^{2}_1+z^{2}_2}}\left[z_2\sin\left(\arctan\frac{z^{}_2}{z^{}_1}\right) +z^{}_1\cos\left(\arctan\frac{z^{}_2}{z^{}_1}\right)\right]
\end{array}\right.\\
&=-z_1\omega^2(t)
\end{align}
\begin{align}
v_{2,s}v_s &= \left\{\begin{array}{l}
\sqrt{z^{2}_1+z^{2}_2}\omega(t)\sin\left(\arctan\frac{z^{}_2}{z^{}_1}\right)\frac{\omega(t)}{\sqrt{z^{2}_1+z^{2}_2}}\left[z_1\cos\left(\arctan\frac{z^{}_2}{z^{}_1}\right) +z_2\sin\left(\arctan\frac{z^{}_2}{z^{}_1}\right)\right]\\\\
+\sqrt{z^{2}_1+z^{2}_2}\omega(t)\cos\left(\arctan\frac{z^{}_2}{z^{}_1}\right)\frac{\omega(t)}{\sqrt{z^{2}_1+z^{2}_2}}\left[z_2\cos\left(\arctan\frac{z^{}_2}{z^{}_1}\right) -z^{}_1\sin\left(\arctan\frac{z^{}_2}{z^{}_1}\right)\right]\\\\
\end{array}\right.\\
&= z_2\omega^2(t)
\end{align}
giving with the second derivative term
\begin{align}
&=\left\{\begin{array}{l}
f_1 = -\dot{\omega}(t)\sqrt{z^{2}_1+z^{2}_2}\sin\left(\arctan\frac{z^{}_2}{z^{}_1}\right)-z_1\omega^2(t)\\\\\\
f_2 = \dot{\omega}(t)\sqrt{z^{2}_1+z^{2}_2}\cos\left(\arctan\frac{z^{}_2}{z^{}_1}\right)+z_2\omega^2(t)\\\\
f_3=0
\end{array}\right.
\end{align}
$$\blacklozenge$$
\newpage


\section{p193 - Exercise}
\begin{tcolorbox}
Verify that the operator $\frac{\partial}{\partial t}$ does not alter tensor character.
\end{tcolorbox}
Be $X^r$ and $Y^r$, two tensors so that $I=X_rY^r$ is an invariant. Obviously, $\frac{\partial I}{\partial t}$ will also be invariant and 
\begin{align}
\frac{\partial I}{\partial t}= \frac{\partial X_r}{\partial t}Y^r+X_r\frac{\partial Y^r}{\partial t}
\end{align}
Meaning that the right side is a sum of two invariants, from which we conclude (see page 20, $\mathbf{1.607}$) that $\frac{\partial X_r}{\partial t}$ and $\frac{\partial Y^r}{\partial t}$ are tensors.
$$\blacklozenge$$
\newpage

\section{p193 - Clarification to 6.112}
\begin{tcolorbox}
$$\mathbf{6.112.}\spatie \int Fn_rdS = \int F_{,r}dV$$
\end{tcolorbox}
Green's theorem is generally presented in the form
$$\int\overline{F}.\overline{n}dS = \int \overline{\nabla}. \overline{F}dV$$or
$$\int F_rn_rdS = \int F_{r,r}dV$$
We can define $$\overline{F} = F\overline{1}_r$$
 $\overline{F}.\overline{n}$  will then become $Fn_r$ while $\overline{\nabla}. \overline{F}$ will become $\partial_r F$, giving the expression $\mathbf{6.112.}$.
 $$\blacklozenge$$
\newpage



\section{p196 - Exercise}
\begin{tcolorbox}
Write out $\mathbf{6.126}$ and $\mathbf{6.127b}$ explicitly for spherical polar coordinates.
\end{tcolorbox}
For spherical polar coordinates we have 
\begin{align}
(v^r) = \left(\begin{matrix}\dot{r}\\r\dot{\theta}\\r\sin\theta\dot{\phi} \end{matrix}\right)
\end{align}
and (see $\mathbf{2.546}$ page 58):
\begin{align}
v^r_{|r} &= \frac{1}{r^2}\partial_r\left(r^2v^1\right)+\frac{1}{\sin\theta}\partial_{\theta}\left(\sin\theta v^2\right)+\partial_{\phi} v^3\\
&=  \frac{1}{r^2}\left(2rv^1+r^2\partial_rv^1\right)+\frac{1}{\sin\theta}\left(v^2\cos\theta + \sin\theta\partial_{\theta}v^2\right)\\
&= \frac{2}{r}\dot{r}+r\dot{\theta}\cot \theta 
\end{align}
and 
\begin{align}
&\partial_t \rho+ \left(\rho v^r\right)_{|r} =0\\
\Leftrightarrow\spatie &\partial_t\rho+ \rho_{|r}v^r+\rho v^r_{|r} =0\\
\Leftrightarrow\spatie &\dv{\rho}{t}+\rho \left(\frac{2}{r}\dot{r}+r\dot{\theta}\cot \theta \right) =0
\end{align}
 $$\blacklozenge$$
\newpage


\section{p198 - Exercise}
\begin{tcolorbox}
If $\epsilon^{rmn}$ is defined in precisely the same way as $\epsilon_{rmn}$, prove that $$ \epsilon^{'uvw}= J\epsilon^{rmn}\frac{\partial x^{'u}}{\partial x^r}\frac{\partial x^{'v}}{\partial x^m}\frac{\partial x^{'w}}{\partial x^n}$$
\end{tcolorbox}
We follow the pretty same line of reasoning as for $\epsilon_{rst}$. Going from $x^{'r}$ to $x^{s}$ we have (expanding the determinant of the inverse Jacobian along the rows instead of the columns):
\begin{align}
J^{-1}=\left|\frac{\partial x^{'p}}{\partial x^{q}}\right|&= \epsilon^{rmn}\frac{\partial x^{'1}}{\partial x^r}\frac{\partial x^{'2}}{\partial x^m}\frac{\partial x^{'3}}{\partial x^n}\\
\times \epsilon^{uvw}\spatie J^{-1}\epsilon^{uvw}&= \epsilon^{rmn}\frac{\partial x^{'u}}{\partial x^r}\frac{\partial x^{'v}}{\partial x^m}\frac{\partial x^{'w}}{\partial x^n}\\
\times J\spatie\epsilon^{uvw}&= J\epsilon^{rmn}\frac{\partial x^{'u}}{\partial x^r}\frac{\partial x^{'v}}{\partial x^m}\frac{\partial x^{'w}}{\partial x^n}\\ 
\end{align}
 $$\blacklozenge$$
\newpage



\section{p198 - Exercise}
\begin{tcolorbox}
Prove that  $\frac{\epsilon^{rmn}}{\sqrt{a}}$ is an (absolute) contravariant tensor of the third order.
\end{tcolorbox}
\begin{align} 
\sqrt{a^{'}} &= J\sqrt{a}\\
\epsilon^{'uvw}&= J\epsilon^{rmn}\frac{\partial x^{'u}}{\partial x^r}\frac{\partial x^{'v}}{\partial x^m}\frac{\partial x^{'w}}{\partial x^n}\\
\text{(1)  in (2)} \spatie \epsilon^{'uvw}&= \frac{\sqrt{a^{'}}}{\sqrt{a^{}}}\epsilon^{rmn}\frac{\partial x^{'u}}{\partial x^r}\frac{\partial x^{'v}}{\partial x^m}\frac{\partial x^{'w}}{\partial x^n}\\
\Rightarrow \spatie \frac{\epsilon^{'uvw}}{\sqrt{a^{'}}}&= \frac{\epsilon^{rmn}}{\sqrt{a}}\frac{\partial x^{'u}}{\partial x^r}\frac{\partial x^{'v}}{\partial x^m}\frac{\partial x^{'w}}{\partial x^n}
\end{align}
which is the required transformation rule for a "normal" (absolute) tensor.
 $$\blacklozenge$$
\newpage


\section{p199 - Clarification to pressure invariance to direction of the surface element.}
\begin{tcolorbox}
Pressure is independent of the direction
\end{tcolorbox}
\begin{figure}[H]%
    \centering
    \subfloat[]{\tdplotsetmaincoords{80}{160}
\begin{tikzpicture}[tdplot_main_coords, >=Latex]
\tikzmath{\aax=3.84;\bby=6;\ccz=4;\a = (\bby+\bby)/4;\d=-8.5;\e=4;};
\aax, \bby,\ccz;
\coordinate (O) at (0,0,0);
\coordinate (A) at (0,0,\ccz);
\coordinate (B) at (0,\bby,0);
\coordinate (C) at (\aax,0,0);
\coordinate (x) at (1.5*\aax,0,0);
\coordinate (y) at (0,1.5*\bby,0);
\coordinate (z) at (0,0,1.5*\ccz);

\coordinate (xy) at (\aax/3,\bby/3,0);
\draw [fill=white] (xy) circle (2pt) node[above right] (n1) {$$};
\coordinate (xz) at (\aax/3,0,\ccz/3);
\draw [fill=white] (xz) circle (2pt) node[above right] (n1) {$$};
\coordinate (yz) at (0,\bby/3,\ccz/3);
\draw [fill=white] (yz) circle (2pt) node[above right] (n1) {$$};
\coordinate (xyz) at (\aax/3,\bby/3,\ccz/3);
\draw [fill=gray] (xyz) circle (2pt) node[above right] (n1) {$$};

\coordinate (nxy) at (\aax/3,\bby/3,-2);
\coordinate (nxyO) at (\aax/3,\bby/3,-0.2);
\draw[dashed, thick](xy)--(nxyO);
\draw[-Latex,very thick](nxyO)--(nxy);
\coordinate (nxz) at (\aax/3,-5,\ccz/3);
\coordinate (nxzO) at (\aax/3,-2.5,\ccz/3);
\draw[dashed,very thick](xz)--(nxzO);
\draw[-Latex,very thick](nxzO)--(nxz);
\coordinate (nyz) at (-2,\bby/3,\ccz/3);
\coordinate (nyzO) at (-0.5,\bby/3,\ccz/3);
\draw[dashed,very thick](yz)--(nyzO);
\draw[-Latex,very thick](nyzO)--(nyz);

\coordinate (nxyz) at (3*\aax/3,3*\bby/3,3*\ccz/3);
\draw[-Latex, very thick](xyz)--(nxyz);
\node[anchor = south east] at (nxy){$dS_z$};
\node[anchor = south east] at (nxz){$dS_y$};
\node[anchor = south east] at (nyz){$dS_x$};
\node[anchor = south east] at (nxyz){$dS_t$};

\coordinate (yzp) at (3*\aax/3,\bby/3,\ccz/3);
\draw[-Latex,](xyz)--(yzp);
\node[anchor = south east] at (yzp){$$};

\coordinate (xyp) at (\aax/3,\bby/3,3.1*\ccz/3);
\draw[-Latex,](xyz)--(xyp);
\node[anchor = south east] at (xyp){$$};

\coordinate (xzp) at (\aax/3,3*\bby/3,\ccz/3);
\draw[-Latex, ](xyz)--(xzp);
\node[anchor = north west] at (xzp){$$};

\coordinate (pz) at (\aax/3,\bby/3,3.1*\ccz/3);
\draw[ dashed](nxyz)--(pz);
\draw [fill=white] (pz) circle (1pt)node[above left] (pz) {$\gamma_z$};

\coordinate (pxy) at (3.*\aax/3,3*\bby/3,\ccz/3);
\draw[ dashed](nxyz)--(pxy);
\draw[ dashed](xyz)--(pxy);
\draw [fill=white] (pxy) circle (1pt) {};

\coordinate (px) at (3*\aax/3,\bby/3,\ccz/3);
\draw[ dashed](pxy)--(px);
\draw [fill=white] (px) circle (1pt) node[above left] (px) {$\gamma_x$};
\coordinate (py) at (\aax/3,3*\bby/3,\ccz/3);
\draw[ dashed](pxy)--(py);
\draw [fill=white] (py) circle (1pt)node[above right] (py) {$\gamma_y$};


\draw[-Latex](O)--(x);
\node[anchor = south east] at (x){x};
\draw[-Latex](O)--(y);
\node[anchor = south east] at (y){y};
\draw[-Latex](O)--(z);
\node[anchor = south east] at (z){z};

\draw[very thick,](A)--(B)--(C)--cycle;

\draw[dashed, very thick](O)--(B);
\draw[dashed, very thick](O)--(C);
\draw[dashed,very thick](O)--(A);

\node[anchor = south west] at (A){A};
\node[anchor = north east] at (O){o};
\node[anchor = north east] at (C){C};
\node[anchor = south west ] at (B){B};

\end{tikzpicture}    }
    \quad
        \subfloat[]{\tdplotsetmaincoords{80}{160}
\begin{tikzpicture}[scale = 0.8,tdplot_main_coords, >=Latex]
\tikzmath{\aax=3.84;\bby=6;\ccz=4;\a = (\bby+\bby)/4;\d=-8.5;\e=4;};
\aax, \bby,\ccz;
\coordinate (O) at (0,0,0);
\coordinate (A) at (0,0,\ccz);
\coordinate (B) at (0,\bby,0);
\coordinate (C) at (\aax,0,0);
\coordinate (x) at (1.5*\aax,0,0);
\coordinate (y) at (0,1.5*\bby,0);
\coordinate (z) at (0,0,1.5*\ccz);

\coordinate (xyz) at (\aax/3,\bby/3,\ccz/3);
\draw [fill=gray] (xyz) circle (2pt) node[above right] (n1) {$$};


\coordinate (nxyz) at (3*\aax/3,3*\bby/3,3*\ccz/3);
\draw[-Latex, very thick](xyz)--(nxyz);
\node[anchor = south east] at (nxyz){$dS_t$};

\coordinate (yzp) at (3*\aax/3,\bby/3,\ccz/3);
\draw[-Latex,](xyz)--(yzp);
\node[anchor = south east] at (yzp){$$};

\coordinate (xyp) at (\aax/3,\bby/3,3.1*\ccz/3);
\draw[-Latex,](xyz)--(xyp);
\node[anchor = south east] at (xyp){$$};

\coordinate (xzp) at (\aax/3,3*\bby/3,\ccz/3);
\draw[-Latex, ](xyz)--(xzp);
\node[anchor = north west] at (xzp){$$};
\node[anchor = north east] at (xyz){$q_{t}$};

\coordinate (pz) at (\aax/3,\bby/3,3.1*\ccz/3);

\draw [fill=white] (pz) circle (1pt)node[above left] (pz) {$\gamma_z$};

\coordinate (pxy) at (3.*\aax/3,3*\bby/3,\ccz/3);


\coordinate (px) at (3*\aax/3,\bby/3,\ccz/3);

\draw [fill=white] (px) circle (1pt) node[above left] (px) {$\gamma_x$};
\coordinate (py) at (\aax/3,3*\bby/3,\ccz/3);

\draw [fill=white] (py) circle (1pt)node[above right] (py) {$\gamma_y$};


\draw[-Latex](O)--(x);
\node[anchor = south east] at (x){x};
\draw[-Latex](O)--(y);
\node[anchor = south east] at (y){y};
\draw[-Latex](O)--(z);
\node[anchor = south east] at (z){z};

\draw[,](A)--(B)--(C)--cycle;
\draw[](A)--(O)--(C)--cycle;
\draw[](A)--(O)--(B)--cycle;
\draw[](O)--(B)--(C)--cycle;

\node[anchor = south west] at (A){A};
\node[anchor = south west] at (O){o};
\node[anchor = north east] at (C){C};
\node[anchor = south west ] at (B){B};
%\draw[-Latex, ultra thick](xyz)--(A);
\coordinate (Ht) at (\aax/2,\bby/2,0);
\draw[dashed,ultra thick](A)--(Ht);
\draw [fill=white] (Ht) circle (1pt)node[below left] (Ht) {$h$};
\draw[dashed,ultra thick](O)--(Ht);

\end{tikzpicture}    }
    \quad
        \subfloat[]{\begin{tikzpicture}[scale = 0.8]
\coordinate (O) at (0,0);
\node[anchor = north east ] at (O){$h$};
\coordinate (ds) at (-2,3);
\draw[-Latex,thick] (O)--(ds);
\node[anchor = south west ] at (ds){$dS_t$};
\coordinate (A) at (4,3.5) {};
\draw[-Latex,thick] (O)--(A);
\node[anchor = south west ] at (A){$A$};
\coordinate (gamma) at (0,4.5) {};
\draw[-Latex,thick] (O)--(gamma);
\node[anchor = south west ] at (gamma){$\gamma_z$};
 \draw pic[draw,,angle radius=3cm,"$\frac{\pi}{2}-\alpha$"] {angle=A--O--gamma};
  \draw pic[draw,,angle radius=2cm,"$\alpha$"] {angle=gamma--O--ds};
  \coordinate (P) at (5,0);
  \draw[dashed] (O)--(P);
  \draw pic[draw,,angle radius=2cm,"$\alpha$"] {angle=P--O--A};
\end{tikzpicture}
}
    \quad
\caption{Pressure on a trirectangular tetrahedron}
\label{fig:fig_p199}
\end{figure}
To see that the pressure is independent of the direction of the surface element on which we measure it, let's consider a trirectangular tetrahedron $OABC$ as  depicted in figure $5.2(a)$. Let's define $P_x,P_y,P_z, P_t$ the pressure measured on the $4$ surfaces with normal vectors $dS_x,dS_y,dS_z,dS_t$. Let's neglect second order terms due to acceleration and external forces. For the forces along axis $z$ (the same reasoning is valid for the two others) we will have $P_zdS_z= P_tdS_t\gamma_z$ where $\gamma_z$ is the cosine of the angle formed by  normal on $dS_t$ and the z-axis.\\
Let's investigate the relationship between $dS_t$ and $dS_x,dS_y,dS_z$.\\
 Be $hA$ the line element lying in the plane $ABC$ (see figure $5.2(b)$) and $hO$ the line element lying in the plane $OBC$. As the area of a triangle $=\half\times base \times perpendicular \ height$ we get $dS_z = \half |hO||BC|$. But $|hO|= |hA|\cos\alpha$  (see figure $5.2(c)$) and so $dS_z = \half |BC||hA|\cos\alpha = dS_t\gamma_z$
 and get 
\begin{align}
&P_zdS_z= P_tdS_t\gamma_z\\ 
\Rightarrow \spatie & P_z dS_t\gamma_z= P_tdS_t\gamma_z\\ 
\Rightarrow \spatie & P_z = P_t
\end{align}

 $$\blacklozenge$$
\newpage
\section{p201 - Exercise}
\begin{tcolorbox}
Write down the contravariant form of $\mathbf{6.147}$
$$\mathbf{6.147} \spatie \partial_t v_r + v_s v_{r|s} = X_r - \rho^{-1} p_{,r}$$
\end{tcolorbox}
\begin{align}
\mathbf{6.147} \spatie \partial_t v_r + v_s v_{r|s} &= X_r - \rho^{-1} p_{,r}\\
\times a^{mr}\spatie \partial_t a^{mr} v_r + v_s a^{mr}v_{r|s} &= a^{mr}X_r - \rho^{-1} a^{mr}p_{,r}
\end{align}
By $\mathbf{2.527}$ page 53 we  have $a^{rs}_{|t}=0$ and thus 
\begin{align}
v^{m}_{\ |s}= \left(a^{mr}v_{r}\right)_{|s} &= \left(a^{mr}\right)_{|n}v_{r}+a^{mr}v_{r|s}= a^{mr}v_{r|s} \\
(2) \Rightarrow \spatie \partial_t  v^m + v_s v^{m}_{\ \ |s} &= X^m - \rho^{-1} a^{mr}p_{,r}\\
 \Rightarrow \spatie \partial_t  v^m + v_s v^{m}_{\ \ |s} &= X^m - \rho^{-1} p^{'}_{,r}
\end{align}
Note that $p_{,r}$ in $(1)$ and $(5)$ are not the same vector function as $p_{,r}$ can be written as $\overline{\nabla}p$ which is coordinate system  dependent.
 $$\blacklozenge$$
\newpage

\section{p202 - Exercise}
\begin{tcolorbox}
Verify by means of  $\mathbf{3.204}$ that $\mathbf{6.157}$ and $\mathbf{6.156}$  are the same equation.
\end{tcolorbox}
\begin{align}
\left\{\begin{array}{lll}
\mathbf{3.204}&\spatie&\Gamma^{n}_{rn} = \half\partial_r\log a = \partial_r\log \sqrt{a}\\
\mathbf{6.157}&\spatie&\left(\sqrt{a}a^{mn}\phi_{,m}\right)_{,n}=0\\
\mathbf{6.156}&\spatie&a^{mn}\phi_{|mn}=0
\end{array}\right.
\end{align}
Considering also
\begin{align}
\left\{\begin{array}{l}
a^{mn}_{|k}=0\\
T^m_{|n}= \partial_n T_m+\Gamma^{m}_{kn}T_k
\end{array}\right.
\end{align}
So $\mathbf{6.156}$ can written as
\begin{align}
& a^{mn}\phi_{|mn}=0\\
\Leftrightarrow\spatie & \left(a^{mn}\phi_{|m}\right)_{|n}=0\\
T^n = a^{mn}\phi_{|m}\spatie\Rightarrow\spatie &\partial_n T^n+\Gamma^{n}_{kn}T^k=0\\
\Rightarrow\spatie &\partial_n \left(a^{mn}\phi_{|m}\right)+\Gamma^{n}_{kn}a^{pk}\phi_{|p}=0\\
\Leftrightarrow\spatie &\partial_n \left(a^{mn}\right) \phi_{,m}+a^{mn}\phi_{,mn}+\Gamma^{n}_{kn}a^{pk}\phi_{,p}=0\\
\text{(2)}\quad \Rightarrow\spatie &\partial_n \left(a^{mn}\right) \phi_{,m}+a^{mn}\phi_{,mn}+\partial_k\log \sqrt{a}a^{pk}\phi_{,p}=0\\
\Leftrightarrow\spatie &\left(a^{mn}_{,n}\right) \phi_{,m}+a^{mn}\phi_{,mn}+\frac{1}{\sqrt{a}}\left(\sqrt{a}a^{pk}\right)_{,k}\phi_{,p}=0\\
\Leftrightarrow\spatie &\sqrt{a} \phi_{,m}\left(a^{mn}\right)_{,n}+\sqrt{a}a^{mn}\phi_{,mn}+a^{mn}\phi_{,m}\left(\sqrt{a}\right)_{,n}=0\\
\Leftrightarrow\spatie &\left(\sqrt{a}a^{mn}\phi_{,m}\right)_{,n}=0
\end{align}
 $$\blacklozenge$$
\newpage


\section{p205 - Exercise}
\begin{tcolorbox}
Show that a small strain is a rigid body displacement if, and only if, $e_{rs}=0$. In the case of finite strain, deduce form $\mathbf{6.206}$ the conditions which must be satisfied by the partial derivatives of the displacement in order that it may b a rigid body displacement.
\end{tcolorbox}
\textbf{Suppose we deal with  a rigid body.} Then the position of two particles of the rigid body are given by

\begin{align}
\left\{\begin{array}{l}
p_r=z_r+u_r(z)\\
p^{'}_r=z^{'}_r+u_r(z^{'})\\
\end{array}\right.
\end{align}
and
\begin{align}
\left\{\begin{array}{l}
L_0= z_r-z^{'}_r \\
L_1 = z^{'}_r+u_r(z^{'})-z_r-u_r(z)  
\end{array}\right.
\end{align}
A rigid body means $L_1=L_0$, giving $u_r(z^{'})=u_r(z^{})$ i.e $u_r(z^{})$ is a constant and thus $u_{r,s}(z^{})=0$.\\
As $e_{rs}=\half\left(u_{r,s}(z^{})+u_{s,r}(z^{})\right)$ we get $e_{rs}=0$\\\\
\textbf{Suppose now that $e_{rs}=0$}\\
We have $e= e_{rs}\lambda_r\lambda_s = 0$ and $e= u_{r,s}(z^{})\lambda_r\lambda_s $
As the $\lambda_r$ are arbitrary, in the sense that we are free to choose whatever curve to approach the initial point, we conclude that $u_{r,s}(z^{})=0$ . So$u_{r}(z^{})$ is a constant, meaning that the mutual distance between two arbitrary points, do not change. The body is a rigid body.
 $$\lozenge$$
 For a finite strain we have $\mathbf{6.206}$:
 \begin{align}
 \lim \frac{L_1^2-L_0^2}{L_0^2}=2 u_{r,s}(z^{})\lambda_r\lambda_s + u_{m,r}(z^{})u_{m,s}(z^{})\lambda_r\lambda_s
 \end{align}
 This limit is $0$ and so we get as condition
  \begin{align}
 \left(2 u_{r,s}(z^{}) + u_{m,r}(z^{})u_{m,s}(z^{})\right)\lambda_r\lambda_s=0
 \end{align}
 As the $\lambda_r$ are arbitrary, in the sense that we are free to choose whatever curve to approach the initial point, we conclude that $2 u_{r,s}(z^{}) + u_{m,r}(z^{})u_{m,s}(z^{})$ must be zero.
 $$2 u_{r,s}(z^{}) + u_{m,r}(z^{})u_{m,s}(z^{})=0$$
 $$\blacklozenge$$
\newpage

\section{p207 - Clarification}
\begin{tcolorbox}
Then, clearly, since the the volume of the tetrahedron is less than $a^3$,\\\\
$\mathbf{6.217}\spatie \displaystyle \lim_{a \to 0} \frac{1}{a^2} \dv{M_r}{t}=0,\quad \displaystyle \lim_{a \to 0} \frac{1}{a^2} \int X_rdV=0$
$$\vdots$$
But  $\displaystyle \lim_{a \to 0} \frac{S}{a^2}$ is not zero,...
\end{tcolorbox}
First, note that the volume of  a trirectangular tetrahedron is $V=\frac{1}{6}abc$ with $a,b,c$ the bases of the 3 rectangular triangles (see clarification for page 199), so if $a\ge b,c$ we have  $V< a^3$.\\
There is no assurance that $\displaystyle \lim_{a \to 0} \frac{1}{a^3} \dv{M_r}{t}=0$. Indeed, consider $\mathbf{5.334}$. For a continuous medium, this equation can be written as
\begin{align}
I_{st} &= \int_{V}\rho\epsilon_{ptq}\epsilon_{psn}z_sz_tdV
\end{align} 
If $V$ goes to zero, the quantities under the integral can be approximated by constant values and hence, the dynamics of the tetrahedron are govern by
\begin{align}
&\displaystyle \lim_{V \to 0}I_{st} = \rho\epsilon_{ptq}\epsilon_{psn}z_sz_t V\\
\mathbf{5.332}\text{:}\spatie &\dv{I_{st}\omega_t}{t}= M_s\\
\Rightarrow\spatie &\lim_{V \to 0} \dv{M_s}{t} = \lim_{V \to 0}V\dv{\left(\rho\epsilon_{ptq}\epsilon_{psn}z_sz_t\omega_t\right)}{t} \\
\Rightarrow\spatie &\lim_{a \to 0} \frac{1}{a^3}\dv{M_s}{t} = \lim_{a \to 0}\frac{1}{a^3}V\dv{\left(\rho\epsilon_{ptq}\epsilon_{psn}z_sz_t\omega_t\right)}{t} \\
\Rightarrow\spatie &\lim_{a \to 0} \frac{1}{a^3}\dv{M_s}{t} < \lim_{a \to 0}\frac{1}{a^3}a^3\dv{\left(\rho\epsilon_{ptq}\epsilon_{psn}z_sz_t\omega_t\right)}{t} \\
\Rightarrow\spatie &\lim_{a \to 0} \frac{1}{a^3}\dv{M_s}{t} < \lim_{a \to 0}\dv{\left(\rho\epsilon_{ptq}\epsilon_{psn}z_sz_t\omega_t\right)}{t}
\end{align}
but there is no reason to admit that $\displaystyle \lim_{V \to 0}\dv{\left(\rho\epsilon_{ptq}\epsilon_{psn}z_sz_t\omega_t\right)}{t}=0$. \\
On the other hand, replacing $a_3$ with $a^2$ in $(5)$ gives 
\begin{align}
&\lim_{a \to 0} \frac{1}{a^2}\dv{M_s}{t} = \lim_{a \to 0}\frac{1}{a^2}V\dv{\left(\rho\epsilon_{ptq}\epsilon_{psn}z_sz_t\omega_t\right)}{t} \\
\Rightarrow\spatie &\lim_{a \to 0} \frac{1}{a^2}\dv{M_s}{t} < \lim_{a \to 0}\frac{1}{a^2}a^3\dv{\left(\rho\epsilon_{ptq}\epsilon_{psn}z_sz_t\omega_t\right)}{t} \\
\Rightarrow\spatie &\lim_{a \to 0} \frac{1}{a^2}\dv{M_s}{t} < \lim_{a \to 0}a\dv{\left(\rho\epsilon_{ptq}\epsilon_{psn}z_sz_t\omega_t\right)}{t}\\
\Rightarrow\spatie &\lim_{a \to 0} \frac{1}{a^2}\dv{M_s}{t} =0
\end{align}
For $ \displaystyle \lim_{a \to 0} \frac{1}{a^2} \int X_rdV=0$ the reasoning is even simpler as for a volume going to zero , we can consider $X_r$ as constant and thus 
\begin{align}
 \displaystyle \lim_{a \to 0} \frac{1}{a^2} \int X_rdV=\displaystyle \lim_{a \to 0} \frac{1}{a^2}  X_r \int dV < \displaystyle \lim_{a \to 0} \frac{1}{a^2}  X_r a^3 = 0
\end{align}
$$\lozenge$$
\textbf{But  $\displaystyle \lim_{a \to 0} \frac{S}{a^2}$ is not zero,...}\\\\
Be $S_t$ the area of the "sloped" triangle in the tetrahedron. Then, the total area of the tetrahedron is:
\begin{align}
&S= \half\left(ab + bc + ac \right) + S_t\\
&S_t > \half ab\text{,   }S_t > \half bc\text{,   }S_t > \half ac\\
\Rightarrow \spatie & S>  \half\left(ab + bc + ac \right) + \half ab\\
\Rightarrow \spatie & \frac{S}{a^2}>  \half\left( \frac{bc}{a^2} + \frac{c}{a} \right) + b
\end{align}
If we shrink the tetrahedron uniformly and put $a=\epsilon a_0, b=\epsilon b_0, c=\epsilon c_0$ then $(16)$ can be written as 
\begin{align}
\lim_{\epsilon \to 0}\frac{S}{a^2}>   \half\left( \frac{b_0c_0}{a_0^2} + \frac{c_0}{a_0} \right) + b_0\lim_{\epsilon \to 0}\epsilon 
\end{align}
which is indeed not zero.
 $$\blacklozenge$$
\newpage

\section{p208 - Exercise}
\begin{tcolorbox}
Show that the stress across a plane $z_1 = \text{const.}$ has the components $E_{11}, \ E_{21}, \ E_{31}$. What are the components across planes $z_2 = \text{const.}$ and $z_3 = \text{const.}$?
\end{tcolorbox}
\begin{align}
\mathbf{6.223} \spatie T_r = E_{rs}n_s
\end{align}
So for the the stress across a plane $z_1 = \text{const.}$, we have $n_1=1, \ n_2=0, \ n_3 = 0$ and so 
\begin{align}
T_r\left(z_1 = \text{const.}\right) = \left(\begin{matrix}E_{11}\\E_{21}\\E_{31}\end{matrix}\right)
\end{align}
For the the stress across a plane $z_2 = \text{const.}$, we have $n_1=0, \ n_2=1, \ n_3 = 0$ and for the the stress across a plane $z_3 = \text{const.}$, we have $n_1=0, \ n_2=0, \ n_3 = 1$ and so
\begin{align}
&T_r\left(z_2 = \text{const.}\right) = \left(\begin{matrix}E_{12}\\E_{22}\\E_{32}\end{matrix}\right)\\
&T_r\left(z_3 = \text{const.}\right) = \left(\begin{matrix}E_{13}\\E_{23}\\E_{33}\end{matrix}\right)\
\end{align}
 $$\blacklozenge$$
\newpage

\section{p210 - Exercise}
\begin{tcolorbox}
Show that the if $\mathbf{6.233}$ is solved for strain, so as to read
$$\mathbf{6.237} \spatie e_{rs}= C_{rsmn}E_{mn}$$
then the symmetry conditions $\mathbf{6.234}, \  \mathbf{6.235}, \text{ and } \mathbf{6.236} $ imply similar conditions on $C_{rsmn}$. (The tensor $C_{rsmn}$ is he second elasticity tensor).
\end{tcolorbox}
\textbf{a) $C_{rsnm}= C_{rsmn}$ and $C_{srmn}= C_{rsmn}$\\}
This is a direct consequence of the symmetries $E_{nm}= E_{mn}$ and $e_{nm}= e_{mn}$. \\E.g.:
\begin{align}
e_{nm}&= e_{mn}\\
\Leftrightarrow\spatie C_{nmrs}E_{rs}&=C_{mnrs}E_{rs}\\
\Rightarrow\spatie C_{nmrs}&=C_{mnrs}
\end{align}\\
\textbf{b) $C_{rsnm}= C_{mnrs}$ \\}
We simplify the notation by using 'compactified' indices (e.g.):
$$e_{a}= C_{ab}E_{b} \quad \Leftrightarrow \quad e_{rs}= C_{rsmn}E_{mn}$$
Let's form the invariant $E_ae_a$, then :
\begin{align}
E_ae_a &=  c_{au}e_{u}C_{av}E_{v}\\
&=  c_{ua}e_{a}C_{uv}E_{v}\quad\text{(renaming dummy indices)}\\
&=  C_{uv}E_u E_{v}\\
&=  C_{vu}E_u E_{v}\quad\text{(renaming dummy indices)}
\end{align}
From $(6)$ and $(7)$ we can conclude 
$$C_{vu}=C_{uv}$$
 $$\blacklozenge$$
\newpage

\section{p212 - Exercise}
\begin{tcolorbox}
Deduce from  $\mathbf{6.250}$ that if an isotropic elastic body is in equilibrium under no body forces, then the expansion $\theta$ is an harmonic function $\left(\mathbf{\Delta}\theta = 0\right)$
\end{tcolorbox}
$$\mathbf{6.250}: \spatie \rho f_r= \rho X_r+\left(\lambda+ \mu\right)\theta_{,r}+\mu\mathbf{\Delta}u_r$$
In equilibrium under no body forces means $f_r=0$ and $X_r=0$. So,
\begin{align}
\left(\lambda+ \mu\right)\theta_{,r}+\mu\mathbf{\Delta}u_r&=0\\
\partial_r \quad \Rightarrow \spatie \left( \lambda+ \mu \right)\underbrace{\theta_{,rr}}_{= \mathbf{\Delta}\theta} +\mu\underbrace{\left(\mathbf{\Delta}u_r \right)_{,r}}_{=\mathbf{\Delta}u_{r,r}} &=0\\
u_{r,r}=\theta\quad\Rightarrow\spatie \left( \lambda+ \mu \right)\mathbf{\Delta}\theta +\mu\mathbf{\Delta}\theta &=0\\
\Rightarrow\spatie\spatie\spatie \mathbf{\Delta}\theta &=0
\end{align}
 $$\blacklozenge$$
\newpage

\section{p213 - Exercise}
\begin{tcolorbox}
Express the equations of motion   $\mathbf{6.250}$ in curvilinear coordinates.
\end{tcolorbox}
$$\mathbf{6.250}: \spatie \rho f_r= \rho X_r+\left(\lambda+ \mu\right)\theta_{,r}+\mu\mathbf{\Delta}u_r$$
We start from $\mathbf{6.252}: \spatie \rho f^r= \rho X^r+E^{rs}_{\ \ |s}$
\begin{align}
& \rho f^r= \rho X^r+E^{rs}_{\ \ |s}\\
\mathbf{6.245}\text{:}\quad\Rightarrow\spatie & E^{rs} = \lambda \delta_{rs} \theta + 2 \mu e^{rs}\\
\Rightarrow\spatie & \rho f^r= \rho X^r+\lambda \delta_{rs} \theta _{ |s} + 2 \mu e^{rs}_{\ \ |s}\\
\Leftrightarrow\spatie & \rho f^r= \rho X^r+\lambda  \theta _{,r} + 2 \mu e^{rs}_{\ \ |s}\\
\mathbf{6.245}\text{: }\spatie \spatie & e^{rs} = \half\left(u^r_{\ |s}+u^s_{\ |r}\right)\\
\Leftrightarrow\spatie & \rho f^r= \rho X^r+\lambda  \theta _{,r} + 2 \mu \half\left(u^r_{\ |ss}+u^s_{\ |rs}\right)\\
\mathbf{6.246}\text{: }\spatie \spatie &\theta = e^{nn} = \half\left(u^n_{\ |n}+u^n_{\ |n}\right) = u^n_{\ |n}\\
\text{(6)}\Rightarrow\spatie & \rho f^r= \rho X^r+\lambda  \theta _{,r} + \mu \left(u^r_{\ |ss}+\theta_{\ |r}\right)\\
\Rightarrow\spatie & \rho f^r= \rho X^r+\left(\lambda +\mu\right) \theta _{,r} + \mu u^r_{\ |ss}
\end{align}
So, $$f^r= \rho X^r+\left(\lambda +\mu\right) \theta _{,r} + \mu \mathbf {\Delta}u^r$$
where we define the Laplacian differential operator as $$\mathbf{\Delta}\left({\cdot}\right) \overset{\underset{\mathrm{def}}{}}{=} \left({\cdot}\right)_{|nn}$$
 $$\blacklozenge$$
\newpage

\section{p215 - Exercise}
\begin{tcolorbox}
Verify that $E_r=z_r, \ H_r=0$ satisfy the wave equation but not Maxwell's equations
\end{tcolorbox}
$$\mathbf{6.306}: \spatie \ \frac{1}{c^2}\frac{\partial^2E_r}{\partial t^2} -E_{r,mn} = 0$$
$$\mathbf{6.307}: \spatie \ \frac{1}{c^2}\frac{\partial^2H_r}{\partial t^2} -H_{r,mn} = 0$$
Obviously, $\mathbf{6.307}$ is trivial as $H_r=0$ is a constant $=0$, and so are the derivatives.\\
For $\mathbf{6.306}$, $ \frac{\partial^2E_r}{\partial t^2}= 0$ as $E_r$ is no function of time.\\
So, the defined field satisfy the wave equation.\\
\begin{align}
\left\{ \begin{array}{ll}
\mathbf{6.301}: &\frac{1}{c}\frac{\partial E_r}{\partial t} = \epsilon_{rmn}H_{n,m}, \quad  \frac{1}{c}\frac{\partial H_r}{\partial t}= -\epsilon_{rmn}E_{n,m}\\
\mathbf{6.302}: & E_{n,n}=0, \spatie \quad H_{n,n}=0
\end{array}\right.
\end{align}
The first equation of $\mathbf{6.302}$ is not satisfied as $E_{n,n}= N$ with $N$ the space dimension.
 $$\blacklozenge$$
\newpage

\section{p220 - Exercise}
\begin{tcolorbox}
Prove a similar statement for the electric and magnetic vectors of the complementary electromagnetic field.
\end{tcolorbox}
If we take the complex conjugate of $\mathbf{6.324}$ and subtract, we obtain after multiplying by $i$:
\begin{align}
E^{**}_r = -\epsilon_{rmn}H^{**}_nV_{,m}, \quad H^{**}_r = \epsilon_{rmn}E^{**}_nV_{,m}, 
\end{align} 

Hence, the vectors $V_{,r}, E^{**}_r$ and $ H^{**}_r$ form a right-handed orthogonal triad, and we obtain the relation
\begin{align}
E^{**}_rE^{**}_r=H^{**}_nH^{**}_n
\end{align}
 $$\blacklozenge$$
\newpage

\section{p221 - Clarification}
\begin{tcolorbox}
Some thoughts about polarization
\end{tcolorbox}


\begin{figure}[H]%
    \centering
    \subfloat[]{\begin{tikzpicture}[x=(15:0.5), y=(90:0.6), z=(-20:2.2)]
  \newcommand*\lateraleye{%
       \scalebox{0.25}{
    \tikzset{every picture/.style={line width=0.75pt}} 
    \begin{tikzpicture}[x=0.75pt,y=0.75pt,yscale=-1,xscale=1]
    \draw  [line width=1.5]  (300,100.33) .. controls (326,122) and (352,135) .. (378,139.33) .. controls (352,143.67) and (326,156.67) .. (300,178.33) ;
    \draw  [fill={rgb, 255:red, 0; green, 0; blue, 0 }  ,fill opacity=1 ] (308.94,116.33) .. controls (313.87,116.33) and (317.86,125.51) .. (317.85,136.83) .. controls (317.84,148.15) and (313.84,157.33) .. (308.91,157.33) .. controls (303.99,157.32) and (300,148.14) .. (300.01,136.82) .. controls (300.02,125.5) and (304.02,116.32) .. (308.94,116.33) -- cycle ;
    \draw  [draw opacity=0][line width=1.5]  (314.84,166.6) .. controls (311.87,164.64) and (309.14,162.18) .. (306.76,159.24) .. controls (295.12,144.82) and (296.6,124.33) .. (310.07,113.45) .. controls (311.48,112.32) and (312.96,111.33) .. (314.5,110.49) -- (331.14,139.55) -- cycle ; \draw  [line width=1.5]  (314.84,166.6) .. controls (311.87,164.64) and (309.14,162.18) .. (306.76,159.24) .. controls (295.12,144.82) and (296.6,124.33) .. (310.07,113.45) .. controls (311.48,112.32) and (312.96,111.33) .. (314.5,110.49) ;
    \draw  [fill={rgb, 255:red, 255; green, 255; blue, 255 }  ,fill opacity=1 ] (304.43,124.2) .. controls (306.09,124.25) and (307.32,128.01) .. (307.18,132.6) .. controls (307.05,137.19) and (305.59,140.88) .. (303.93,140.83) .. controls (302.27,140.78) and (301.03,137.02) .. (301.17,132.43) .. controls (301.31,127.83) and (302.76,124.15) .. (304.43,124.2) -- cycle ;
    \end{tikzpicture}
    }\,}
  \def\A{1.4}
  \def\L{3.0}
  \def\M{7.5}
  \def\nwave{4}
  \def\k{(360*\nwave/\M)} % 2pi*n / L = 360*n / L
  \def\nvec{40} % vectors per wavelength
 % SECTION 1
  
   \draw[thick,-Latex] (0,0,0) -- (0,0,1.2*\M);
  \draw[thick] (0,0,0) -- (0,0,0.4*\L);
  %\foreach \ang in {45,90,...,360}{
    %\draw[<->,very thick] (0,0,0.4*\L)++(\ang:\A) --++ (\ang+180:2*\A);
 % }
 

  \draw[thick] (0,0,0.4*\L) -- (0,0,\L);
  \draw[->,] (0,0,1.*\L)++(60:1.1*\A) --++ (0,0,0.2*\L) node[right] {$\vb{v}$};
  \node[scale=0.9,yslant=tan(0)] at (-0.8*\L,-0.8*\L,0.4*\L) {P };
  \node[scale=0.9,yslant=tan(0)] at (0,0.6*\L,0.4*\L) { $E_1$};
   \node[scale=0.9,yslant=tan(0)] at (-2.0,-0.50,0.4*\L) { $E_2$};
  \node[scale=0.9,yslant=tan(-10)] at (0,0,9.5) { $ \lateraleye$};
 
  
  % SECTION 2
        \draw[fill=white,] (2.2,0,0.4*\L) --++ (0:0) circle (1.5pt);
    \draw[fill=gray!50,] (0,-0.88,0.4*\L) --++ (0:0) circle (1.5pt);
  \draw[thin,dashed,] (1*\L,1*\L,0.4*\L) -- (1*\L,-1*\L,0.4*\L)--  (-1*\L,-1*\L,0.4*\L)-- (-1*\L,1*\L,0.4*\L)-- cycle;
   \draw[->, thin,] (0,0,0.4*\L)++(0:0) --++ (0+180:2*\A);
  \draw[->, thin,] (0,0,0.4*\L)++(90:0) --++ (90+180:\A);
  \draw[->, thin,] (0,0,0.4*\L)++(90+180:0) --++ (90+0:\A);
  \draw[->,thin,] (0,0,0.4*\L)++(180:0) --++ (180+180:2*\A);
  \begin{scope}[canvas is xy plane at z=0.4*\L]
    \draw[thick,dashed] {(0,0) ellipse (2*1.4 and 1.4)};
  \end{scope}
  %Vertical components
  \begin{scope}[shift={(0,0,0*\L/2)}]
   
    \foreach \i [evaluate={\z=\i*\M/\nvec; \c=int(\i!=\nvec);}] in {0,...,\nvec}{
      \ifnum\c=1
        \draw[gray!30,thin] (0,0,\z) --++ (90:{\A*cos(\k*\z)});
      \fi
    }
 
    \draw[samples=100,smooth,variable=\z,domain=0.4*\L:1.*\M,very thick]
      plot(0,{\A*cos(\k*\z)},\z);

    \draw[,samples=100,smooth,variable=\z,domain=0:0.4*\L,,gray!50,thin]
      plot(0,{\A*cos(\k*\z)},\z);
    %Horizontal components
     \foreach \i [evaluate={\z=\i*1.*\M/\nvec; \c=int(\i!=\nvec/2);}] in {0,...,\nvec}{
      \ifnum\c=1
        \draw[gray!30,thin] (0,0,\z) --++ (0:{2*\A*cos(\k*\z+90)});
      \fi
    }
  \draw[,samples=100,smooth,variable=\z,domain=0.4*\L:1.*\M,,very thick]
      plot({2*\A*cos(\k*\z+90)},0,\z);
   
     \draw[,samples=100,smooth,variable=\z,domain=0:0.4*\L,gray!50,thin]
      plot({2*\A*cos(\k*\z+90)},0,\z);
           \foreach \i [evaluate={\z=\i*\M/\nwave/2}] in {2,...,8}{
         \draw[fill=white]  (0,0,\z) --++ (0:0) circle (1.5pt);
    }
    \foreach \i [evaluate={\p=\i*\M/\nwave/2+\M/\nwave/4}] in {1,...,7}{
         \draw[fill=gray]  (0,0,\p) --++ (0:0) circle (1.5pt);
    }

  \end{scope}

\end{tikzpicture}    }
    \quad
        \subfloat[]{
\begin{tikzpicture}[x={(0.8cm, 0.4cm)}, y={(0.9cm, -0.3cm)}, z={(0cm,1cm)}, line cap=round, line join=round]
\tikzset{>=latex}
\tikzset{axis/.style={black, very thick, ->}}
\tikzset{ef/.style={very thick, red}}
\tikzset{vec/.style={black, -{Latex[length=0.8mm]}}}
\tikzset{every text node part/.style={align=center}}

\newcommand*\lateraleye{%
       \scalebox{0.25}{
    \tikzset{every picture/.style={line width=0.75pt}} 
    \begin{tikzpicture}[x=0.75pt,y=0.75pt,yscale=-1,xscale=1]
    \draw  [line width=1.5]  (300,100.33) .. controls (326,122) and (352,135) .. (378,139.33) .. controls (352,143.67) and (326,156.67) .. (300,178.33) ;
    \draw  [fill={rgb, 255:red, 0; green, 0; blue, 0 }  ,fill opacity=1 ] (308.94,116.33) .. controls (313.87,116.33) and (317.86,125.51) .. (317.85,136.83) .. controls (317.84,148.15) and (313.84,157.33) .. (308.91,157.33) .. controls (303.99,157.32) and (300,148.14) .. (300.01,136.82) .. controls (300.02,125.5) and (304.02,116.32) .. (308.94,116.33) -- cycle ;
    \draw  [draw opacity=0][line width=1.5]  (314.84,166.6) .. controls (311.87,164.64) and (309.14,162.18) .. (306.76,159.24) .. controls (295.12,144.82) and (296.6,124.33) .. (310.07,113.45) .. controls (311.48,112.32) and (312.96,111.33) .. (314.5,110.49) -- (331.14,139.55) -- cycle ; \draw  [line width=1.5]  (314.84,166.6) .. controls (311.87,164.64) and (309.14,162.18) .. (306.76,159.24) .. controls (295.12,144.82) and (296.6,124.33) .. (310.07,113.45) .. controls (311.48,112.32) and (312.96,111.33) .. (314.5,110.49) ;
    \draw  [fill={rgb, 255:red, 255; green, 255; blue, 255 }  ,fill opacity=1 ] (304.43,124.2) .. controls (306.09,124.25) and (307.32,128.01) .. (307.18,132.6) .. controls (307.05,137.19) and (305.59,140.88) .. (303.93,140.83) .. controls (302.27,140.78) and (301.03,137.02) .. (301.17,132.43) .. controls (301.31,127.83) and (302.76,124.15) .. (304.43,124.2) -- cycle ;
    \end{tikzpicture}
    }\,}
%Styles

\tikzset{vec/.style={black, -{Latex[length=0.8mm]}}}
\tikzset{every text node part/.style={align=center}}
	\begin{scope}[canvas is yz plane at x=6]
		\draw[thick, dashed] (-1.2,-1.2) rectangle (1.52,1.2);
		\draw[very thick,  dashed] (0,0) ellipse (0.8cm and 0.6cm);
	\end{scope}
%	% Main Axes

	
	% Propagation Direction Vector
	\draw[axis] (1,0,0) -- (8.3,0,0) node[right, black] { $ $};
	\draw[axis] (1,0,2) -- (2,0,2) node[right, black] { $\vb{v}$};
	
	
	% Correction so as the Result to Seem 3d
	\draw[very thick] (1,0,0) -- (6,0,0);
	
	% Red Line
	\draw[very thick,  variable=\t, domain=1:6, samples=300] plot (\t, {0.8*sin(deg(\t*4+90))}, {0.6*cos(deg(\t*4+90))});
	
	% Vectors from Axis to Red Line
	\foreach \i [evaluate={\k = \i*4; \ii = \i;}] in {1,1.05,...,6}
	{
		\draw[very thin] (\ii,0,0) -- +(0, {0.8*sin(deg(\k+90))}, {0.6*cos(deg(\k+90))});
	}
	\begin{scope}[canvas is yz plane at x=9]
	\node[scale=0.9,xslant = tan(-30),yslant =tan(-0)] at  (0,0,6) { $ \lateraleye$};
	\end{scope};
\end{tikzpicture}
}
    \quad
        \subfloat[]{\begin{tikzpicture}[x=(15:0.5), y=(90:0.6), z=(-20:2.2)]
\newcommand*\lateraleye{%
       \scalebox{0.25}{
    \tikzset{every picture/.style={line width=0.75pt}} 
    \begin{tikzpicture}[x=0.75pt,y=0.75pt,yscale=-1,xscale=1]
    \draw  [line width=1.5]  (300,100.33) .. controls (326,122) and (352,135) .. (378,139.33) .. controls (352,143.67) and (326,156.67) .. (300,178.33) ;
    \draw  [fill={rgb, 255:red, 0; green, 0; blue, 0 }  ,fill opacity=1 ] (308.94,116.33) .. controls (313.87,116.33) and (317.86,125.51) .. (317.85,136.83) .. controls (317.84,148.15) and (313.84,157.33) .. (308.91,157.33) .. controls (303.99,157.32) and (300,148.14) .. (300.01,136.82) .. controls (300.02,125.5) and (304.02,116.32) .. (308.94,116.33) -- cycle ;
    \draw  [draw opacity=0][line width=1.5]  (314.84,166.6) .. controls (311.87,164.64) and (309.14,162.18) .. (306.76,159.24) .. controls (295.12,144.82) and (296.6,124.33) .. (310.07,113.45) .. controls (311.48,112.32) and (312.96,111.33) .. (314.5,110.49) -- (331.14,139.55) -- cycle ; \draw  [line width=1.5]  (314.84,166.6) .. controls (311.87,164.64) and (309.14,162.18) .. (306.76,159.24) .. controls (295.12,144.82) and (296.6,124.33) .. (310.07,113.45) .. controls (311.48,112.32) and (312.96,111.33) .. (314.5,110.49) ;
    \draw  [fill={rgb, 255:red, 255; green, 255; blue, 255 }  ,fill opacity=1 ] (304.43,124.2) .. controls (306.09,124.25) and (307.32,128.01) .. (307.18,132.6) .. controls (307.05,137.19) and (305.59,140.88) .. (303.93,140.83) .. controls (302.27,140.78) and (301.03,137.02) .. (301.17,132.43) .. controls (301.31,127.83) and (302.76,124.15) .. (304.43,124.2) -- cycle ;
    \end{tikzpicture}
    }\,}
\tikzstyle{platecol}=[gray!20!black!20,opacity=0.8]
\tikzstyle{platetopcol}=[gray!20!black!20,opacity=0.8]
\tikzstyle{platesidEcol}=[gray!20!black!20,opacity=0.8]
\tikzstyle{mydashed}=[dash pattern=on 1.2 off 0.7,line width=0.3]
% POLARIZER
\def\W{3.5}  % width polarizer
\def\w{0.05} % width slit
\def\l{2.9}  % length slit
\def\t{0.05}
\def\N{7}    % number of slits
  \tikzset{
  plate/.pic={
    \ifnumless{45}{#1}{
      \def\topang{#1}
    }{
      \def\topang{#1+90}
    }
    \fill[platetopcol]
      (\topang:\W/2)++(\topang-90:\W/2) --++ (0,0,-\t) --++ (\topang+90:\W) --++ (0,0,\t) -- cycle;
    \fill[platesidEcol]
      (\topang+90:\W/2)++(\topang:\W/2) --++ (0,0,-\t) --++ (\topang+180:\W) --++ (0,0,\t) -- cycle;
    \fill[platecol]
      (#1:\W/2)++(#1-90:\W/2) --++ (#1-180:\W) --++ (#1+90:\W) --++ (#1:\W) -- cycle
      \foreach \i [evaluate={\x=-\W/2+\i*\W/(\N+1);}] in {1,...,\N}{
        (#1:\l/2)++(#1+90:\x+\w/2) --++ (#1-180:\l) --++ (#1-90:\w) --++ (#1:\l) -- cycle
      };
  }
}
  \def\A{1.4}
  \def\L{3.2}
  \def\M{4.5}
  \def\nwave{4}
  \def\k{(360*\nwave/\M)} % 2pi*n / L = 360*n / L
  %\def\dx{90/\k}
  \def\nvec{40} % per wavelength
  
  % SECTION 1
  \draw[thin,dashed] (1*\L,1*\L,0.4*\L) -- (1*\L,-1*\L,0.4*\L)--  (-1*\L,-1*\L,0.4*\L)-- (-1*\L,1*\L,0.4*\L)-- (1*\L,1*\L,0.4*\L);
   \draw[thick,-Latex] (0,0,0) -- (0,0,8);
  \draw[thick] (0,0,0) -- (0,0,0.4*\L);
  %\foreach \ang in {45,90,...,360}{
    %\draw[<->,very thick,Ecol] (0,0,0.4*\L)++(\ang:\A) --++ (\ang+180:2*\A);
 % }
  \draw[-Latex,very thick,] (0,0,0.4*\L)++(0:0) --++ (0+180:1.2*\A);
  \draw[-Latex,very thick,] (0,0,0.4*\L)++(90:0) --++ (90+180:2*\A);
  \draw[-Latex,very thick,] (0,0,0.4*\L)++(90+180:0) --++ (90+0:2*\A);
  \draw[-Latex,very thick,] (0,0,0.4*\L)++(180:0) --++ (180+180:1.2*\A);
  \draw[-Latex,very thick,] (0,0,0.4*\L)++(45:0) --++ (45:0.85*\A);
  \draw[-Latex,very thick,] (0,0,0.4*\L)++(45:0) --++ (45+180:0.85*\A);
   \draw[-Latex,very thick,] (0,0,0.4*\L)++(135:0) --++ (135+180:2*\A);
    \draw[-Latex,very thick,] (0,0,0.4*\L)++(135:0) --++ (135:2*\A);

  \draw[thick] (0,0,0.4*\L) -- (0,0,\L);
  \draw[-Latex,very thick,] (0,0,1.6*\L)++(60:1.1*\A) --++ (0,0,0.2*\L) node[right] {$\vb{v}$};
  \node[scale=0.9,yslant=tan(0)] at (-0.8*\L,-0.8*\L,0.4*\L) {P };
  \node[scale=0.9,yslant=tan(10)] at (0.6,-0.75*\A,0.0*\L) { $E$};
  \node[scale=0.9,yslant=tan(-10)] at (0,0,9) { $ \lateraleye$};
  
  % SECTION 2
  \begin{scope}[shift={(0,0,\L)}]
    \pic at (0,0) {plate={90}};
    \node[scale=0.9,yslant=tan(-10),right=7,below] at (-135:0.7*\W) {Q};
    \draw[thick] (0,0,0) -- (0,0,\M/2);
    \draw[Latex-Latex,very thick,] (0,0,\M/2)++(90:\A) --++ (-90:2*\A); %-\dx
    \draw[thick,samples=100,smooth,variable=\z,domain=0:\M]
      plot(0,{\A*cos(\k*\z)},\z);
    \foreach \i [evaluate={\z=\i*\M/\nvec; \c=int(\i!=\nvec/2);}] in {0,...,\nvec}{
      \ifnum\c=1
        \draw[very thin, gray!50] (0,0,\z) --++ (90:{\A*cos(\k*\z)});
      \fi
    }
    \draw[thick] (0,0,\M/2) -- (0,0,\M);
    \node[scale=0.9,yslant=tan(-10),below=-7,align=center] at (0,-1.4*\A,0.45*\M)
      {};
  \end{scope}
  
 
\end{tikzpicture}}
    \quad
\caption{Polarization of light}
\label{fig:fig_p221}
\end{figure}
In the above figures, only the electric field is represented (the magnetic field has to be imagined perpendicular to the $E_r$ vector).\\
In figure (a) we see an elliptical polarization which occurs when the EMW can be split into two perpendicular components. When $\left|\overline{E_1}\right|= \left|\overline{E_2}\right|$, one can speak about circular polarization. Figure (b) gives a view at a certain time $t$ of the result of $\overline{E_1}+\overline{E_2}$.//
In figure (c) an observer 'sees' in the phase wave situated at $P$, an unpolarized EMV. After passing a linear polarizing material at $Q$ the observer will 'see' the EMV oscillating only in the vertical plane along $\mathbf{v}$ 
 $$\blacklozenge$$
\newpage


\section{p221 - Exercise}
\begin{tcolorbox}
What conditions must be imposed on the fixed complex vectors $E^{(0)}_r$ and $H^{(0)}_r$ in order that the wave may be plane-polarized
\end{tcolorbox}
As stated a plane-polarized wave will have  its vectors $E^{*}_r$ and $E^{**}_r$ have the same directions and moreover the $E^{*}_r$ vector maintains a fixed directions.\\
This means that $E^{*}_r$ can be written as $E^{*}_r =  \alpha\left(z,t\right) \mathcal{E}_r$ and $E^{**}_r =  \beta\left(z,t\right) \mathcal{E}_r$ with $\alpha, \ \beta$  real valued functions and $\mathcal{E}_r$ a constant. 
Note that from the definition of $E^{*}_r$ and $E^{**}_r$ we have
\begin{align}
E^{}_r &= E^{*}_r+iE^{**}_r
\end{align}
and 
\begin{align}
 \begin{array}{l}
\text{ 6.330}\\\\
\text{  6.331}\\
\end{array}
\quad &\left\{ \begin{array}{l}
\frac{\partial E^{*}_r}{\partial t} = -\frac{2\pi c}{\lambda} E^{**}_r\\\\
\frac{\partial E^{**}_r}{\partial t} = \frac{2\pi c}{\lambda}  E^{*}_r\\
\end{array}\right.\\
\Rightarrow &\left\{\begin{array}{l}
\frac{\partial \alpha}{\partial t} = -\frac{2\pi c}{\lambda} \beta\\\\
\frac{\partial \beta}{\partial t} = \frac{2\pi c}{\lambda}  \alpha\\
\end{array}\right.\\
\Rightarrow &\left\{\begin{array}{l}
\frac{\partial^2 \alpha}{\partial t^2} = -\left(\frac{2\pi c}{\lambda}\right) ^2 \alpha\\\\
\frac{\partial^2 \beta}{\partial t^2} = -\left(\frac{2\pi c}{\lambda}\right) ^2 \beta\\\\
\end{array}\right.\\
\Rightarrow &\left\{\begin{array}{l}
\alpha = A\left(z\right)\cos \frac{2\pi c}{\lambda}  t + B\left(z\right)\sin \frac{2\pi c}{\lambda}  t \\\\
\beta = C\left(z\right)\cos \frac{2\pi c}{\lambda}  t + D\left(z\right)\sin \frac{2\pi c}{\lambda}  t   
\end{array}\right.\\
\Rightarrow \quad & \frac{E^{}_r}{\mathcal{E}_r}=A\cos \frac{2\pi c}{\lambda} t + B\sin \frac{2\pi c}{\lambda} t+i\left(C\left(z\right)\cos \frac{2\pi c}{\lambda}  t + D\left(z\right)\sin \frac{2\pi c}{\lambda}  t \right)
\end{align}
With $A,B,C,D:\mathbb{R}^3\rightarrow \mathbb{R}$.\\
Let's write $E^{(0)}_r$ as $E^{(0)}_r = a+ib$ with $a, b$ real valued functions depending on the position only.
Then, $\textbf{6.308}$ can be rewritten as
\begin{align}
E^{}_r &= \left(a+ib\right)\left(\cos S+i\sin S\right)\\
&= a\cos S-b\sin S+i\left(b\cos S+a\sin S\right)
\end{align}
We have, with $S= \frac{2\pi V}{\lambda}-\frac{2\pi c}{\lambda}t$
\begin{align}
&\left\{\begin{array}{l}
\cos S= \cos \frac{2\pi V}{\lambda}\cos \frac{2\pi c}{\lambda}t+\sin \frac{2\pi V}{\lambda}\sin \frac{2\pi c}{\lambda}t\\\\
\sin S= \sin \frac{2\pi V}{\lambda}\cos \frac{2\pi c}{\lambda}t-\cos \frac{2\pi V}{\lambda}\sin\frac{2\pi c}{\lambda}t\\
\end{array}\right.\\
\text{(8)}\Rightarrow \quad E^{}_r =& \left\{\begin{array}{l}a\cos \frac{2\pi V}{\lambda}\cos \frac{2\pi c}{\lambda}t+a\sin \frac{2\pi V}{\lambda}\sin \frac{2\pi c}{\lambda}t+b\cos \frac{2\pi V}{\lambda}\sin \frac{2\pi c}{\lambda}t-b\sin \frac{2\pi V}{\lambda}\cos \frac{2\pi c}{\lambda}t\\\\
+i\left(
b\cos \frac{2\pi V}{\lambda}\cos \frac{2\pi c}{\lambda}t+b\sin \frac{2\pi V}{\lambda}\sin \frac{2\pi c}{\lambda}t - a\cos \frac{2\pi V}{\lambda}\sin \frac{2\pi c}{\lambda}t+a\sin \frac{2\pi V}{\lambda}\cos \frac{2\pi c}{\lambda}t
\right)
\end{array}\right.\\
=& \left\{\begin{array}{l}\left(a\cos \frac{2\pi V}{\lambda}-b\sin \frac{2\pi V}{\lambda}\right)\cos \frac{2\pi c}{\lambda}t+\left(a\sin \frac{2\pi V}{\lambda}+b\cos \frac{2\pi V}{\lambda}\right)\sin \frac{2\pi c}{\lambda}t\\\\
+i\left[
\left(b\cos \frac{2\pi V}{\lambda}+a\sin \frac{2\pi V}{\lambda}\right)\cos \frac{2\pi c}{\lambda}t+\left(b\sin \frac{2\pi V}{\lambda} - a\cos \frac{2\pi V}{\lambda}\right)\sin \frac{2\pi c}{\lambda}t
\right]\\\\
\end{array}\right.
\end{align}
Let's now compare equations $(6)$ and $(11)$
\begin{align}
&E^{}_r=A\mathcal{E}_r\cos \frac{2\pi c}{\lambda} t + B\mathcal{E}_r\sin \frac{2\pi c}{\lambda} t+i\left(C\mathcal{E}_r\cos \frac{2\pi c}{\lambda}  t + D\mathcal{E}_r\sin \frac{2\pi c}{\lambda}  t \right)\\
&E^{}_r=\left\{\begin{array}{l}\left(a\cos \frac{2\pi V}{\lambda}-b\sin \frac{2\pi V}{\lambda}\right)\cos \frac{2\pi c}{\lambda}t+\left(a\sin \frac{2\pi V}{\lambda}+b\cos \frac{2\pi V}{\lambda}\right)\sin \frac{2\pi c}{\lambda}t\\\\
+i\left[
\left(a\sin \frac{2\pi V}{\lambda}+b\cos \frac{2\pi V}{\lambda}\right)\cos \frac{2\pi c}{\lambda}t-\left(a\cos \frac{2\pi V}{\lambda}-b\sin \frac{2\pi V}{\lambda}\right)\sin \frac{2\pi c}{\lambda}t
\right]\\\\
\end{array}\right.
\end{align}
We see that 
\begin{align}
&A\mathcal{E}_r= a\cos \frac{2\pi V}{\lambda}-b\sin \frac{2\pi V}{\lambda}\\
&B\mathcal{E}_r= a\sin \frac{2\pi V}{\lambda}+b\cos \frac{2\pi V}{\lambda}\\
&C=B\\
&D=-A\\
\text{(14), (15)}\quad\Rightarrow \quad &\left\{\begin{array}{l}a= A\mathcal{E}_r\cos \frac{2\pi V}{\lambda}+B\mathcal{E}_r\sin \frac{2\pi V}{\lambda}\\
b= -A\mathcal{E}_r\sin \frac{2\pi V}{\lambda}+B\mathcal{E}_r\cos \frac{2\pi V}{\lambda}\\
\end{array}\right.\\
\Rightarrow \quad E^{(0)}_r&= a+ib\\
&=\mathcal{E}_r\left(A\left(z\right)+iB\left(z\right)\right)e^{-\frac{2\pi V}{\lambda}}
\end{align}
So, the direction of $E^{(0)}_r$ does not change as $\frac{E^{(0)}_r}{E^{(0)}_s} = \frac{\mathcal{E}_r}{\mathcal{E}_s}$  and the magnitude varies with the position but in such a way that the effect of $V$ is annihilated.
$$\blacklozenge$$
\newpage

\section{p223 - Clarification}
\begin{tcolorbox}
Interrelationship between the identities $\mathbf{6.337}$ to $\mathbf{6.340}$
\end{tcolorbox}
\begin{align}
\mathbf{6.337} \spatie &\frac{1}{c^2}\frac{\partial^2 \phi_r}{\partial t^2}+ \frac{1}{c} \pdv{\psi_{,r}}{t} = \phi_{r,mm}-\phi_{m,mr}\\
\mathbf{6.338} \spatie &\frac{1}{c}\frac{\partial}{\partial t} \phi_{m,m}+ \psi_{,mm}=0\\
\mathbf{6.339(a)} \spatie &\frac{1}{c^2}\frac{\partial^2 \phi_r}{\partial t^2}-  \phi_{r,mm}=0\\
\mathbf{6.339(b)} \spatie &\frac{1}{c^2}\frac{\partial^2 \psi}{\partial t^2}-  \psi_{,mm}=0\\
\mathbf{6.340} \spatie &\frac{1}{c}\frac{\partial \psi}{\partial t}+ \phi_{m,m}=0
\end{align}
then
\begin{align}
\text{(3) in (1)}\spatie & \frac{1}{c} \pdv{\psi_{,r}}{t} =-\phi_{m,mr}\\
\pdv{\text{(5)}}{z_r}\spatie &\frac{1}{c}\frac{\partial \psi_{,r}}{\partial t}+ \phi_{m,mr}=0\quad \Leftrightarrow \quad \text{(6)}
\end{align}
What about $\text{(4)}$ ?\\
\begin{align}
\text{(4) in (2)}\spatie & \frac{1}{c}\frac{\partial}{\partial t} \phi_{m,m}=-\frac{1}{c^2}\frac{\partial^2 \psi}{\partial t^2}\\
\int_{t} \text{(8)}\Rightarrow\spatie &\frac{1}{c}\frac{\partial \psi}{\partial t}+ \phi_{m,m}=C
\end{align}
In $\text{(9)}$, $C$ is a function, constant in $t$. Imposing $C=0$ gives still a valid solution and is equivalent with $\text{(5)}$.
 $$\blacklozenge$$
\newpage


\section{p223 - Clarification}
\begin{tcolorbox}
 $$\mathbf{6.342}\spatie \frac{1}{c^2}\frac{\partial^2 \Pi_r}{\partial t^2}- \Pi_{r,mm}=0$$
\end{tcolorbox}
\begin{align}
\mathbf{6.341(a)} \spatie &\mathbf{E_n= \Pi_{m,mn}-\frac{1}{c^2}\frac{\partial^2 \Pi_n}{\partial t^2}}\\
\mathbf{6.341(b)} \spatie &\mathbf{H_n= \frac{1}{c}\epsilon_{npq}\frac{\partial}{\partial t}\Pi_{q,p}}
\end{align}
We first check, under which conditions $\text{(1) and (2)}$ satisfy the Maxwell equations 
\begin{align}\left\{ \begin{array}{ll}
\text{6.301(a),(b)}&E_{m,m}=0,\spatie H_{m,m}=0\\\\
\text{6.302(a)}&\frac{1}{c}\frac{\partial E_r}{\partial t}= \epsilon_{rmn}H_{n,m}\\\\
\text{6.302(b)}&\frac{1}{c}\frac{\partial H_r}{\partial t}= -\epsilon_{rmn}E_{n,m}
\end{array}\right.
\end{align}
with the wave equations: 
\begin{align}\left\{ \begin{array}{ll}
\text{6.306}&\frac{1}{c^2}\frac{\partial^2 E_r}{\partial t^2}- E_{r,mm}=0\\\\
\text{6.307}&\frac{1}{c^2}\frac{\partial^2 H_r}{\partial t^2}- H_{r,mm}=0
\end{array}\right.
\end{align}
We have for $\mathbf{6.301(b)}$:
\begin{align}
\text{(2)}_{,n}\spatie H_{n,n} &= \frac{1}{c}\epsilon_{npq}\frac{\partial \Pi_{q,pn}}{\partial t}\\&= 0 \quad \left(  \Pi_{q,pn}=\Pi_{q,np } \text{ and }\epsilon_{npq}=-\epsilon_{pnq} \right)\end{align}
So, $\mathbf{6.301(b)}$ is satisfied.\\\\
For $\mathbf{6.302(b)}$ we have:
\begin{align}               
\text{6.302(b)}\spatie &\frac{1}{c}\frac{\partial H_r}{\partial t}= -\epsilon_{rmn}E_{n,m}\\
\text{(2)}_{,t}\quad \Rightarrow\spatie &\frac{1}{c}\epsilon_{npq}\frac{\partial^2}{\partial t^2}\Pi_{q,p}= -\epsilon_{rmn}E_{n,m}\\
\text{(1) in (8)}\quad \Rightarrow\spatie &\frac{1}{c}\epsilon_{npq}\frac{\partial^2}{\partial t^2}\Pi_{q,p}= -\underbrace{\epsilon_{rmn}\Pi_{q,qnm}}_{=0}+\epsilon_{rmn}\frac{1}{c^2}\frac{\partial^2 \Pi_{n,m}}{\partial t^2}\\
\Rightarrow\spatie &\frac{1}{c}\epsilon_{npq}\frac{\partial^2}{\partial t^2}\Pi_{q,p}= \epsilon_{rmn}\frac{1}{c^2}\frac{\partial^2 \Pi_{n,m}}{\partial t^2}
\end{align}
So, $\mathbf{6.302(b)}$ is satisfied.\\\\
We still have to prove that the expression $\text{(1) and (2)}$ are consistent with $\text{6.301(a)}$ and $\text{6.302(a)}$\\\\
also for $\mathbf{6.301(a)}$:
\begin{align}
\text{(1)}_{,n}\spatie E_{n,n} &= \Pi_{m,mnn}-\frac{1}{c^2}\frac{\partial^2 \Pi_{n,n}}{\partial t^2}
\end{align}
and for $\mathbf{6.302(a)}$:
\begin{align}
\text{6.302(a)}\spatie \frac{1}{c}\frac{\partial E_r}{\partial t}&= \epsilon_{rmn}H_{n,m}\\
\text{(2)}\quad \Rightarrow\spatie&= \epsilon_{rmn}\frac{1}{c}\epsilon_{npq}\frac{\partial}{\partial t}\Pi_{q,pm}\\
\Rightarrow\spatie \frac{\partial E_r}{\partial t}&= \epsilon_{rmn}\epsilon_{npq}\frac{\partial}{\partial t}\Pi_{q,pm}\\
&= \left(\delta_{rp}\delta_{mq}-\delta_{rq}\delta_{mp}\right)\frac{\partial}{\partial t}\Pi_{q,pm}\\
&= \partial_t\Pi_{m,rm}-\partial_t\Pi_{r,mm}
\end{align}
Let's impose the condition on $\Pi_n$:
\begin{align}
\mathbf{6.342} \spatie &\mathbf{\Pi_{r,mm}=\frac{1}{c^2}\frac{\partial^2 \Pi_r}{\partial t^2}}
\end{align}
From $(1)$ we have
\begin{align}
\frac{1}{c^2}\frac{\partial^2 \Pi_n}{\partial t^2}&=\Pi_{m,mn}-E_n\\
\text{(17) becomes}\spatie  \Pi_{r,mm}&=\Pi_{m,mn}-E_n\\
\Rightarrow\spatie E_n &= \Pi_{m,mn}-\Pi_{r,mm}\\
\Rightarrow\spatie \partial_tE_n &=\partial_t\Pi_{m,mn}- \partial_t\Pi_{r,mm}
\end{align}
Which is consistent with $\mathbf{6.302(a)}$ following $\text{(16)}$.\\\\

For $\mathbf{6.301(a)}$ we have
\begin{align}
\text{(16)}_{,n}\quad\Rightarrow\spatie \frac{1}{c^2}\frac{\partial^2 \Pi_{n,n}}{\partial t^2}&=\Pi_{n,mmn}\\
\text{(11)}\quad\Rightarrow\spatie E_{n,n} &= \Pi_{m,mnn}-\Pi_{n,mmn}\\
&=0
\end{align}
Which is consistent with $\mathbf{6.301(a)}$\\\\

 $$\blacklozenge$$
\newpage



\section{p226 - Clarification}
\begin{tcolorbox}
 $$\mathbf{6.361}\spatie \Pi^{(0)}_r\left(z\right)= \int_{V_{\zeta}}P_r\left(\zeta\right)F\left(z, \zeta\right)dV_{\zeta}$$
\end{tcolorbox}
The reason why we can find a solution in the form $\mathbf{6.361}$ is because the condition $\mathbf{6.358}: \quad \Pi^{(0)}_{r,mm}+ k^2 \Pi^{(0)}_r = 0$ is a linear homogeneous differential equation. So any linear combination of solutions of this equation will also be a solution:\\
Be
\begin{align}
\Pi^{(0)}_r = \overset{N}{\sum} C_n\overset{n}{\Pi}^{(0)}_r
\end{align}
where the $\overset{n}{\Pi}^{(0)}_r$ satisfy the condition  $\overset{n}{\Pi}^{(0)}_{r,mm}+ k^2 \overset{n}{\Pi}^{(0)}_r = 0$ an are evaluated at the same point $z_r$ but at different $\zeta_n$.\\
This is the situation as illustrated in the figure $(a)$ below.
\begin{figure}[H]%
    \centering
    \subfloat[]{\begin{tikzpicture}
\coordinate (O) at (-0,-0);
\coordinate (X) at (-0.4-0,-0.4-0);
\coordinate (Y) at (1-0,0-0);
\coordinate (Z) at (0-0,1-0);
\draw [-](O)--(X);
\draw [-](O)--(Y);
\draw [-](O)--(Z);
\coordinate (P) at (1,1);

%First volume
\node[right] at (P) {$z_m$};
\draw  [fill= white](P)circle (1pt);
\coordinate (ksi1) at (1.5,2.5) {} {};
\coordinate (ksi2) at (1,3) {} {};
\coordinate (ksi3) at (3,3) {} {};
\coordinate (ksi4)  at (3.5,2.5) {};
\coordinate (ksi5)  at (3,1.5) {};

\node[above] at (ksi1) {$\zeta^{'(1)}$};
\node[above] at (ksi2) {$\zeta^{'(2)}$};
\node[above] at (ksi3) {$\zeta^{'(3)}$};
\node[below] at (ksi4) {$\zeta^{'(..)}$};
\node[below] at (ksi5) {$\zeta^{'(...)}$};
\draw [-Latex, dotted](P)--(ksi1);
\draw  [-Latex, dotted](P)--(ksi2);
\draw  [-Latex, dotted](P)--(ksi3);
\draw  [-Latex, dotted](P)--(ksi4);
\draw  [-Latex, dotted](P)--(ksi5);
\draw  [fill= white](ksi1)circle (1pt);
\draw  [fill= white](ksi2)circle (1pt);
\draw  [fill= white](ksi3)circle (1pt);
\draw  [fill= white](ksi4)circle (1pt);
\draw  [fill= white](ksi5)circle (1pt);
\draw  plot[smooth cycle, tension=.7] coordinates {(0.5,3) (0.5,2.5) (1,2) (2,1.5) (3,0.5) (3.5,1) (4,1.5) (4,2)  (3.5,3.5) (3,4) (2,4)   (1,4) };
\node[below] at (3,0.5){$V^{'}_{\zeta}$};

\end{tikzpicture}}
    \quad
     \subfloat[]{\begin{tikzpicture}
\coordinate (O) at (-2,-2);
\coordinate (X) at (-0.4-2,-0.4-2);
\coordinate (Y) at (1-2,0-2);
\coordinate (Z) at (0-2,1-2);
\draw [-](O)--(X);
\draw [-](O)--(Y);
\draw [-](O)--(Z);
\coordinate (P) at (1,1);

%First volume
\node[right] at (P) {$z_m$};
\draw  [fill= white](P)circle (1pt);
\coordinate (ksi1) at (1.5,2.5) {} {};
\coordinate (ksi2) at (1,3) {} {};
\coordinate (ksi3) at (3,3) {} {};
\coordinate (ksi4)  at (3.5,2.5) {};
\coordinate (ksi5)  at (3,1.5) {};

\node[above] at (ksi1) {$\zeta^{'(1)}$};
\node[above] at (ksi2) {$\zeta^{'(2)}$};
\node[above] at (ksi3) {$\zeta^{'(3)}$};
\node[below] at (ksi4) {$\zeta^{'(..)}$};
\node[below] at (ksi5) {$\zeta^{'(...)}$};
\draw [-Latex, dotted](P)--(ksi1);
\draw  [-Latex, dotted](P)--(ksi2);
\draw  [-Latex, dotted](P)--(ksi3);
\draw  [-Latex, dotted](P)--(ksi4);
\draw  [-Latex, dotted](P)--(ksi5);
\draw  [fill= white](ksi1)circle (1pt);
\draw  [fill= white](ksi2)circle (1pt);
\draw  [fill= white](ksi3)circle (1pt);
v\draw  [fill= white](ksi4)circle (1pt);
\draw  [fill= white](ksi5)circle (1pt);
\draw  plot[smooth cycle, tension=.7] coordinates {(0.5,3) (0.5,2.5) (1,2) (2,1.5) (3,0.5) (3.5,1) (4,1.5) (4,2)  (3.5,3.5) (3,4) (2,4)   (1,4) };
\node[below] at (3,0.5){$V^{'}_{\zeta}$};
%second volume
\coordinate (xksi1) at (-0.5,0.5) {} {} {};
\coordinate (xksi2) at (-0.5,-1) {} {} {};
\coordinate (xksi3) at (-1.5,-0.5) {} {} {};
\coordinate (xksi4) at (0.5,-1) {} {};
\coordinate (xksi5) at (0,-1.5) {} {};

\node[above] at (xksi1) {$\zeta^{{"}(1)}$};
\node[above] at (xksi2) {$\zeta^{{"}(2)}$};
\node[above] at (xksi3) {$\zeta^{{"}(3)}$};
\node[below] at (xksi4) {$\zeta^{{"}(..)}$};
\node[below] at (xksi5) {$\zeta^{{"}(...)}$};
\draw [-Latex, dotted](P)--(xksi1);
\draw  [-Latex, dotted](P)--(xksi2);
\draw  [-Latex, dotted](P)--(xksi3);
\draw  [-Latex, dotted](P)--(xksi4);
\draw  [-Latex, dotted](P)--(xksi5);
\draw  [fill= white](xksi1)circle (1pt);
\draw  [fill= white](xksi2)circle (1pt);
\draw  [fill= white](xksi3)circle (1pt);
\draw  [fill= white](xksi4)circle (1pt);
\draw  [fill= white](xksi5)circle (1pt);
\draw  plot[smooth cycle, tension=.7] coordinates {(-2,-0.5) (-1,-1)  (-0.5,-2) (0,-2.5)  (1,-1.5) (1,-1)  (1,0)  (0,0.5) (-0.5,1.5)  (-1.5,1) (-2,0.5) (-2,0) };
\node[below] at(0,-2.5){$V^{"}_{\zeta}$};
\end{tikzpicture}}
\caption{Integral form of Hertz vectors }
\label{fig:fig_p221}
\end{figure}
If we take more and more points $\zeta_r$ and take the $C_k$ as a weight  factor, then in the limit, we get $\Pi^{(0)}_r\left(z\right)= \int_{V_{\zeta}}P_r\left(\zeta\right)F\left(z, \zeta\right)dV_{\zeta}$ where $P_r\left(\zeta\right)$ is a kind of density vector field.
On page $227$, it is mentioned that the vector field $P_r\left(\zeta\right)$ does not need to be continuous. This situation is illustrated in the figure $(b)$  where $V_{\zeta} =  V^{'}_{\zeta}\oplus V^{"}_{\zeta}$
 $$\blacklozenge$$
\newpage



\section{p227 - Exercise}
\begin{tcolorbox}
Show that Maxwell's equation in the form $\mathbf{6.356}$ are satisfied by 
 $$\begin{array}{ll}& E^{(0)}_r= ik\epsilon_{rpq}\int_{V_{\zeta}}Q_q\left(\zeta\right)F_{,p}dV_{\zeta}\\
 \mathbf{6.363}&\\
 &H^{(0)}_r= \int_{V_{\zeta}}Q_m\left(\zeta\right)\left(F_{,mr}+k^2F\delta_{mr}\right) dV_{\zeta}
 \end{array}$$
 where $Q_r\left(\zeta\right)$ is an arbitrary vector field, and $F$ is as in $\mathbf{6.344}$
\end{tcolorbox}
Let's first look at this with a discrete point and define 
\begin{align}
 E^{(0)}_r&= ik\epsilon_{rpq}\Pi^{(0)}_{q,p}\\
 H^{(0)}_r&= \Pi^{(0)}_{m,mr}+k^2\Pi^{(0)}_r
\end{align}
We check whether that Maxwell's equation in the form $\mathbf{6.356}$ are satisfied:
\begin{align}
&-ikE^{(0)}_r= \epsilon_{rmn}H^{(0)}_{n,m}\\
\text{(1) and (2)}\quad\Rightarrow \spatie &-ikik\epsilon_{rpq}\Pi^{(0)}_{q,p}= \underbrace{\epsilon_{rmn}\Pi^{(0)}_{q,qmn}}_{=0}+\epsilon_{rmn}k^2\Pi^{(0)}_{n,m}\\
\Rightarrow\spatie &k^2\epsilon_{rpq}\Pi^{(0)}_{q,p}= \epsilon_{rmn}k^2\Pi^{(0)}_{n,m}
\end{align}
So the first Maxwell equation is satisfied.\\
For the second
\begin{align}
&ikH^{(0)}_r= \epsilon_{rmn}E^{(0)}_{n,m}\\
\text{(1) and (2)}\quad\Rightarrow \spatie &ik\Pi^{(0)}_{m,mr}+ik^3\Pi^{(0)}_r= \epsilon_{rmn}ik\epsilon_{npq}\Pi^{(0)}_{q,pm}\\
\Rightarrow\spatie &\Pi^{(0)}_{m,mr}+k^2\Pi^{(0)}_r= \left(\delta_{rp}\delta_{mq}-\delta_{rq}\delta_{mp}\right)\Pi^{(0)}_{q,pm}\\
\Rightarrow\spatie &\Pi^{(0)}_{m,mr}+k^2\Pi^{(0)}_r= \Pi^{(0)}_{m,mr}-\Pi^{(0)}_{r,mm}\\
\Rightarrow\spatie &\Pi^{(0)}_{r,mm}+k^2\Pi^{(0)}_r= \Pi^{(0)}_{m,mr}-\Pi^{(0)}_{m,mr}=0
\end{align}
Conclusion, $(1)$ and $(2)$ are valid expressions of a EMW provided that the condition (10) is respected.
The rest of the reasoning is identical as for  the previous form for $E^{(0)}_r$ and $H^{(0)}_r$ when we take a linear combination of $\overset{n}{\Pi}^{(0)}_r$, each satisfying (10) and taking the limit on volume $V_{\zeta}$ while expressing $\Pi^{(0)}_{r}$ as $Q_{r}\left(\zeta \right)F\left(z,\zeta\right)$.
 $$\blacklozenge$$
\newpage


\section{p228 - Exercise}
\begin{tcolorbox}
Write out Maxwell's equations in terms of a magnetic vector and a skew-symmetric electric tensor.
\end{tcolorbox}
Let's  define 
\begin{align}
 &E_{rm}= \epsilon_{rmn}E_n\\
 \Rightarrow\spatie &E_r = \half \epsilon_{rmn}E_{mn}
\end{align}
Maxwell's equation :
\begin{align}\left\{ \begin{array}{ll}
\text{6.301(a),(b)}&E_{m,m}=0,\spatie H_{m,m}=0\\\\
\text{6.302(a)}&\frac{1}{c}\frac{\partial E_r}{\partial t}= \epsilon_{rmn}H_{n,m}\\\\
\text{6.302(b)}&\frac{1}{c}\frac{\partial H_r}{\partial t}= -\epsilon_{rmn}E_{n,m}
\end{array}\right.
\end{align}\\\\
$\mathbf{6.302(a)}$:
\begin{align}
\text{6.302(a)}\times\epsilon_{rmn}\Rightarrow\spatie \frac{1}{c}\frac{\partial E_{rm}}{\partial t}&= -\epsilon_{rmn}\epsilon_{npq}H_{q,p}\\
&= \left(\delta{mp}\delta{rq}-\delta{mq}-\delta{rp}\right)H_{q,p}\\
&= H_{r,m}-H_{m,r}
\end{align}\\
So,
$$\mathbf{\frac{\partial E_{rm}}{\partial t}=H_{r,m}-H_{m,r}}$$\\
$\mathbf{6.302(b)}$:\\
\begin{align}
\text{(2) in 6.302(b)}\Rightarrow\spatie \frac{1}{c}\frac{\partial H_r}{\partial t}&= \half \epsilon_{rmn}\epsilon_{npq}E_{pq,m}\\
&= \half \left( \delta_{rp}\delta_{mq}-\delta_{rq}\delta_{pm}\right)E_{pq,m}\\
&= \half \left( E_{rm,m} -E_{mr,m}\right)\\
&= E_{rm,m} \quad\text{(}E_{rm}\text{ is skew-symmetric)}
\end{align}\\
So,
$$\mathbf{\frac{1}{c}\frac{\partial H_r}{\partial t}= E_{rm,m}}$$\\
\\\\
$\mathbf{6.301(a)}$:\\
\begin{align}
\text{(2) in 6.301(a)}\quad\Rightarrow \spatie \epsilon_{rmn}E_{mn,r}=0
\end{align}
As this equation is homogeneous, we can permute the indices in $E_{mn,r}$ and write $\epsilon_{rmn}E_{rn,m}=0$ and $\epsilon_{rmn}E_{mr,n}=0$.\\
Adding this three equation together we get 
$$\mathbf{E_{mn,r}+E_{rn,m}+E_{mr,n}=0}$$\\
\\\\
$\mathbf{6.302(b)}$:\\
As in this case, nothing changes for $H_r$ we have\\
$$\mathbf{H_{n,n}=0}$$\\
 $$\blacklozenge$$
 \newpage
 

\section{p229 - Clarification}
\begin{tcolorbox}
$$\mathbf{6.371}\spatie F_{\alpha\beta} = H_{\alpha\beta}, \ =-F_{4\alpha}=E_{\alpha}, \ F_{44}=0$$
$$\mathbf{6.374}\spatie g_{\alpha\beta} = a_{\alpha\beta}, \ g_{\alpha4}=0, \ g_{44}=-1$$
\end{tcolorbox}
For clarity this give in matrix form
\begin{align}
\left(F_{mn}\right)=\left(\begin{matrix}
0&H_3&H_2&E_1\\
-H_3&0&H_1&E_2\\
H_2&-H_1&0&E_3\\
-E_1&-E_2&-E_3&0\\
\end{matrix}\right)
\end{align}
\begin{align}
\left(g_{mn}\right)=\left(\begin{matrix}
a_{11}&a_{12}&a_{13}&0\\
a_{12}&a_{22}&a_{23}&0\\
a_{13}&a_{23}&a_{33}&0\\
0&0&0&-1\\
\end{matrix}\right)
\end{align}
 $$\blacklozenge$$
\newpage


\section{p231 - Exercise}
\begin{tcolorbox}
Show that with homogeneous coordinates $z-r$ ($z_1,z_2,z_3$ being rectangular Cartesians in space and $z_4= ict=ix^4$ ) Maxwell's equations read
$$ F_{rm,n}+F_{mn,r}+F_{nr,m}=0, \ F_{rm,m}=0$$
Write out the components of $F_{mn}$ in terms of the real electric and magnetic vectors, noting which components are real and which are imaginary.
\end{tcolorbox}
We use the same convention as in the book: Greek indices are restricted to the space manifold. Extending this manifold to a $4$ dimensional manifold, latin suffixes will be used.
We have $\mathbf{6.369}$:
\begin{align}
\left\{\begin{array}{ll}
\text{(a)}&\frac{1}{c}\frac{\partial E_r}{\partial t}= a^{mn}H_{rm|n}\\\\
\text{(b)}&\frac{1}{c}\frac{\partial H_{rm}}{\partial t}= E_{r,m}-E_{m,r}\\\\
\text{(c)}&a^{mn}E_{n|m}=0\\\\
\text{(d)}&H_{rm,n}+H_{mn,r}+H_{nr,m}=0
\end{array}\right.
\end{align}
We rewrite (1):
\begin{align}
\left\{\begin{array}{ll}
\text{(a)}&\frac{\partial E_{\alpha}}{\partial (ict)}= -a^{\beta\gamma}H_{\alpha\beta|\gamma}\\\\
\text{(b)}&i\frac{\partial H_{\alpha\beta}}{\partial (ict)}= E_{\alpha,\beta}-E_{\beta,\alpha}\\\\
\text{(c)}&a^{\alpha\beta}E_{\beta|\alpha}=0\\\\
\text{(d)}&H_{\alpha\beta,\gamma}+H_{\beta\gamma,\alpha}+H_{\gamma\alpha,\beta}=0
\end{array}\right.
\end{align}
instead of  $\mathbf{6.371}$ let's define:
\begin{align}
&F_{\alpha\beta} =H_{\alpha\beta} , \  F_{\alpha4} =-F_{4\alpha} =-iE_{\alpha}, \ F_{44}=0\\
\Rightarrow \spatie &E_{\alpha}= iF_{\alpha4} =-iF_{4\alpha}
\end{align}

Using $(4)$, (2) becomes
\begin{align}
\left\{\begin{array}{ll}
\text{(a)}&-\frac{\partial F_{\alpha4}}{\partial (ict)}= a^{\beta\gamma}F_{\alpha\beta|\gamma}\\\\
\text{(b)}&i\frac{\partial F_{\alpha\beta}}{\partial (ict)}= iF_{\alpha4,\beta}-iF_{\beta4,\alpha}\\\\
\text{(c)}&a^{\alpha\beta}F_{\beta4,\alpha}=0\\\\
\text{(d)}&F_{\alpha\beta,\gamma}+F_{\beta\gamma,\alpha}+F_{\gamma\alpha,\beta}=0
\end{array}\right.
\end{align}
For homogeneous coordinates we have $a^{\alpha\beta}=1$ and $a^{\alpha4}=1$ so:
\begin{align}
\left\{\begin{array}{ll}
\text{(a)}&F_{\alpha4,4}= -F_{\alpha\beta|\beta}\\\\
\text{(b)}& F_{\alpha\beta,4}= F_{\alpha4,\beta}-F_{\beta4,\alpha}\\\\
\text{(c)}&F_{\beta4,\beta}=0\\\\
\text{(d)}&F_{\alpha\beta,\gamma}+F_{\beta\gamma,\alpha}+F_{\gamma\alpha,\beta}=0
\end{array}\right.
\end{align}

Let's look what happens when we extend the range  $\alpha,\beta,\gamma$ to $4$. First let's extend  $\gamma$ to $4$. The left part of equation (6d) can be written as 
\begin{align}
F_{\alpha\beta,\gamma}+F_{\beta\gamma,\alpha}+F_{\gamma\alpha,\beta}+ F_{\alpha\beta,4}+F_{\beta4,\alpha}+F_{4\alpha,\beta}
\end{align}
Consider 
\begin{align}
P= F_{\alpha\beta,4}+F_{\beta4,\alpha}+F_{4\alpha,\beta}
\end{align}
Let's extend $\alpha$ to $4$:
\begin{align}
P^{'}&= F_{4\beta,4}+F_{\beta4,4}+\underbrace{F_{44,\beta}}_{=0}\\
&= \underbrace{F_{4\beta,4}+F_{\beta4,4}}_{=0}\quad \text{(}F_{mn} \text{ is skew-symmetric)}
\end{align}
Extend $\beta$ to $4$:
\begin{align}
P^{"}&= F_{44,4}+F_{44,4}+F_{44,4}\\
&=0\quad \text{(}F_{mn} \text{ is skew-symmetric)}
\end{align}
So, we only have to prove that $P= F_{\alpha\beta,4}+F_{\beta4,\alpha}+F_{4\alpha,\beta}=0$.\\
From $(6b)$ we get:
\begin{align}
P&= F_{\alpha\beta,4}+F_{\beta4,\alpha}+F_{4\alpha,\beta}\\
&= F_{\alpha4,\beta}-F_{\beta4,\alpha}+F_{\beta4,\alpha}-F_{\alpha4,\beta}\\
&=0
\end{align}
We get so,
\begin{align}
F_{rm,n}+F_{mn,r}+F_{nr,m}=0
\end{align}
Consider now equation $(6c)$ and extend the suffixes from $3$ to $4$.\\
What is the value of the following expression?
\begin{align}
Q&=F_{4m,m}=F_{4\beta,\beta} + F_{44,4}
\end{align}
Obviously, $F_{mn}$ being skew-symmetric we have $F_{44}=0\quad\Rightarrow\quad F_{44,4}=0 \quad \Rightarrow\quad Q=0$.\\
So, the Maxwell equations reduce to 
\begin{align}\left\{\begin{array}{l}
F_{rm,n}+F_{mn,r}+F_{nr,m}=0\\
F_{rm,m}=0
\end{array}\right.
\end{align}
For the explicit expression of $F_{mn}$ we get from $(3)$ and $(4)$:
\begin{align}
F_{mn}=
\left(
\begin{matrix}
0&H_3&-H_1&-iE_1\\
-H_3&0&H_2&-iE_2\\
H_1&-H_2&0&-iE_3\\
iE_1&iE_2&iE_3&0\\
\end{matrix}
\right)
\end{align}
 $$\blacklozenge$$
\newpage



\section{p234 - Exercise 1}
\begin{tcolorbox}
For a fluid in motion referred to curvilinear coordinates, the kinetic energy of the fluid in any region $R$ is
$$T=\half\int_R\rho v_rv^rdV$$
Use the equation of motion $\mathbf{6.147}$ to show that, if we follow the particles which compose $R$, we have 
$$\dv{T}{t}= -\int_S p n_r v^rdS + \int_R \theta pdV + \int_R\rho v_rX^rdV$$
where $S$ is the bounding surface to $R$, and $n_r$ the unit vector normal to $S$ and drawn outward. Show further that if, instead of following the particles, we calculate the rate of change of $T$ for a fixed portion of space, we get the above expression with the following additional term $$ -\half\int_S\rho n_rv^rv_sv^sdS$$
\end{tcolorbox}
Let's recall that 
\begin{align}
\theta = v^r_{\ |r}\spatie  \mathbf{(6.126}\text{ page 196.)}
\end{align}
and
\begin{align}
\left\{\begin{array}{lll}
\text{(a)}&\frac{\partial v_r}{\partial t}+v^s v_{r|s} = X_r-\rho^{-1}p_{,r}&\quad \text{see (6.147)}\\\\
\text{(b)}&\frac{\partial v^r}{\partial t} + v_s v^{r}_{\ \ |s} = X^r - \rho^{-1} a^{rm}p_{,m}&\quad \text{ see exercise page 201}
\end{array}\right.
\end{align}
First let us note that, for the first part of the question, we move with the particles which means that for the considered region the mass contained in this region will remain unchanged and hence $\rho dV$ can be considered as a constant when bringing the derivation operator inside the volume integral.\\
So,
\begin{align}
T=\half\int_R\rho v_rv^rdV
\end{align}
\begin{align}
\Rightarrow\spatie\dv{T}{t} &=\half\int_R \left(\fdv{v_r}{t}v^r+\fdv{v_r}{t}v^r\right) \rho dV\\
&=\half\int_R \left[\left(\partial_t v_r+v^s v_{r|s}\right)v^r+v_r\left( \partial_t v^r + v_s v^{r}_{\ \ |s} \right)\right] \rho dV\\
&=\half\int_R \left[\left(X_r-\rho^{-1}p_{,r}\right)v^r+v_r\left(X^r - \rho^{-1} a^{rm}p_{,m}\right)\right] \rho dV\\
&=\half\int_R \left(\underbrace{X_rv^r}_{=X^rv_r}-\rho^{-1}p_{,r}v^r+X^rv_r - \rho^{-1} a^{rm}v_r p_{,m}\right) \rho dV\\
&=-\half\int_R \left(p_{,r}v^r + \underbrace{a^{rm}v_r}_{= v^m} p_{,m}\right)  dV+\int_R\rho X^rv_r dV\\
&=-\int_R p_{,r}v^r dV+\int_R \rho X^rv_r dV
\end{align}
Let's look at the expression $p_{,r}v^r$ in the first integral in (9).\\
Obviously:
\begin{align}
&\left(pv^r\right)_{,r} = p_{,r}v^r+pv^r_{,r}\\
\Rightarrow \spatie&p_{,r}v^r = \left(pv^r\right)_{,r} -pv^r_{,r}
\end{align}
Let's note also that
\begin{align}
\left(pv^r\right)_{|r} -pv^r_{\ |r} &=  \left(pv^r\right)_{,r} +\Gamma^{r}_{mr}\left(pv^m\right)-pv^r_{,r}-p\Gamma^{r}_{mr}v^m \\
&=  \left(pv^r\right)_{,r}-pv^r_{,r} \\
\text{(11) becomes}\spatie p_{,r}v^r&=\left(pv^r\right)_{|r} -pv^r_{\ |r}
\end{align}
Substituting in (9):
\begin{align}
\dv{T}{t} &=-\int_R \left(pv^r\right)_{|r} dV+\int_R pv^r_{\ |r} dV+\int_R \rho X^rv_r dV 
\end{align}
Using Green's theorem $ \int F_rn^rdS = \int F^r_{\ |r}dV$, and putting $ F^r=pv^r$:  
\begin{align}
\dv{T}{t} &=-\int_S p\underbrace{v_r n^r}_{=v^r n_r } dS+\int_R p\underbrace{v^r_{\ |r}}_{=\theta} dV+\int_R \rho X^rv_r dV 
\end{align}
giving
\begin{align}
\mathbf{\dv{T}{t} =-\int_S p v^r n_rdS+\int_R p\theta dV+\int_R \rho X^rv_r dV} 
\end{align}
$$\lozenge$$\\\\
What if we look at the rate of change of $T$ in a fixed region?\\
Then $\rho dV$ can't be considered as a constant and bringing the derivative operator under the integral will generate an additional term
\begin{align}
\half\int_R v_rv^r \fdv{\rho}{t} dV
\end{align}
As we fix the spatial coordinates, we have $\fdv{\rho}{t}= \partial_t \rho$ and by $\mathbf{6.127b}$ we have 
\begin{align}
\partial_t \rho= -\left(\rho v^r\right)_{|r}
\end{align}
So $(18)$ becomes
\begin{align}
\text{(18)}=-\half\int_R v_sv^s \left(\rho v^r\right)_{|r} dV
\end{align}
Using again Green's theorem $ \int F_rn^rdS = \int F^r_{\ |r}dV$ with   $ F^r=\rho v^r$ and nothing that $v_sv^s$ is an invariant: 
\begin{align}
\text{(18)}=-\half\int_S v_sv^s \rho v^rn_r dV
\end{align}
 $$\blacklozenge$$
\newpage



\section{p235 - Exercise 2}
\begin{tcolorbox}
Consider a fluid in which $\rho$ is a function of $p$, moving under a conservative body force. Show that if the motion is steady, but not necessarily irrotational, then the following quantity is constant along each stream line:
$$\half v_rv^r+P+U$$
(a stream line is a curve which, at each point , has the direction of the velocity vector $v^r$).
Compare and contrast this result with $\mathbf{6.154}$.
\end{tcolorbox}
What is given:

\begin{align}
\left\{\begin{array}{lll}
\text{(a)}&\partial_t v_r + v^sv_{r|s} = X_r- \rho^{-1} p_{,r}&\quad \text{see (6.147)}\\\\
\text{(b)}&P_{,r}= \rho^{-1} p_{,r}&\quad \text{ see (6.150)}\\\\
\text{(c)}&X_{r}= -U_{,r}&\quad \text{ see (6.151)}
\end{array}\right.
\end{align}
As the motion is stationary: $ \partial_t v_r =0$ and so .
\begin{align}
v^sv_{r|s} + U_{,r}+P_{,r}= 0
\end{align}

Let's consider a stream line given by the set of equations $x^r = x^r(u)$, where $u$ is a parameter. Then, by definition of a streamline, and considering a steady flow, we have $\dv{x^r}{u} = kv^r$. Considering $(2)$ we have 
\begin{align}
&v^sv_{r|s} + U_{,r}+P_{,r}= 0\\
\times \dv{x^r}{u} \spatie &v^sv_{r|s}\dv{x^r}{u} + U_{,r}\dv{x^r}{u}+P_{,r}\dv{x^r}{u}= 0\\
\spatie &v^sv_{r|s}\dv{x^r}{u} + \dv{U}{u}+\dv{P}{u}= 0
\end{align}
Let's look at the first term:
\begin{align}
v^sv_{r|s}\dv{x^r}{u}&=kv^sv_{r|s}v^r\\
&=kv^s\left(a_{rm}v^m\right)_{|s}v^r\\
&=kv^sv^m_{\  |s}a_{rm}v^r\\
&=kv^sv^m_{\ |s}v_m\\
&=kv^sv^r_{\ |s}v_r
\end{align}
So (5) can be equivalently written as $kv^sv_{r|s}v^r + \dv{U}{u}+\dv{P}{u}= 0$ and $kv^sv^r_{\  |s}v_r+ \dv{U}{u}+\dv{P}{u}= 0$. Summing these two gives
\begin{align}
&kv^s\left(v_{r|s}v^r+ v^r_{\  |s}v_r\right) + 2\dv{U}{u}+2\dv{P}{u}= 0\\
\Leftrightarrow\spatie &kv^s\left(v_{r}v^r\right)_{|s} + 2\dv{U}{u}+2\dv{P}{u}= 0\\
\Leftrightarrow\spatie &kv^s\left(v_{r}v^r\right)_{,s} + 2\dv{U}{u}+2\dv{P}{u}= 0\spatie v_{r}v^r \text{ is an invariant}\\
\Leftrightarrow\spatie &\dv{x^s}{u}\left(v_{r}v^r\right)_{,s} + 2\dv{U}{u}+2\dv{P}{u}= 0\\
\Leftrightarrow\spatie &\dv{\left(v_{r}v^r\right)}{u} + 2\dv{U}{u}+2\dv{P}{u}= 0
\end{align}
Integrating expression (15) gives 
$$\mathbf{\half v_{r}v^r + U+P= C}$$
with $C$ a constant along a streamline.
$$\lozenge$$
Compare and contrast this result with $\mathbf{6.154}$ (irrotational motion).
$$-\partial_t\phi + \half a^{mn}\phi_{,m}\phi_{,n}+P+U=F(t)$$
For a stationary motion ($\partial_t\phi=0, \ F(t)=constant$) the expression reduces to 
$$\mathbf{\half a^{mn}\phi_{,m}\phi_{,n}+P+U=C}$$
Replacing in the general expression $\half v_{r}v^r + U+P= C$, $v_{r}v^r $ by $v_r = -\phi_{,r}$ and $v^r = -a^{rm}\phi_{,m}$ gives obviously $\half a^{mn}\phi_{,m}\phi_{,n}+P+U=C$.
 $$\blacklozenge$$
\newpage



\section{p235 - Exercise 3}
\begin{tcolorbox}
For the general motion of the fluid described in Exercise 2, prove that $$\dv{}{t}\int_C v_rdx^r =0$$
where the integral is take round any closed curve, and $\dv{}{t}$ is the co-moving time derivative.
\end{tcolorbox}

\begin{figure}[H]%
    \centering
\begin{tikzpicture}[scale = 0.8]
\coordinate (O) at (-15,-5) ;
\coordinate (X) at (-17,-7) ;
\coordinate (Y) at (-11,-5) ;
\coordinate (Z) at (-15,-1) ;
\draw[-Latex] (O)--(X);
\draw [-Latex](O)--(Y);
\draw [-Latex](O)--(Z);
\coordinate (p1) at (-10,-3)  ;
\coordinate (p2) at (-12,-1)  ;
\coordinate (p3) at (-14,-0.5)  ;
\coordinate (p4) at (-12.5,-2) ;
\coordinate(pin) at (-12.5,-0.5) {};
\coordinate(pinb) at (-10,2.5) {};

\coordinate (p1b) at (-7,-1) {} {} {};
\coordinate (p2b) at (-9,1.5) {} {} {} {} {};
\coordinate (p3b) at (-12,2.5) {} {} {} {};
\coordinate (p4b) at (-9,0.5) {} {};
\draw[dashed]  [decoration={markings, mark=at position 0.52 with {\arrow[scale = 0.1]{Latex[length=20mm]}}},    postaction={decorate}]  plot[smooth, tension=.7]   coordinates {(pin) (-12,0.5) (pinb)};
\draw[dashed]     plot[smooth cycle, tension=1.] coordinates {(p1) (p2) (p3) (p4) };

\draw[dashed]  [decoration={markings, mark=at position 0.52 with {\arrow[scale = 0.1]{Latex[length=20mm]}}},    postaction={decorate}]  plot[smooth, tension=.7]   coordinates {(p2) (-11,0) (p2b)};
\draw  [fill= white](p2)circle (1.5pt);
\draw  [fill= white](p2b)circle (1.5pt);

\draw[dashed, thick,decoration={markings, mark=at position 0.92 with {\arrow[scale = 0.1]{Latex[length=20mm]}}},    postaction={decorate}]   plot[smooth cycle, tension=1] coordinates {(p1b) (p2b) (p3b) (p4b)};
\path[fill=gray!20,opacity=0.7]  plot[smooth, tension=.7] coordinates {(p3) (-14.003,-0.6401) (-13.9067,-0.8602) (-13.7967,-1.0184) (-13.6179,-1.1834) (-13.3428,-1.4172) (-13.1227,-1.5823) (-12.8682,-1.7542) (p4) (-12.1736,-2.1944) (-11.6991,-2.4626) (-11.4584,-2.5933) (-11.1764,-2.724) (-10.9082,-2.8546) (-10.6606,-2.944) (-10.3924,-3.0334) (-10.1792,-3.0541) (p1) (-9.865,-2.6831) (-9.9558,-1.8885)   (-10.4377,-0.8215) (-11.2083,-0.0814) (-12.0901,0.2502)  (-12.8008,0.1747) (-13.3825,-0.0349)   (-13.776,-0.2481) (-13.8723,-0.3306) (-13.9961,-0.4957) (-13.9961,-0.4957)};
%\draw[thin,gray,]  plot[smooth, tension=.7] coordinates { (p1) (-10.0193,-2.5059) (-10.131,-2.16) (-10.3511,-1.7828) (-10.6262,-1.3829) (-10.86,-1.039) (-11.2451,-0.6263) (-11.7128,-0.3237) (-12.1874,-0.1656) (-12.5381,-0.1037) (-12.9094,-0.0418) (-13.2396,-0.0624) (-13.556,-0.1449) (-13.776,-0.2481) (-13.8723,-0.3306) (-13.9961,-0.4957) (-13.9961,-0.4957)};
\path [fill=gray!20,opacity=0.7,line width=0pt]  plot[smooth cycle, tension=.7] coordinates {(p3b) (-11.8795,2.312) (-11.6342,2.1106) (-11.2576,1.8828) (-10.3204,1.3047) (-9.5934,0.8668) (p4b) (-8.3408,-0.0179) (-7.7014,-0.5435) (-7.3773,-0.8238) (-7.2634,-0.9551) (-7.1583,-1.0252) (p1b) (-7.0883,-0.1561) (-7.2547,0.5797) (-7.535,1.3242) (-7.9467,1.9003) (-8.7701,2.552)    (-10.1102,2.9323) (-11.1788,2.9614) (-11.8269,2.7229) (-11.9321,2.6266) (-12.0021,2.4952)};
%\draw[thin,gray,] plot[smooth , tension=.7] coordinates {(p1b) (-6.9306,-0.5172) (-7.1671,0.5952) (-7.4211,1.296) (-7.9467,1.9003) (-8.4634,2.3558) (-9.2167,2.8814) (-9.8649,3.0828) (-10.5744,3.0916) (-11.1262,2.9945) (-11.4941,2.8543) (-11.8269,2.7229) (-11.9321,2.6266) (-12.0021,2.4952)};

%\node[above] at (p1) {$p_1$};
%\node[above] at (p2) {$p_2$};
%\node[above] at (p3) {$p_3$};
%\node[above] at (p4) {$p_4$};

%\node[above] at (p1b) {$p_1$};
%\node[above] at (p2b) {$p_2$};
%\node[above] at (p3b) {$p_3$};
%\node[above] at (p4b) {$p_4$};


\coordinate (Cdt) at (-6.626,-1.5028) {} {} {} {} {};
\node[right] at (Cdt) {$\partial\Sigma_{t+dt}$};

\coordinate (C) at (-11.5,-3.5) {} {} {} {} {} {} {};
\node[left] at (C) {$\partial\Sigma_t$};
\node[right] at (-8.5,-2) {stream lines};
\node[left] at (-13,1.5) {$v^rdt$};
\coordinate (sigma_dt) at (-6.8209,2.3645) ;
\coordinate (sigma_t)  at (-8.645,-2.9141) ;
\coordinate (sigma_s)  at(-13.2133,2.5744);
\node[right] at (sigma_t) {$\Sigma_t$};
\node[above] at (sigma_dt) {$\Sigma_{t+dt}$};
\node[left] at (sigma_s) {$\Sigma_{s}$};

%\draw[dashed]    plot[smooth, tension=.7]   coordinates {(pin) (-12,0.5) (-10.8634,1.6719)};
\draw [decoration={markings, mark=at position 0.52 with {\arrow[scale = 0.1]{Latex[length=20mm]}}},    postaction={decorate}]  plot[smooth, tension=.7]   coordinates {(p3) (-13.5,0.5) (p3b)};
\draw [decoration={markings, mark=at position 0.52 with {\arrow[scale = 0.1]{Latex[length=20mm]}}},    postaction={decorate}]  plot[smooth, tension=.7]   coordinates {(p4) (-11.5,-1) (p4b)};
\draw[decoration={markings, mark=at position 0.52 with {\arrow[scale = 0.1]{Latex[length=20mm]}}},    postaction={decorate}]  plot[smooth, tension=.7]  coordinates {(p1) (-8.5,-2) (p1b)};

\draw  [fill= black](pinb)circle (1.5pt);
\draw  [fill= black](pin)circle (1.5pt);
\draw [very thick]  plot[smooth , tension=0.7] coordinates {(p3b) (-11.8795,2.312) (-11.6342,2.1106) (-11.2576,1.8828) (-10.3204,1.3047) (-9.5934,0.8668) (p4b) (-8.3408,-0.0179) (-7.7014,-0.5435) (-7.3773,-0.8238) (-7.2634,-0.9551) (-7.1583,-1.0252) (-7.1,-1)(p1b) };
\draw [very thick,decoration={markings, mark=at position 0.62 with {\arrow[scale = 0.1]{Latex[length=20mm]}}},    postaction={decorate}]   plot[smooth, tension=1.] coordinates {(p3) (-14.003,-0.6401) (-13.9067,-0.8602) (-13.7967,-1.0184) (-13.6179,-1.1834) (-13.3428,-1.4172) (-13.1227,-1.5823) (-12.8682,-1.7542) (p4) (-12.1736,-2.1944) (-11.6991,-2.4626) (-11.4584,-2.5933) (-11.1764,-2.724) (-10.9082,-2.8546) (-10.6606,-2.944) (-10.3924,-3.0334) (-10.1792,-3.0541) (p1) };



\draw  [fill= white](p1)circle (1.5pt);

\draw  [fill= white](p3)circle (1.5pt);
\draw  [fill= white](p4)circle (1.5pt);
\draw  [fill= white](p1b)circle (1.5pt);
\draw  [fill= white](p3b)circle (1.5pt);
\draw  [fill= white](p4b)circle (1.5pt);

\draw[ultra thin]  plot[smooth, tension=.7] coordinates {(-10.4045,-3.0513) (-10.7193,-3.358) (C)};
\draw[ultra thin]  plot[smooth, tension=.7] coordinates {(Cdt) (-7.4666,-1.1949) (-7.7087,-0.533)};
\draw[ultra thin]  plot[smooth, tension=.7] coordinates {(sigma_t) (-9.2503,-2.0908) (-9.9606,-1.9052)};
\draw [ultra thin] plot[smooth, tension=.7] coordinates {(sigma_dt) (-6.8854,1.6301) (-7.5473,1.3153)};

\draw  (-12.39,1.5009) node (v1) {} ellipse (0.1291 and 0.0969);

\draw  plot[smooth, tension=.7] coordinates {(sigma_s) (-12.6806,2.316) (v1)};
\coordinate (n2a) at (-9.5,2.4) {};
\coordinate (n2b) at (-9.6,3.6) {} {};
\node[above] at (n2b) {$\hat{n}$};
\draw  [fill= white](n2a)ellipse (0.1291 and 0.0469);
\draw[thick, -Latex](n2a)--(n2b);
\coordinate (n1a)  at (-10.5,-1.3) {};
\coordinate (n1b) at (-9.5,-1.2) {} {};
\node[right] at (n1b) {$\hat{n}$};
\draw  [fill= white](n1a)ellipse ( 0.0469 and 0.1291);
\draw[thick, -Latex](n1a)--(n1b);
\end{tikzpicture}
\caption{Evolution of a closed co-moving curve }
\label{fig:fig_p2235}
\end{figure}
Let's see what happens when a closed  curve moves during a time $dt$. All the particles along that curve $\partial\Sigma_t$ will move along stream lines and will form a new closed curve $\partial\Sigma_{t+dt}$. Also all particles lying on a surface $\Sigma_t$ enclosed by the closed curve will end on the surface $\Sigma_{t+dt}$.
So we can use the Kelvin-Stokes theorem and form
\begin{align}
\oint_{\partial\Sigma_{t+dt} }{v_rdx^r }-\oint_{\partial\Sigma_{t}} v_rdx^r&= \iint_{\Sigma_{t+dt}}\epsilon_{rjk}v_{k,j}n^rdS-\iint_{\Sigma_{t}}\epsilon_{rjk}v_{k,j}(n^r)dS
\end{align}

Consider now the surface formed by the envelope of the stream lines starting from the curve $\partial\Sigma_t$.

What we try to achieve is to go from a curve integral to a volume integral using first the Kelvin-Stokes theorem followed by the divergence theorem and adding a term which we have to prove it is equal to zero
\begin{align}
\oint_{\partial\Sigma_{t+dt} }{v_rdx^r }-\oint_{\partial\Sigma_{t}} v_rdx^r&= \iint_{\Sigma_{t+dt}}\epsilon_{rjk}v_{k,j}n^rdS-\iint_{\Sigma_{t}}\epsilon_{rjk}v_{k,j}(-n^r)dS+\underbrace{\iint_{\Sigma_{s}}\epsilon_{rjk}v_{k,j}n^rdS}_{=0?}\\
&= \iint_{\Sigma_{t+dt}}\epsilon_{rjk}v_{k,j}n^rdS+\iint_{\Sigma_{t}}\epsilon_{rjk}v_{k,j}n^rdS+\underbrace{\iint_{\Sigma_{s}}\epsilon_{rjk}v_{k,j}n^rdS}_{=0?}\\
&= \iiint_V \epsilon_{rjk}v_{k,jr}dV\\
&=0
\end{align}
Note that in the second term of the right expression, we changed the sign of the normal vector $\hat{n}$ as we want the normal vector on the surface point outward of the considered volume.\\\\

So we have to prove that indeed,

\begin{align}\iint_{\Sigma_{s}}\epsilon_{rjk}v_{k,j}n^rdS=0\end{align}
Nothing that $n^rdS$ can be expressed as the cross product of $\overline{v}dt$ and $d\overline{l}$ (the last being the line segment along the curve $\partial_t\Sigma$), we have  
\begin{align}
n^rdS&= \epsilon_{rpq}v^pdx^qdt
\end{align}
And get
\begin{align}
\iint_{\Sigma_{s}}\epsilon_{rjk}v_{k,j}n^rdS&= \iint_{\Sigma_{s}}\epsilon_{rjk}v_{k,j}\epsilon_{rpq}v^pdx^qdt\\
&= \iint_{\Sigma_{s}}\left(\delta_{jp}\delta_{kq}-\delta_{jq}\delta_{kp}\right)v_{k,j}v^pdx^qdt\\
&= \iint_{\Sigma_{s}}\left(v_{q,p}v^pdx^q-v_{p,q}v^pdx^q\right)dt\\\left(v_pdt = dx^p\right)\quad\Rightarrow\spatie
&= \iint_{\Sigma_{s}}\left(v_{q,p}dx^pdx^q-v_{p,q}dx^pdx^q\right)\\
&=0
\end{align}
So $(5)$ is proven and get $d\oint_{\partial\Sigma_{t}} v_rdx^r=0$ during an infinitesimal time $dt$ giving 
$$\dv{}{t}\oint_{\partial\Sigma_{t}} v_rdx^r=0$$
 $$\blacklozenge$$
\newpage


\section{p235 - Exercise 4}
\begin{tcolorbox}
Curves having at each point the direction of the vorticity vector $\omega^r$ are called "vortex lines". Prove that $\int_{C} v_rdx^r$ has the same value for all closed curves $C$ which lie on the surface of the tube of vortex lines, and go once around the tube in the same sense. (Use Stokes' theorem; cf. 7.502.).
\end{tcolorbox}
\begin{figure}[H]%
    \centering
\begin{tikzpicture}[scale=0.8]
\coordinate (O) at (-15,-5) ;
\coordinate (X) at (-17,-7) ;
\coordinate (Y) at (-11,-5) ;
\coordinate (Z) at (-15,-1) ;
\draw[-Latex] (O)--(X);
\draw [-Latex](O)--(Y);
\draw [-Latex](O)--(Z);
\coordinate (p0) at(-14.0384,-4.8633) {}  ;
\coordinate (p1) at (-13.5,-4) {}  ;
\coordinate (p2) at (-12,-2.5) {} {}  ;
\coordinate (p3) at (-10,-1) {}  ;
\coordinate (p4) at (-8.715,-0.3396);


\coordinate (m0) at(-13.056,-5.286)  {}  ;
\coordinate (m1) at(-12.6333,-4.5206) {}  ;
\coordinate (m2) at (-11.3653,-3.2747) {}  {}  ;
\coordinate (m3) at (-9.389,-1.8018) {} {}  ;
\coordinate (m4) at(-8.4523,-1.2877);

\coordinate (q0) at (-14.8152,-4.2007)  {}  ;
\coordinate (q1) at (-14.3697,-3.4696) {}  ;
\coordinate (q2) at (-12.8618,-1.983) {} {}  {}  ;
\coordinate (q3) at (-10.6228,-0.2025)  {}  ;
\coordinate (q4) at(-9.2862,0.4372);

\coordinate(pin) at (-11.2607,-4.0646) {} {} {} {} {} {} {};
\coordinate (pinb) at (-13.4505,-0.6099) {} {} {} {} {} {} {};
\draw [rotate=-40,black,thick,dashed] (m2) arc (0:180:1 and 0.5);\draw[dashed][thick]  plot[smooth, tension=.7] coordinates {(p0) (p1) (p2) (p3) (p4)};
\draw [dashed, ultra thin] plot[smooth, tension=.7] coordinates {(-14.5068,-4.1824) (-13.7871,-3.3828) (-12.2449,-1.9548) (-10.1201,-0.3213) (-9.1377,0.1585)};
\draw[dashed,ultra thin]  plot[smooth, tension=.7] coordinates {(-13.3416,-5.1306) (-12.6562,-4.1596) (-11.331,-2.9715) (-9.3091,-1.4408) (-8.5665,-1.0752)};


\draw  [fill= white](p1)circle (1.5pt);
\draw  [fill= white](p2)circle (1.5pt);
\draw  [fill= white](p3)circle (1.5pt);
\path[fill=gray!20,opacity=0.7] plot[smooth cycle, tension=.7] coordinates { (-12.9303,-2.3525) (-12.8618,-2.5236) (-12.6793,-2.7746) (-12.4054,-3.0484) (-12.3142,-3.1055) (-12.1088,-3.2424) (-11.8122,-3.3451) (-11.5954,-3.3679) (-11.4585,-3.3679) (m2) (-11.2417,-3.0256) (-11.2645,-2.5921)  (-11.5497,-1.9417) (-11.8806,-1.8048) (-12.2229,-1.7934) (-12.4739,-1.8162) (q2) (-12.8618,-1.9645) (-12.9351,-2.191) (-12.8846,-2.4437) (-12.8504,-2.5236)};

\path [fill=gray!20,opacity=0.7] plot[smooth cycle, tension=.7] coordinates {(-10.7595,-0.7696) (-10.7108,-0.8915) (-10.5889,-1.119) (-10.3858,-1.4115) (-10.2477,-1.5496) (-10.0689,-1.6796) (-9.9227,-1.769) (-9.6952,-1.8502) (-9.6058,-1.8665) (-9.4839,-1.8502) (m3) (-9.2402,-1.5334) (-9.2402,-1.1921) (-9.2483,-0.8428) (-9.3458,-0.6234) (-9.4758,-0.4853) (-9.6383,-0.3309) (-9.8577,-0.2822) (-10.077,-0.2334) (-10.3289,-0.209) (-10.4995,-0.209) (q3)  (-10.7433,-0.3959) (-10.792,-0.5421) (-10.7514,-0.7859)};
\coordinate (omega) at (-11.7872,-1.3915) {};
\draw [rotate=-45,black] (omega)   circle(0.051cm and 0.1cm);
\node (omega2) at (-12.544,-0.3591) {};
\draw [-Latex](omega)--(omega2);
\node[right]  at (omega2){$\hat{n}_{\omega}$};
\draw [rotate=-30,black,thick,decoration={markings, mark=at position 0.52 with {\arrow[scale = 0.1]{Latex[length=-20mm]}}},    postaction={decorate}] (m1) arc (0:-180:1 and 0.5);
\draw [rotate=-30,black,thick,dashed] (m1) arc (0:180:1 and 0.5);

\draw [rotate=-40,black,thick,decoration={markings, mark=at position 0.52 with {\arrow[scale = 0.1]{Latex[length=-20mm]}}},    postaction={decorate}] (m2) arc (0:-180:1 and 0.5);


\draw [rotate=-50,black,thick,decoration={markings, mark=at position 0.52 with {\arrow[scale = 0.1]{Latex[length=-20mm]}}},    postaction={decorate}] (m3) arc (0:-180:1 and 0.5);
\draw [rotate=-50,black,thick,dashed] (m3) arc (0:180:1 and 0.5);



\node[right] at (pin) {$\partial \Sigma_C$};
\node[left] at (pinb) {$\Sigma_{\omega}$};


\draw[thick] plot[smooth, tension=.7] coordinates { (m0) (m1) (m2) (m3) (m4) };
\draw[thick]  plot[smooth, tension=.7] coordinates { (q0) (q1) (q2) (q3) (q4) };
%\draw[thick]  plot[smooth, tension=.7] coordinates {(-14.5296,-4.5137) (-14.2098,-4.0796) (-12.8161,-2.6517) (-10.7142,-0.9838) (-9.0577,0.01)};
%\draw[thick]  plot[smooth, tension=.7] coordinates {(-13.7071,-5.1648) (-13.3758,-4.6165) (-12.005,-3.3028) (-10.0402,-1.7264) (-8.6122,-0.8467)};
\draw [thick] plot[smooth, tension=.7] coordinates {(-14.404,-4.6095) (-14.1184,-4.1868) (-12.7475,-2.736) (-10.6456,-1.0453) (-8.9892,-0.12)};
\draw[thick]  plot[smooth, tension=.7] coordinates {(-13.73,-5.0893) (-13.4444,-4.5866) (-12.2221,-3.193) (-10.1544,-1.6394) (-8.6693,-0.7141)};

\draw []  plot[smooth, tension=.7] coordinates {(pinb) (-13.0741,-1.1253) (omega)};
\node (v2) at (-12.0164,-3.2729) {};
\draw [Latex-]  plot[smooth, tension=.7] coordinates {(v2) (-11.6394,-3.7413) (pin)};
\node[right] (v3) at (-8.7036,1.4295) {\text{vortex lines}};
\draw[Latex-]  plot[smooth, tension=.7] coordinates { (-9.8688,0.2293) (-9.2291,0.9883)(-8.6922,1.1775) };

\draw  plot[smooth, tension=.7] coordinates {(p0)};
\draw  plot[smooth, tension=.7] coordinates {};
\coordinate (v4) at (-13.8929,-1.5641) {} ;
\node[above] at (v4)  {$\Sigma_C$};
\node (v5) at (-12.6444,-2.1021) {};

\draw  plot[smooth, tension=.7] coordinates {(v4) (-13.5052,-2.0732) (v5)};

\coordinate (v5) at (-11.427,0.7816) {};
\node[above] at (v5)  {$\Sigma_{C^{'}}$};
\node (v7) at (-10.4433,-0.4949) {};
\draw  plot[smooth, tension=.7] coordinates {(v7) (-10.6189,0.3483) (v5)};
\coordinate (v8) at (-9.0263,-2.1695) {} ;
\node[right] at (v8) {$\partial \Sigma_{C^{'}}$};;
\node (v6) at (-9.9397,-1.7948) {};
\draw[-Latex]  plot[smooth, tension=.7] coordinates {(v8) (-9.5064,-2.2515) (v6)};
\coordinate (el1) at (-11.6488,-3.0231) {};
\coordinate (el2) at (-9.5853,-1.55) {};
\draw [rotate=-10,black] (el1) node (v1) {}  circle(0.051cm and 0.1cm);
\draw [rotate=3,black] (el2) node (v1) {}  circle(0.051cm and 0.1cm);
\coordinate (el1b) at (-10.4064,-3.2345) {} {};
\coordinate (el2b) at (-8.1674,-1.4681) {};
\draw [-Latex](el1)--(el1b);
\node[right]  at (el1b){$\hat{n}$};
\draw [-Latex](el2)--(el2b);
\node[right]  at (el2b){$\hat{n}$};



\end{tikzpicture}
\caption{Vortex line tube }
\label{fig:fig_p2235}
\end{figure}
We have in rectangular Cartesian coordinates
\begin{align}
\textbf{(6.128)}\quad\spatie &\omega_{jk}=\half\left(v_{k,j}-v_{j,k}\right)\\
\textbf{(6.129)}\quad\spatie &\omega_r=\half\epsilon_{rjk}\omega_{jk}
\end{align}
So we can use the Kelvin-Stokes theorem and form
\begin{align}
\oint_{\partial\Sigma_{C^{'}}} {v_rdx^r }-\oint_{\partial\Sigma_{C}} v_rdx^r&= \iint_{\Sigma_{C^{'}}}\epsilon_{rjk}v_{k,j}n^rdS-\iint_{\Sigma_{C}}\epsilon_{rjk}v_{k,j}n^rdS
\end{align}
Obviously, we have on the surface $\Sigma_{\omega}$ of the tube,  $\hat{\omega}\cdot\hat{n}_{\omega}=0$ and thus $\iint_{\Sigma_{\omega}}\omega_rn^rdS=0$.
Expanding the terms under the integral gives (by $(6.129)$ and $(6.128)$)
\begin{align}
\iint_{\Sigma_{\omega}}\omega_rn^rdS&=\frac{1}{4}\iint_{\Sigma_{\omega}}\epsilon_{rjk}\left(v_{k,j}-v_{j,k}\right)n^rdS
\end{align}
As $\iint_{\Sigma_{\omega}}\omega_rn^rdS=0$ we can put 
\begin{align}
\iint_{\Sigma_{\omega}}\omega_rn^rdS&=\frac{1}{2}\iint_{\Sigma_{\omega}}\epsilon_{rjk}v_{k,j}n^rdS-\frac{1}{2}\iint_{\Sigma_{\omega}}\epsilon_{rjk}v_{j,k}n^rdS\\
&=\frac{1}{2}\iint_{\Sigma_{\omega}}\epsilon_{rjk}v_{k,j}n^rdS-\frac{1}{2}\iint_{\Sigma_{\omega}}\epsilon_{rkj}v_{k,j}n^rdS\\
&=\frac{1}{2}\iint_{\Sigma_{\omega}}\epsilon_{rjk}v_{k,j}n^rdS+\frac{1}{2}\iint_{\Sigma_{\omega}}\epsilon_{rjk}v_{k,j}n^rdS\\
&=\iint_{\Sigma_{\omega}}\epsilon_{rjk}v_{k,j}n^rdS\quad\left(=0\right)
\end{align}
Adding $(8)$ in $(3)$ we get 
\begin{align}
\oint_{\partial\Sigma_{C^{'}}} {v_rdx^r }-\oint_{\partial\Sigma_{C}} v_rdx^r&= \iint_{\Sigma_{C^{'}}}\epsilon_{rjk}v_{k,j}n^rdS+\iint_{\Sigma_{C}}\epsilon_{rjk}v_{k,j}n^rdS+\iint_{\Sigma_{\omega}}\epsilon_{rjk}v_{k,j}n^rdS
\end{align}
Note that in the second term of the right expression, we changed the sign of the normal vector $\hat{n}$ as we want the normal vector on the surface point outward of the considered volume.\\
Using Green's theorem:
\begin{align}
\oint_{\partial\Sigma_{C^{'}}} {v_rdx^r }-\oint_{\partial\Sigma_{C}} v_rdx^r&= \iint_{\Sigma_{C^{'}}}\epsilon_{rjk}v_{k,j}n^rdS+\iint_{\Sigma_{C}}\epsilon_{rjk}v_{k,j}n^rdS+\iint_{\Sigma_{\omega}}\epsilon_{rjk}v_{k,j}n^rdS\\
 &= \iiint_V \epsilon_{rjk}v_{k,jr}dV\\
 &=0
\end{align}
where $V$ is the volume enclosed by the surfaces $\Sigma_{C}$, $\Sigma_{C^{'}}$ and $\Sigma_{\omega}$.\\
And get $d\oint_{\partial\Sigma_{t}} v_rdx^r=0$ along  an infinitesimal displacement along the vortex line  giving 
$$\dv{}{s}\oint_{\partial\Sigma_{t}} v_rdx^r=0$$ where $s$ is a parameter upon which a vortex line can be expressed. \\
Hence $\oint_{\partial\Sigma_{t}} v_rdx^r$ is constant along a vortex line.

 $$\blacklozenge$$
\newpage



\section{p235 - Exercise 5}
\begin{tcolorbox}
Prove that for the type of fluid described in Exercise 2, the voricity tensor satisfies the differential equations
$$\dv{}{t}\omega_{rs} = \omega_{pr}v_{p,s} - \omega_{ps}v_{p,r}$$
the coordinates being rectangular Cartesians. Write these equations for curvilinear coordinates.\\
Deduce from these equations that, if $\omega_{rs}=0$ initially at some point $P$ in the fluid, these quantities will remain zero permanently for the particle which was initially at $P$.
\end{tcolorbox}

We have in rectangular Cartesian coordinates
\begin{align}
\textbf{(6.128)}\quad\spatie &\omega_{jk}=\half\left(v_{k,j}-v_{j,k}\right)\\
\textbf{(6.129)}\quad\spatie &\omega_r=\half\epsilon_{rjk}\omega_{jk}\\
\Rightarrow\spatie \omega_r&=\half\half\epsilon_{rjk}\left(v_{k,j}-v_{j,k}\right)\\
&=\half\half\epsilon_{rjk}v_{k,j}-\half\half\epsilon_{rjk}v_{j,k}\\
&=\half\half\epsilon_{rjk}v_{k,j}+\half\half\epsilon_{rjk}v_{k,j}\\
&=\half\epsilon_{rjk}v_{k,j}\\
\Rightarrow\spatie \omega_{r,r}&=\half\epsilon_{rjk}v_{k,jr}\\
&=0
\end{align}
So we have, as the flow is steady
\begin{align}
\dv{}{t}\omega_{rs}&= \omega_{rs,p}v_p\\
\text{(1)}\quad \Rightarrow\spatie &= \half \left(v_{s,rp}-v_{r,sp}\right)v_p\\
\end{align}

*******************************************\\\\

We have 
\begin{align}
\textbf{(6.146)}\spatie \partial_t v_r +v_sv_{r,s} &= X_r - \rho^{-1}p_{,r}
\end{align}
In the fluid considered here ($\partial_t v_r=0$; $\rho=\rho(p)\Rightarrow \rho^{-1}p_{,r}\coloneqq P_{,r}$; $X_r= -U_{,r}$) we get
\begin{align}
 v_sv_{r,s} + U_{,r} + P_{,r}=0
\end{align}

Consider the following expression $\epsilon_{nmr}\epsilon_{nst}v_mv_{t,s}$
This can be expressed as 
\begin{align}
\epsilon_{nrm}\epsilon_{nst}v_mv_{t,s}&=\left(\delta_{rs}\delta_{mt}-\delta_{rt}\delta_{ms}\right)v_mv_{t,s}\\
&=v_mv_{m,r}-v_mv_{r,m}
\end{align}
giving
\begin{align}
v_sv_{r,s}&= v_sv_{s,r}-\epsilon_{nrm}\epsilon_{nst}v_mv_{t,s}
\end{align}
So (11) can be expressed as 
\begin{align}
&v_sv_{s,r}-\epsilon_{nrm}\epsilon_{nst}v_mv_{t,s} + U_{,r} + P_{,r}=0\\
\text{(6)}\quad\Rightarrow\spatie &v_sv_{s,r}-2\epsilon_{nrm}\omega_nv_m + U_{,r} + P_{,r}=0\\
&v_sv_{s,r}+2\epsilon_{rnm}\omega_nv_m + U_{,r} + P_{,r}=0
\end{align}
Let's take the curl of $(17)$:
\begin{align}
&\epsilon_{pkr}\left(v_sv_{s,r}\right)_{,k}+2\epsilon_{pkr}\epsilon_{rnm}\left(\omega_nv_m\right)_{,k} + \epsilon_{pkr}U_{,r} + \epsilon_{pkr}P_{,r}=0
\end{align}
The terms with $U$ and $P$ are of the kind $\nabla\times\nabla(G)$ with $G$ a scalar function $G:\mathbb{R}^3\rightarrow \mathbb{R}$. These terms are equal to zero.\\ 
So, $(18)$ becomes 
\begin{align}
&\underbrace{\epsilon_{pkr}v_{s,k}v_{s,r}}_{=0}+\underbrace{\epsilon_{pkr}v_sv_{s,kr}}_{=0}+2\epsilon_{pkr}\epsilon_{rnm}\left(\omega_nv_m\right)_{,k} =0
\end{align}
and 
\begin{align}
\epsilon_{pkr}\epsilon_{rnm}\left(\omega_nv_m\right)_{,k} 
&=\epsilon_{rpk}\epsilon_{rnm}\left(\omega_nv_m\right)_{,k}\\
&=\delta_{pn}\delta_{km}\left(\omega_nv_m\right)_{,k}-\delta_{pm}\delta_{kn}\left(\omega_nv_m\right)_{,k}\\
&=\left(\omega_pv_k\right)_{,k}-\left(\omega_kv_p\right)_{,k}\\
&=\omega_{p,k}v_k+\omega_pv_{k,k}-\omega_{k,k}v_p-\omega_kv_{p,k}
\end{align}

Note that the divergence of the vorticity is zero (see $(8)$), so 
\begin{align}
 \epsilon_{pkr}\epsilon_{rnm}\left(\omega_n v_m\right)_{,k}
&= \omega_{p,k}v_k+\omega_pv_{k,k}-\omega_kv_{p,k}
\end{align}
$(24)$ in $(19)$ gives 
\begin{align}
&\omega_{p,k}v_k+\omega_pv_{k,k}-\omega_kv_{p,k} =0
\end{align}
We have $\text{(2): } \omega_p=\half\epsilon_{pmn}\omega_{mn}$
 giving
\begin{align}
&\epsilon_{pmn}\omega_{mn,k}v_k+\epsilon_{pmn}\omega_{mn}v_{k,k}-\epsilon_{kmn}\omega_{mn}v_{p,k} =0
\end{align}
We have $\text{(7): } \dv{}{t}\omega_{rs}= \omega_{rs,p}v_p$
 $$\blacklozenge$$
\newpage