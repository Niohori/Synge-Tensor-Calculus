\setcounter{chapter}{7}
\chapter{Non-Riemannian spaces.}
\pagebreak[4]
\begin{comment}
\section{p283 - Clarification}
\begin{tcolorbox}
$$\mathbf{8.101}\spatie \dv{T^{\rho}}{u} =\dv{T_r}{u}X^{\rho}_{r} + X^{\rho}_{rs}T^r \dv{x^s}{u}$$
\end{tcolorbox}
$T^r$ being a tensor, we have $ T^{\rho} = T^r X^{\rho}{r}$
, giving
\begin{align}
\dv{T^{\rho}}{u} &= X^{\rho}_{r}\dv{T_r}{u} + T^r dv{X^{\rho}_{r}}{u}\\
&= \dv{T_r}{u}X^{\rho}_{r} + T^r X^{\rho}_{rs}\dv{x^s}{u}
\end{align}
$$\blacklozenge$$
\newpage

\section{p284 - Clarification}
\begin{tcolorbox}
..., we immediately see that
$$\fdv{TS^r}{u} = \dv{T}{u} S^r + T \fdv{S^r}{u}$$
\end{tcolorbox}
We have 
\begin{align}
&\left\{\begin{array}{ll}
\fdv{S^r}{u} = \dv{S^r}{u} +\Gamma^r_{mn}S^m\dv{x^n}{u}& \ \\\\
\fdv{TS^r}{u} = \dv{TS^r}{u} +\Gamma^r_{mn}TS^m\dv{x^n}{u}& \ \\
\end{array}\right.\\
\Rightarrow \spatie &\left\{\begin{array}{ll}
T\fdv{S^r}{u} = T\dv{S^r}{u} +\Gamma^r_{mn}TS^m\dv{x^n}{u}&(a)\\\\
\fdv{TS^r}{u} = S^r\dv{T}{u} + T\dv{S^r}{u} +\Gamma^r_{mn}TS^m\dv{x^n}{u}&(b)\\
\end{array}\right.\\
(a)-(b)\Rightarrow\spatie & T \fdv{S^r}{u} - \fdv{TS^r}{u} =-  S^r\dv{T}{u} \\
\Rightarrow\spatie &\fdv{TS^r}{u} = T\fdv{S^r}{u}  +  S^r\dv{T}{u} 
\end{align}
$$\blacklozenge$$
\newpage

\section{p285 - Exercise}
\begin{tcolorbox}
Show that any tensor $T^{mn}$ may be written in the form
$$T^{mn}=X_{(p)}^{\ \ m}Y_{(p)}^{\ \ n}$$
and use this result to prove problem 11, Exercise III.
\end{tcolorbox}
The proof is the same as for $\mathbf{(8.107)}$ . Let's define $X^{(p) m}_{}=T^{mp}$ and $Y^{\ \ n}_{(p)}= \delta^n_p$. Then 
\begin{align}
S^{mn}&= X^{(p) m}_{} \delta^n_p\\
 &= X^{(n) m}_{}\\
 &=T^{mn}
\end{align}
$$\lozenge$$
Problem 11, Exercise III: $ T^{mn}_{\quad | mn}= T^{mn}_{\quad | nm}$\\
We can compose $T^{mn}$ in two equivalent ways:   $T^{mn} = X^{(p) m}_{}Y^{\ \ n}_{(p)}$ with $X^{(p) m}_{}=T^{mp}$and $Y^{\ \ n}_{(p)}= \delta^n_p$ and also $T^{mn} = X^{(p) n}_{}Y^{\ \ m}_{(p)}$ with $X^{(p) n}_{}=T^{pn}$and $Y^{\ \ m}_{(p)}= \delta^m_p$. \\
For the first we have obviously
\begin{align}
T^{mn}_{\quad | m}&= X^{(p) m}_{\quad \quad |m} \delta^n_p+ X^{(p) m}_{} \underbrace{\delta^n_{p|m}}_{=0}\\
&= X^{(p) m}_{\quad \quad | m} \delta^n_p\\
T^{mn}_{\quad \quad | mn}&= X^{(p) m}_{\quad \quad |mn} \delta^n_p+ X^{(p) m}_{\quad |m} \underbrace{\delta^n_{p|m}}_{=0}\\
&= X^{(p) m}_{\quad \quad |mn} \delta^n_p\\
&= X^{(n) m}_{\quad \quad |mn} 
= T^{mn}_{\quad | mn}
\end{align}
and for the second 
\begin{align}
T^{mn}_{\quad | n}&= X^{(p) n}_{\quad \quad |n} \delta^m_p+ X^{(p) n}_{} \underbrace{\delta^m_{p|n}}_{=0}\\
&= X^{(p) n}_{\quad \quad | n} \delta^m_p\\
T^{mn}_{\quad \quad | nm}&= X^{(p) n}_{\quad \quad |nm} \delta^m_p+ X^{(p) n}_{\quad |n} \underbrace{\delta^m_{p|m}}_{=0}\\
&= X^{(p) n}_{\quad \quad |nm} \delta^m_p\\
&= X^{(m) n}_{\quad \quad |nm} = T^{mn}_{\quad | nm}
\end{align}
$$\blacklozenge$$
\newpage


\section{p286 - Exercise}
\begin{tcolorbox}
Show that, if the parameter along the curve $x^r=x^r(u)$ is changed from $u$ to $v$, then the absolute derivative of a tensor field with respect to $v$ is $\dv{u}{v}$ times the abssolute derivative with rspect to $u$. Symbolically
$$\fdv{}{v}= \dv{u}{v}\fdv{}{u}$$
\end{tcolorbox}
Be $T^{rs\dots}_{mn\dots}$ an arbitrary tensor. Then,
\begin{align}
\fdv{}{v}T^{rs\dots}_{mn\dots} &= \dv{}{v}T^{rs\dots}_{mn\dots}+ \sum_{p} A^p_t \dv{x^t}{v} - \sum_{p} B_{pt} \dv{x^t}{v} 
\end{align}
with $A^p_t$ and $B_{pt}$ being the coefficients as functions of the Christoffels symbols and the tensor.
This expression can be expressed as 
\begin{align}
\fdv{}{v}T^{rs\dots}_{mn\dots} &= \dv{u}{v}\dv{}{u}T^{rs\dots}_{mn\dots}+ \sum_{p} A^p_t  \dv{u}{v}\dv{x^t}{u} - \sum_{p} B_{pt}  \dv{u}{v}\dv{x^t}{u} \\
&=\dv{u}{v}\fdv{}{u}T^{rs\dots}_{mn\dots}
\end{align}
$$\blacklozenge$$
\newpage

\section{p286 - Exercise}
\begin{tcolorbox}
Prove that $C^r_{mn}$ defined by 
$$\mathbf{8.114}\spatie C^r_{mn}=\Gamma^r_{mn}-\overline{\Gamma}^r_{mn}$$
is a tensor 
\end{tcolorbox}
Using $\mathbf{(8.112)}$ and $\mathbf{(8.113)}$ we get for a coordinate transformation
\begin{align}
C^{\rho}_{\mu \nu }&=\Gamma^{\rho}_{\mu \nu }-\overline{\Gamma}^{\rho}_{\mu \nu }\\
&= \Gamma^{r}_{mn}X^{\rho}_{r}X^{m}_{\mu}X^{n}_{\nu}+ X^{r}_{\mu \nu}X^{\rho}_{r}-\overline{\Gamma}^{r}_{mn}X^{\rho}_{r}X^{m}_{\mu}X^{n}_{\nu}- X^{r}_{\mu \nu}X^{\rho}_{r}\\
&= \left(\Gamma^{r}_{mn}- \overline{\Gamma}^{r}_{mn}\right) X^{\rho}_{r}X^{m}_{\mu}X^{n}_{\nu}
\end{align}
which is the transformation rule for a tensor 
$$\blacklozenge$$
\newpage



\section{p286 - Exercise}
\begin{tcolorbox}
Show that the right-hand side of $\mathbf{8.110}$ may also be obtained formally by the method of $\mathbf{(2.516)}$, i.e., by differentiating the invariant $T^{..r}_{mn}X^mY^nZ_r$, and using $\fdv{X^m}{u}=0$, $\fdv{Y^n}{u}=0$, $\fdv{Z_r}{u}=0$.
\end{tcolorbox}
Be $Q= T^{..r}_{mn}X^mY^nZ_r$ an invariant, then $\fdv{Q}{u}$ is also invariant and $\fdv{Q}{u} = \dv{Q}{u}$.
So,
\begin{align}
\dv{Q}{u}&= \dv{T^{..r}_{mn}}{u}X^mY^nZ_r+ T^{..r}_{mn}\left(\dv{X^m}{u}Y^nZ_r+X^m\dv{Y^n}{u}Z_r +X^mY^n\dv{Z_r}{u}\right)
\end{align}
where $X^m, \ Y^n, \ Z_r$ are transported parallely, hence  $$\fdv{X^m}{u} = 0, \ \fdv{Y^n}{u}=0 , \ \fdv{Z_r}{u}=0  $$
and 
$$\dv{X^m}{u} = -\Gamma^m_{pq}X^p\dv{x^q}{u}, \ \dv{Y^n}{u}=-\Gamma^n_{pq}Y^p\dv{x^q}{u} , \ \dv{Z_r}{u}=\Gamma^p_{rq}Z_p\dv{x^q}{u}  $$
giving
\begin{align}
\dv{Q}{u} &= \dv{T^{..r}_{mn}}{u}X^mY^nZ_r+ T^{..r}_{mn}\left(-\Gamma^m_{pq}X^p\dv{x^q}{u}Y^nZ_r  -\Gamma^n_{pq}Y^p\dv{x^q}{u}X^mZ_r+\Gamma^p_{rq}Z_p\dv{x^q}{u}X^mY^n\right)\\
&= \underbrace{\left(\dv{T^{..r}_{mn}}{u}-\Gamma^p_{mq}T^{..r}_{pn}\dv{x^q}{u} -\Gamma^p_{nq}T^{..r}_{mp}\dv{x^q}{u}+\Gamma^r_{pq}T^{..p}_{mn}\dv{x^q}{u}\right)}_{= \fdv{T^{..r}_{mn}}{u}}X^mY^nZ_r
\end{align}
Putting $   \fdv{T^{..r}_{mn}}{u} = \dv{T^{..r}_{mn}}{u}-\Gamma^p_{mq}T^{..r}_{pn}\dv{x^q}{u} -\Gamma^p_{nq}T^{..r}_{mp}\dv{x^q}{u}+\Gamma^r_{pq}T^{..p}_{mn}\dv{x^q}{u}$ is correct as from $$\fdv{Q}{u}= \fdv{T^{..r}_{mn}}{u}X^mY^nZ_r+ T^{..r}_{mn}\fdv{X_m}{u}Y^nZ_r+T^{..r}_{mn}\fdv{Y_{n}}{u}X^mZ_r+T^{..r}_{mn}\fdv{Z^{r}}{u}X^mY^n$$ and 
$$\fdv{X^m}{u} = 0, \ \fdv{Y^n}{u}=0 , \ \fdv{Z_r}{u}=0  $$
we get 
$$\fdv{Q}{u}= \fdv{T^{..r}_{mn}}{u}X^mY^nZ_r$$
as in $(3)$.  
$$\blacklozenge$$
\newpage




\section{p288 - Exercise}
\begin{tcolorbox}
Prove that $\delta^r_{s|n} = C^r_{sn}$ where $C^r_{sn}$ is defined by $\mathbf{(8.114)}$.
\end{tcolorbox}
\begin{align}
 \delta^r_{s|n} &=\underbrace{\delta^r_{s,n} }_{=0}+ \Gamma^r_{kn}\delta^k_s - \overline{\Gamma}^k_{sn}\delta^r_k\\
 &=\Gamma^r_{sn} - \overline{\Gamma}^r_{sn}\\
 &= C^r_{sn}
\end{align}
$$\blacklozenge$$
\\\\
\section{p288 - Exercise}
\begin{tcolorbox}
Prove that an invariant remains constant under parallel propagation.
\end{tcolorbox}
 $T$ invariant and transported parallely means $\fdv{T}{u}=\dv{T}{u}=0$ which implies $T=\text{constant}$
$$\blacklozenge$$
\newpage

\section{p289 - Exercise}
\begin{tcolorbox}
Show that by a suitable choice of the parameter $u$ along a path, the differential equation $\mathbf{(8.119)}$ simplifies to $\fdv{\lambda^r}{u}=0$.
\end{tcolorbox}
 The proof is similar to the clarification $2.17$ in part I and generalises the idea of null-geodesics.\\
 Suppose that we have indeed as in $\mathbf{(8.119)}$
 \begin{align}
 \fdv{\lambda^r}{u} = \mu(u)\lambda^r
 \end{align}
 Suppose now that we can find a suitable parameter $s$ for which
 \begin{align}
 \fdv{\lambda^r}{s} = 0
 \end{align}
 so 
 \begin{align}
 \fdv{\lambda^r}{s} = \dv[2]{x^r}{s}+\Gamma^r_{mn}\dv{x^m}{s}\dv{x^n}{s}=0
 \end{align}
 and suppose that there is an homeomorphism between $u$ and $s$, then we can write (see section $2.17$ for the details):
 \begin{align}
\dv[2]{x^r}{u}+\Gamma^r_{mn}\dv{x^m}{s}\dv{x^n}{s}=- \frac{\dv[2]{u}{s}}{\left(\dv{x^r}{s}\right)^2}\dv{x^r}{u}
 \end{align}
 and from$(1)$ we deduce 
 \begin{align}
 -\frac{\dv[2]{u}{s}}{\left(\dv{u}{s}\right)^2}= \mu(u)
 \end{align}
 This is a differential equation which gives (see section $2.17$ in part I for the details)
 \begin{align}
 \frac{\dv[2]{s}{u}}{\left(\dv{s}{u}\right)^2}= \mu(u)
 \end{align}
 giving a solution 
 \begin{align}
 s(u)&= \int_{u_0}^{u}\left(exp\left(\int_{v_0}^v\mu(v)dw\right)\right)dv
 \end{align}
 as a suitable parameter for which $$\fdv{\lambda^r}{s} = 0$$
$$\blacklozenge$$
\newpage



\section{p290 - Exercise}
\begin{tcolorbox}
Show that, in a space with ortho-invariant linear connection, the Kronecker delta is propagated parallely along curves satisfying $C_n\lambda^n = 0$.
\end{tcolorbox}
 \begin{align}
 \fdv{}{u}\delta^r_s &= \underbrace{\dv{}{u}\delta^r_s}_{=0}+\Gamma^r_{mn}\delta^m_s\dv{x^n}{u} - \overline{\Gamma}^m_{sn}\delta^r_m\dv{x^n}{u} \\
 &= \left(\Gamma^r_{mn}\delta^m_s - \overline{\Gamma}^m_{sn}\delta^r_m\right)\dv{x^n}{u} \\
 &=\underbrace{ \left(\Gamma^r_{sn} - \overline{\Gamma}^r_{sn}\right)}_{=C^r_{sn}}\dv{x^n}{u}\\
 &=C^r_{sn}\lambda^n 
 \end{align}
 Ortho-invariant linear connection means $C^r_{sn}=\delta^r_s C_n$, giving
  \begin{align}
 \fdv{}{u}\delta^r_s &= \delta^r_s C_n\lambda^n \\
 &=0
 \end{align}
 provided that $C_n\lambda^n = 0$.
$$\blacklozenge$$\\\\



\section{p292 - Exercise}
\begin{tcolorbox}
Show that in the case of a single connection,
$$\mathbf{8.127}  \spatie \fdv{}{u}\delta^r_s =0$$
\end{tcolorbox}
 From exercise page $290$, we have
 $$\fdv{}{u}\delta^r_s =  \left(\Gamma^r_{sn} - \overline{\Gamma}^r_{sn}\right)\dv{x^n}{u}$$ and for a single connection $\Gamma^r_{sn} = \overline{\Gamma}^r_{sn}$ from which we get
 $$\fdv{}{u}\delta^r_s =  0$$
$$\blacklozenge$$
\newpage



\section{p294 - Exercise}
\begin{tcolorbox}
Deduce immediately that from $F2$ that
$$T_{|mn}= T_{|nm}$$ where $T$ is an invariant.
\end{tcolorbox}
 From $F2$ we know that there exists a coordinate system for which the absolute derivative reduces to the ordinary derivative, hence $T_{,mn}= T_{,nm}$ . As $T$ is an invariant the identity holds for every coordinate system.
$$\blacklozenge$$\\

\section{p295 - Exercise}
\begin{tcolorbox}
\begin{align*}
\begin{array}{ll}
\mathbf{8.215}& R^s_{.rmn} = -R^s_{.rnm}\\
\mathbf{8.216}& R^s_{.rmn} + R^s_{.mnr}+ R^s_{.nrm}=0\\
\mathbf{8.217}& R^s_{.rmn|k} + R^s_{.rnk|m}+ R^s_{.rkm|n}=0
\end{array}
\end{align*}
Prove the above identities by using a coordinate system considered in $F2$.
\end{tcolorbox}
 At the origin of the coordinate syetm of the type considered in $F2$. We have 
 \begin{align}
 R^s_{.rmn}&= \Gamma^s_{rn,m}-\Gamma^s_{rm,n}\\
 \Rightarrow \spatie R^s_{.rnm}&= \Gamma^s_{rm,n}-\Gamma^s_{rn,m} = -R^s_{.rmn}
 \end{align}
 and 
 \begin{align}
 R^s_{.rmn} + R^s_{.mnr}+ R^s_{.nrm}&= \Gamma^s_{rn,m}-\Gamma^s_{rm,n} +\Gamma^s_{mr,n}-\Gamma^s_{mn,r} +\Gamma^s_{nm,r}-\Gamma^s_{nr,m} \\
 &=0
 \end{align}
 and 
 \begin{align}
 R^s_{.rmn|k} + R^s_{.rnk|m}+ R^s_{.rkm|n}&= \Gamma^s_{rn,mk}-\Gamma^s_{rm,nk} +\Gamma^s_{rk,nm}-\Gamma^s_{rn,km} +\Gamma^s_{rm,kn}-\Gamma^s_{rk,mn}\\
 &=0
 \end{align}
$$\blacklozenge$$
\newpage

\section{p295 - Exercise}
\begin{tcolorbox}
Verify that $F_{mn}$ is skew-symmetric and that $R_{mn}+F_{mn}$ is symmetric. Show also directly from $8.219$ that $F_{mn}$ vanishes in a Riemannian space?
\end{tcolorbox}
\begin{align}
F_{mn} &= \half\left( \Gamma^s_{ns,m}-\Gamma^s_{ms,n} \right)\\
F_{nm} &= \half\left( \Gamma^s_{ms,n}-\Gamma^s_{ns,m} \right)\\
&= -\half\left( \Gamma^s_{ns,m}-\Gamma^s_{ms,n} \right)
\end{align}
And, as for $m=n$, $F_{mn} = 0$, we can conclude that $F_{mn}$ is skew-symmetric.
$$\lozenge$$\\
\begin{align}
R_{mn} +F_{mn} &=  \Gamma^s_{ms,n}-\Gamma^s_{mn,s}+\half\left( \Gamma^s_{ns,m}-\Gamma^s_{ms,n} \right)+\Gamma^k_{ms}\Gamma^s_{kn}-\Gamma^k_{mn}\Gamma^s_{ks}\\
&=  \underbrace{\half \Gamma^s_{ms,n}-\Gamma^s_{mn,s}+\half \Gamma^s_{ns,m}}_{=\text{symmetric in } m,n} +\underbrace{\Gamma^k_{ms}\Gamma^s_{kn}}_{=\text{symmetric in } m,n}-\underbrace{\Gamma^k_{mn}\Gamma^s_{ks}}_{=\text{symmetric in } m,n}\\
\end{align}
Conclusion: $R_{mn} +F_{mn}$ is symmetric.
$$\lozenge$$\\
In Riemannian space we have $$F_{mn} = \half R^s_{.smn} = \half a^{sk}R_{ksmn}$$
As $a^{sk}= a^{ks}$ and $R_{ksmn}= - R_{skmn}$, this implies
$$F_{mn}=0$$ as wehave doubleterms of the form
\begin{align*}
a^{SK}R_{KSmn} + a^{KS}R_{SKmn} = a^{SK}R_{KSmn} - a^{SK}R_{KSmn} = 0
\end{align*}
$$\blacklozenge$$\\
\newpage

\section{p298 - Exercise}
\begin{tcolorbox}
Prove that $\mathbf{(8.301)}$ implies
$$\mathbf{8.310}\spatie a^{mn}_{\quad |r} - a^{mn}\phi_r=0$$
\end{tcolorbox}
We have 
\begin{align}
 a_{mn|r} + a_{mn}\phi_r=0
 \end{align}
 As $a^{mn}a_{mn}= N$ we have 
 \begin{align}
 &\left(a^{mn}a_{mn}\right)_{|r}=0\\
 \Rightarrow \spatie &a^{mn}a_{mn|r}=-a^{mn}_{\quad |r}a_{mn}\\
 (3) \text{ and } (1) \times a^{mn}:\spatie &-a^{mn}_{\quad |r}a_{mn}+ a^{mn}a_{mn}\phi_ra_{mn}=0\\
 \Rightarrow \spatie &a^{mn}_{\quad |r}-a^{mn}a_{mn}\phi_r-=0
 \end{align}
$$\blacklozenge$$\\

\section{p299 - Exercise}
\begin{tcolorbox}
Is $\delta^r_s$ a gauge invariant tensor?
\end{tcolorbox}
We have $\delta^n_r = a^{'}_{rk} a^{'kn} $ with (see $\mathbf{(8.311)}$) $a^{'}_{rk}= \lambda a^{'}_{rk} $.\\
From chapter II $\mathbf{(2.203)}$ we have 
\begin{align}
a^{'kn} &= \frac{\Delta^{'kn}}{a^{'}}
\end{align}
with
\begin{align}
\Delta^{'kn} &= \lambda^{N-1}\Delta^{kn}\\
a^{'}&= \lambda^N a\\
\Rightarrow\spatie a^{'kn} &= \frac{ 1}{\lambda} a^{kn}\\
\Rightarrow\spatie \delta^n_r &= \lambda a_{rk}\frac{ 1}{\lambda} a^{kn}\\
&= \delta^{kn} 
\end{align}
Hence, $ \delta^{kn} $ is gauge invariant.
$$\blacklozenge$$
\newpage
\section{p299 - Exercise}
\begin{tcolorbox}
Show that, under the gauge transformation $\mathbf{(8.311)}$, $a^{mn}$ and $a$ transforms as follows:
$$a^{'kn} = \frac{ 1}{\lambda} a^{kn},\quad a^{'}= \lambda^N a $$
\end{tcolorbox}
See previous exercise.
$$\blacklozenge$$\\

\section{p300 - Exercise}
\begin{tcolorbox}
The covariant curvature tensor is defined by $$R_{srmn} = a_{sp}R^s_{.rmn}$$.
How does it behave under gauge transformations?
\end{tcolorbox}
 $$R_{srmn} = a_{sp}R^s_{.rmn}$$
 $R^p_{.rmn} $ is gauge invariant, hence $R^{'p}_{.rmn} =R^p_{.rmn} $ and $a^{'}_{sp}= \lambda a^{}_{sp} $ giving
 
 \begin{align}
 R^{'}_{srmn} &=\lambda a^{}_{sp}R^p_{.rmn}\\
 &=\lambda R_{srmn}
 \end{align}
 Hence, $R_{srmn}$ is not gauge invariant.
$$\blacklozenge$$\\
\newpage


\section{p307 - Exercise 1}
\begin{tcolorbox}
In a space with a general linear connection,show that the expressions
$$\Gamma^r_{mn}- \Gamma^r_{nm}, \quad \overline{\Gamma}^r_{mb}- \overline{\Gamma}^r_{nm}$$ are tensors of the tpe indicated by the position of he suffixes.
\end{tcolorbox}
 We use $\mathbf{(8.112)}$ page $286$:
 \begin{align}
 \left\{\begin{array}{l}
 \Gamma^{\rho}_{\mu\nu} = \Gamma^{r}_{mn} X^{\rho}_r X^m_{\mu} X^n_{\nu} + X^r_{\mu \nu} X^{\rho}_r\\\\
 \overline{\Gamma}^{\rho}_{\mu\nu} = \overline{\Gamma}_{mn} X^{\rho}_r X^m_{\mu} X^n_{\nu} + X^r_{\mu \nu} X^{\rho}_r\\
 \end{array}\right.\\
 \Rightarrow\spatie \Gamma^{\rho}_{\mu\nu} -\Gamma^{\rho}_{\nu\mu}&=\Gamma^{r}_{mn} X^{\rho}_r X^m_{\mu} X^n_{\nu}-\Gamma^{r}_{nm} X^{\rho}_r X^n_{\mu} X^m_{\nu}\\
 &= \left(\Gamma^{r}_{mn}-\Gamma^{r}_{nm}\right)X^{\rho}_r X^m_{\mu} X^n_{\nu}
 \end{align}
 So, $\Gamma^{r}_{mn}-\Gamma^{r}_{nm}$ transforms as a tensor.
$$\blacklozenge$$\\
\newpage

\section{p308 - Exercise 2}
\begin{tcolorbox}
 If $T$ is an invariant in a space with a single linear connection, show that
 $$ T_{|mn} - T_{|nm} = -2 T_{|r} L^r_{mn}$$
 where 
 $$ L^r_{mn} = -L^r_{nm} = \half\left(\Gamma^r_{mn} - \Gamma^r_{nm}\right)$$
\end{tcolorbox}
For an invariant, yields
\begin{align}
&T_{|m} = T_{,m}\\
\Rightarrow\spatie T_{|mn}-T_{|nm} &= \underbrace{T_{,mn}-T_{,nm}}_{=0} -\overline{\Gamma}^k_{mn}T_{,k} + \overline{\Gamma}^k_{nm}T_{,k} \\
&= -\left(\Gamma^k_{mn}- \Gamma^k_{nm} \right) \underbrace{T_{,k}}_{= T_{|k}}\\
&= 2T_{|k}L^k_{mn}
\end{align}
with $ L^k_{mn} = \half\left(\Gamma^k_{mn}- \Gamma^k_{nm} \right) $

$$\blacklozenge$$\\
\newpage


\section{p308 - Exercise 3}
\begin{tcolorbox}
If $T_r$ is a covariant vector field in in a space with a single linear connection, show that
 $$ T_{r|mn} - T_{r|nm} = T_s R^s_{.rmn}-2 T_{|r} L^r_{mn}$$
 where $ L^r_{mn}$ is defined in Ex. $2$, and where 
 $$R^s_{.pm} = \Gamma^s_{rn,m} - \Gamma^s_{rm,n} + \Gamma^p_{rn}\Gamma^s_{pm}-\Gamma^p_{rm}\Gamma^s_{pn}$$
\end{tcolorbox}
Starting from $ T_{r|m} = T_{r,m} - \Gamma^p_{rm}T_p$ we get
\begin{align}
&\left\{\begin{array}{l}
T_{r|mn} = T_{r,mn} - \Gamma^p_{rm,n}T_p- \Gamma^p_{rm}T_{p,n} - \Gamma^q_{rn}T_{q,m} + \Gamma^q_{rn}\Gamma^p_{qm}T_{p} - \Gamma^q_{mn}T_{r,q} +  \Gamma^q_{mn}\Gamma^p_{rq}T_{p} \\\\
T_{r|nm} = T_{r,mn} - \Gamma^p_{rn,m}T_p- \Gamma^p_{rn}T_{p,m} - \Gamma^q_{rm}T_{q,n} + \Gamma^q_{rm}\Gamma^p_{qn}T_{p} - \Gamma^q_{nm}T_{r,q} +  \Gamma^q_{nm}\Gamma^p_{rq}T_{p} \\
\end{array}
\right.
\end{align}

\begin{align}
\Rightarrow\quad T_{r|mn} -T_{r|nm} &= \left\{\begin{array}{l}
 \left(\Gamma^p_{rn,m} - \Gamma^p_{rm,n} 
- \Gamma^q_{mn}\Gamma^p_{rq}+ \Gamma^q_{rn}\Gamma^p_{qm}- \Gamma^q_{rm}\Gamma^p_{qn}+  \Gamma^q_{mn}\Gamma^p_{rq}\right)T_p\\\
- \cancel{\Gamma^p_{rm}T_{p,n}} - \cancel{\Gamma^q_{rn}T_{q,m}}  - \Gamma^q_{mn}T_{r,q}   +  \cancel{\Gamma^p_{rn}T_{p,m}} + \cancel{\Gamma^q_{rm}T_{q,n}} + \Gamma^q_{mn}T_{r,q}  \\
\end{array}
\right.\\
&=  \left\{\begin{array}{l}
 \underbrace{\left(\Gamma^p_{rn,m} - \Gamma^p_{rm,n} 
+ \Gamma^q_{rn}\Gamma^p_{qm}- \Gamma^q_{rm}\Gamma^p_{qn}\right)}_{= R^p_{.rmn}}T_p\\\\
 +  \Gamma^q_{mn}\underbrace{\left(T_{r,q}- \Gamma^p_{rq}T_p\right)}_{= T_{r|q}} -  \Gamma^q_{nm}\underbrace{\left(T_{r,q}-\Gamma^p_{rq}T_p\right)}_{= T_{r|q}}
   \\
\end{array}
\right.\\
&= R^p_{.rmn}T_p-2T_{r|q}L^q_{mn}
\end{align}
$$\blacklozenge$$\\
\newpage



\section{p308 - Exercise 4}
\begin{tcolorbox}
In a space with a single linear connection, show that \textit{if} there exists a coordinate system for each point $O$ of space such that, at $O$ and for all curves through $O$, the absolute and ordinary derivatives of any tensor differ by a multiple of the tensor, \textit{then }the coefficients of linear connection satisfy a relationship of the form 
$$\Gamma^r_{mn}-\Gamma^r_{nm} = \delta^r_mA_n - \delta^r_n A_m$$
where $A_r$ is some covariant vector. (Such a single linear connection is said to be \textit{semi-symmetric}).
\end{tcolorbox}
Let's put $\Delta T_k = \dv{T_k}{u} - \fdv{T_k}{u}$, so
\begin{align}
\Delta T_k&= \Gamma^m_{kt}T_m\lambda^t\spatie (\lambda^t = \dv{x^t}{u}\text{ for any curve.})
\end{align}
Given is: $ \Delta T_k = \mu T_k$. So $(1)$ can be written as 
\begin{align}
\mu T_k&= \Gamma^m_{kt}T_m\lambda^t\\
\Leftrightarrow\spatie &= \left(\Gamma^m_{kt}\lambda^t\right)T_m\\
\Rightarrow\spatie \Gamma^m_{kt}\lambda^t &= \mu \delta^m_k
\end{align}
We can also write $(4)$ as $ \Gamma^m_{tk}\lambda^k = \mu \delta^m_t$
\begin{align}
\Rightarrow\spatie \Gamma^m_{kt}\lambda^t\lambda^k-\Gamma^m_{tk}\lambda^t\lambda^k &= \mu \left(\delta^m_k\lambda^k- \delta^m_t\lambda^t\right)\\
\Leftrightarrow\spatie \left(\Gamma^m_{kt}-\Gamma^m_{tk}\right)\lambda^t\lambda^k &= \mu \left(\lambda^m- \lambda^m\right)\\
&=0
\end{align}
As $(7)$ is true for every curve (every $\lambda^t$), we can conclude as a  trivial solution that $\Gamma^m_{kt}=\Gamma^m_{tk}$. But another solution makes $\left(\Gamma^m_{kt}-\Gamma^m_{tk}\right)\lambda^t\lambda^k  =0$, i.e. $ \Gamma^m_{kt}-\Gamma^m_{tk}= \delta^m_kA_t - \delta^m_tA_k $, indeed
\begin{align}
\left(\delta^m_kA_t - \delta^m_tA_k\right)\lambda^t\lambda^k &=  \lambda^mA_t\lambda^t-\lambda^mA_k\lambda^k\\
&=0
\end{align}
This  solution is less restrictive than the trivial solution. \\
As any tensor can be expressed as the outer product of $1-$forms, the proof is valid for any tensor of any rank.
$$\blacklozenge$$\\
\newpage



\section{p308 - Exercise 5}
\begin{tcolorbox}
Prove the converse of Exercise $4$.\\
Hint: Consider the coordinate transformation (in notation of $\mathbf{8.208}$):
$$x^{\rho} = \delta^{\rho}_r\left(x^r-x^r_0\right) + \half\left\{\half\delta^{\rho}_r\left(\Gamma^r_{mn} +\Gamma^r_{nm}\right)_0 - \frac{1}{N-1}\delta^{\rho}_n\left(\Gamma^p_{pm}-\Gamma^p_{mp}\right)_0 \right\}\left(x^m-x^m_0\right)\left(x^n-x^n_0\right)$$
\end{tcolorbox}
Be $T_r$ a random tensor at a point $O$ and a random curve $x^r=x^r(u)$.\\ Then 
\begin{align}\Delta T_r=  \dv{T_r}{u}-\fdv{T_r}{u}  =  \Gamma^m_{rn}T_m\lambda^n
\spatie (\lambda^n=\dv{x^n}{u})\end{align} 
let's used the proposed coordinate transformation and using that $\Gamma^p_{pm}-\Gamma^p_{mp} = \delta^p_pA_m - \delta^p_n A_p= \left(N-1\right)A_m$  at the point $O$, so the transformation reduces to
\begin{align}
x^{\rho} = \delta^{\rho}_r\left(x^r-x^r_0\right) + \half\left\{\half\delta^{\rho}_r\left(\Gamma^r_{mn} +\Gamma^r_{nm}\right)_0 - \delta^{\rho}_nA_{n(0)} \right\}\left(x^m-x^m_0\right)\left(x^n-x^n_0\right)
\end{align} 
Using $\mathbf{(8.211)}$ we get for a coordinate transformation:
\begin{align}
\Delta T_{\mu} &= \left( \Gamma^r_{mn}X_r^{\rho}X^m_{\mu}X^n_{\nu}- X_{mn}^{\rho}X^m_{\mu}X^n_{\nu} \right)X^s_{\rho}X^{\nu}_tT_s\lambda^t\\
&= \left( \Gamma^r_{mn}\delta_r^{s}X^m_{\mu}X^n_{\nu}\delta_t^{n}- X_{mn}^{\rho}X^s_{\rho}X^m_{\mu}\delta^n_{t} \right)T_s\lambda^t\\
&= \left( \Gamma^s_{mt}X^m_{\mu}- X_{mt}^{\rho}X^s_{\rho}X^m_{\mu} \right)T_s\lambda^t
\end{align}
and, using $(2)$  for the considered coordinate transformation:
\begin{align}
X^{\rho}_t &= \delta^{\rho}_r\delta^{r}_t + \half\left\{\half\delta^{\rho}_r\left(\Gamma^r_{mn} +\Gamma^r_{nm}\right)_0 - \delta^{\rho}_nA_{n(0)} \right\}\left(\delta^m_t x^n+\delta^n_t x^m-\delta^m_t x^n_0-\delta^n_t x^m_0\right)\\
&= \delta^{\rho}_t + \half\delta^{\rho}_r\left(\Gamma^r_{tn} +\Gamma^r_{nt}\right)_0\left(x^n - x^n_0 \right) - \half\left(\delta^{\rho}_n\left(x^n-x^n_0\right)A_{t(0)}+\delta^{\rho}_t\left(x^n-x^n_0\right)A_{n(0)} \right)
\end{align}
and 
\begin{align}
X^{\rho}_{tm} &= \half\left(\Gamma^r_{tm} +\Gamma^r_{mt}\right)_0\delta^{\rho}_r- \half \delta^{\rho}_m A_{t(0)}- \half \delta^{\rho}_t A_{m(0)}
\end{align}
giving for $(5)$:
\begin{align}
\Delta T_{\mu} &= \left[ \Gamma^s_{mt}X^m_{\mu}- \left(\half\left(\Gamma^r_{tm} +\Gamma^r_{mt}\right)_0\delta^{\rho}_r- \half \delta^{\rho}_m A_{t(0)}- \half \delta^{\rho}_t A_{m(0)}\right)X^s_{\rho}X^m_{\mu} \right]T_s\lambda^t\\
 &= \left[ \Gamma^s_{mt}X^m_{\mu}- \half\left(\Gamma^r_{tm} -\Gamma^r_{mt}\right)_0\delta^{s}_rX^m_{\mu}+ \half \delta^{s}_m A_{t(0)}X^m_{\mu}+ \half \delta^{s}_t A_{m(0)}X^m_{\mu}\right]T_s\lambda^t\\
 &= \left[ \half\underbrace{\left(\Gamma^s_{mt}-\Gamma^s_{tm} -\right)_0}_{=\delta^s_mA_{t(0)} - \delta^s_t A_{m(0)}}X^m_{\mu}+ \half \delta^{s}_m A_{t(0)}X^m_{\mu}+ \half \delta^{s}_t A_{m(0)}X^m_{\mu}\right]T_s\lambda^t\\
 &= \left[ \half\delta^s_mA_{t(0)} X^m_{\mu}- \half\delta^s_t A_{m(0)}X^m_{\mu}+ \half \delta^{s}_m A_{t(0)}X^m_{\mu}+ \half \delta^{s}_t A_{m(0)}X^m_{\mu}\right]T_s\lambda^t\\
 &=  \delta^s_mA_{t(0)} X^m_{\mu}T_s\lambda^t\\
 &= \underbrace{\left(A_{t(0)} \lambda^t\right)}_{\text{invariant}}\underbrace{T_sX^s_{\mu}}_{= T_{\mu}}\\
\end{align}
and we get $$\Delta T_{\mu}= kT_{\mu}$$.
$$\blacklozenge$$\\
\newpage



\section{p308 - Exercise 6}
\begin{tcolorbox}
Show that $$T_{r|s} - T_{s|r}= T_{r,s}-T_{s,r}$$
$T_r$ being a covariant vector, if the connection is symmetric but not, if the connection is unsymmetric.
\end{tcolorbox}
$T_{r|s} = T_{r,s}-\Gamma^m_{rs}T_m$ and $T_{s|r} = T_{s,r}-\Gamma^m_{sr}T_m$
So if the linear connection is symmetric
\begin{align}
T_{r|s} - T_{s|r}&= T_{r,s}-\Gamma^m_{rs}T_m-T_{s,r}+\Gamma^m_{sr}T_m
&= T_{r,s}-T_{s,r}
\end{align}\\
This is obviously not true for a non-symmetric linear connection.
$$\blacklozenge$$\\
\newpage

\section{p308 - Exercise 7}
\begin{tcolorbox}
Show that, in the generalized Stokes' theorem $\mathbf{(7.505)}$, the partial derivatives in the integrand on the left-hand side can be replaced by the covariant derivative if the space has a symmetric connection.
\end{tcolorbox}
\begin{align}
\mathbf{(7.505):}\spatie \int_{R_{(M)}} T_{k_1\dots k_{M-1},k_M}d\tau_{(M1)}^{k_1\dots k_{M-1}}&= \int_{R_{(M-1)}} T_{k_1\dots k_{M-1},k_M}d\tau_{(M-1)}^{k_1\dots k_{M-1}}
\end{align}
and we have 
\begin{align}
\int_{R_{(M)}} T_{k_1\dots k_{M-1}|k_M}d\tau_{(M1)}^{k_1\dots k_{M-1}}&= \left\{\begin{array}{l}
\int_{R_{(M)}} T_{k_1\dots k_{M-1},k_M}d\tau_{(M1)}^{k_1\dots k_{M-1}}\\\\
-\int_{R_{(M)}} \Gamma^{k_{\rho}}_{k_1 k_M}T_{k_{\rho}\dots k_{M-1},k_M}d\tau_{(M1)}^{k_1\dots k_{M-1}}\\\\
-\int_{R_{(M)}} \Gamma^{k_{\rho}}_{k_2 k_M}T_{k_1 k_{\rho}\dots k_{M-1},k_M}d\tau_{(M1)}^{k_1\dots k_{M-1}}\\\\
\spatie \spatie \vdots\\\\
-\int_{R_{(M)}} \Gamma^{k_{\rho}}_{k_{M-1} k_M}T_{k_1\dots k_{\rho},k_M}d\tau_{(M1)}^{k_1\dots k_{M-1}}
\end{array}\right.
\end{align}
We note that in the terms with the negative sign that $d\tau_{(M1)}^{k_1\dots k_{M-1}}$ is skew-symmetric. So, under the integral we will have two cancelling terms of sums of the form 

$$ \Gamma^{u}_{k_p k_M}T_{\dots u \dots ,k_M}d\tau_{(M1)}^{k_1\dots (k_p)\dots (k_{M})}+\Gamma^{u}_{k_M k_p }T_{\dots u \dots ,k_M}d\tau_{(M1)}^{k_1\dots (k_{M})\dots (k_p)}$$
As $\Gamma^{u}_{k_p k_M}$ is symmetric and  $d\tau_{(M1)}^{k_1\dots (k_{M})\dots (k_p)}$ skew-symmetric, the two sums cancel each other and we get the straight form of the Stokes theorem.
$$\blacklozenge$$\\
\newpage


\section{p309 - Exercise 8}
\begin{tcolorbox}
In a Weyl space, the rate of change of the squared magnitude $X^2$of a vector $X^r$ under parallel propagation of $X^r$ along some curve $x^r=x^r(u)$ is given  by
$$ \frac{\Delta}{\Delta u} X^2 = \dv{}{u}\left(\epsilon a_{mn}X^mX^n\right) = \epsilon \dv{a_{mn}}{u}X^mX^n +2\epsilon a_{mn} X^m\dv{X^n}{u}$$
where $\dv{a_{mn}}{u}$ is the ordinary derivative along the curve of $a_{mn}$ (which is defined throughout the space), whereas $\dv{X^r}{u}$ is obtained by parallel propagation, i.e.,
$$\dv{X^n}{u} = -\Gamma^n_{sp}X^s\dv{x^p}{u}$$
Show that 
$$\frac{\Delta}{\Delta u} X^2 = -X^2\phi_r\dv{x^r}{u}$$
Hence, prove that the change in $X^2$ under parallel propagation of $X^r$ around an infinitesimal circuit, bounding a $2-$element of extension $d\tau_{(2)}^{mn}$, is given by
$$\Delta X^2= \frac{2}{N}X^2F_{mn}d\tau_{(2)}^{mn}$$
(Hint: Use Stokes' theorem).
\end{tcolorbox}
We have in a space with symmetric linear connections
\begin{align}
\fdv{a_{mn}}{u} &= \dv{a_{mn}}{u} - \Gamma^k_{mt} a_{kn}\lambda^t - \Gamma^k_{nt} a_{mk}\lambda^t
\end{align}
and 
\begin{align}
\fdv{a_{mn}}{u} &= a_{mn|k}\lambda^k
\end{align}
and (Weyl space)
\begin{align}
a_{mn|k}+ a_{mn}\phi_k=0
\end{align}
Combining $(1),\ (2),\ (3)$ we get
\begin{align*}
-a_{mn}\phi_k\lambda_k &= \dv{a_{mn}}{u} - \Gamma^k_{mt} a_{kn}\lambda^t - \Gamma^k_{nt} a_{mk}\lambda^t\\
\Rightarrow\spatie \dv{a_{mn}}{u} &= \left(-a_{mn}\phi_t + \Gamma^k_{mt} a_{kn} + \Gamma^k_{nt} a_{mk}\right)\lambda^t
\end{align*}
From this expression and $ \dv{X^n}{u} = -\Gamma^n_{sp}X^s\lambda^p$ we get
\begin{align*}
\frac{\Delta}{\Delta u} X^2 &=\epsilon\left(-a_{mn}\phi_t + \Gamma^k_{mt} a_{kn} + \Gamma^k_{nt} a_{mk}\right)\lambda^tX^mX^n -2\epsilon a_{mn} X^m\Gamma^n_{sp}X^s\lambda^p\\
&=-\epsilon a_{mn}\phi_t\lambda^tX^mX^n  + \cancel{\epsilon\Gamma^k_{mt} a_{kn}\lambda^tX^mX^n }+ \cancel{\epsilon \Gamma^k_{nt} a_{mk}X^mX^n\lambda^t } -\cancel{\epsilon a_{mn}\Gamma^n_{sp X^m}X^s\lambda^p}-\cancel{\epsilon a_{mn} \Gamma^n_{sp}X^mX^s\lambda^p}\\
&=-\epsilon a_{mn}X^mX^n \phi_t \lambda^t
\end{align*}
but $X^2= \epsilon a_{mn} X^mX^n$, so
$$\frac{\Delta}{\Delta u} X^2 = -X^2\phi_t  \lambda^t$$


$$\lozenge$$


Applying Stokes' theorem to $\frac{\Delta}{\Delta u} X^2$
\begin{align}
 \int_{R_{(1)}} \Delta X^2 =\int_{R_{(1)}} \frac{\Delta}{\Delta u} X^2 du=-\int_{R_{(1)}}X^2\phi_r  dx^{r}= - \int_{R_{(2)}} \left( X^2\phi_r\right)_{,s}d\tau_{(2)}^{rs}
\end{align}
Taking the limit to an infinitesimal $2-$space $R_{(1)}\rightarrow dx^{r}$ and $R_{(2)}\rightarrow d\tau_{(2)}^{rs}$ we can write
\begin{align}
\Delta X^2  &= - \left( X^2\phi_r\right)_{,s}d\tau_{(2)}^{rs}\\
&= - \left[\left( X^2\right)_{,s}\phi_r + X^2\phi_{r,s} \right]d\tau_{(2)}^{rs}\\
&= - \left[ \left( X^2\right)_{,s}\phi_r + \half X^2\phi_{r,s} \right]d\tau_{(2)}^{rs}\\
&= \half X^2\left(\phi_{s,r}- \phi_{r,s} \right)d\tau_{(2)}^{rs} -  \left( X^2\right)_{,s}\phi_rd\tau_{(2)}^{rs} 
\end{align}
(the last step used the skew-symmetricity of $d\tau_{(2)}^{rs}$).\\

We have $F_{rs} = \frac{N}{4}\left(\phi_{s,r}- \phi_{r,s} \right)$, giving
\begin{align}
\Delta X^2  &= \frac{2}{N}F_{rs}  X^2 d\tau_{(2)}^{rs} -  \left( X^2\right)_{,s}\phi_rd\tau_{(2)}^{rs} 
\end{align}
Let's put $K\equiv \left( X^2\right)_{,s}\phi_rd\tau_{(2)}^{rs} $ and as we have $X^2= a_{mn}X^mX^n$
\begin{align}
K&=\left(a_{mn}X^mX^n\right)_{,s}\phi_r d\tau_{(2)}^{rs}\\
&=a_{mn,s}X^mX^n \phi_r d\tau_{(2)}^{rs}+2a_{mn}X^mX^n_{,s}\phi_r d\tau_{(2)}^{rs}
\end{align}
From $\mathbf{(8.304)}$ we have
\begin{align}
a_{mn,s} -\Gamma^k_{ms}a_{kn}- \Gamma^k_{nk}a_{km}+a_{mn}\phi_s=0
\end{align} 
and $(11)$ can be written as 

\begin{align}
K&= \Gamma^k_{ms}a_{kn}X^mX^n \phi_r d\tau_{(2)}^{rs}+ \Gamma^k_{ns}a_{km}X^mX^n \phi_r d\tau_{(2)}^{rs}-a_{mn}\phi_s X^mX^n \phi_r d\tau_{(2)}^{rs}+2a_{mn}X^mX^n_{,s}\phi_r d\tau_{(2)}^{rs}\\
&= 2\Gamma^k_{ms}a_{kn}X^mX^n \phi_r d\tau_{(2)}^{rs}-\underbrace{a_{mn}\phi_s X^mX^n \phi_r d\tau_{(2)}^{rs}}_{=0}+2a_{mn}X^mX^n_{,s}\phi_r d\tau_{(2)}^{rs}\\
&= 2\underbrace{\left(\Gamma^m_{ks}X^k +X^m_{,s}\right)}_{X^m_{\ |s}}a_{mn}X^n\phi_r d\tau_{(2)}^{rs}
\end{align}
We can put $X^m_{\ |s}=0$ as we can  span the infinitesimal $2-$space by a bundle of curves along which we parallely transport $X^m$.
So, $K=0$ and $(9)$ becomes
$$\Delta X^2  = \frac{2}{N}F_{rs}  X^2 d\tau_{(2)}^{rs}$$

$$\blacklozenge$$\\
\newpage


\section{p309 - Exercise 9}
\begin{tcolorbox}
Verify that the projective curvature tensor $\mathbf{8.338}$ is invariant under all projective transformations.
\end{tcolorbox}
We have
\begin{align}
\left\{\begin{array}{ll}
\mathbf{(8.338)}&W^s_{.rmn}=R^s_{.rmn}-\frac{2}{N+1}\delta_r^s F_{mn}+\frac{1}{N-1}\left(\delta^s_m R_{rn}-\delta^s_n R_{rm}\right)-\frac{2}{N^2-1}\left(\delta^s_n F_{rm}-\delta^s_mF_{rn}\right)\\\\
\mathbf{(8.214)}&R^{s}_{.rmn} = \Gamma^{s}_{rn,m} - \Gamma^{s}_{rm,n}+\Gamma^{p}_{rn}\Gamma^{s}_{pm}-\Gamma^{p}_{rm}\Gamma^{s}_{pn}\\\\
\mathbf{(8.219)}&F_{mn} = \half\left(\Gamma^{s}_{ns,m} - \Gamma^{s}_{ms,n}\right)\\
\end{array}
\right.
\end{align}
using a projective transformation of the form $\mathbf{(8.337)}$, we get 

\begin{align}
\left\{\begin{matrix}
\Gamma^{'s}_{rn,m} = \Gamma^{s}_{rn,m} +\delta^s_n\psi_{r,m} +\delta^s_r\psi_{n,m}\\\\
\Gamma^{'s}_{rm,n} = \Gamma^{s}_{rm,n} +\delta^s_m\psi_{r,n} +\delta^s_r\psi_{m,n}\\\\
\Gamma^{'s}_{ns,m} = \Gamma^{s}_{ns,m} +\delta^s_s\psi_{n,m} +\delta^s_n\psi_{s,m}= \Gamma^{s}_{ns,m} +\left(N+1\right)\psi_{n,m} \\\\
\Gamma^{'s}_{ms,n} = \Gamma^{s}_{ms,n} +\delta^s_s\psi_{m,n} +\delta^s_m\psi_{s,n}= \Gamma^{s}_{ms,n} +\left(N+1\right)\psi_{m,n} \\\\
\Gamma^{'s}_{rs,m} = \Gamma^{s}_{rs,m} +\delta^s_s\psi_{r,m} +\delta^s_r\psi_{s,m}= \Gamma^{s}_{rs,m} +\left(N+1\right)\psi_{r,m} \\\\
\Gamma^{'s}_{rs,n} = \Gamma^{s}_{rs,n} +\delta^s_s\psi_{r,n} +\delta^s_r\psi_{s,n}= \Gamma^{s}_{rs,n} +\left(N+1\right)\psi_{r,n} \\\\
\Gamma^{'p}_{rn} = \Gamma^{p}_{rn} +\delta^p_n\psi_r +\delta^p_r\psi_n\\\\
\Gamma^{'s}_{pm} = \Gamma^{s}_{pm} +\delta^s_m\psi_p +\delta^s_p\psi_m\\\\
\Gamma^{'p}_{rm} = \Gamma^{p}_{rm} +\delta^p_m\psi_r +\delta^p_r\psi_m\\\\
\Gamma^{'s}_{pn} = \Gamma^{s}_{pn} +\delta^s_n\psi_p +\delta^s_p\psi_n\\\\
\end{matrix}\right.\\
\end{align}
giving
\begin{align}
R^{'s}_{.rmn} &= \Gamma^{'s}_{rn,m} - \Gamma^{'s}_{rm,n}+\Gamma^{'p}_{rn}\Gamma^{'s}_{pm}-\Gamma^{'p}_{rm}\Gamma^{'s}_{pn}\\
&=\left\{\begin{matrix}
R^{s}_{.rmn}\\\\
+\delta^s_n\psi_{r,m} +\delta^s_r\psi_{n,m}-\delta^s_m\psi_{r,n} -\delta^s_r\psi_{m,n}\\\\
+\left(\delta^p_n\psi_r +\delta^p_r\psi_n\right) \Gamma^{s}_{pm}\\\\
+\Gamma^{p}_{rn}\left(\delta^s_m\psi_p +\delta^s_p\psi_m\right)\\\\
-\left(\delta^p_m\psi_r +\delta^p_r\psi_m\right)\Gamma^{s}_{pn}\\\\
-\Gamma^{p}_{rm}\left(\delta^s_n\psi_p +\delta^s_p\psi_n\right)\\\\
+\left(\delta^p_n\psi_r +\delta^p_r\psi_n\right) \left(\delta^s_m\psi_p +\delta^s_p\psi_m\right)\\\\
-\left(\delta^p_m\psi_r +\delta^p_r\psi_m\right)\left(\delta^s_n\psi_p +\delta^s_p\psi_n\right)\\
\end{matrix}\right.\\
&=\left\{\begin{matrix}
R^{s}_{.rmn}\\\\
+\delta^s_n\psi_{r,m} +\delta^s_r\psi_{n,m}-\delta^s_m\psi_{r,n} -\delta^s_r\psi_{m,n}\\\\
+\delta^p_n\psi_r\Gamma^{s}_{pm} +\delta^p_r\psi_n \Gamma^{s}_{pm}\\\\
+\delta^s_m\psi_p\Gamma^{p}_{rn} +\delta^s_p\psi_m\Gamma^{p}_{rn}\\\\
-\delta^p_m\psi_r\Gamma^{s}_{pn} -\delta^p_r\psi_m\Gamma^{s}_{pn}\\\\
-\Gamma^{p}_{rm}\delta^s_n\psi_p -\Gamma^{p}_{rm}\delta^s_p\psi_n\\\\
+\delta^p_n\psi_r\delta^s_m\psi_p +\delta^p_n\psi_r\delta^s_p\psi_m +\delta^p_r\psi_n\delta^s_m\psi_p  +\delta^p_r\psi_n \delta^s_p\psi_m\\\\
-\delta^p_m\psi_r \delta^s_n\psi_p -\delta^p_m\psi_r \delta^s_p\psi_n-\delta^p_r\psi_m\delta^s_n\psi_p -\delta^p_r\psi_m\delta^s_p\psi_n\\
\end{matrix}\right.
\end{align}
\begin{align}
&=\left\{\begin{matrix}
R^{s}_{.rmn}\\\\
+\delta^s_n\psi_{r,m} +\delta^s_r\psi_{n,m}-\delta^s_m\psi_{r,n} -\delta^s_r\psi_{m,n}\\\\
+\cancel{\psi_r\Gamma^{s}_{nm}} +\cancel{\psi_n \Gamma^{s}_{rm}}\\\\
+\delta^s_m\psi_p\Gamma^{p}_{rn} +\cancel{\psi_m\Gamma^{s}_{rn}}\\\\
-\cancel{\psi_r\Gamma^{s}_{mn}} -\cancel{\psi_m\Gamma^{s}_{rn}}\\\\
-\Gamma^{p}_{rm}\delta^s_n\psi_p -\cancel{\Gamma^{s}_{rm}\psi_n}\\\\
+\cancel{\delta^s_m\psi_n\psi_r} +\cancel{\delta^s_n\psi_m\psi_r }+\delta^s_m\psi_r\psi_n  +\cancel{\delta^s_r\psi_m\psi_n} \\\\
-\cancel{\delta^s_n\psi_r \psi_m }-\cancel{\delta^s_m\psi_r \psi_n}-\delta^s_n\psi_r\psi_m -\cancel{\delta^s_r\psi_m\psi_n}\\
\end{matrix}\right.\\
&=\left\{\begin{matrix}
R^{s}_{.rmn}\\\\
+\delta^s_n\psi_{r,m} +\delta^s_r\psi_{n,m}-\delta^s_m\psi_{r,n} -\delta^s_r\psi_{m,n}\\\\
+\delta^s_m\psi_p\Gamma^{p}_{rn} 
-\delta^s_n\psi_p \Gamma^{p}_{rm}
+\delta^s_m\psi_r\psi_n  -\delta^s_n\psi_r\psi_m 
\end{matrix}\right.
\end{align}
and
\begin{align}
F^{'}_{mn} &= \half\left(\Gamma^{s}_{ns,m} - \Gamma^{s}_{ms,n}\right)\\
&=F^{}_{mn} + \half\left(N+1\right)\left(\psi_{n,m} -\psi_{m,n}\right)
\end{align}
and
\begin{align}
R^{'}_{rm} &= \left\{\begin{matrix}
R^{}_{.rm}\\\\
+\delta^s_s\psi_{r,m} +\delta^s_r\psi_{s,m}-\delta^s_m\psi_{r,s} -\delta^s_r\psi_{m,s}\\\\
+\delta^s_m\psi_p\Gamma^{p}_{rs} 
-\delta^s_s\psi_p \Gamma^{p}_{rm}
+\delta^s_m\psi_r\psi_s  -\delta^s_s\psi_r\psi_m 
\end{matrix}\right.\\
&=\left\{\begin{matrix}
R^{}_{.rm}\\\\
+N\psi_{r,m} +\cancel{\psi_{r,m}}-\cancel{\psi_{r,m}} -\psi_{m,r}\\\\
+\psi_p\Gamma^{p}_{rm} 
-N\psi_p \Gamma^{p}_{rm}
+\psi_r\psi_m  -N\psi_r\psi_m 
\end{matrix}\right.\\
&=R^{}_{.rm}
+N\psi_{r,m} -\psi_{m,r} -\left(N-1\right) \Gamma^{p}_{rm}\psi_p
-\left(N-1\right)\psi_r\psi_m  
\end{align}
and
\begin{align}
R^{'}_{rn} &=R^{}_{.rn}
+N\psi_{r,n} -\psi_{n,r} -\left(N-1\right) \Gamma^{p}_{rn}\psi_p
-\left(N-1\right)\psi_r\psi_n  
\end{align}


When using a projective transformation of the form $\mathbf{(8.337)}$ (where $\psi_k$ is a random vector) and we calculate $\Delta W = W^{'s}_{.rmn}-W^s_{.rmn}$ , only terms containing $\psi_k$ will remain and we get
\begin{align}
\Delta W&= \left\{ \begin{matrix}
\delta^s_n\psi_{r,m} +\delta^s_r\psi_{n,m}-\delta^s_m\psi_{r,n} -\delta^s_r\psi_{m,n}\\\\
+\delta^s_m\psi_p\Gamma^{p}_{rn} 
-\delta^s_n\psi_p \Gamma^{p}_{rm}
+\delta^s_m\psi_r\psi_n  -\delta^s_n\psi_r\psi_m \\\\
-\frac{2}{N+1}\delta^s_r  \half\left(N+1\right)\left(\psi_{n,m} -\psi_{m,n}\right)\\\\
+\frac{1}{N-1}\left(\delta^s_m \left(N\psi_{r,n} -\psi_{n,r} -\left(N-1\right) \Gamma^{p}_{rn}\psi_p
-\left(N-1\right)\psi_r\psi_n \right)\right)\\\\-\frac{1}{N-1}\left(\delta^s_n \left(N\psi_{r,m} -\psi_{m,r} -\left(N-1\right) \Gamma^{p}_{rm}\psi_p
-\left(N-1\right)\psi_r\psi_m\right)\right)\\\\
-\frac{2}{N^2-1}\left(\delta^s_n  \half\left(N+1\right)\left(\psi_{r,m} -\psi_{m,r}\right)-\delta^s_m \half\left(N+1\right)\left(\psi_{n,r} -\psi_{r,n}\right)\right)
\end{matrix}\right.\\
&= \left\{ \begin{matrix}
\delta^s_n\psi_{r,m} +\cancel{\delta^s_r\psi_{n,m}}-\delta^s_m\psi_{r,n} -\cancel{\delta^s_r\psi_{m,n}}\\\\
+\delta^s_m\psi_p\Gamma^{p}_{rn} -\delta^s_n\psi_p \Gamma^{p}_{rm}
+\delta^s_m\psi_r\psi_n  -\delta^s_n\psi_r\psi_m \\\\
-\cancel{\delta^s_r\psi_{n,m}}+\cancel{\delta^s_r\psi_{m,n}}\\\\
+\frac{N}{N-1}\delta^s_m \psi_{r,n} -\frac{1}{N-1}\delta^s_m \psi_{n,r} - \delta^s_m \Gamma^{p}_{rn}\psi_p
-\delta^s_m \psi_r\psi_n \\\\
-\frac{1}{N-1}\delta^s_nN\psi_{r,m} +\frac{1}{N-1}\delta^s_n\psi_{m,r} + \delta^s_n\Gamma^{p}_{rm}\psi_p
+\delta^s_n\psi_r\psi_m\\\\
-\frac{1}{N-1}\delta^s_n  \psi_{r,m} +\frac{1}{N-1}\delta^s_n  \psi_{m,r}+\frac{1}{N-1}\delta^s_m \psi_{n,r} -\frac{1}{N-1}\delta^s_m \psi_{r,n}
\end{matrix}\right.\\
&= \left\{ \begin{matrix}
\cancel{\delta^s_m\psi_p\Gamma^{p}_{rn}} -\cancel{\delta^s_n\psi_p \Gamma^{p}_{rm}}+ \cancel{\delta^s_n\Gamma^{p}_{rm}\psi_p}- \cancel{\delta^s_m \Gamma^{p}_{rn}\psi_p}\\\\
 +\cancel{\delta^s_m\psi_r\psi_n}  -\cancel{\delta^s_n\psi_r\psi_m} +\cancel{\delta^s_n\psi_r\psi_m}-\cancel{\delta^s_m \psi_r\psi_n}\\\\
+\frac{N}{N-1}\delta^s_m \psi_{r,n} -\frac{1}{N-1}\delta^s_m \psi_{n,r} 
 \\\\
+\delta^s_n\psi_{r,m} -\delta^s_m\psi_{r,n}\\\\
-\frac{1}{N-1}\delta^s_nN\psi_{r,m} +\frac{1}{N-1}\delta^s_n\psi_{m,r} 
\\\\
-\frac{1}{N-1}\delta^s_n  \psi_{r,m} +\frac{1}{N-1}\delta^s_n  \psi_{m,r}+\frac{1}{N-1}\delta^s_m \psi_{n,r} -\frac{1}{N-1}\delta^s_m \psi_{r,n}
\end{matrix}\right.
\end{align}
which simplifies to
\begin{align}
\left(N-1\right)\Delta W&=\left\{ \begin{matrix}
\cancel{N\delta^s_m \psi_{r,n}} -\cancel{\delta^s_m \psi_{n,r} }
 \\\\
+\cancel{N\delta^s_n\psi_{r,m}} -\cancel{\delta^s_n\psi_{r,m}} -\cancel{N\delta^s_m\psi_{r,n}}+\cancel{\delta^s_m\psi_{r,n}}\\\\
-\cancel{N\delta^s_n\psi_{r,m}} +\cancel{\delta^s_n\psi_{m,r}} 
\\\\
-\cancel{\delta^s_n  \psi_{r,m}} +\cancel{\delta^s_n  \psi_{m,r}}+\cancel{\delta^s_m \psi_{n,r}} -\cancel{\delta^s_m \psi_{r,n}}
\end{matrix}\right.\\
&=0
\end{align}
proving that $ = W^{'s}_{.rmn}=W^s_{.rmn}$ for any (random vector $\psi_k$) and thus invariant for that type of projective transformation.
$$\blacklozenge$$\\
\newpage


\end{comment}



\section{p309 - Exercise 10}
\begin{tcolorbox}
In a projective space, \textit{the coefficients of projective connection $P^{r}_{mn}$ }are defined as follows
$$P^{r}_{mn} = \Gamma ^{r}_{mn}-\frac{1}{N+1}\left(\delta^r_m \Gamma^p_{pn}+  \delta^r_n \Gamma^p_{pm} \right)$$
Show that $P^{r}_{mn} $ is invariant under projective transformations of $\Gamma^{r}_{mn}$. verify that $P^{r}_{nm}=P^{r}_{mn}$, $P^{r}_{rn}=0$. Find the transformation properties of $P^{r}_{mn}$ under changes of coordinate system. (T.Y. Thomas.)
\end{tcolorbox}
Using $(2)$from the previous exercise, we have

\begin{align}
P^{'r}_{mn} &= \Gamma ^{'r}_{mn}-\frac{1}{N+1}\left(\delta^r_m \Gamma^{'p}_{pn}+  \delta^r_n \Gamma^{'p}_{pm} \right)\\
&=\left\{\begin{matrix}
 \Gamma^{r}_{mn} +\delta^r_n\psi_m +\delta^r_m\psi_n\\\\
 -\frac{1}{N+1}\delta^r_m \left(\Gamma^{p}_{pn} +\delta^p_n\psi_p +\delta^p_p\psi_n\right)\\\\
 -\frac{1}{N+1}  \delta^r_n \left(\Gamma^{p}_{pm} +\delta^p_m\psi_p +\delta^p_p\psi_m \right)\\
\end{matrix}\right.\\\\
&=\left\{\begin{matrix}
 \Gamma^{r}_{mn} +\delta^r_n\psi_m +\delta^r_m\psi_n\\\\
 -\frac{1}{N+1}\delta^r_m \Gamma^{p}_{pn} -\frac{1}{N+1}\delta^r_m \psi_n -\frac{N}{N+1}\delta^r_m \psi_n\\\\
 -\frac{1}{N+1}  \delta^r_n \Gamma^{p}_{pm} -\frac{1}{N-1}  \delta^r_n \psi_m -\frac{N}{N+1}  \delta^r_n \psi_m \\
\end{matrix}\right.\\
&=\left\{\begin{matrix}
 \Gamma^{r}_{mn} -\frac{1}{N+1}  \delta^r_n \Gamma^{p}_{pm}-\frac{1}{N-1}\delta^r_m \Gamma^{p}_{pn}\\\\
 +\delta^r_n\psi_m +\delta^r_m\psi_n -\frac{1}{N+1}  \delta^r_n \psi_m -\frac{N}{N+1}  \delta^r_n \psi_m \\\\
 -\frac{1}{N+1}\delta^r_m \psi_n -\frac{N}{N+1}\delta^r_m \psi_n\\\\
\end{matrix}\right.\\\\
&=\left\{\begin{matrix}
 \underbrace{\Gamma^{r}_{mn} -\frac{1}{N+1} \left( \delta^r_m \Gamma^{p}_{pn}+\delta^r_n \Gamma^{p}_{pm}\right)}_{=P^{r}_{mn}  }\\\\
 +\cancel{\frac{1}{N+1}\delta^r_n\psi_m }+\cancel{\frac{1}{N+1}\delta^r_m\psi_n} +\cancel{\frac{N}{N+1}\delta^r_n\psi_m} +\cancel{\frac{N}{N+1}\delta^r_m\psi_n}-\cancel{\frac{1}{N+1}  \delta^r_n \psi_m} -\cancel{\frac{N}{N+1}  \delta^r_n \psi_m }\\\\
 -\cancel{\frac{1}{N+1}\delta^r_m \psi_n} -\cancel{\frac{N}{N+1}\delta^r_m \psi_n}\\\\
\end{matrix}\right.\\
&= P^{r}_{mn}
\end{align}
proving the invariance of $P^{r}_{mn}$ under projective transformations of the linear symmetric connections.
$$\lozenge$$
As we are dealing with linear symmetric connections and $\delta^r_m \Gamma^{p}_{pn}-  \delta^r_n \Gamma^{p}_{pm} $ also is symmetric, we can conclude that $P^{r}_{mn}$ is symmetric. \\
Also,
\begin{align}
P^{r}_{rn} &= \Gamma ^{r}_{rn}-\frac{1}{N+1}\left(\delta^r_r \Gamma^p_{pn}+  \delta^r_n \Gamma^p_{pr} \right)\\
&= \Gamma ^{r}_{rn}-\frac{1}{N+1}\left(N \Gamma^p_{pn}+  \Gamma^p_{pn} \right)\\
&= \Gamma ^{r}_{rn}-\Gamma^p_{pn}\\
&=0
\end{align}
$$\lozenge$$
Let's perform a coordinate transformation, and use $$\mathbf{(8.112)}: \quad \Gamma^{\rho}_{\mu\nu} = \Gamma^{r}_{mn}X^{\rho}_{r}X^{m}_{\mu}X^{n}_{\nu}+ X^{r}_{\mu\nu}X^{\rho}_{r}$$ from which we obtain

\begin{align}
P^{\rho}_{\mu\nu} &= \Gamma^{\rho}_{\mu\nu}-\frac{1}{N+1}\left(\delta^{\rho}_{\mu} \Gamma^{\tau}_{\tau \nu}+  \delta^{\rho}_{\nu} \Gamma^{\tau}_{\tau \mu} \right)\\
&= \left\{\begin{matrix}\Gamma^{r}_{mn}X^{\rho}_{r}X^{m}_{\mu}X^{n}_{\nu}+ X^{r}_{\mu\nu}X^{\rho}_{r}\\\\
-\frac{1}{N+1}\delta^{\rho}_{\mu} \left(\Gamma^{r}_{mn}\underbrace{X^{\tau}_{r}X^{m}_{\tau}}_{=\delta^m_r}X^{n}_{\nu}+ \underbrace{X^{r}_{\tau\nu}X^{\tau}_{r}}_{=0}\right)\\\\
-\frac{1}{N+1} \delta^{\rho}_{\nu} \left(\Gamma^{r}_{mn}\underbrace{X^{\tau}_{r}X^{m}_{\tau}}_{=\delta^m_r}X^{n}_{\mu}+\underbrace{ X^{r}_{\tau\mu}X^{\tau}_{r}}_{=0}\right)
\end{matrix}\right.\\
&= \left\{\begin{matrix}\Gamma^{r}_{mn}X^{\rho}_{r}X^{m}_{\mu}X^{n}_{\nu}+ X^{r}_{\mu\nu}X^{\rho}_{r}\\\\
-\frac{\Gamma^{r}_{rn}}{N+1}\left(\delta^{\rho}_{\mu} X^{n}_{\nu}
+\delta^{\rho}_{\nu}X^{n}_{\mu}\right)
\end{matrix}\right.\\
\text{(2.542):}\spatie&= \left\{\begin{matrix}\Gamma^{r}_{mn}X^{\rho}_{r}X^{m}_{\mu}X^{n}_{\nu}+ X^{r}_{\mu\nu}X^{\rho}_{r}\\\\
-\frac{\pdv{}{x^r}\sqrt{a}}{\sqrt{a}(N+1)}\left(\delta^{\rho}_{\mu} X^{n}_{\nu}
+\delta^{\rho}_{\nu}X^{n}_{\mu}\right)
\end{matrix}\right.
\end{align}
$$\blacklozenge$$
\newpage



\section{p309 - Exercise 11}
\begin{tcolorbox}
Show that the differential equation of a path can be written in the form
$$\lambda^s\left(\dv{\lambda^r}{u}+P^r_{mn}\lambda^m\lambda^n\right) = \lambda^r\left(\dv{\lambda^s}{u}+P^s_{mn}\lambda^m\lambda^n\right)$$
where $P^r_{mn}$ are the coefficients of projective connection defined in Ex. $10$. Deduce that no change in the $\Gamma^r_{mn}$ other than a projective transformation leaves the $P^r_{mn}$ invariant.
\end{tcolorbox}
Starting from $\mathbf{(8.328)}$ we have the differential equation of a path
\begin{align}
\lambda^s\left(\dv{\lambda^r}{u}+\Gamma^r_{mn}\lambda^m\lambda^n\right) = \lambda^r\left(\dv{\lambda^s}{u}+\Gamma^s_{mn}\lambda^m\lambda^n\right)
\end{align}
and using the definition of $P^r_{mn}$ we check the following expression
\begin{align}
\lambda^s\left(\dv{\lambda^r}{u}+\left[\Gamma^r_{mn}-\frac{1}{N+1}\left(\delta^r_m \Gamma^p_{pn}+  \delta^r_n \Gamma^p_{pm} \right)\right]\lambda^m\lambda^n\right) &\overset{\mathrm{?}}{=} \lambda^r\left(\dv{\lambda^s}{u}+\left[\Gamma^s_{mn}-\frac{1}{N+1}\left(\delta^s_m \Gamma^p_{pn}+  \delta^s_n \Gamma^p_{pm} \right)\right]\lambda^m\lambda^n\right)\\
\Rightarrow \spatie  \Gamma^p_{pn}\lambda^r\lambda^n\lambda^s+  \Gamma^p_{pm} \lambda^m\lambda^r\lambda^s &\overset{\mathrm{?}}{=} \Gamma^p_{pn}\lambda^s\lambda^n\lambda^r+   \Gamma^p_{pm}\lambda^m\lambda^s\lambda^r
\end{align}
Obviously the last expression is true, proving the equivalence of the equation of a path.
$$\lozenge$$
As $$\lambda^s\left(\dv{\lambda^r}{u}+P^r_{mn}\lambda^m\lambda^n\right) = \lambda^r\left(\dv{\lambda^s}{u}+P^s_{mn}\lambda^m\lambda^n\right)$$ describes a path and is equivalent to $\mathbf{(8.328)}$ we can follow the exact same reasoning from $\mathbf{(8.328)}$ on, to $\mathbf{(8.332)}$ where 
\begin{align}
B^r_{mn} &=P^{'r}_{mn}-P^{r}_{mn}\\
&=\Gamma ^{'r}_{mn}-\frac{1}{N+1}\left(\delta^r_m \Gamma^{'p}_{pn}+  \delta^r_n \Gamma^{'p}_{pm} \right)-\Gamma ^{r}_{mn}+\frac{1}{N+1}\left(\delta^r_m \Gamma^p_{pn}+  \delta^r_n \Gamma^p_{pm} \right)\\
&=\underbrace{\Gamma ^{'r}_{mn}-\Gamma ^{r}_{mn}}_{= A^r_{mn}}-\frac{1}{N+1}\left[\delta^r_m \Gamma^{'p}_{pn}+  \delta^r_n \Gamma^{'p}_{pm} -\delta^r_m \Gamma^p_{pn}-  \delta^r_n \Gamma^p_{pm} \right]\\
&=A^r_{mn}-\frac{1}{N+1}\left[\delta^r_m\left( \Gamma^{'p}_{pn}-\Gamma^p_{pn}\right) +  \delta^r_n \left(\Gamma^{'p}_{pm} -   \Gamma^p_{pm} \right)\right]
\end{align}
As we want $P^r_{mn}$ to be invariant under projective transformations, we have $B^r_{mn}=0$ and hence $(7)$ can be written as
\begin{align}
A^r_{mn}&=\frac{1}{N+1}\left[\delta^r_m\left( \Gamma^{'p}_{pn}-\Gamma^p_{pn}\right) +  \delta^r_n \left(\Gamma^{'p}_{pm} -   \Gamma^p_{pm} \right)\right]
\end{align}
From this we see that $A^r_{mn}$ must of the form
$$A^r_{mn}=\delta^r_n\phi_m+\delta^r_n\phi_n$$
with $\phi_k = \frac{1}{N+1}\left( \Gamma^{'p}_{pk}-\Gamma^p_{pk}\right) $, which leads to the type of transformations as defined in $\mathbf{(8.337)}$.
$$\blacklozenge$$
\newpage


\section{p310 - Exercise 12}
\begin{tcolorbox}
Defining  $$\begin{matrix}
P^s_{.rmn}=P^s_{rn,m}-P^s_{rm,n}+P^p_{rn}P^s_{pm}-P^p_{rm}P^s_{pn} \\\\
P_{rm}=P^s_{.rms}
\end{matrix}$$
where  $P^r_{mn}$ are the coefficients of projective connection of Ex. $10$, show that
$$P^s_{.smn}=0,\quad P_{rm}=-P^s_{rm,s}+P^p_{rs}P^s_{pm}  $$
Prove that 

$$W^s_{.rmn}=P^s_{.rmn}+\frac{1}{N-1}\left(\delta^s_m P_{rn}-\delta^s_nP_{rm}\right) $$
where $W^s_{.rmn}$ is the projective curvature tensor.
\end{tcolorbox}

\begin{align}
P^s_{.smn}&=P^s_{sn,m}-P^s_{sm,n}+\underbrace{P^p_{sn}P^s_{pm}-P^p_{sm}P^s_{pn} }_{=0}
\end{align}
from which we see that $P^s_{.smn}=-P^s_{.snm}$. From the definition of $P^s_{.rmn}$ it is easy to see that this quantity is symmetric in the last two suffixes.
Hence we can conclude from $P^s_{.smn}=-P^s_{.snm}$ and $P^s_{.smn}=P^s_{.snm}$ that $P^s_{.smn}=0$.
$$\lozenge$$
\begin{align}
P_{rm}&=P^s_{.rms}\\
&=\underbrace{P^s_{rs,m}}_{=0}-P^s_{rm,s}+P^p_{rs}P^s_{pm}-P^p_{rm}\underbrace{P^s_{ps} }_{=0}\\
&=-P^s_{rm,s}+P^p_{rs}P^s_{pm}
\end{align}
$$\lozenge$$
The last assignment requires about 5 pages of tedious and boring basic algebraic and suffix manipulations. There was no added value to transcript this in Latex.
\begin{comment} 
\begin{align}
\left\{\begin{array}{ll}
\mathbf{(8.338)}&W^s_{.rmn}=R^s_{.rmn}-\frac{2}{N+1}\delta_r^s F_{mn}+\frac{1}{N-1}\left(\delta^s_m R_{rn}-\delta^s_n R_{rm}\right)-\frac{2}{N^2-1}\left(\delta^s_n F_{rm}-\delta^s_mF_{rn}\right)\\\\
\mathbf{(8.214)}&R^{s}_{.rmn} = \Gamma^{s}_{rn,m} - \Gamma^{s}_{rm,n}+\Gamma^{p}_{rn}\Gamma^{s}_{pm}-\Gamma^{p}_{rm}\Gamma^{s}_{pn}\\\\
\mathbf{(8.317)}&R^{}_{rm} = \Gamma^{k}_{rk,m} - \Gamma^{k}_{rm,k}+\Gamma^{p}_{rk}\Gamma^{k}_{pm}-\Gamma^{p}_{rm}\Gamma^{k}_{pk}\\\\
\mathbf{(8.219)}&F_{mn} = \half\left(\Gamma^{k}_{nk,m} - \Gamma^{k}_{mk,n}\right)\\
\end{array}
\right.
\end{align}
We get 
\begin{align}
W^s_{.rmn}&=\left\{\begin{matrix}
\Gamma^{s}_{rn,m} - \Gamma^{s}_{rm,n}+\Gamma^{p}_{rn}\Gamma^{s}_{pm}-\Gamma^{p}_{rm}\Gamma^{s}_{pn}\\\\
-\frac{1}{N+1}\delta_r^s \left(\Gamma^{k}_{nk,m} - \Gamma^{k}_{mk,n}\right)\\\\
+\frac{1}{N-1}\delta^s_m \left(\Gamma^{k}_{rk,n} - \Gamma^{k}_{rn,k}+\Gamma^{p}_{rk}\Gamma^{k}_{pn}-\Gamma^{p}_{rn}\Gamma^{k}_{pk}\right)\\\\
-\frac{1}{N-1}\delta^s_n \left(\Gamma^{k}_{rk,m} - \Gamma^{k}_{rm,k}+\Gamma^{p}_{rk}\Gamma^{k}_{pm}-\Gamma^{p}_{rm}\Gamma^{k}_{pk}\right)\\\\
-\frac{1}{N^2-1}\left(\delta^s_n \left(\Gamma^{k}_{mk,r} - \Gamma^{k}_{rk,m}\right)-\delta^s_m\left(\Gamma^{k}_{nk,r} - \Gamma^{k}_{rk,n}\right)\right)\\
\end{matrix}
\right.\\
&=\left\{\begin{matrix}
\left(\Gamma^{s}_{rn} -\frac{1}{N+1}\delta_r^s \Gamma^{k}_{nk} -\frac{1}{N-1}\delta^s_n \Gamma^{k}_{rk}+\frac{1}{N^2-1} \delta^s_n \Gamma^{k}_{rk}\right)_{,m}\\\\
- \left(\Gamma^{s}_{rm}-\frac{1}{N+1}\delta_r^s \Gamma^{k}_{mk}-\frac{1}{N-1}\delta^s_m \Gamma^{k}_{rk}+\frac{1}{N^2-1}\delta^s_m \Gamma^{k}_{rk}\right)_{,n} \\\\
  +\frac{1}{N-1}\left(\delta^s_n  \Gamma^{k}_{rm}- \delta^s_m \Gamma^{k}_{rn}\right)_{,k}\\\\
 +\frac{1}{N^2-1}\left(\delta^s_m\Gamma^{k}_{nk}  -\delta^s_n \Gamma^{k}_{mk}\right)_{,r} 
  \\\\
+\Gamma^{p}_{rn}\Gamma^{s}_{pm}-\Gamma^{p}_{rm}\Gamma^{s}_{pn}\\\\
+\frac{1}{N-1}\delta^s_m\Gamma^{p}_{rk}\Gamma^{k}_{pn}-\frac{1}{N-1}\delta^s_m\Gamma^{p}_{rn}\Gamma^{k}_{pk}\\\\
-\frac{1}{N-1}\delta^s_n \Gamma^{p}_{rk}\Gamma^{k}_{pm}+\frac{1}{N-1}\delta^s_n \Gamma^{p}_{rm}\Gamma^{k}_{pk}\\\\
\\
\end{matrix}
\right.\\
&=\left\{\begin{matrix}
\left(\Gamma^{s}_{rn} -\frac{1}{N^2-1}\left[(N-1)\delta_r^s \Gamma^{k}_{nk} +(N+1)\delta^s_n \Gamma^{k}_{rk}- \delta^s_n \Gamma^{k}_{rk}\right]\right)_{,m}\\\\
- \left(\Gamma^{s}_{rm} -\frac{1}{N^2-1}\left[(N-1)\delta_r^s \Gamma^{k}_{mk} +(N+1)\delta^s_m \Gamma^{k}_{rk}- \delta^s_m \Gamma^{k}_{rk}\right]\right)_{,n}\\\\
  +\frac{1}{N-1}\left(\delta^s_n  \Gamma^{k}_{rm}- \delta^s_m \Gamma^{k}_{rn}\right)_{,k}\\\\
 +\frac{1}{N^2-1}\left(\delta^s_m\Gamma^{k}_{nk}  -\delta^s_n \Gamma^{k}_{mk}\right)_{,r} 
  \\\\
+\Gamma^{p}_{rn}\Gamma^{s}_{pm}-\Gamma^{p}_{rm}\Gamma^{s}_{pn}\\\\
+\frac{1}{N-1}\delta^s_m\Gamma^{p}_{rk}\Gamma^{k}_{pn}-\frac{1}{N-1}\delta^s_m\Gamma^{p}_{rn}\Gamma^{k}_{pk}\\\\
-\frac{1}{N-1}\delta^s_n \Gamma^{p}_{rk}\Gamma^{k}_{pm}+\frac{1}{N-1}\delta^s_n \Gamma^{p}_{rm}\Gamma^{k}_{pk}\\\\
\\
\end{matrix}
\right.
\end{align}
\begin{align}
W^s_{.rmn}
&=\left\{\begin{matrix}
\left(\Gamma^{s}_{rn} -\frac{1}{N^2-1}\left[(N-1)\delta_r^s \Gamma^{k}_{nk} +(N-1)\delta^s_n \Gamma^{k}_{rk}+ \delta^s_n \Gamma^{k}_{rk}\right]\right)_{,m}\\\\
- \left(\Gamma^{s}_{rm} -\frac{1}{N^2-1}\left[(N-1)\delta_r^s \Gamma^{k}_{mk} +(N-1)\delta^s_m \Gamma^{k}_{rk}+ \delta^s_m \Gamma^{k}_{rk}\right]\right)_{,n}\\\\
  +\frac{1}{N-1}\left(\delta^s_n  \Gamma^{k}_{rm}- \delta^s_m \Gamma^{k}_{rn}\right)_{,k}\\\\
 +\frac{1}{N^2-1}\left(\delta^s_m\Gamma^{k}_{nk}  -\delta^s_n \Gamma^{k}_{mk}\right)_{,r} 
  \\\\
+\Gamma^{p}_{rn}\Gamma^{s}_{pm}-\Gamma^{p}_{rm}\Gamma^{s}_{pn}\\\\
+\frac{1}{N-1}\delta^s_m\Gamma^{p}_{rk}\Gamma^{k}_{pn}-\frac{1}{N-1}\delta^s_m\Gamma^{p}_{rn}\Gamma^{k}_{pk}\\\\
-\frac{1}{N-1}\delta^s_n \Gamma^{p}_{rk}\Gamma^{k}_{pm}+\frac{1}{N-1}\delta^s_n \Gamma^{p}_{rm}\Gamma^{k}_{pk}\\
\end{matrix}
\right.\\
&=\left\{\begin{matrix}
\left(\underbrace{\Gamma^{s}_{rn} -\frac{1}{N+1}\left[\delta_r^s \Gamma^{k}_{nk} +\delta^s_n \Gamma^{k}_{rk}\right]}_{= P^s_{rn}}-\frac{1}{N^2-1} \delta^s_n \Gamma^{k}_{rk}\right)_{,m}\\\\
- \left(\underbrace{\Gamma^{s}_{rm} -\frac{1}{N+1}\left[\delta_r^s \Gamma^{k}_{mk} +\delta^s_m \Gamma^{k}_{rk}\right]}_{= P^s_{rm}}-\frac{1}{N^2-1} \delta^s_m \Gamma^{k}_{rk}\right)_{,n}\\\\
  +\frac{1}{N-1}\left(\delta^s_n  \Gamma^{k}_{rm}- \delta^s_m \Gamma^{k}_{rn}\right)_{,k}\\\\
 +\frac{1}{N^2-1}\left(\delta^s_m\Gamma^{k}_{nk}  -\delta^s_n \Gamma^{k}_{mk}\right)_{,r} 
  \\\\
+\Gamma^{p}_{rn}\Gamma^{s}_{pm}-\Gamma^{p}_{rm}\Gamma^{s}_{pn}\\\\
+\frac{1}{N-1}\delta^s_m\Gamma^{p}_{rk}\Gamma^{k}_{pn}-\frac{1}{N-1}\delta^s_m\Gamma^{p}_{rn}\Gamma^{k}_{pk}\\\\
-\frac{1}{N-1}\delta^s_n \Gamma^{p}_{rk}\Gamma^{k}_{pm}+\frac{1}{N-1}\delta^s_n \Gamma^{p}_{rm}\Gamma^{k}_{pk}\\
\end{matrix}
\right.
\end{align}
\begin{align}
W^s_{.rmn}&=\left\{\begin{matrix}
P^s_{rn,m }-P^s_{rm,n}\\\\
+\frac{1}{N^2-1} \left[\delta^s_m \Gamma^{k}_{rk,n}- \delta^s_n \Gamma^{k}_{rk,m}
 +\delta^s_m\Gamma^{k}_{nk,r}  -\delta^s_n \Gamma^{k}_{mk,r} \right]
  \\\\
  +\frac{1}{N-1}\left(\delta^s_n  \Gamma^{k}_{rm}- \delta^s_m \Gamma^{k}_{rn}\right)_{,k}\\\\
+\Gamma^{p}_{rn}\Gamma^{s}_{pm}-\Gamma^{p}_{rm}\Gamma^{s}_{pn}\\\\
+\frac{1}{N-1}\delta^s_m\left(\underbrace{\Gamma^{p}_{rk}\Gamma^{k}_{pn}-\Gamma^{p}_{rn}\Gamma^{k}_{pk}}_{=R^k_{.rnk}- \Gamma^k_{rk,n}+\Gamma^k_{rn,k}}\right)-\frac{1}{N-1}\delta^s_n\left(\underbrace{\Gamma^{p}_{rk}\Gamma^{k}_{pm} -\Gamma^{p}_{rm}\Gamma^{k}_{pk}}_{=R^k_{.rmk}- \Gamma^k_{rk,m}+\Gamma^k_{rm,k}}\right)\\
\end{matrix}
\right.\\
&=\left\{\begin{matrix}
P^s_{rn,m }-P^s_{rm,n}\\\\
+\frac{1}{N^2-1} \left[\delta^s_m \Gamma^{k}_{rk,n}- \delta^s_n \Gamma^{k}_{rk,m}
 +\delta^s_m\Gamma^{k}_{nk,r}  -\delta^s_n \Gamma^{k}_{mk,r} \right]
  \\\\
  +\frac{1}{N-1}\left(\cancel{\delta^s_n  \Gamma^{k}_{rm}}- \cancel{\delta^s_m \Gamma^{k}_{rn}}\right)_{,k}\\\\
+\Gamma^{p}_{rn}\Gamma^{s}_{pm}-\Gamma^{p}_{rm}\Gamma^{s}_{pn}\\\\
-\frac{1}{N-1}\delta^s_m\left( \Gamma^k_{rk,n}-\cancel{\Gamma^k_{rn,k}}\right)+\frac{1}{N-1}\delta^s_n\left(\Gamma^k_{rk,m}-\cancel{\Gamma^k_{rm,k}}\right)\\\\
+\frac{1}{N-1}\left(\delta^s_m\underbrace{R^k_{.rnk}}_{=R_{rn}}-\underbrace{\delta^s_nR^k_{.rmk}}_{=R_{rm}}\right)
\end{matrix}
\right.\\
&=\left\{\begin{matrix}
P^s_{rn,m }-P^s_{rm,n}\\\\
+\frac{N}{N^2-1} \left[\delta^s_m \Gamma^{k}_{rk,n}- \delta^s_n \Gamma^{k}_{rk,m}\right]\\\\
 +\frac{1}{N^2-1} \left[\delta^s_m\Gamma^{k}_{nk,r}  -\delta^s_n \Gamma^{k}_{mk,r} \right]
  \\\\
+\Gamma^{p}_{rn}\Gamma^{s}_{pm}-\Gamma^{p}_{rm}\Gamma^{s}_{pn}\\\\
+\frac{1}{N-1}\left(\delta^s_mR_{rn}-\delta^s_nR_{rm}\right)
\end{matrix}
\right.\\
\end{align}
We have
\begin{align}
R_{rn}&= \Gamma^k_{rk,n} - \Gamma^k_{rn,k} +\Gamma^p_{rk} \Gamma^k_{pn} -\Gamma^p_{rn} \Gamma^k_{pk} 
\end{align}
giving
\begin{align}
W^s_{.rmn}&=\left\{\begin{matrix}
P^s_{rn,m }-P^s_{rm,n}\\\\
+\frac{N}{N^2-1} \left[\delta^s_m \Gamma^{k}_{rk,n}- \delta^s_n \Gamma^{k}_{rk,m}\right]\\\\
 +\frac{1}{N^2-1} \left[\delta^s_m\Gamma^{k}_{nk,r}  -\delta^s_n \Gamma^{k}_{mk,r} \right]
  \\\\
+\Gamma^{p}_{rn}\Gamma^{s}_{pm}-\Gamma^{p}_{rm}\Gamma^{s}_{pn}\\\\
+\frac{1}{N-1}\left[\delta^s_m\left(\Gamma^k_{rk,n} - \Gamma^k_{rn,k} +\Gamma^p_{rk} \Gamma^k_{pn} -\Gamma^p_{rn} \Gamma^k_{pk}\right)-\delta^s_n\left(\Gamma^k_{rk,m} - \Gamma^k_{rm,k} +\Gamma^p_{rk} \Gamma^k_{pm} -\Gamma^p_{rm} \Gamma^k_{pk}\right)\right]
\end{matrix}
\right.\\
&=\left\{\begin{matrix}
\underbrace{P^s_{rn,m }-P^s_{rm,n}}_{= P^s_{.rmn}-P^p_{rn}P^s_{pm}+P^p_{rm}P^s_{pn}}\\\\
+\frac{2N+1}{N^2-1} \left[\delta^s_m \Gamma^{k}_{rk,n}- \delta^s_n \Gamma^{k}_{rk,m}\right]\\\\
 +\frac{1}{N^2-1} \left[\delta^s_m\Gamma^{k}_{nk,r}  -\delta^s_n \Gamma^{k}_{mk,r} \right]
  \\\\
+\Gamma^{p}_{rn}\Gamma^{s}_{pm}-\Gamma^{p}_{rm}\Gamma^{s}_{pn}\\\\
+\frac{1}{N-1}\left[\delta^s_m\left( - \Gamma^k_{rn,k} +\Gamma^p_{rk} \Gamma^k_{pn} -\Gamma^p_{rn} \Gamma^k_{pk}\right)-\delta^s_n\left( - \Gamma^k_{rm,k} +\Gamma^p_{rk} \Gamma^k_{pm} -\Gamma^p_{rm} \Gamma^k_{pk}\right)\right]
\end{matrix}
\right.\\
&=\left\{\begin{matrix}
P^s_{.rmn}\\\\
-P^p_{rn}P^s_{pm}+P^p_{rm}P^s_{pn}\\\\
+\frac{2N+1}{N^2-1} \left[\delta^s_m \Gamma^{k}_{rk,n}- \delta^s_n \Gamma^{k}_{rk,m}\right]\\\\
 +\frac{1}{N^2-1} \left[\delta^s_m\Gamma^{k}_{nk,r}  -\delta^s_n \Gamma^{k}_{mk,r} \right]
  \\\\
+\Gamma^{p}_{rn}\Gamma^{s}_{pm}-\Gamma^{p}_{rm}\Gamma^{s}_{pn}\\\\
+\frac{1}{N-1}\left[\delta^s_m\left( - \Gamma^k_{rn,k} +\Gamma^p_{rk} \Gamma^k_{pn} -\Gamma^p_{rn} \Gamma^k_{pk}\right)-\delta^s_n\left( - \Gamma^k_{rm,k} +\Gamma^p_{rk} \Gamma^k_{pm} -\Gamma^p_{rm} \Gamma^k_{pk}\right)\right]
\end{matrix}
\right.\\
\end{align}

We note that from $P^{r}_{mn} = \Gamma ^{r}_{mn}-\frac{1}{N+1}\left(\delta^r_m \Gamma^p_{pn}+  \delta^r_n \Gamma^p_{pm} \right)$
we get
\begin{align}
P^{p}_{rn} P^{s}_{pm} &= \left[\Gamma ^{p}_{rn}-\frac{1}{N+1}\left(\delta^p_r \Gamma^k_{kn}+  \delta^p_n \Gamma^k_{kr} \right)\right]\left[\Gamma ^{s}_{pm}-\frac{1}{N+1}\left(\delta^s_p \Gamma^k_{km}+  \delta^s_m \Gamma^k_{kp} \right)\right] \\
&=\left\{\begin{matrix}
\Gamma ^{p}_{rn}\Gamma ^{s}_{pm}\\\\
-\frac{1}{N+1}\Gamma ^{p}_{rn}\left(\delta^s_p \Gamma^k_{km}+  \delta^s_m \Gamma^k_{kp} \right)\\\\
-\frac{1}{N+1}\Gamma ^{s}_{pm}\left(\delta^p_r \Gamma^k_{kn}+  \delta^p_n \Gamma^k_{kr} \right)\\\\
+\frac{1}{(N+1)^2}\left(\delta^s_p \Gamma^k_{km}+  \delta^s_m \Gamma^k_{kp} \right)\left(\delta^p_r \Gamma^k_{kn}+  \delta^p_n \Gamma^k_{kr} \right)
\end{matrix}\right.\\
&=\left\{\begin{matrix}
\Gamma ^{p}_{rn}\Gamma ^{s}_{pm}\\\\
-\frac{1}{N+1} \Gamma ^{s}_{rn}\Gamma^k_{km}-\frac{1}{N+1}  \delta^s_m \Gamma ^{p}_{rn}\Gamma^k_{kp} \\\\
-\frac{1}{N+1}\Gamma ^{s}_{rm} \Gamma^k_{kn}-\frac{1}{N+1}  \Gamma ^{s}_{nm}\Gamma^k_{kr} \\\\
+\frac{1}{(N+1)^2}\left( \delta^s_r \Gamma^k_{km}\Gamma^k_{kn} +\delta^s_n \Gamma^k_{km}\Gamma^k_{kr}+\delta^s_m \Gamma^k_{kr} \Gamma^k_{kn}+\delta^s_m \Gamma^k_{kn}  \Gamma^k_{kr}  \right)
\end{matrix}\right.
\end{align}
and 
\begin{align}
P^{p}_{rm} P^{s}_{pn} 
&=\left\{\begin{matrix}
\Gamma ^{p}_{rm}\Gamma ^{s}_{pn}\\\\
-\frac{1}{N+1} \Gamma ^{s}_{rm}\Gamma^k_{kn}-\frac{1}{N+1}  \delta^s_n \Gamma ^{p}_{rm}\Gamma^k_{kp} \\\\
-\frac{1}{N+1}\Gamma ^{s}_{rn} \Gamma^k_{km}-\frac{1}{N+1}  \Gamma ^{s}_{mn}\Gamma^k_{kr} \\\\
+\frac{1}{(N+1)^2}\left( \delta^s_r \Gamma^k_{kn}\Gamma^k_{km} +\delta^s_m \Gamma^k_{kn}\Gamma^k_{kr}+\delta^s_n \Gamma^k_{kr} \Gamma^k_{km}+\delta^s_n \Gamma^k_{km}  \Gamma^k_{kr}  \right)
\end{matrix}\right.
\end{align}
hence,
\begin{align}
P^{p}_{rn} P^{s}_{pm}- P^{p}_{rm} P^{s}_{pn} 
&=\left\{\begin{matrix}
\Gamma ^{p}_{rn}\Gamma ^{s}_{pm}-\Gamma ^{p}_{rm}\Gamma ^{s}_{pn}\\\\
+\cancel{\frac{1}{N+1} \Gamma ^{s}_{rm}\Gamma^k_{kn}}+\frac{1}{N+1}  \delta^s_n \Gamma ^{p}_{rm}\Gamma^k_{kp} -\cancel{\frac{1}{N+1} \Gamma ^{s}_{rn}\Gamma^k_{km}}-\frac{1}{N+1}  \delta^s_m \Gamma ^{p}_{rn}\Gamma^k_{kp}\\\\
+\cancel{\frac{1}{N+1}\Gamma ^{s}_{rn} \Gamma^k_{km}}+\cancel{\frac{1}{N+1}  \Gamma ^{s}_{mn}\Gamma^k_{kr}}-\cancel{\frac{1}{N+1}\Gamma ^{s}_{rm} \Gamma^k_{kn}}-\cancel{\frac{1}{N+1}  \Gamma ^{s}_{nm}\Gamma^k_{kr} } \\\\
+\frac{1}{(N+1)^2}\left( \cancel{\delta^s_r \Gamma^k_{km}\Gamma^k_{kn} +\delta^s_n \Gamma^k_{km}\Gamma^k_{kr}+\delta^s_m \Gamma^k_{kr} \Gamma^k_{kn}}+\delta^s_m \Gamma^k_{kn}  \Gamma^k_{kr}  \right)\\\\
-\frac{1}{(N+1)^2}\left( \cancel{\delta^s_r \Gamma^k_{kn}\Gamma^k_{km} +\delta^s_m \Gamma^k_{kn}\Gamma^k_{kr}+\delta^s_n \Gamma^k_{kr} \Gamma^k_{km}}+\delta^s_n \Gamma^k_{km}  \Gamma^k_{kr}  \right)
\end{matrix}\right.\\
&=
\Gamma ^{p}_{rn}\Gamma ^{s}_{pm}-\Gamma ^{p}_{rm}\Gamma ^{s}_{pn}+\frac{1}{N+1}  \left(\delta^s_n \Gamma ^{p}_{rm}\Gamma^k_{kp} -  \delta^s_m \Gamma ^{p}_{rn}\Gamma^k_{kp}\right)
+\frac{1}{(N+1)^2}\left(\delta^s_m \Gamma^k_{kn}  \Gamma^k_{kr} 
-\delta^s_n \Gamma^k_{km}  \Gamma^k_{kr}  \right)
\end{align}
also
\begin{align}
P_{rm}&=-P^s_{rm,s}+P^p_{rs}P^s_{pm}
\end{align}
with
\begin{align}
P^p_{rs}P^s_{pm}&=\left[\Gamma ^{p}_{rs}-\frac{1}{N+1}\left(\delta^p_r \Gamma^k_{ks}+  \delta^p_s\Gamma^k_{kr} \right)\right]\left[\Gamma ^{s}_{pm}-\frac{1}{N+1}\left(\delta^s_p \Gamma^k_{km}+  \delta^s_m \Gamma^k_{kp} \right)\right] \\
&=\left\{\begin{matrix}
\Gamma ^{p}_{rs}\Gamma ^{s}_{pm}\\\\
-\frac{1}{N+1} \Gamma ^{s}_{rs}\Gamma^k_{km}-\frac{1}{N+1}  \Gamma ^{p}_{rm}\Gamma^k_{kp}\\\\
-\frac{1}{N+1} \Gamma ^{s}_{rm}\Gamma^k_{ks}-\frac{1}{N+1} \Gamma ^{s}_{sm}\Gamma^k_{kr}\\\\
+\frac{1}{(N+1)^2}\left(\Gamma^k_{kr} \Gamma^k_{km}+\Gamma^k_{kr} \Gamma^k_{km}+ N\Gamma^k_{kr}\Gamma^k_{km} +\Gamma^k_{kr}  \Gamma^k_{km}  \right)
\end{matrix}\right.\\
\end{align}
giving
\begin{align}
W^s_{.rmn}&=\left\{\begin{matrix}
P^s_{.rmn}\\\\
\cancel{-\Gamma ^{p}_{rn}\Gamma ^{s}_{pm}+\Gamma ^{p}_{rm}\Gamma ^{s}_{pn}}\\\\
-\frac{1}{N+1}  \left(\delta^s_n \Gamma ^{p}_{rm}\Gamma^k_{kp} +  \delta^s_m \Gamma ^{p}_{rn}\Gamma^k_{kp}\right)\\\\
-\frac{1}{(N+1)^2}\left(\delta^s_m \Gamma^k_{kn}  \Gamma^k_{kr} 
+\delta^s_n \Gamma^k_{km}  \Gamma^k_{kr}  \right)\\\\
+\frac{2N+1}{N^2-1} \left[\delta^s_m \Gamma^{k}_{rk,n}- \delta^s_n \Gamma^{k}_{rk,m}\right]\\\\
 +\frac{1}{N^2-1} \left[\delta^s_m\Gamma^{k}_{nk,r}  -\delta^s_n \Gamma^{k}_{mk,r} \right]
  \\\\
\cancel{+\Gamma^{p}_{rn}\Gamma^{s}_{pm}-\Gamma^{p}_{rm}\Gamma^{s}_{pn}}\\\\
+\frac{1}{N-1}\left[\delta^s_m\left( - \Gamma^k_{rn,k} +\Gamma^p_{rk} \Gamma^k_{pn} -\Gamma^p_{rn} \Gamma^k_{pk}\right)-\delta^s_n\left( - \Gamma^k_{rm,k} +\Gamma^p_{rk} \Gamma^k_{pm} -\Gamma^p_{rm} \Gamma^k_{pk}\right)\right]
\end{matrix}
\right.\\
&=\left\{\begin{matrix}
P^s_{.rmn}\\\\
-\frac{1}{N^2-1}  \left[(N-1)\delta^s_n \Gamma ^{p}_{rm}\Gamma^k_{kp} +  (N-1)\delta^s_m \Gamma ^{p}_{rn}\Gamma^k_{kp}-(N+1)\delta^s_n\Gamma^p_{rm} \Gamma^k_{pk} +(N+1)\delta^s_m\Gamma^p_{rn} \Gamma^k_{pk}\right]\\\\
-\frac{1}{(N+1)^2}\left(\delta^s_m \Gamma^k_{kn}  \Gamma^k_{kr} 
+\delta^s_n \Gamma^k_{km}  \Gamma^k_{kr}  \right)\\\\
+\frac{2N+1}{N^2-1} \left[\delta^s_m \Gamma^{k}_{rk,n}- \delta^s_n \Gamma^{k}_{rk,m}\right]\\\\
 +\frac{1}{N^2-1} \left[\delta^s_m\Gamma^{k}_{nk,r}  -\delta^s_n \Gamma^{k}_{mk,r} \right]
  \\\\
+\frac{1}{N-1}\left[\delta^s_m\left( - \Gamma^k_{rn,k} +\Gamma^p_{rk} \Gamma^k_{pn}\right)-\delta^s_n\left( - \Gamma^k_{rm,k} +\Gamma^p_{rk} \Gamma^k_{pm} \right)\right]
\end{matrix}
\right.\\
&=\left\{\begin{matrix}
P^s_{.rmn}\\\\
-\frac{1}{N^2-1}  \left[( 2N\delta^s_m \Gamma ^{p}_{rn}\Gamma^k_{kp}-2\delta^s_n\Gamma^p_{rm} \Gamma^k_{pk} \right]\\\\
-\frac{1}{(N+1)^2}\left(\delta^s_m \Gamma^k_{kn}  \Gamma^k_{kr} 
+\delta^s_n \Gamma^k_{km}  \Gamma^k_{kr}  \right)\\\\
+\frac{2N+1}{N^2-1} \left[\delta^s_m \Gamma^{k}_{rk,n}- \delta^s_n \Gamma^{k}_{rk,m}\right]\\\\
 +\frac{1}{N^2-1} \left[\delta^s_m\Gamma^{k}_{nk,r}  -\delta^s_n \Gamma^{k}_{mk,r} \right]
  \\\\
+\frac{1}{N-1}\left[\delta^s_m\left( - \Gamma^k_{rn,k} +\Gamma^p_{rk} \Gamma^k_{pn}\right)-\delta^s_n\left( - \Gamma^k_{rm,k} +\Gamma^p_{rk} \Gamma^k_{pm} \right)\right]
\end{matrix}
\right.\\
\end{align}
\begin{align}
P^p_{rs}P^s_{pm}&=\left[\Gamma ^{p}_{rs}-\frac{1}{N+1}\left(\delta^p_r \Gamma^k_{ks}+  \delta^p_s\Gamma^k_{kr} \right)\right]\left[\Gamma ^{s}_{pm}-\frac{1}{N+1}\left(\delta^s_p \Gamma^k_{km}+  \delta^s_m \Gamma^k_{kp} \right)\right] \\
&=\left\{\begin{matrix}
\Gamma ^{p}_{rs}\Gamma ^{s}_{pm}\\\\
-\frac{1}{N+1} \Gamma ^{s}_{rs}\Gamma^k_{km}-\frac{1}{N+1}  \Gamma ^{p}_{rm}\Gamma^k_{kp}\\\\
-\frac{1}{N+1} \Gamma ^{s}_{rm}\Gamma^k_{ks}-\frac{1}{N+1} \Gamma ^{s}_{sm}\Gamma^k_{kr}\\\\
+\frac{1}{(N+1)^2}\left(\Gamma^k_{kr} \Gamma^k_{km}+\Gamma^k_{kr} \Gamma^k_{km}+ N\Gamma^k_{kr}\Gamma^k_{km} +\Gamma^k_{kr}  \Gamma^k_{km}  \right)
\end{matrix}\right.\\
\end{align}
giving
\begin{align}
W^s_{.rmn}&=\left\{\begin{matrix}
P^s_{.rmn}\\\\
-\frac{1}{N^2-1}  \left[ 2N\delta^s_m \Gamma ^{p}_{rn}\Gamma^k_{kp}-2\delta^s_n\Gamma^p_{rm} \Gamma^k_{pk} \right]\\\\
-\frac{1}{(N+1)^2}\left(\delta^s_m \Gamma^k_{kn}  \Gamma^k_{kr} 
+\delta^s_n \Gamma^k_{km}  \Gamma^k_{kr}  \right)\\\\
+\frac{2N+1}{N^2-1} \left[\delta^s_m \Gamma^{k}_{rk,n}- \delta^s_n \Gamma^{k}_{rk,m}\right]\\\\
 +\frac{1}{N^2-1} \left[\delta^s_m\Gamma^{k}_{nk,r}  -\delta^s_n \Gamma^{k}_{mk,r} \right]
  \\\\
+\frac{1}{N^2-1}\left[-(N+1) \delta^s_m\Gamma^k_{rn,k} +(N+1)\delta^s_m\Gamma^p_{rk} \Gamma^k_{pn}+(N+1)\delta^s_n\Gamma^k_{rm,k} -(N+1)\delta^s_n\Gamma^p_{rk} \Gamma^k_{pm} \right]
\end{matrix}
\right.\\
&=\left\{\begin{matrix}
P^s_{.rmn}\\\\
-\frac{1}{N^2-1}  \left[ 2N\delta^s_m \Gamma ^{p}_{rn}\Gamma^k_{kp}-2\delta^s_n\Gamma^p_{rm} \Gamma^k_{pk}-(N+1)\delta^s_m\Gamma^p_{rk} \Gamma^k_{pn} +(N+1)\delta^s_n\Gamma^p_{rk} \Gamma^k_{pm} \right]\\\\
-\frac{1}{(N+1)^2}\left(\delta^s_m \Gamma^k_{kn}  \Gamma^k_{kr} 
+\delta^s_n \Gamma^k_{km}  \Gamma^k_{kr}  \right)\\\\
+\frac{2N+1}{N^2-1} \left[\delta^s_m \Gamma^{k}_{rk,n}- \delta^s_n \Gamma^{k}_{rk,m}\right]\\\\
 +\frac{1}{N^2-1} \left[\delta^s_m\Gamma^{k}_{nk,r}  -\delta^s_n \Gamma^{k}_{mk,r}-(N+1)\delta^s_m\Gamma^k_{rn,k} +(N+1)\delta^s_n\Gamma^k_{rm,k}\right]
  \\
\end{matrix}
\right.\\
\end{align}
\end{comment}
$$\blacklozenge$$
\newpage



\section{p310 - Exercise 13}
\begin{tcolorbox}
Show that 
$$\begin{matrix}
W^s_{.smn} =0,\quad W^s_{.rsn} =0,\quad W^s_{.rms} =0,\\
W^s_{.rmn} =-W^s_{.rnm} ,\quad W^s_{.rmn}+W^s_{.mnr}+W^s_{.nrm}=0,
\end{matrix}$$
\end{tcolorbox}
\begin{align*}
W^s_{.smn}&=\underbrace{P^s_{.smn}}_{=0}+\frac{1}{N-1}\left(\underbrace{\delta^s_m P_{sn}-\delta^s_nP_{sm}}_{=P_{mn}-P_{nm} =0}\right)\\
&=0
\end{align*}
\begin{align*}
W^s_{.rsn}&=P^s_{.rsn}+\frac{1}{N-1}\left(\delta^s_s P_{rn}-\delta^s_nP_{rs}\right)\\
&=P^s_{.rsn}+\underbrace{P_{rn}}_{= P^s_{.rns}}\\
\end{align*}
but $P^s_{.rmn}$ is skew-symmetric in the last two suffixes, so
$$W^s_{.rsn}=0$$
The same reasoning holds for $$W^s_{.rms}=0$$
$$\lozenge$$
\begin{align*}
W^s_{.rmn}+W^s_{.mnr}+W^s_{.nrm}&=\left\{\begin{matrix}
\frac{1}{N-1} \left(\delta^s_mP_{rn}-\delta^s_nP_{rm}+\delta^s_nP_{mr}-\delta^s_rP_{mn}+\delta^s_rP_{nm}-\delta^s_mP_{nr} \right)\\\\
+P^s_{.rn,m}-P^s_{.rm,n}+P^p_{.rn}P^s_{.pm}-P^p_{.rm}P^s_{.pn}\\\\
+P^s_{.mr,n}-P^s_{.mn,r}+P^p_{.mr}P^s_{.pm}-P^p_{.mn}P^s_{.pr}\\\\
+P^s_{.nm,r}-P^s_{.nr,m}+P^p_{.nm}P^s_{.pr}-P^p_{.nr}P^s_{.pm}\\
\end{matrix}\right.\\
&=0
\end{align*}
$$\blacklozenge$$
\newpage



\section{p310 - Exercise 14}
\begin{tcolorbox}
In a space with linear connection, we say that the \textit{directions} of two vectors, $X^r$ at a point $A$ and $Y^r$ at a point $B$, are parallel with respect to a curve $C$ which joins $A$ and $B$ if the vector obtained by parallel propagation of $X^r$ along $C$ from $A$ to $B$ is a multiple of $Y^r$. prove that the most general change of linear connection which preserves parallelism of directions (with respect to all curves) is given by 
$$\Gamma^{'r}_{\ mn} = \Gamma^r_{mn} + \delta^r_m \psi_n$$
where $\psi_n$ is an arbitrary vector. If $\Gamma^r_{mn}$ are the coefficients of a symmetric connection, show that $\Gamma^{'r}_{\ mn}$ are semi-symmetric(cf. Exercise 4).
\end{tcolorbox}
Let's propagate parallely the vector $X^r$ along the same curve $C$ but with two different linear connections $\Gamma^r_{mn}$ and $\Gamma^{'r}_{\ mn}$ . 
So we have

\begin{align}
&\left\{\begin{matrix}
X^r_{\ ,n}+ \Gamma^r_{mn}X^m=0\\\\
X^{r}_{, mn}+ \Gamma^{'r}_{mn}X^m=0
\end{matrix}\right.\\
\end{align}
Let's evaluate these expressions at the point $B$, requiring in both cases that at this point $X^r=\lambda_{(1)} Y^r$ for $\Gamma^r_{mn}$ and $X^r=\lambda_{(2)} Y^r$ for $\Gamma^{'r}_{\ mn}$ where $Y^r$ is a single valued vector at this point. 
We get,
\begin{align}
&\left\{\begin{matrix}
\left(\lambda_{(1)} Y^r\right)_{, n}+ \Gamma^r_{mn}\lambda_{(1)} Y^m=0\\\\
\left(\lambda_{(2)}Y^r\right)_{, n}+ \Gamma^{'r}_{mn}\lambda_{(2)} Y^m=0
\end{matrix}\right.
\end{align}
As $Y^r$ is fixed, the partial derivatives can be reduced to $\left(\lambda_{(.)} \right)_{, n}Y^r + 2^{nd}\text{  order terms}$ and we rewrite $(3)$ as
\begin{align}
&\left\{\begin{matrix}
\left(\Gamma^r_{mn}\lambda_{(1)}+\delta^r_m\left(\lambda_{(1)}\right)_{, n}  \right) Y^m=0\\\\
\left(\Gamma^{'r}_{mn}\lambda_{(2)}+\delta^r_m\left(\lambda_{(2)}\right)_{, n}  \right) Y^m=0
\end{matrix}\right.\\
\Leftrightarrow \spatie &\left\{\begin{matrix}
\left(\Gamma^r_{mn}+\delta^r_m\frac{\left(\lambda_{(1)}\right)_{, n}}{\lambda_{(1)}}  \right) Y^m=0\\\\
\left(\Gamma^{'r}_{mn}+\delta^r_m\frac{\left(\lambda_{(2)}\right)_{, n}}{\lambda_{(2)}} \right) Y^m=0
\end{matrix}\right.\\
\Rightarrow \spatie &\Gamma^{'r}_{mn}= \Gamma^r_{mn}+\delta^r_m\left[\frac{\left(\lambda_{(1)}\right)_{, n}}{\lambda_{(1)}}-\frac{\left(\lambda_{(2)}\right)_{, n}}{\lambda_{(2)}}  \right]
\end{align}
Put $\psi_n = \half\left(\frac{\left(\lambda_{(1)}\right)_{, n}}{\lambda_{(1)}}-\frac{\left(\lambda_{(2)}\right)_{, n}}{\lambda_{(2)}}\right) $
and we get $$\Gamma^{'r}_{mn}= \Gamma^r_{mn}+2\delta^r_m\psi_n  $$
$$\lozenge$$
If $\Gamma^r_{mn}$ is symmetric then,
$$\Gamma^{'r}_{mn}-\Gamma^{'r}_{nm}= \delta^r_m\left[\frac{\left(\lambda_{(1)}\right)_{, n}}{\lambda_{(1)}}-\frac{\left(\lambda_{(2)}\right)_{, n}}{\lambda_{(2)}}  \right]-\delta^r_n\left[\frac{\left(\lambda_{(1)}\right)_{, m}}{\lambda_{(1)}}-\frac{\left(\lambda_{(2)}\right)_{, m}}{\lambda_{(2)}}  \right] $$
which is of the form $\Gamma^r_{mn}-\Gamma^r_{nm} = \delta^r_mA_n - \delta^r_n A_m$ as required by Exercise 4. 
$$\blacklozenge$$
\newpage


\section{p310 - Exercise 15}
\begin{tcolorbox}
In a space with symmetric connection, show that 
$$T^r_{\ |mn} - T^r_{\ |nm } = - T^s R^r_{.smn}$$
\end{tcolorbox}
We have 
\begin{align*}
&\left\{\begin{matrix}
T^r_{\ |mn} = \pdv{}{x^n}T^r_{\ |m} + \Gamma^r_{qn}T^q_{\ |m} -\Gamma^q_{mn}T^r_{\ |q}\\\\
T^r_{\ |nm} = \pdv{}{x^m}T^r_{\ |n} + \Gamma^r_{qm}T^q_{\ |n} -\Gamma^q_{nm}T^r_{\ |q} 
\end{matrix}\right.
\end{align*}
giving with $ T^q_{\ |m}= T^q_{,m}+ \Gamma^q_{km}T^k$ and  $ T^q_{\ |n}= T^q_{,n}+ \Gamma^q_{kn}T^k$
\begin{align*}
T^r_{\ |mn}-T^r_{\ |nm}= &\left\{\begin{matrix}
T^s\left(\pdv{}{x^n}\Gamma^r_{ms}-\pdv{}{x^m}\Gamma^r_{ns}\right)\\\\
+ \cancel{\Gamma^r_{mq}T^q_{,n}-\Gamma^r_{nq}T^q_{,m}}
\\\\
  +\cancel{\Gamma^r_{qn}T^q_{,m}+\Gamma^r_{pn} \Gamma^p_{sm}T^s}\\\\
 +\Gamma^r_{pn}\Gamma^p_{sm}T^s- \Gamma^r_{pm}\Gamma^p_{sn}T^s
\end{matrix}\right.\\
\mathbf{(8.214)}\spatie &= -T^sR^r_{.smn}
\end{align*}
$$\blacklozenge$$
\newpage
